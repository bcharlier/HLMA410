\documentclass{tp_um}
\makeatletter
%--------------------------------------------------------------------------------

\usepackage[frenchb]{babel}

\usepackage{amsmath}
\usepackage{amsbsy}
\usepackage{amsfonts}
\usepackage{amssymb}
\usepackage{amscd}
\usepackage{amsthm}
\usepackage{mathtools}
\usepackage{eurosym}
\usepackage{nicefrac}

\usepackage{latexsym}
\usepackage[a4paper,hmargin=20mm,vmargin=25mm]{geometry}
\usepackage{dsfont}
\usepackage[utf8]{inputenc}
\usepackage[T1]{fontenc}

\usepackage{multicol}
\usepackage[inline]{enumitem}
%\setlist{nosep}
\setlist[itemize,1]{,label=$-$}

\usepackage{sectsty}
%\sectionfont{}
\allsectionsfont{\normalfont\sffamily\bfseries\normalsize}

\usepackage{graphicx}
\usepackage{tikz}

\usepackage{pgfplots}
\usepgfplotslibrary{fillbetween}
\pgfplotsset{compat=newest}
%\usepgfplotslibrary{external} 
%\tikzexternalize[prefix=./output_latex/]
%\DeclareSymbolFont{RalphSmithFonts}{U}{rsfs}{m}{n}
%\DeclareSymbolFontAlphabet{\mathscr}{RalphSmithFonts}
%\def\mathcal#1{{\mathscr #1}}

\newcounter{zut}
\setcounter{zut}{1}
\newcommand{\exo}[1]{\noindent {\sffamily\bfseries Exercice~\thezut. #1} \
		   \addtocounter{zut}{1}}



\providecommand{\abs}[1]{\left|#1\right|}
\providecommand{\norm}[1]{\left\Vert#1\right\Vert}
\providecommand{\U}{\mathcal{U}}
\providecommand{\R}{\mathbb{R}}
\providecommand{\Cc}{\mathcal{C}}
\providecommand{\reg}[1]{\mathcal{C}^{#1}}
\providecommand{\1}{\mathds{1}}
\providecommand{\N}{\mathbb{N}}
\providecommand{\Z}{\mathbb{Z}}
\providecommand{\E}{\mathbb{E}}
\providecommand{\p}{\partial}
\providecommand{\one}{\mathds{1}}
\renewcommand{\P}{\mathbb{P}}


%Operateur
\providecommand{\abs}[1]{\left\lvert#1\right\rvert}
\providecommand{\sabs}[1]{\lvert#1\rvert}
\providecommand{\babs}[1]{\bigg\lvert#1\bigg\rvert}
\providecommand{\norm}[1]{\left\lVert#1\right\rVert}
\providecommand{\bnorm}[1]{\bigg\lVert#1\bigg\rVert}
\providecommand{\snorm}[1]{\lVert#1\rVert}
\providecommand{\prs}[1]{\left\langle #1\right\rangle}
\providecommand{\sprs}[1]{\langle #1\rangle}
\providecommand{\bprs}[1]{\bigg\langle #1\bigg\rangle}

\DeclareMathOperator{\deet}{Det}
\DeclareMathOperator{\vol}{Vol}
\DeclareMathOperator{\aire}{Aire}
\DeclareMathOperator{\hess}{Hess}
\DeclareMathOperator{\var}{Var}

%------------------------------------------------------------------------------
\DeclareUnicodeCharacter{00A0}{~}
\makeatother


\newcommand{\miniscule}{\@setfontsize\miniscule{5}{6}}
%-----------------------------------------------------------------------------

\title{\Large \sffamily\bfseries Intégrales doubles, triples et curvilignes}
\ue{HLMA410}


%-----------------------------------------------------------------------------
\begin{document}

\maketitle


\bigskip

Les exercices ou les questions marqués d'une étoile ne sont pas prioritaires.


%\section{Intégrales à paramètres}

%\exo{(Intégrale de Gauss)} %Le but de l'exercice est de calculer la valeur de l'intégrale de Gauss 
%%\[
	%%
%%\]
%Soient $f,g$ deux fonctions de $\R$ dans $\R$ définies par 
%\[
	%f(x) = \int_0^x e^{-t^2} dt \qquad \text{ et } \qquad g(x)= \int_0^1 \frac{e^{-(t^2+1)x^2}}{t^2+1} dt.
%\]
%\begin{enumerate}
	%\item Montrer que $g(x)  + f^2(x) = \frac{\pi}{4}$ pour tout $x\in\R$.
	%\item Montrer que $\lim_{x\to +\infty} g(x) = 0 $.
	%\item En déduire la valeur de $I = \int_0^{+\infty} e^{-t^2} dt$.
%\end{enumerate}
%%bibmath exo 3

%\bigskip


%%\exo{(Fonction de Bessel)} Soit $f:\R\to\R$ définie par $f(x) = \int_0^\pi \cos(x\sin\theta) d\theta$.
%%\begin{enumerate}
	%%\item Montrer que $f$ est de classe $\reg{2}$ sur $\R$.
	%%\item Vérifier que $f$ est solution de l'équation différentielle suivante 
		%%\[
			%%xf''(x) + f(x) + xf(x) =0,  \quad \text{ pour tout } x\in\R.
		%%\]
%%\end{enumerate}
%%%bibmath exo 1

\section{Intégrales multiples}



\exo[*]{} Soit $D=[0,1]^2$. Calculer: $\displaystyle{\iint_D \frac{dx\,dy}{(x+y+1)^2}}$.
%L2PC ex1
	\begin{center}
			\begin{tikzpicture}[scale=.5]
				\begin{axis}[xlabel=$x$,ylabel=$y$]%,xtick=\empty,ytick=\empty,ztick=\empty ]
					\addplot3[surf,opacity=.7,samples=50,domain=0:1] gnuplot {1 / (1 +x + y)**2};
				\end{axis}
			\end{tikzpicture}
		\end{center}
\bigskip


\exo{ } Calculer l'intégrale $\iint_D e^{x^2} dx dy$ où $D = \left\{ (x,y) \in \R^2 | 0\leq y \leq x \leq 1 \right\}.$ 
%LM p253

\exo{} Calculer les intégrales multiples suivantes :
\begin{enumerate}
	\item $I_1 = \displaystyle \iint_{D} (x + y)e^{-x}e^{-y}\, dx\, dy$ o\`u
		$D  = \left\{ (x, y)\in \R^{2} \; | \; x, y\geq 0, x + y \leq 1\right\}$.
	\begin{center}
			\begin{tikzpicture}[scale=.5]
				\begin{axis}[xlabel=$x$,ylabel=$y$]%,xtick=\empty,ytick=\empty,ztick=\empty ]
					\addplot3[surf,opacity=.7,samples=50,domain=0:2] gnuplot {(x + y)*exp(-x)*exp(-y)};
				\end{axis}
			\end{tikzpicture}
		\end{center}

	\item $I_2 = \displaystyle \iint_D x\sin y \,dx \, dy$, o\`u 
		$D=\{(x,y)\in \R^2 \; | \; 0\leq y\leq\frac{\pi}{2},0\leq x\leq \cos y\}$.
		Dessiner !
	\begin{center}
			\begin{tikzpicture}[scale=.5]
				\begin{axis}[xlabel=$x$,ylabel=$y$]%,xtick=\empty,ytick=\empty,ztick=\empty ]
					\addplot3[surf,opacity=.7,samples=50,domain=-pi:pi] gnuplot {x * sin(y)};
				\end{axis}
			\end{tikzpicture}
		\end{center}

	%\item $I_3 = \displaystyle \iint_{D} \frac{xy}{1 + x^{2} + y^{2}}\, dx\, dy$ où $D  = \left\{ (x, y)\in [0, 1]^{2} \; | \; x^{2} + y^{2} \geq 1\right\}$.
	%\begin{center}
			%\begin{tikzpicture}[scale=.5]
				%\begin{axis}[xlabel=$x$,ylabel=$y$]%,xtick=\empty,ytick=\empty,ztick=\empty ]
					%\addplot3[surf,opacity=.7,samples=50,domain=-3:3] gnuplot {x * y /(1+x**2+y**2)};
				%\end{axis}
			%\end{tikzpicture}
		%\end{center}
		
	%\item	$I_4 = \displaystyle \iiint_{D} z \, dx\, dy\, dz$ o\`u 
		%$D  = \left\{ (x, y, z)\in (\R_{ + })^{3} \;  | \; z \leq \min ( 1 -x^{2}, 1 -y^2 ) \right\}$.
\end{enumerate}
%L2PC ex2

\exo{} Soit $D=\{ (x,y,z)\in \R^3 \; | \; x\ge 0, y\ge 0, z\ge 0, x+y+z\le 1\}$. Calculer:
$\displaystyle \iiint_D \frac{dx\, dy\, dz}{(1+x+y+z)^3}$.
%L2PC 3 

\bigskip

\exo{}Calculer l'aire de la région grisée de la figure suivante :
\begin{center}
	\begin{tikzpicture}[scale=.7]
			\def\xone{-4};\def\xtwo{4};\def\yone{-4};\def\ytwo{4}
% grid
			\draw[step=1cm,help lines] (\xone,\yone) grid (\xtwo,\ytwo);
			\draw[thick,->] (\xone-.3, 0) -- (\xtwo+.3, 0) node (a) [right] {$x$};
			\draw[thick,->] (0, \yone-.3) -- (0, \ytwo+.3) node[above] {$y$};

			\draw[dashed,black] (-4,-2) -- (2,4) node[above] {$y=x+2$};
			\draw[dashed,black] (-4,-2/3) -- (4,-10/3) node[below] {$3y=-x-6$};
			\draw[dashed,black] (-4,-.5) -- (4,7/2) node[below right] {$2y=x+3$};

			\coordinate (a) at (-1,1);
			\coordinate (b) at (3,3);
			\coordinate (c) at (3,-3);
			\coordinate (d) at (-3,-1);
		
			\draw[black,fill=gray!50,fill opacity=.5] (a) -- (b) -- (c) -- (d) -- cycle;
	\end{tikzpicture}
\end{center}
% faccanoni ex 6.11 p 210


\bigskip


%\exo{} Soit le domaine $\mathcal D = \left\{ (x,y) \in \R^2\   | \    (2x^2 -1) - 2 \abs{x} (y+1) + (y+1)^2 \leq 0 \right\}$
%\begin{center}
	%\begin{tikzpicture}[scale=.8]
		%\begin{axis}
			%\addplot[very thick, domain=-1:1,samples=500]{-1 + abs(x) + (1 - x^2)^.5 };
			%\addplot[very thick,domain=-1:1,samples=500]{-1 + abs(x) - (1 - x^2)^.5 };
		%\end{axis}
	%\end{tikzpicture}
%\end{center}
%\begin{enumerate}
	%\item Déterminer deux fonctions $\varphi,\phi:[-1,1] \to \R$ telles que \[\mathcal D = \left\{ (x,y)\in\R^2 | -1\leq x \leq 1, \varphi(x)\leq y \leq \phi(x) \right\}.\]
	%\item Calculer une primitive de $x\mapsto \sqrt{1-x^2}$.
	%\item Calculer l'aire de $\mathcal D$
%\end{enumerate}
%% faccanoni ex 6.19 p 215

\section{Changement de coordonées}

\exo{} En utilisant un changement de variables, calculer l'int\'egrale de $f$ sur $D$ avec
\begin{modenumerate}
\item $D=\{ (x,y)\in \R^2 \mid \pi^2 < x^2+y^2 \leq 4\pi^2 \}\; ;\; f(x,y)=\sin \sqrt{x^2+y^2}$ ;
%\item $D=\left\{ (x,y)\in \R^2 \mid \frac{x^2}{a^2}+\frac{y^2}{b^2} \leq 1 \right\}\; \mbox{avec }a\, ,\, b>0\; ;\; f(x,y)=x^2+y^2$ ;
\item $D=\{ (x,y,z)\in \R^3 \mid x^2+y^2 \leq 1\, ,\, 0\leq z\leq h \}\; \mbox{avec }h>0 \; ;\; f(x,y,z)=z$ ;
%\item $D=\{ (x,y)\in \R^2 \mid 
%0 < x^2\leq y\leq 2x^2\, ,\, 1/x\leq y \leq 2/x \}\; ;\; f(x,y)=x+y$  
%(changement de variable $u=y/x^2\, ,\, v=xy$) ;
%\item $D=\{ (x,y,z)\in \R^3 \mid 
%x\geq 0\, ,\, y\geq 0\, ,\, z\geq 0\, ,\, x^2+y^2+z^2 \leq 1 \}\; ;\; 
%f(x,y,z)=xyz $ ;
    \moditem{*} $D=\{ (x,y,z)\in \R^3 \mid 1\leq x^2+y^2+z^2 \leq 4 \}\; ;\; f(x,y,z)=(x^2+y^2+z^2)^\alpha $.
\end{modenumerate}
%L2PC 9

\bigskip

%\exo{} Soient $0<a\le b, 0<c\le d,$ et $D= \{ax^2\le y\le bx^2, \frac cx\le y\le \frac d x\}$. Calculer l'aire de $D$.
%{\it Indication: poser $u=\frac y{x^2}$ et $v=xy$.}
%%L2PC 13

\bigskip

\exo{}
Soit $R > 1$. On considère le domaine du plan  \[\mathcal D = \left\{ (x,y) \in \R^2 | y \geq 0 \text{ et } 1 \leq x^2 + y^2  \leq R^2 \right\} \cup \left\{ (x,y)\in\R^2 | xy\geq 0, \text{ et } 1 \leq x^2+y^2 \leq R^2\right\}.\]
\begin{enumerate}
	\item Représenter graphiquement le domaine $\mathcal D$.
	\item Calculer les coordonnées du centre de gravité de  $\mathcal D$ (on suppose que $\mathcal D$ est de densité uniforme).
	\item À partir de quelle valeur de $R$ le centre de gravité appartient à $\mathcal D$ ?
\end{enumerate}
% faccanoni ex 6.32 p 228

\bigskip

\exo{}Calculer le volume :
\begin{modenumerate}
	\item du solide en dessous du c\^one $C : z=\sqrt{ x^2+y^2}$ et au dessus de la couronne $A: z=0$ et $ 4\leq x^2+y^2\leq 25$. Dessiner!
            \moditem{*} du solide qui est \`a la fois \`a l'int\'erieur du cylindre $x^2+y^2=4$ et de l'ellipsoide $4x^2+4y^2+z^2=64$. Dessiner!
\end{modenumerate}
%L2PC 8 et 9

\bigskip



\exo[*]{(Intégrale de Gauss)} Soit $R>0$, $D_R=\{x^2+y^2\le R^2, x>0, y>0\}$ et $K_R=[0,R]^2$.
\begin{enumerate}
    \item Montrer que~: 
        \[\iint_{D_R} e^{-(x^2+y^2)}\, dx\, dy \le \iint_{K_R} e^{-(x^2+y^2)}\, dx\, dy \le
        \iint_{D_{2R}} e^{-(x^2+y^2)}\, dx\, dy.\]

    \item En d\'eduire l'existence et la valeur de $\displaystyle \lim_{R\rightarrow +\infty} \int_0^R e^{-t^2}\, dt$.
\end{enumerate}
%%L2PC ex 15


\section{Champ de gradient}

\exo{}
Soit $A = (1,0)$ et $B = (0,1)$ deux points du plan. On considère le champ de vecteurs $V(x,y) = (2xy+y^2-1 , 2xy + x^2) $ défini sur $\R^2$ :
\begin{enumerate}
	\item Exprimer l'intégrale curviligne de $V$ le long du segment reliant les points $A$ et  $B$ (orienté de $A$ vers $B$), puis calculer cette intégrale.
	\item Trouver une  fonction $f:\R^2 \to \R$ qui satisfait $ \nabla f = V$.
	\item Calculer alors l'intégrale curviligne de $V$ le long de la courbe $\Gamma$ de paramétrisation $\phi :t \to (\cos^5(t),\sin^4(t))$, $t \in [0; \frac{\pi}{2}]$.
\end{enumerate}
%exo 8.22 faccanoni

\bigskip

\exo{ }Calculer la circulation du champ de vecteur  $F(x, y, z) = (yz, zx, xy)$ le long de l'hélice $H$ paramétrée par $t \mapsto (\cos t, \sin t, t)$ où $t$ varie de $0$ à $\frac \pi 4$ . 
% fic0158 exo 7

\bigskip

\exo{}On considère le champ de vecteurs $V(x,y) = (-y , x) \frac{1}{x^2+y^2}$.
\begin{center}
	\begin{tikzpicture}[scale=.8]
		\begin{axis}[,xtick=\empty,ytick=\empty,ztick=\empty ,xlabel=$x$,ylabel=$y$,xlabel=$x$,ylabel=$y$,domain=-3:3, view={0}{90}]
			\addplot3[blue, quiver={u={y/(x^2+y^2)}, v={-x/(x^2+y^2)}, scale arrows=.4}, -stealth,samples=10] {0};
		\end{axis}
	\end{tikzpicture}
\end{center}
\begin{enumerate}
	\item Quel est le domaine de définition de $V$ ?
	\item Calculer la circulation du champ de vecteurs $V$ le long du cercle unité parcouru dans le sens direct.
	%\item Le champ de vecteurs $V$ est-il un champs de gradient?
\end{enumerate}
%exo 10 de fic 00158

%\section{Application de la formule de Green-Riemann}

%\exo{}
%En utilisant la formule de Green-Riemann, calculer $I = \iint_D xy dxdy$ où $ D = \{ (x,y) \in \R^2 | x>0,\    y>0,\    x+y < 1\}$
%%exo 9 de fic00158
\end{document}

