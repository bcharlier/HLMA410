\documentclass{tp_um}
\makeatletter
%--------------------------------------------------------------------------------

\usepackage[french]{babel}
\usepackage{amsmath}
\usepackage{amsbsy}
\usepackage{amsfonts}
\usepackage{amssymb}
\usepackage{amscd}
\usepackage{amsthm}
\usepackage{mathtools}
\usepackage{eurosym}
\usepackage{nicefrac}

\usepackage{latexsym}
\usepackage[a4paper,hmargin=20mm,vmargin=25mm]{geometry}
\usepackage{dsfont}
\usepackage[utf8]{inputenc}
\usepackage[T1]{fontenc}
\usepackage{lmodern}

\usepackage{multicol}
\usepackage[inline]{enumitem}
\setlist{nosep}
\setlist[itemize,1]{,label=$-$}


\newenvironment{modenumerate}
  {\enumerate\setupmodenumerate}
  {\endenumerate}

\newif\ifmoditem
\newcommand{\setupmodenumerate}{%
  \global\moditemfalse
  \let\origmakelabel\makelabel
  \def\moditem##1{\global\moditemtrue\def\mesymbol{##1}\item}%
  \def\makelabel##1{%
    \origmakelabel{##1\ifmoditem\rlap{\mesymbol}\fi\enspace}%
    \global\moditemfalse}%
}


\usepackage{sectsty}
%\sectionfont{}
\allsectionsfont{\color{astral}\normalfont\sffamily\bfseries\normalsize}

\usepackage{graphicx}
\usepackage{tikz}
\usetikzlibrary{babel}
\usepackage{tikz,tkz-tab}

\usepackage[babel=true, kerning=true]{microtype}


\usepackage{pgfplots}
\usepgfplotslibrary{fillbetween}
\pgfplotsset{compat=newest}
\usepgfplotslibrary{external} 
\tikzexternalize[prefix=./output_latex/]
%\DeclareSymbolFont{RalphSmithFonts}{U}{rsfs}{m}{n}
%\DeclareSymbolFontAlphabet{\mathscr}{RalphSmithFonts}
%\def\mathcal#1{{\mathscr #1}}



\providecommand{\abs}[1]{\left|#1\right|}
\providecommand{\norm}[1]{\left\Vert#1\right\Vert}
\providecommand{\U}{\mathcal{U}}
\providecommand{\R}{\mathbb{R}}
\providecommand{\Cc}{\mathcal{C}}
\providecommand{\reg}[1]{\mathcal{C}^{#1}}
\providecommand{\1}{\mathds{1}}
\providecommand{\N}{\mathbb{N}}
\providecommand{\Z}{\mathbb{Z}}
\providecommand{\p}{\partial}
\providecommand{\one}{\mathds{1}}
\providecommand{\E}{\mathbb{E}}\providecommand{\V}{\mathbb{V}}
\renewcommand{\P}{\mathbb{P}}


%Operateur
\providecommand{\abs}[1]{\left\lvert#1\right\rvert}
\providecommand{\sabs}[1]{\lvert#1\rvert}
\providecommand{\babs}[1]{\bigg\lvert#1\bigg\rvert}
\providecommand{\norm}[1]{\left\lVert#1\right\rVert}
\providecommand{\bnorm}[1]{\bigg\lVert#1\bigg\rVert}
\providecommand{\snorm}[1]{\lVert#1\rVert}
\providecommand{\prs}[1]{\left\langle #1\right\rangle}
\providecommand{\sprs}[1]{\langle #1\rangle}
\providecommand{\bprs}[1]{\bigg\langle #1\bigg\rangle}

\DeclareMathOperator{\deet}{Det}
\DeclareMathOperator{\hess}{Hess}
\DeclareMathOperator{\jac}{Jac}


\newcommand\rst[2]{{#1}_{\restriction_{#2}}}



% generate breakable white space allowing students to write notes.

\usepackage[framemethod=tikz]{mdframed}

\mdfdefinestyle{response}{
	leftmargin=.01\textwidth,
	rightmargin=.01\textwidth,
	linewidth=1pt
	hidealllines=false,
	leftline=true,
	rightline=true,topline=true,bottomline=true,
	skipabove=0pt,
	%innertopmargin=-5pt,
	%innerbottommargin=2pt,
	linecolor=black,
	innerrightmargin=0pt,
	}



\newcommand*{\DivideLengths}[2]{%
  \strip@pt\dimexpr\number\numexpr\number\dimexpr#1\relax*65536/\number\dimexpr#2\relax\relax sp\relax
}

\providecommand{\rep}[1]{$ $ \newline \begin{mdframed}[style=response]  
	
	\vspace*{\DivideLengths{#1}{3cm}cm}
	\pagebreak[1]	
	\vspace*{\DivideLengths{#1}{3cm}cm}
	\pagebreak[1]		
	\vspace*{\DivideLengths{#1}{3cm}cm}   \end{mdframed}}

\providecommand{\blanc}[1]{$ $ \newline 
	
	\vspace*{\DivideLengths{#1}{3cm}cm}
	\pagebreak[1]	
	\vspace*{\DivideLengths{#1}{3cm}cm}
	\pagebreak[3]		
	\vspace*{\DivideLengths{#1}{3cm}cm}}

\usepackage{ifthen}

\newcommand{\eno}[1]{%
	\ifthenelse{\equal{\version}{eno}}{#1}{}%
}
\newcommand{\cor}[1]{%
        \ifthenelse{\equal{\version}{cor}}{
\medskip 

{\small \color{gray} #1}

\medskip 
}{}
}

%------------------------------------------------------------------------------
%\DeclareUnicodeCharacter{00A0}{~}
\makeatother


\newcommand{\miniscule}{\@setfontsize\miniscule{5}{6}}
%-----------------------------------------------------------------------------

\title{\Large \sffamily\bfseries  Fonctions de classe $\reg{1}$}
\ue{HLMA410}


%-----------------------------------------------------------------------------
\begin{document}

\maketitle

\bigskip

Les exercices ou les questions marqués d'une étoile ne sont pas prioritaires.

\section{Fonctions de classe $\reg{1}$ et différentielle}

\exo{} Soit $f:\R^2 \to \R$ la fonction définie par $f(x,y) = \frac{x^2y}{x^2+y^2}$ si $(x,y) \neq(0,0)$ et $f(0,0) = 0$. 
		\begin{center}
			\begin{tikzpicture}[scale=.5]
				\begin{axis}[,xlabel=$x$,ylabel=$y$]%,xtick=\empty,ytick=\empty,ztick=\empty ]
					\addplot3[surf,opacity=.7,samples=50] gnuplot {(x**2 *y) /(x**2 + y**2)};
				\end{axis}
			\end{tikzpicture}
		\end{center}
\begin{enumerate}
	\item La fonction $f$ est-elle continue ?
	\item La fonction $f$ admet-elle des dérivées partielles ?%en Calculer $\nabla f (x,y)$ pour tout $(x,y) \in \R^2 $.
	\item La fonction $f$ est elle de classe $\reg{1}$ ?
	\item La fonction $f$ est-elle différentiable sur $ \R^2$ ?
\end{enumerate}
%Faccanoni exo 4.22 p91


\exo[*]{} Soit $\U = \left\{ (x,y) \in \R^2 | y\neq 0 \right\}$. On considère l'application $f:\R^2 \to \R$ définie pour $(x,y) \in \R^2$ par $ f(x,y) =\begin{cases}
	y^2 \sin\left( \tfrac x y \right) & \text{ si } (x,y) \in \U \\ 0 & \text{ sinon.}
\end{cases}$.
		\begin{center}
			\begin{tikzpicture}[scale=.5]
				\begin{axis}[,xlabel=$x$,ylabel=$y$]%,xtick=\empty,ytick=\empty,ztick=\empty ]
					\addplot3[surf,opacity=.9,samples=50,domain=-3:3,y domain=-1:1] gnuplot {(y**2 *sin(x/y))};
				\end{axis}
			\end{tikzpicture}
		\end{center}

\begin{enumerate}
	\item Montrer que $\U$ est un ouvert de $\R^2$ et que $f$ est de classe $\reg{1}$ sur $\U$.
	\item Montrer que $f$ est continue sur $\R^2$.
	\item Montrer que $f$ a des dérivées partielles en tout point de $\R^2$. Les calculer.
	\item Montrer que la fonction $\frac{\partial f}{\partial x}$ est continue.
	\item Montrer que la fonction $f$ n'est pas de classe $\reg{1}$.
\end{enumerate}
% Liret-Martinais:  exo 5 p 190

\exo{} Calculer la différentielle de chacune des fonctions suivantes:

\begin{enumerate*}[itemjoin={{, \hspace*{1em} }}]
		\item $m = p^5q^3$  \   
		%\item $u = \sqrt{x^2 +3y^2}$  \   
		\item $R= \alpha \beta^2 \cos(\gamma)$  \   
		%\item $T = \frac{ \nu }{1 + uvw}$  \   
		%\item $L = xzye^{ - y^2 - z^2}$
	\end{enumerate*}


\section{Règle de la chaîne}

\exo{}
On consid\`ere les fonctions $f\colon \R^2\longrightarrow \R^3$ et
$g\colon \R^3\longrightarrow \R$ d\'efinies par
\[
f(x,y)=(\cos(xy), y, x\exp(y^2)),\quad 
g(u,v,w)= uvw .
\]
\begin{enumerate}
 \item  Calculer explicitement $g\circ f$.
 \item En utilisant l'expression trouv\'ee en (1), calculer les d\'eriv\'ees partielles de $g\circ f$.
 \item  D\'eterminer les matrices jacobiennes $J_f(x,y)$ et $J_g(u,v,w)$ de $f$ et de $g$. 
 \item Retrouver le r\'esultat de la question 2. en utilisant un produit appropri\'e de matrices jacobiennes. 
\end{enumerate}
% fic00062 exo 4


\exo{} Soit $f:\mathcal U \to \R$ une fonction de classe $\reg{1}$ sur un ouvert $\mathcal U$ de $\R^2$. Calculer $\left( \frac{\partial f}{\partial x}  \right)^2 + \left( \frac{\partial f}{\partial y}  \right)^2 $ en coordonnées polaires.
{\it Indication:  On demande en fait d'exprimer les dérivées partielles de $f$ en fonction des dérivées partielles de  $g:(r,\theta) \mapsto f(r\cos(\theta), r \sin(\theta))$. }
%exo 16 p44 dans Niglio


%\exo{} On consid\`ere l'application $\varphi:\R^3\to \R^3$ d\'efinie par $$\varphi (x,y,z)= (x^2-y^2, y^2-z^2, z^2-x^2)$$ et une application $f:\R^3\longrightarrow \R$ de classe $\reg{1}$. 

%\begin{enumerate}
	%\item \'Ecrire les matrices jacobiennes de $\varphi $ et  de l'application $g=f\circ \varphi:\R^3\longrightarrow \R$. 
	%\item  Calculer $\frac{\partial g}{\partial x}(t,t,t)+ \frac{\partial g}{\partial y}(t,t,t)+ \frac{\partial g}{\partial z}(t,t,t) $ pour tout r\'eel $t$.
%\end{enumerate}


\section{ Fonctions implicites}

\exo{} On considère l'équation $xe^y + ye^x = 0$:
\begin{center}
    
    \begin{tikzpicture}[scale=1]
        \begin{axis}[z post scale=1.5,zlabel style={rotate=-90},zlabel=$z$,xlabel = $x$,ylabel=$y$,width=.3\textwidth, ,domain=-.5:.5,zmax=1.648721271,zmin=-1,view/h=70] 
            
            \addplot3[ contour gnuplot={ % cdata should not be affected by z filter:
                output point meta=rawz,
                levels={0},
                labels=false,
            }, samples=71, z filter/.code=\def\pgfmathresult{-1}, thick] {x*exp(y) + y*exp(x) };
            \addplot3[surf,opacity=.7,samples=20] gnuplot {x*exp(y) + y*exp(x)  };
        \end{axis} 
    \end{tikzpicture}
\end{center}
\begin{enumerate}
    \item Vérifier qu'elle définie une et une seule fonction $y=\varphi(x)$ au voisinage de $(0,0)$.
    \item Calculer le développement de Taylor de $\varphi$ à l'ordre 2 centré en $x = 0$.
\end{enumerate}

%%Facononi: exerci 4.45 p 114


\exo[*]{} On considère la fonction définie par $f(x,y) = x^2 y + \ln(1+y^2)$ dont voici le graphe:
\begin{center}
    
    \begin{tikzpicture}[scale=1]
        \begin{axis}[zlabel style={rotate=-90},zlabel=$z$,xlabel = $x$,ylabel=$y$,width=.3\textwidth, ,domain=-1:1,zmax=2,zmin=-1] 
            
            \addplot3[ contour gnuplot={ % cdata should not be affected by z filter:
                output point meta=rawz,
                levels={0},
                labels=false,
            }, samples=71, z filter/.code=\def\pgfmathresult{-1}, thick ] {x^2 *y + ln(1+y^2) };
            \addplot3[surf,opacity=.7,samples=20] gnuplot {x**2 *y + log(1+y**2) };
        \end{axis} 
    \end{tikzpicture}
\end{center}
\begin{enumerate}
    \item Vérifier que le théorème des fonctions implicites ne s'applique pas à l'origine.
    \item Montrer que, au voisinage de l'origine, l'ensemble $ L_0 = \left\{ (x,y)  \in \R^2 | f(x,y) =0\right\}$ est constitué de l'axe des abscisses et d'une courbe dont on déterminera une expression. %a tangente à l'origine.
    %\item Déterminer des fonctions $x \mapsto \varphi(x)$ de classe  $\reg{1}$ ou $\reg{2}$ sur un intervalle $I$ contenant 0 et telles que $f(x,\varphi(x)) = 0$ pour tout $x\in I$.
\end{enumerate}
%%exercice 22 p. 46 Noirot

\section{\'Equations aux dérivée partielles}

\exo{(Primitives)}
Déterminer toutes les fonctions $f: \R^2 \to \R$ solutions des systèmes suivants:

\begin{enumerate*}
	\item $\begin{cases}
				\frac{\partial f}{\partial x} (x,y) = xy^2  \\
				\frac{\partial f}{\partial y} (x,y) = yx^2 
		\end{cases}$
	\item $\begin{cases}
				\frac{\partial f}{\partial x} (x,y) = e^x y  \\
				\frac{\partial f}{\partial y} (x,y) = e^x + 2y 
		\end{cases}$
	%\item $\frac{\partial f}{\partial x} (x,y) = xy^2$
\end{enumerate*}
% exo 9 de bibmath


\exo{(Fonctions invariantes par translation)} On cherche à déterminer les fonctions $\R^2 \to \R$ de classe $\reg{1}$ vérifiant pour tout $x,y,t\in\R$:
\[
	f(x+t,y+t) = f(x,y)
\]
\begin{enumerate}
	\item Démontrer que, pour tout $(x,y) \in\R^2$,
		\[
			\frac{\partial f}{\partial x} (x,y) + \frac{\partial f}{\partial y} (x,y) = 0.
		\]
	\item On pose $u=x+y$ et $v=x-y$ et $F(u,v) = f(x,y)$. Démontrer que $\frac{\partial F}{\partial u} = 0$.
	\item Conclure.
\end{enumerate}
% exo bibmath 
{\it Indication:  Voici un exemple de solution:}

\begin{minipage}{6cm}
			\begin{tikzpicture}[scale=.5]
				\begin{axis}[,xlabel=$x$,ylabel=$y$,view={-60}{45}]%,xtick=\empty,ytick=\empty,ztick=\empty ]
					\addplot3[surf,opacity=.7,samples=50,domain=-5:5,y domain=-5:5] gnuplot { exp(- abs(x - y)**2  )};
				\end{axis}
			\end{tikzpicture}
		\end{minipage}
		$f(x,y) =  \exp(-(x - y)^2)$




\exo[*]{}
Trouver toutes les applications $f$ de $\R^2$ dans $\R$ vérifiant  
\begin{equation}
	x\frac{\partial f}{\partial x}+ y\frac{\partial f}{\partial y}=\sqrt{x^2+y^2} \tag{$*$}
\end{equation}
	sur $D=\{(x,y)\in\R^2/\;x>0\}$.
{\it Indication: on pourra passer en coordonnées polaires. Voici un exemple de solution:}
% Exo 10 fiche 000116

\begin{minipage}{6cm}
			\begin{tikzpicture}[scale=.5]
				\begin{axis}[,xlabel=$x$,ylabel=$y$,view={30}{40}]%,xtick=\empty,ytick=\empty,ztick=\empty ]
					\addplot3[surf,opacity=.7,samples=50,domain=0:1,y domain=-1:1] { sqrt(x^2 + y^2) + (- atan( y/x))^2};
				\end{axis}
			\end{tikzpicture}
		\end{minipage}
		$f(x,y) = \sqrt{x^2 + y^2} -  (\arctan(y/x))^2$


%\exo{(\'Equation de transport)}
%Trouver toutes les fonctions $f: \R^2 \to \R$, $\reg{1}$ sur $\R^2$ telles que
%\[
%\frac{\partial f}{\partial x}+ 2x\frac{\partial f}{\partial y}=0
%\]
%{\it  Indication:  on pourra effectuer le changement de variables $x=u,y=v+u^2$ . Voici un exemple de solution:}

%\begin{minipage}{6cm}

			%\begin{tikzpicture}[scale=.5]
				%\begin{axis}[,xlabel=$x$,ylabel=$y$,view={-60}{75}]%,xtick=\empty,ytick=\empty,ztick=\empty ]
					%%\addplot3[surf,samples=50,domain=-3:7,y domain=-5:5] gnuplot { exp(- abs(x - y**2)**2  )};
					%\addplot3[surf,opacity=.7,samples=50,domain=-3:7,y domain=-5:5] { exp(- abs(x - y^2)^2  )};
				%\end{axis}
			%\end{tikzpicture}
		%\end{minipage}
%$f(x,y) = \exp(-(x - y^2)^2 ) $

\end{document}

