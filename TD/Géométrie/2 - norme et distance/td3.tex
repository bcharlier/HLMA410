\documentclass{tp_um}
\makeatletter
%--------------------------------------------------------------------------------

\usepackage[frenchb]{babel}

\usepackage{amsmath}
\usepackage{amsbsy}
\usepackage{amsfonts}
\usepackage{amssymb}
\usepackage{amscd}
\usepackage{amsthm}
\usepackage{mathtools}
\usepackage{eurosym}
\usepackage{nicefrac}

\usepackage{latexsym}
\usepackage[a4paper,hmargin=20mm,vmargin=25mm]{geometry}
\usepackage{dsfont}
\usepackage[utf8]{inputenc}
\usepackage[T1]{fontenc}

\usepackage{multicol}
\usepackage[inline]{enumitem}
%\setlist{nosep}
\setlist[itemize,1]{,label=$-$}

\usepackage{sectsty}
%\sectionfont{}
\allsectionsfont{\normalfont\sffamily\bfseries\normalsize}

\usepackage{graphicx}
\usepackage{tikz}

\usepackage{pgfplots}
\usepgfplotslibrary{fillbetween}
\pgfplotsset{compat=newest}
%\usepgfplotslibrary{external} 
%\tikzexternalize[prefix=./output_latex/]
%\DeclareSymbolFont{RalphSmithFonts}{U}{rsfs}{m}{n}
%\DeclareSymbolFontAlphabet{\mathscr}{RalphSmithFonts}
%\def\mathcal#1{{\mathscr #1}}

\newcounter{zut}
\setcounter{zut}{1}
\newcommand{\exo}[1]{\noindent {\sffamily\bfseries Exercice~\thezut. #1} \
		   \addtocounter{zut}{1}}



\providecommand{\abs}[1]{\left|#1\right|}
\providecommand{\norm}[1]{\left\Vert#1\right\Vert}
\providecommand{\U}{\mathcal{U}}
\providecommand{\R}{\mathbb{R}}
\providecommand{\Cc}{\mathcal{C}}
\providecommand{\reg}[1]{\mathcal{C}^{#1}}
\providecommand{\1}{\mathds{1}}
\providecommand{\N}{\mathbb{N}}
\providecommand{\Z}{\mathbb{Z}}
\providecommand{\E}{\mathbb{E}}
\providecommand{\p}{\partial}
\providecommand{\one}{\mathds{1}}
\renewcommand{\P}{\mathbb{P}}


%Operateur
\providecommand{\abs}[1]{\left\lvert#1\right\rvert}
\providecommand{\sabs}[1]{\lvert#1\rvert}
\providecommand{\babs}[1]{\bigg\lvert#1\bigg\rvert}
\providecommand{\norm}[1]{\left\lVert#1\right\rVert}
\providecommand{\bnorm}[1]{\bigg\lVert#1\bigg\rVert}
\providecommand{\snorm}[1]{\lVert#1\rVert}
\providecommand{\prs}[1]{\left\langle #1\right\rangle}
\providecommand{\sprs}[1]{\langle #1\rangle}
\providecommand{\bprs}[1]{\bigg\langle #1\bigg\rangle}

\DeclareMathOperator{\deet}{Det}
\DeclareMathOperator{\vol}{Vol}
\DeclareMathOperator{\aire}{Aire}
\DeclareMathOperator{\hess}{Hess}
\DeclareMathOperator{\var}{Var}

%------------------------------------------------------------------------------
\DeclareUnicodeCharacter{00A0}{~}
\makeatother


\newcommand{\miniscule}{\@setfontsize\miniscule{5}{6}}
%-----------------------------------------------------------------------------

\title{\large \sffamily\bfseries Normes dans $\R^n$ et limites}
\ue{HLMA410}


%-----------------------------------------------------------------------------
\begin{document}

\maketitle

\bigskip

Les exercices ou les questions marqués d'une étoile ne sont pas prioritaires.


%\exo{}Montrer que pour tout $x$ dans $\R^n$ on a $\lim_{p\to +\infty} \norm{x}_p = \norm{x}_\infty$

\section{Normes et distances sur $\R^n$}

\exo{(Équivalence des normes usuelles)} Démontrer que les normes $\norm{\cdot}_1$, $\norm{\cdot}_2$ et $\norm{\cdot}_\infty$ de $\R^n$ sont équivalentes.

\bigskip
\newpage

\exo{(Une norme plus exotique)} Soit $a,b \in \R$ deux réels fixés avec $a\neq 0$. Si $(x,y)\in\R^2$, on pose \[N_{a,b}(x,y) = \max \left\{ \abs{bx+y} , \abs{(a+b)x+y} \right\}.\] 
\begin{enumerate}
	\item Montrer que l'application $N_{a,b}:\R^2 \to\R$ définie bien une norme sur $\R^2$.
	\item Dessiner la boule unité dans le cas où $(a,b) = (1,0)$.
	{ \it  Indication: montrer que $N_{1,0} (x,y) \leq 1$ ssi $-1\leq y \leq 1$ et $-1\leq x+y\leq 1$.}
\end{enumerate}

 %A = { (x,y) ² | -1 < x + y < 1 } , B = { (x,y) ² | -1 < x - 2y < 1 } 
% bouleConverge exo 39 p 25

\bigskip

\newpage
\exo{(Convexité et inégalité triangulaire)}
Montrer que la boule unité d'un espace vectoriel normé $E$ est un convexe de cet espace. {\it Indication: une partie $A\subset E$ est convexe ssi $\forall x,y\in A$ on a $\{\lambda x + (1-\lambda) y| \lambda \in [0,1]\} \subset A $  }

\bigskip

\newpage
\exo[*]{(Distance SNCF)} On note $\norm{\cdot}_2$ la norme euclidienne usuelle sur $\R^2$. On
note $O=(0,0)$ l'origine du plan et on définit
\[
	d(A,B) = \begin{cases}
		%\snorm{u-v}_2, &\text{si les points $u$, $v$ et $O$ sont alignés  } \\
		%\snorm{u}_2 + \snorm{v}_2, &\text{sinon.}
		\snorm{\overrightarrow{AB}}_2, &\text{si les points $A$, $B$ et $O$ sont alignés  } \\
		\snorm{\overrightarrow{OA}}_2 + \snorm{\overrightarrow{OB}}_2, &\text{sinon.}
	\end{cases}
\]
\begin{enumerate}
	\item Montrer que $d$ est une distance sur $\R^2$.
	\item Dessiner l'ensemble des points situés à distance inférieure à 3 du point $C = (2,0)$.
	\item On considère la suite $u = ( 1, \frac{1}{m+1})_{m\in\N}$ dans $\R^2$. Montrer que $u$ converge vers $\ell=(1,0)$ pour la norme $2$ mais que la suite $d(u_m,\ell)$ ne tend pas vers 0 quand $m \to +\infty$.
	\item Existe-t-il une norme $\snorm{\cdot}$ sur $\R^2$ telle que $d(A,B) = \snorm{\overrightarrow{AB}}$ pour tout $A,B\in\R^2$? 
\end{enumerate}


\newpage

%\section{Exercices supplémentaires}
%
%\noindent Les deux exercices suivants ne sont à traiter qu'en dernier et on pourra admettre les résultats du premier pour traiter le second.
%
%\bigskip
%
\exo[*]{(Inégalités de \textsc{Hölder} et de \textsc{Minkowski})}. 
Soit $(p,q)\in [1,+\infty[^2$ tel que $\frac{1}{p}+\frac{1}{q}=1$.
        \begin{enumerate}
                \item Montrer que pour $(x,y) \in [0,+\infty[^2$, $xy \leq \frac{x^p}{p}+ \frac{y^q}{q}$.
                                {\it Indication: étudier le minimum de la fonction $x\mapsto \frac{x^p}{p} + \frac{y^q}{q} -yx$ pour $x\geq 0$}
                \item En déduire que $\forall (a_1,\cdots,a_n),(b_1,\cdots,b_n)\in \R^n$, 
                        \[
                                \bigg\lvert \sum_{k=1}^n a_kb_k\bigg\rvert \leq \bigg(\sum_{k=1}^n \abs{a_k}^p\bigg)^{1/p}\bigg(\sum_{k=1}^n \abs{b_k}^q\bigg)^{1/q}
                        \]
                \item 	En déduire que $\forall (a_1,\cdots,a_n),(b_1,\cdots,b_n) \in \R^n$, 
                        \[
                                \bigg(\sum^n_{k=1}\abs{a_k+b_k}^p\bigg)^{1/p}\leq\bigg(\sum^n_{k=1}\abs{a_k}^p\bigg)^{1/p}+\bigg(\sum^n_{k=1}\abs{b_k}^p\bigg)^{1/p}.
        \]
                \end{enumerate}

\newpage

%
%
\exo[*]{(Les normes $\boldsymbol{N_{ \hspace*{-1pt}p}}$)} 
Soit $p\in [1,+\infty[$, pour $x=(x_1,\cdots,x_n) \in \R^n$, on définit $N_p(x)=(\sum_{k=1}^n|x_k|^p)^{1/p}$.
\begin{enumerate}
	\item Montrer que $\forall p \geq 1$, $N_p$ est une norme sur $\R^n$.
	\item Dessiner les boules unités de $\R^2$ dans le cas où $p =1,3/2,2, 5,+\infty$.
	\item Montrer que, pour $x\in\R^n$ fixé, $\lim_{p\to+\infty} \limits N_p(x)=\max \{|x_1|,\cdots,\abs{x_n}\}=N_\infty(x)$.
	\item Montrer que si $0<p<1$, $N_p$ n'est pas une norme sur $\R^n$ (si $n \geq 2$). 
\end{enumerate}


%Jauge d'un convexe 
%http://denis.monasse.free.fr/livre-html/coursse31.html
\newpage


\section{Limites de suites}

\exo{} Étudier la nature des suites $(u_{n})_{n\in\N}$ de $\R^3$ ci-dessous et déterminer des sous-suites convergentes en fonction des paramètres réels $a$, $b$ et $c$ 
\begin{enumerate}
	\item $u_n = \left( \frac{(-1)^n}{n^a}, \frac{1}{n^b}, c^n \right)$
	\item $u_n = \left( \sum_{k=1}^n\limits \frac{1}{k^a}, \sum_{k=1}^n\limits \frac{b^k}{k!},c \right)$
	\item $u_n = \left( \sum_{k=1}^n\limits a^k, \sum_{k=1}^n\limits a^{2k},\sum_{k=1}^n\limits a^{3k}  \right) $
\end{enumerate}

\section{Ouverts et fermés des espaces vectoriels normés $\R^n$}


\exo{(Ouvert ou fermé)} Dessiner et déterminer la nature (ouvert ou fermé) du domaine de définition des fonctions $\R^2 \to \R$ suivantes:
\begin{multicols}{2}
	\begin{modenumerate}
		\item  $f(x,y)= \frac{\ln(x) + \ln(y)}{x-y}$.
		\item  $f(x,y) = \ln\left( ( 16 -x^2 - y^2)(x^2 + y^2 - 4) \right)$
		\moditem{*}  $f(x,y) = \frac{x^2 - y^2}{\sqrt{x^2 - y}}$
            \moditem{*}  $f(x,y) = \sqrt{\frac{x+y}{x-y}}$
	\end{modenumerate}
\end{multicols}
\bigskip
\newpage

\exo{(Stabilité par intersection finie)} Montrer qu'une intersection finie d'ouverts de $\R^n$ est un ouvert de $\R^n$. Montrer qu'une intersection infinie d'ouvert de $\R^n$ n'est pas nécessairement un ouvert. Qu'en est-il pour les parties fermées de $\R^n$?



\bigskip
\newpage

\exo{(Adhérence)} Dessiner l'adhérence des ensembles de $\R^2$  suivants: 
\begin{modenumerate}
	\item   $A = \left\{ (x,y) \in \R^2| x^2 + 2y^2 <1 \right\}$ 
    \item	$C = \{\frac{t}{t+1}\left( \cos(t), \sin(t)\right) , t>0 \}$
        %\begin{center}
            %\begin{tikzpicture}
                %\begin{axis}
                    %\addplot[blue,domain = 0:15 ,samples=500,thick] ({x*cos(deg(x))/(x+1)},{x*sin(deg(x))/(x+1) });
                %\end{axis}
            %\end{tikzpicture}
        %\end{center}
	\moditem{*}   $B = \left\{ (t,\cos(1/t)) \in \R^2| t>0 \right\}$ 
        \begin{center}
            \begin{tikzpicture}
                \begin{axis}[height=6cm, width=15cm, grid]
                    %\addplot[domain = 0:6.28 ,samples=500,thick,blue] ({x},{cos(deg(1/x))});
                    \addplot[domain =0.021:.1,very thick,samples=4500,blue] gnuplot {cos(1/x)};
                    \addplot[domain =0.1:1,very thick,samples=800,blue] gnuplot {cos(1/x)};
                \end{axis}
            \end{tikzpicture}
        \end{center}
        \moditem{*}  $D = \left\{ (x,y) \in\R^2 | x \in \mathbb{Q} \cap [0,1], y\in\mathbb Q \cap [0,1] \right\}$ 
\end{enumerate}


\end{document}
