\documentclass{tp_um}
\makeatletter
%--------------------------------------------------------------------------------

\usepackage[frenchb]{babel}

\usepackage{amsmath}
\usepackage{amsbsy}
\usepackage{amsfonts}
\usepackage{amssymb}
\usepackage{amscd}
\usepackage{amsthm}
\usepackage{mathtools}
\usepackage{eurosym}
\usepackage{nicefrac}

\usepackage{latexsym}
\usepackage[a4paper,hmargin=20mm,vmargin=25mm]{geometry}
\usepackage{dsfont}
\usepackage[utf8]{inputenc}
\usepackage[T1]{fontenc}

\usepackage{multicol}
\usepackage[inline]{enumitem}
%\setlist{nosep}
\setlist[itemize,1]{,label=$-$}

\usepackage{sectsty}
%\sectionfont{}
\allsectionsfont{\normalfont\sffamily\bfseries\normalsize}

\usepackage{graphicx}
\usepackage{tikz}

\usepackage{pgfplots}
\usepgfplotslibrary{fillbetween}
\pgfplotsset{compat=newest}
%\usepgfplotslibrary{external} 
%\tikzexternalize[prefix=./output_latex/]
%\DeclareSymbolFont{RalphSmithFonts}{U}{rsfs}{m}{n}
%\DeclareSymbolFontAlphabet{\mathscr}{RalphSmithFonts}
%\def\mathcal#1{{\mathscr #1}}

\newcounter{zut}
\setcounter{zut}{1}
\newcommand{\exo}[1]{\noindent {\sffamily\bfseries Exercice~\thezut. #1} \
		   \addtocounter{zut}{1}}



\providecommand{\abs}[1]{\left|#1\right|}
\providecommand{\norm}[1]{\left\Vert#1\right\Vert}
\providecommand{\U}{\mathcal{U}}
\providecommand{\R}{\mathbb{R}}
\providecommand{\Cc}{\mathcal{C}}
\providecommand{\reg}[1]{\mathcal{C}^{#1}}
\providecommand{\1}{\mathds{1}}
\providecommand{\N}{\mathbb{N}}
\providecommand{\Z}{\mathbb{Z}}
\providecommand{\E}{\mathbb{E}}
\providecommand{\p}{\partial}
\providecommand{\one}{\mathds{1}}
\renewcommand{\P}{\mathbb{P}}


%Operateur
\providecommand{\abs}[1]{\left\lvert#1\right\rvert}
\providecommand{\sabs}[1]{\lvert#1\rvert}
\providecommand{\babs}[1]{\bigg\lvert#1\bigg\rvert}
\providecommand{\norm}[1]{\left\lVert#1\right\rVert}
\providecommand{\bnorm}[1]{\bigg\lVert#1\bigg\rVert}
\providecommand{\snorm}[1]{\lVert#1\rVert}
\providecommand{\prs}[1]{\left\langle #1\right\rangle}
\providecommand{\sprs}[1]{\langle #1\rangle}
\providecommand{\bprs}[1]{\bigg\langle #1\bigg\rangle}

\DeclareMathOperator{\deet}{Det}
\DeclareMathOperator{\vol}{Vol}
\DeclareMathOperator{\aire}{Aire}
\DeclareMathOperator{\hess}{Hess}
\DeclareMathOperator{\var}{Var}

%------------------------------------------------------------------------------
\DeclareUnicodeCharacter{00A0}{~}
\makeatother


\newcommand{\miniscule}{\@setfontsize\miniscule{5}{6}}
%-----------------------------------------------------------------------------




	\title{\Large \sffamily\bfseries Espaces euclidiens,  formes bilinéaires et quadratiques}
\ue{HLMA410}
%-----------------------------------------------------------------------------
\begin{document}


\maketitle

\bigskip

Les exercices ou les questions marqués d'une étoile ne sont pas prioritaires.


\section{Norme Euclidienne}

Dans ces exercices, les normes considérées sont des normes euclidiennes.


\exo{} Soit $(E,\prs{\cdot,\cdot})$ un espace euclidien.  Démontrer que deux vecteurs $u$ et $v$ de $E$ qui satisfont $\norm{u-v} = \norm{u+v}$ sont orthogonaux.
% Yves Noirot: cours de physique mathématiques: exo 6 chap 1.

\newpage

\exo{} Soit $u$ un vecteur d'un espace euclidien $E$. Déterminer l'ensemble $\{ x \in E\  | \  \prs{x,x-u} =0  \}$.
 {\it Indication: faire un dessin dans le cas $E=\R^2$. }

\newpage

\exo{(Inversion)} Soit $E$ un espace vectoriel euclidien. On définit l'application: $ i(u) = \begin{cases} \frac{u}{\snorm{u}^2} & \text{si $u\neq 0$} \\ \  \   0 & \text{si $u=0$} \end{cases}$.
\begin{enumerate}
	\item Montrer que $i$ est une involution ({\it i.e. } $i(i(u)) = u$ pour tout $u\in E \setminus \left\{ 0 \right\}$) et déterminer les points fixes de $i$ ({\it i.e. } les $u\in E$ tels que $i(u) = u$).
	\item Vérifier que pour tout $u,v \in E \setminus \left\{ 0 \right\}$ on a $\norm{i(u) - i(v)} = \frac{\norm{u-v}}{\norm{u}\norm{v}}$.
	\item On considère le cas où $E=\R^2$. Déterminer l'image par $i$:
		\begin{enumerate}
			\item d'une droite qui passe par $0$.
			\item d'un cercle passant par $0$,
			\item d'une droite affine ne passant pas par $0$,
			%\item d'un cercle ne passant pas $0$.
		\end{enumerate}
\end{enumerate}
% Solution: Exercice 3663 de fic00080.pdf

\newpage

\section{Formes quadratiques}

Dans tous les exercices de cette partie, on précisera le signe (positif, négatif ou aucun des deux) de chaque forme quadratique.



\exo{} Mettre les formes quadratiques suivantes sous forme de sommes et de différences de carrés de formes linéaires indépendantes: 
\begin{enumerate}
	\item $q(x,y) = x^2 + y^2 -3xy$ 
	\item $q(x,y,z) = 2x^2 - 2y^2 - 6z^2 + 3xy - 4 xz + 7 yz$
%	\item $q(x_1,x_2,x_3,x_4,x_5) = 2(x_{1}^2+x_{1}x_{2}+x_{2}^2+x_{2}x_{3}+x_{3}^2+x_{3}x_{4}+x_{4}^2+x_{4}x_{5}+x_{5}^2)$
	\item $q(x,y,z,t) = xy+yz+zt+tx$ %Solution q(x,y,z,t) =( (x+y+z+t)^2 - (x-y-t+z)^2 ) /4  $
\end{enumerate}

\newpage

\exo{} Déterminer si les formes quadratiques suivantes sont sous forme de sommes et de différences de carrés de formes linéaires indépendantes (sinon les y mettre):   
\begin{enumerate}
	\item $q(x,y) = 9\left( \frac{x+2y}{2} \right)^2 + \left( \frac{x - 3y}{2} \right)^2$
	\item $q(x,y,z) = (x-6y+4z)^2 - (y-4z)^2 + 2z^2$
	%\item $q(x,y,z,t) = \left( \frac{x+y+t}{2} \right)^2 - \left( \frac{x-y-t}{2} \right)^2 + \left( \frac{y+z + t}{2} \right)^2 - \left( \frac{-y +z-t}{2} \right)^2 $
	\item $q(x,y) = (x+y)^2 - (x - y)^2 + x^2 +2y^2$
	\item $q(x,y,z) = (x+y+z)^2 + (-x +y +z)^2 - x^2 $
	%\item  $q(x_1,x_2,x_3,x_4,x_5) = (x_1+x_4+x_5)^2 + x_3^2 - x_2^2 $ 
\end{enumerate}

\newpage


\exo[*]{} Déterminer si les formes quadratiques suivantes sont définies et positives
\begin{enumerate}
\item $q(x,y) = (1-\lambda)x^2 + 2\mu xy + (1+\lambda)y^2$ où $\lambda,\mu\in\R$,
    %\reponse{Oui ssi $\lambda^2 + \mu^2 \le 1$.}
\item $q(x,y,z) = x^2 + y^2 + 2z(x\cos\alpha + y\sin \alpha)$ où $\alpha\in\R$.
    %\reponse{Non, disc $=-1$.}
%\item $q(x,y,z,t) = x^2 + 3y^2 + 4z^2 + t^2 + 2xy + xt$.
    %\reponse{Oui, $= \dfrac {5x^2}{12} + 3\left(y+\dfrac x3\right)^2 + 4z^2
                   %+ \left(t+\dfrac x2\right)^2$.}
\end{enumerate}

\newpage

\exo{(Décomposition dans différentes bases)}
Soit la forme quadratique $q(x_1,x_2) = 3(x_1^2 + x_2^2) + 2 x_1 x_2$ définie pour tout $(x_1,x_2)\in\R^2$. On note $A$ la matrice de $q$ dans la base canonique.
%Soit $A = \begin{pmatrix}
	%1&1\\1&2
%\end{pmatrix}$.
\begin{enumerate}
	\item Vérifier que $a = (1,1)\frac{1}{\sqrt{2}} $ et $b = (1,-1)\frac{1}{\sqrt{2}}$ sont des vecteurs propres de $A$ et qu'ils forment une base orthonormale de $\R^2$. Écrire alors $q$ sous forme de sommes et de différences de carrés de formes linéaires indépendantes.
	\item En utilisant l'algorithme de Gauss: mettre $q$ sous forme de sommes et de différences de carrés de formes linéaires indépendantes et écrire cette décomposition sous forme matricielle
	\item En utilisant les deux questions précédentes, trouver d'autres représentations en sommes et différences de carrés de formes linéaires indépendantes.
\end{enumerate}



\newpage


\exo[*]{} Soit $\phi$ la forme bilinéaire symétrique sur $\R^3$ définie par
\[
	\phi(x,y) = (x_1 - 2x_2)(y_1-2y_2) + x_2y_2 + (x_2+x_3)(y_2 + y_3)
\]
\begin{enumerate}
	\item Vérifier que $\phi$ est un produit scalaire sur $\R^3$ et on note $\norm{\cdot}$ la norme associée.
	\item Soit $i=(1,0,0)$, $j=(0,1,0)$ et $k =(0,0,1)$. Calculer 
		$$
		e_1 = \frac{i}{\norm{i}}, \quad  e_2 = \frac{j - \phi(e_1,j)e_1}{\snorm{j - \phi(e_1,j)e_1}}, \quad  e_3 = \frac{k - \phi(e_1,k)e_1 - \phi(e_2,k)e_2}{\norm{k - \phi(e_1,k)e_1 - \phi(e_2,k)e_2}}
		$$
	\item Vérifier que $(e_1,e_2,e_3)$ est une base orthonormale pour $\phi$.
	\item Déterminer (sans calcul) la matrice de $\phi$ dans la base $(e_1,e_2,e_3)$.
\end{enumerate}
% Solution: ex 19 de fic00130.pdf

\newpage
\section{Pour aller plus loin}

\exo[*]{} Soit $n\in\N$ et $n>2$. La forme bilinéaire dont la matrice est 
\[
        A =\frac{1}{2} \begin{pmatrix}
                2(n-1) & -1 & \cdots & -1 \\
                -1 & \ddots & \ddots & \vdots \\
                \vdots & \ddots & \ddots & -1 \\
                -1 & \cdots & -1 & 2(n-1) \\
        \end{pmatrix}
\]
est-elle positive? définie?

\newpage

\exo[*]{(Identité du parallélogramme)}
Soit $E$ un $\R$ espace vectoriel de dimension finie. Soit $\snorm{\cdot}$ une norme sur $E$ vérifiant l'identité du parallèlogramme, c'est-à-dire~:
\[
\forall(u,v)\in E^2,\;\norm{u+v}^2+\norm{u-v}^2=2(\norm{u}^2+\norm{v}^2).
\]
On se propose de démontrer qu'une telle norme $\norm{\cdot}$ est associée à un produit scalaire. On définit sur $E^2$ une application $f$ par~:~
\[
\forall(u,v)\in E^2,\;f(u,v)=\frac{1}{4}(\norm{u+v}^2-\norm{u-v}^2).
\]
\begin{enumerate}
\item  Montrer que pour tout $(u,v,w)$ de $E^3$, on a~:~$f(u+w,v)+f(u-w,v)=2f(u,v)$.
\item  Montrer que pour tout $(u,v)$ de $E^2$, on a~:~$f(2u,v)=2f(u,v)$.
\item  Montrer que pour tout $(u,v)$ de $E^2$ et tout rationnel $r$, on a~:~$f(ru,v)=rf(u,v)$.

On admettra que pour tout réel $\lambda$ et tout $(u,v)$ de $E^2$ on a~:~$f(\lambda u,v)=\lambda f(u,v)$ (ce résultat provient de la continuité de $f$).
\item  Montrer que pour tout $(u,v,w)$ de $E^3$, $f(u,w)+f(v,w)=f(u+v,w)$.
\item  Montrer que $f$ est bilinéaire.
\item  Montrer que $\norm{\cdot}$ est une norme euclidienne.
\end{enumerate}
% Solution: exercice 2 fic0010.pdf

%correction: http://www.mathlaayoune.webs.com/bilineaire/prehilbertien/identiteduparallelogramme.html


\end{document}
