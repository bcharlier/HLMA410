\documentclass{tp_um}
\makeatletter
%--------------------------------------------------------------------------------

\usepackage[frenchb]{babel}

\usepackage{amsmath}
\usepackage{amsbsy}
\usepackage{amsfonts}
\usepackage{amssymb}
\usepackage{amscd}
\usepackage{amsthm}
\usepackage{mathtools}
\usepackage{eurosym}
\usepackage{nicefrac}

\usepackage{latexsym}
\usepackage[a4paper,hmargin=20mm,vmargin=25mm]{geometry}
\usepackage{dsfont}
\usepackage[utf8]{inputenc}
\usepackage[T1]{fontenc}

\usepackage{multicol}
\usepackage[inline]{enumitem}
%\setlist{nosep}
\setlist[itemize,1]{,label=$-$}

\usepackage{sectsty}
%\sectionfont{}
\allsectionsfont{\normalfont\sffamily\bfseries\normalsize}

\usepackage{graphicx}
\usepackage{tikz}

\usepackage{pgfplots}
\usepgfplotslibrary{fillbetween}
\pgfplotsset{compat=newest}
%\usepgfplotslibrary{external} 
%\tikzexternalize[prefix=./output_latex/]
%\DeclareSymbolFont{RalphSmithFonts}{U}{rsfs}{m}{n}
%\DeclareSymbolFontAlphabet{\mathscr}{RalphSmithFonts}
%\def\mathcal#1{{\mathscr #1}}

\newcounter{zut}
\setcounter{zut}{1}
\newcommand{\exo}[1]{\noindent {\sffamily\bfseries Exercice~\thezut. #1} \
		   \addtocounter{zut}{1}}



\providecommand{\abs}[1]{\left|#1\right|}
\providecommand{\norm}[1]{\left\Vert#1\right\Vert}
\providecommand{\U}{\mathcal{U}}
\providecommand{\R}{\mathbb{R}}
\providecommand{\Cc}{\mathcal{C}}
\providecommand{\reg}[1]{\mathcal{C}^{#1}}
\providecommand{\1}{\mathds{1}}
\providecommand{\N}{\mathbb{N}}
\providecommand{\Z}{\mathbb{Z}}
\providecommand{\E}{\mathbb{E}}
\providecommand{\p}{\partial}
\providecommand{\one}{\mathds{1}}
\renewcommand{\P}{\mathbb{P}}


%Operateur
\providecommand{\abs}[1]{\left\lvert#1\right\rvert}
\providecommand{\sabs}[1]{\lvert#1\rvert}
\providecommand{\babs}[1]{\bigg\lvert#1\bigg\rvert}
\providecommand{\norm}[1]{\left\lVert#1\right\rVert}
\providecommand{\bnorm}[1]{\bigg\lVert#1\bigg\rVert}
\providecommand{\snorm}[1]{\lVert#1\rVert}
\providecommand{\prs}[1]{\left\langle #1\right\rangle}
\providecommand{\sprs}[1]{\langle #1\rangle}
\providecommand{\bprs}[1]{\bigg\langle #1\bigg\rangle}

\DeclareMathOperator{\deet}{Det}
\DeclareMathOperator{\vol}{Vol}
\DeclareMathOperator{\aire}{Aire}
\DeclareMathOperator{\hess}{Hess}
\DeclareMathOperator{\var}{Var}

%------------------------------------------------------------------------------
\DeclareUnicodeCharacter{00A0}{~}
\makeatother


\newcommand{\miniscule}{\@setfontsize\miniscule{5}{6}}
%-----------------------------------------------------------------------------

\title{\Large \sffamily\bfseries Limites, continuité et différentiabilité}
\ue{HLMA410}

%-----------------------------------------------------------------------------
\begin{document}

\maketitle

\bigskip

Les exercices ou les questions marqués d'une étoile ne sont pas prioritaires.

\section{Limites de fonctions}

\exo{} Donner le domaine de définition et étudier la limite en l'origine des fonctions suivantes: 
\begin{enumerate}
	\item $f(x,y) = \frac{x^2 - 3y^2}{x^2 + y^2}$ 
		\begin{center}
			\begin{tikzpicture}[scale=.5]
				\begin{axis}[,xlabel=$x$,ylabel=$y$]%,xtick=\empty,ytick=\empty,ztick=\empty ]
					\addplot3[surf,opacity=.7,samples=50] gnuplot {(x**2 -3*y**2) /(x**2 + y**2)};
				\end{axis}
			\end{tikzpicture}
		\end{center}
	\item $f(x,y,z) = \frac{xyz}{x^2 + y^2 +z^2}$
\end{enumerate}

\newpage 

%\exo{} Soit $f:\R^2 \to \R$ la fonction définie par $\frac{xy}{x^2+y^2}$.
%\begin{center}
	%\begin{tikzpicture}[scale=.5]
		%\begin{axis}[,xlabel=$x$,ylabel=$y$]%,xtick=\empty,ytick=\empty,ztick=\empty ]
			%\addplot3[surf,opacity=.7,samples=50] gnuplot {(x*y) /(x**2 + y**2)};
		%\end{axis}
	%\end{tikzpicture}
%\end{center}
%\begin{enumerate}
	%\item Étudier la limite à l'origine de la restriction de $f$ à la droite d'équation $y=ax$. 
	%\item Montrer que $f$ n'admet pas de limite à l'origine.
%\end{enumerate}

\exo{} Soit $f:\R^2 \to \R$ la fonction définie par $\frac{xy}{x+y}$.
%\begin{center}
	%\begin{tikzpicture}[scale=.5]
		%\begin{axis}[,xlabel=$x$,ylabel=$y$,xtick=\empty,ytick=\empty,ztick=\empty,zmin=-1,zmax=1]%,view={47}{42}]
			%\addplot3[surf,samples=50,domain=-1:1] gnuplot {( x>-y? (x*y)/(x + y): 0)};
		%\end{axis}
	%\end{tikzpicture}
%\end{center}
\begin{enumerate}
	\item Étudier la limite à l'origine de la restriction de $f$ à la droite d'équation $y=ax$. 
	\item Calculer la limite à l'origine de de la restriction de $f$ à la parabole d'équation $x+y=x^2$ (faire un dessin!)
	\item Montrer que $f$ n'admet pas de limite à l'origine.
\end{enumerate}

\newpage

\exo[*]{} Montrer que la fonction $f(x,y) = \frac{xy+y^2}{\sqrt{x^2+y^2}}$ tend vers 0 quand $(x,y)$ tend vers l'origine. 
\begin{center}
    \begin{tikzpicture}[scale=.5]
        \begin{axis}[,xlabel=$x$,ylabel=$y$]%,xtick=\empty,ytick=\empty,ztick=\empty]
            \addplot3[surf,opacity=.7,samples=50] gnuplot {(x*y + y**2) / sqrt(x**2 + y**2)};
        \end{axis}
    \end{tikzpicture}
\end{center}

\newpage
\exo{} Donner le domaine de définition et étudier la limite en $a=(0,0)$ de la fonction $f(x,y) = \frac{\ln\left( 1  + xy\right)}{x^2 + y^2}$.
%\begin{center}
	%\begin{tikzpicture}[scale=.5]
		%\begin{axis}[,xlabel=$x$,ylabel=$y$,xtick=\empty,ytick=\empty,ztick=\empty,zmin=0,zmax=2]
			%\addplot3[surf,samples=50] gnuplot {log(1+ x*y) / sqrt(x**2 + y**2)};
		%\end{axis}
	%\end{tikzpicture}
%\end{center}

\newpage
\exo[*]{} Soit $f$ la fonction définie sur le plan privé de la droite $\Delta$ d'équation $y=x$ par la formule $f(x,y) = \frac{\sin(x) - \sin(y)}{x - y}$. Étudier la limite de $f$ en tout point de $\Delta$. % cv vers \cos
	\begin{center}%\shorthandoff{?}
			\begin{tikzpicture}[scale=.5]
				\begin{axis}[,xlabel=$x$,ylabel=$y$]%,xtick=\empty,ytick=\empty,ztick=\empty ]
					\addplot3[surf,opacity=.7,samples=40,domain=-10:10] gnuplot { (x==y? cos(x):  (sin(x) - sin(y)) /(x - y )) }; %\usepackage{microtype}
				\end{axis}
			\end{tikzpicture}
		\end{center}

\newpage
\section{Continuité}

\exo{}
\'Etudier la continuité de la fonction $f(x,y) = \max \left\{ x,y \right\}$.
{\it Indication: On pourra montrer que $\max\{a,b\} = \frac 1 2 (a+b + |a-b|)$ pour tout $a,b\in\R$. }
% Niglio exo 2 et 17 p. 23

\newpage
\exo[*]{} Montrer que la fonction définie par $f(x,y) = \frac{\sin(x+y)}{x+y}$ est continue sur son domaine de définition et qu'elle peut se prolonger par continuité à $\R^2$ tout entier.
	\begin{center}%\shorthandoff{?}
			\begin{tikzpicture}[scale=.5]
				\begin{axis}[,xlabel=$x$,ylabel=$y$]%,xtick=\empty,ytick=\empty,ztick=\empty ]
					\addplot3[surf,opacity=.7,samples=40,domain=-10:10] gnuplot { (x==-y? 1:  (sin(x +y)) /(x + y )) }; %\usepackage{microtype}
				\end{axis}
			\end{tikzpicture}
		\end{center}% Niglio exo 13 p. 23

%\exo{} \'Etudier la continuité pour tout $\alpha>0$ de la fonction $f(x,y) = \frac{\abs{x}^\alpha y}{x^2 + y^4}$ si $(x,y) \neq 0$ et $f(0,0) = 0 $.
%{\it Indication: Pour la continuité en 0, on pourra considérer $f(y^2,y)$ puis majorer $\abs{f(x,y)}$ en distinguant les cas $\abs{x} > y^2$ et $\abs{x} < y^2$.  }

\newpage

\section{Dérivées partielles}


\exo{} Calculer les dérivées partielles d'ordre 1 des fonctions suivantes:
\begin{enumerate}
	\item $f (x, y) = y^5-3xy $ 
\item $f(x, y) = x \cos(e^{xy} )$
\item $F(x,y) = \int_y^x \cos(e^t) dt$
\end{enumerate}

\newpage
\exo[*]{} Existence et calcul des dérivées partielles de la fonction $f$ définie par $\arccos \big( 1+( x-y )^2 \big)$.
% Niglio  exo 3 p41

\newpage
\exo{} Soit $f:\R^2\rightarrow \R$ la fonction
d\'efinie par $f(x,y)=(x^2+y^2)^x$ pour $(x,y)\not=(0,0)$ et
$f(0,0)= 1$.
\begin{enumerate}
 \item La fonction $f$ est-elle continue en $(0,0)$?
 \item D\'eterminer les d\'eriv\'ees partielles de $f$ en un point
quelconque distinct de l'origine.
 \item  La fonction $f$ admet-elle des d\'eriv\'ees partielles par
rapport \`a $x$, \`a $y$ en $(0,0)$?
\end{enumerate}
% ex 1 de fic00062

\newpage

\exo{(Gaz parfait)}Pour un gaz parfait, l'énergie interne $\varepsilon$ s'écrit en fonction du volume spécifique $\tau$ et de l'entropie spécifique $s$ comme
\begin{align*}
	\varepsilon: \R^2 & \to \R \\
	(\tau,s) & \mapsto \tau^{1-\gamma} e^{s/c_\nu}
\end{align*}
où $\gamma>1$ et $C_\nu>0$ sont deux constantes. 
\begin{enumerate}
	\item	Calculer la pression $P = -\frac{\partial \varepsilon}{\partial \tau} $ et la température $T = \frac{\partial\varepsilon}{\partial s}$
	\item  Retrouver la loi des gaz parfaits: $\frac{P\tau}{T}$ est constant.
\end{enumerate}

\newpage
\section{Différentiabilité}

\exo{} On considère la fonction $f: \R^2 \to \R$ définie par $f(x,y) =  \frac{xy}{\sqrt{x^2 + y^2}}$ si $(x,y) \neq (0,0)$ et  $f(0,0)=0$.
	\begin{center}
			\begin{tikzpicture}[scale=.5]
				\begin{axis}[,xlabel=$x$,ylabel=$y$]%,xtick=\empty,ytick=\empty,ztick=\empty ]
					\addplot3[surf,opacity=.7,samples=50] gnuplot {(x*y) /sqrt(x**2 + y**2)};
				\end{axis}
			\end{tikzpicture}
		\end{center}
\begin{enumerate}
	\item Montrer que $f$ est continue sur $\R^2$.
	\item Montrer que $f$ n'est pas différentiable en $(0,0)$.%sur  $\R^2$ {\it Indication: pour la différentiabilité en l'origine, Calculer les dérivées partielles de $f$ en $(0,0)$ et en déduire que $f$  n'est pas différentiable.}
\end{enumerate}
% Niglio  exo 4 p41

\newpage

\exo[*]{}
On considère la fonction $f: \R^2 \to \R$ définie par $f(x,y) =  \begin{cases}\frac{x^2 y}{x^2 + \abs{y}}, & \text{si $(x,y) \neq (0,0)$} \\ 0 , & \text{si $(x,y) = (0,0)$}\end{cases} $. 
	\begin{center}
			\begin{tikzpicture}[scale=.5]
				\begin{axis}[,xlabel=$x$,ylabel=$y$]%,xtick=\empty,ytick=\empty,ztick=\empty ]
					\addplot3[surf,opacity=.7,samples=50] gnuplot {(x**2 *y) /(x**2 + abs(y))};
				\end{axis}
			\end{tikzpicture}
		\end{center}

\begin{enumerate}
	\item Montrer que $f$ est continue et différentiable en l'origine.
        \item \'Etudier la différentiabilité de $f$ au point $(a,0)$, $a\neq 0$.
\end{enumerate}
%Niglio exo 12 et 13 p44

%\exo{}
%\'Etudier suivant $\alpha\in\R$ la différentiabilité à l'origine de la fonction définie par: $f(x,y) =  \frac{\abs{x}^\alpha y}{x^2 + \abs{y}}$, est continue et différentiable en l'origine.
%%Niglio exo 14 p44

\newpage

\section{Applications de la différentiabilité}

\exo[*]{} Soit $f: \R^2 \to\R$ la fonction définie par $f(x,y) = \frac{x+y}{1+x^2+y^2}$.
	\begin{center}
			\begin{tikzpicture}[scale=.5]
				\begin{axis}[,xlabel=$x$,ylabel=$y$]%,xtick=\empty,ytick=\empty,ztick=\empty ]
					\addplot3[surf,opacity=.7,samples=50] gnuplot {(x+y) /(1 +x**2 + y**2)};
				\end{axis}
			\end{tikzpicture}
		\end{center}
\begin{enumerate}
	\item Déterminer et représenter ses courbes de niveau.
	\item Calculer les dérivées partielles premières.
	\item Écrire l'équation du plan tangent à $f$ en $(0,0)$
\end{enumerate}
% Facanonni exo 4.17 p 85

\newpage
\exo{} Sachant que la fonction $f: \R^2 \to \R$ est différentiable et que $f(2,5) = 6$, $\frac{\partial f}{ \partial x}(2,5) = 1 $ et $\frac{\partial f}{\partial y} (2,5) = -1$. Donner une valeur approchée de $f(2.2,4.9)$.
% Facanonni exo 4.12 p 85

\newpage

\exo{} Soit $\alpha>0$ et $a = (1,2)$. On pose $f(x)=\snorm{x-a}^\alpha$ pour tout $x\in\R^2$. 
\begin{enumerate}
	\item Représenter les courbes de niveau et le champ de gradient de $f$ pour $\alpha=2$ 
	\item Même question en $\alpha =1$ en précisant bien le domaine de définition.
\end{enumerate}

\newpage

\exo{} Calculer le jacobien en tout point de $\R^3$ des applications suivantes:
\begin{enumerate}
	\item $f(r,\theta,z) = (r\cos\theta,r\sin\theta,z)$
	\item $g(r,\varphi,\theta) = (r\cos\varphi\cos\theta , r \cos\varphi\sin\theta , r \sin \varphi) $
\end{enumerate}

\end{document}
