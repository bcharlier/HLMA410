\chapter{Espace probabilisé}

\sld{\pagebreak[5]}%%%%%%%%%%%%%%%

\section{Introduction}

La théorie des probabilités utilise le vocabulaire ensembliste pour modéliser le résultat d'expériences aléatoires (\ie soumis au \emph{hasard}).

\begin{remark}
	Le mot \textit{aléatoire} vient du latin \textit{alea} qui signifie dé, la chance (\url{http://www.cnrtl.fr/etymologie/al%C3%A9atoire}).
	Le mot \textit{hasard} vient de l'arable \textit{al-zahr} qui signifie dé. \`A rapprocher aussi de \textit{hazard} risque, danger (\url{http://www.cnrtl.fr/etymologie/hasard}).
\end{remark}

%De façon générale, une \textbf{expérience aléatoire} est une expérience renouvelable, au moins en théorie, et qui, renouvelée dans des conditions identiques, ne donne pas à chaque essai le même résultat.

%Pour étudier un phénomène dépendant du hasard, il faut isoler une expérience aléatoire particulière pour en construir un modèle probabiliste %lui permettant de faire certaines prévisions; disons pour être concret qu'il pourra calculer la probabilité de certains résultats ou de certains événements. La première étape de la modélisation mathématique d'une expérience aléatoire consiste à 
%\begin{enumerate}
	%\item spécifier \emph{l'ensemble des résultats possibles} (ou réalisations) de cette expérience, que l'on note $\Omega$. On désigne également $\Omega$ sous le nom d'\textbf{univers}. 
	%\item calculer la probabilité des éléments de $\Omega$,  notés $\omega$ et appelés résultats élémentaire possibles,.
%\end{enumerate}


{\bf\sffamily But:} prédire les fréquences d'apparition d'événements et non le résultat précis d'une expérience.

%\begin{exemple}
	%Le jeu de dés à 6 faces.
%\end{exemple}
\sld{\vfill\pagebreak[5]}%%%%%%%%%%%%%%%


\section[Ensembles]{Ensembles et opérations sur les ensembles}

\subsection{Définitions}

Un ensemble $\Omega$ est constitué de points $\omega$ tous distincts. L'expression logique 
\[
\omega \in \Omega
\]
signifie que $ \omega$ appartient à $\Omega$. On dit qu'un ensemble $A$ est inclu dans $\Omega$ si
\[
	\omega \in A \Rightarrow \omega\in\Omega.
\]
On note $A \subset \Omega$ et on dit que $A$ est un \emph{sous-ensemble} de $\Omega$.

\begin{definition}
	On note $ \emptyset$ l'ensemble vide (qui ne contient aucun élément).
\end{definition}

\sld{\vfill\pagebreak[5]}%%%%%%%%%%%%%%%


\subsection{Opérations}

\begin{definition}
	Soit $A,B,C \subset \Omega$:
	\begin{enumerate}
		\item \emph{$\boldsymbol A$ inter $\boldsymbol B$} est noté $A \cap B = \left\{ \omega \in \Omega | \omega \in A \text{ et } \omega\in B \right\}$. \pl{\rep{2cm}}
		\item \emph{$\boldsymbol A$ union $\boldsymbol B$} est noté $A \cup B = \left\{ \omega \in \Omega | \omega \in A \text{ ou } \omega\in B \right\}$. \pl{\rep{2cm}}
		\item \emph{$\boldsymbol A$ complémentaire} est noté $A^c = \left\{ \omega \in \Omega | \omega \notin A \right\}$.\pl{\rep{2cm}}
	\end{enumerate}
On retrouve aussi d'autres notations:
\begin{enumerate}
	\item \emph{$\boldsymbol A$ privé de $\boldsymbol B$} noté $A \setminus B = A \cap B^c$.\pl{\rep{2cm}}

	\item \emph{différence symétrique entre $\boldsymbol A$ et $\boldsymbol B$} noté $A \Delta B = (A \cap B^c) \cup ( A^c \cap B) = (A \cup B)  \setminus (A \cap B)$.\pl{\rep{2cm}}

\end{enumerate}
\end{definition}

\begin{proposition}
Les opérations $\cup $ et $\cap $ sont:
	\begin{enumerate}
		\item commutatives: $ A \cup B = B\cup A$ et $ A \cap B = B\cap A$
		\item associatives: $(A \cup B) \cup C = A \cup (B \cup C)$ et $(A \cap B) \cap C = A \cap (B \cap C)$
	 \item élément neutre:$A \cup \emptyset=A$ et $A \cap \Omega=A$
	 \item distributives: $(A \cup B) \cap C = (A \cap C)  \cup (B \cap C)$ et  $(A \cap B) \cup C = (A \cup C)  \cap (B \cup C)$.\pl{\rep{2cm}}
 \end{enumerate}

\end{proposition}

\sld{\vfill\pagebreak[5]\vfill}%%%%%%%%%%%%%%%


On peut alors en déduire des règles de calculs suivantes:
\begin{proposition}
	\begin{enumerate}
		\item \emph{Loi de De Morgan:} $(A \cap B)^c=A^c \cup B^c$, $(A \cup B)^c=A^c \cap B^c$
		\item \emph{Partition} $A \cup A^c = \Omega$ et  $A \cap A^c = \emptyset$
		\item  $A \cup \Omega = \Omega$ et  $A \cap \emptyset=\emptyset$
		\item  $A \cup A=A$ et  $A \cap A=A$
	\end{enumerate}
\end{proposition}

\begin{proof}
	\pl{\rep{7cm}}
\end{proof}

\subsection{Fonctions indicatrices}


\begin{definition}
	Soit $A\subset \Omega$. La \emph{fonction indicatrice} de $A$ est définie par
	\begin{align*}
		\one_A: \Omega &\to \left\{ 0,1 \right\}\\
			\omega & \mapsto \begin{cases}
				1, & \text{ si } \omega \in A\\
				0, & \text{ sinon }
			\end{cases}
	\end{align*}
\end{definition}

\begin{proposition}
	Soit $A, B \subset \Omega$ on a
	\begin{align*}
		\one_{A\cup B} &= \max \left\{ \one_A, \one_B \right\} = \one_A + \one_B - \one_A \one_B\\
		\one_{A\cap B} &= \min \left\{ \one_A, \one_B \right\}  = \one_A \one_B\\
		\one_{A^c} &= 1 - \one_A
	\end{align*}
\end{proposition}
\begin{proof}
	
	\pl{\rep{5cm}}
	\pl{\rep{4cm}}
\end{proof}



\begin{definition}
L'ensemble des parties de $\Omega$ est noté $\Pp(\Omega)$. 
\end{definition}

\begin{exemple}
	Si $\Omega = \left\{ 1,2 \right\}$  alors $\Pp(\Omega) = \left\{ \left\{ 1 \right\},\left\{ 2 \right\},\left\{ 1,2 \right\}, \emptyset \right\}$.
\end{exemple}

\sld{\pagebreak[5]}%%%%%%%%%%%%%%%

\subsection{Cardinalité}

\begin{definition}
	Le nombre d'éléments d'un ensemble $\Omega$ est le \emph{cardinal} de $\Omega$ et est noté $\card(\Omega)$ ou $ \abs{\Omega}$.
	\begin{enumerate}
	\item Lorsque $\card(\Omega) < +\infty$ on parle d'\emph{ensemble fini}.
	\item Lorsque $\card(\Omega) = +\infty$ on distingue les cas suivants
		\begin{enumerate}
		\item si $\Omega$ peut être mis en bijection avec $\N$, on parle d'\emph{ensemble dénombrable}.
		\item dans le cas contraire on parle d'\emph{ensemble non-dénombrable}.
		\end{enumerate}
	\end{enumerate}
\end{definition}

	\begin{exemple}
\sld{
		\begin{enumerate}
			\item $ \left\{ 1,\cdots,n \right\}$ est un ensemble fini.
			\item $\N$, $\Z$, $\Q$, $\N^{2003}$ sont des ensembles dénombrables.
			\item $\R$, $[0,1]$, $\R^{\N}$ sont des ensembles non dénombrables.
		\end{enumerate}
}\pl{\rep{3cm}}
	\end{exemple}


\sld{\pagebreak[5]}%%%%%%%%%%%%%%%

\subsection{Dénombrement}

Dans cette section, les ensembles considérés sont finis.

\subsubsection{Principe de bijection}

Lorsque l'on veut compter les éléments d'un ensemble, on montre que cet ensemble est en bijection avec un ensemble dont on connaît le cardinal. Le reste de cette section énonce un certains nombre de résultats qu'il faut connaître.

\subsubsection{Principe d'indépendance}

Soit $A$ et $B$ deux ensembles finis, on a $A \times B = \left\{ (a,b) | a\in A, b\in B \right\} $. On a alors
\[
| A\times B| = |A|\cdot[B|
\]

\subsubsection{Principe de partition}

On dit que les ensembles $(A_i)_{i\in I}$ forme une partition de $A$ si $A = \bigcup_{i\in I} A_i$ et $i\neq j \Rightarrow A_i\cap A_j = \emptyset$. On a alors 
\[
	|A| = \sum_{i\in I} |A_i|
\]

\sld{\vfill\pagebreak[5]}%%%%%%%%%%%%%%%


\subsubsection{Lemme des bergers}
Ce résultat généralise la procédure de comptage suivante: Quand un berger veut compter ses moutons, il compte le nombre de pattes  puis divise par quatre.

\begin{proposition}
	[(Lemme des bergers)]
	Soit $\phi:D \to A$ une application surjective. On suppose qu'il existe un entier $a \geq 1$ tel que pour tout $y\in A$
	\[
		|\left\{ x \in D | \phi(x) =y \right\}| = a
	\]
	C'est à dire si tout $y\in A$ a exactement $a$ antécédent dans $D$. Alors on a 
	\[
		|A| = \frac{|D|}{a}
	\]
\end{proposition}

\begin{proof}
	On applique le principe de partition à l'ensemble $D$: les ensembles $(D_y)_{y\in A}$ où $D_y = \{ x\in D | \phi(x) =y \}$ forment une partition de $D$. D'où
	\[
		|D| = \sum_{y\in A} |D_y| = \sum_{y\in A} a = a |A|. \qedhere
\]
\end{proof}


\subsubsection{Quelques résultats incontournables}

\paragraph{Nombre d'applications de $D$ dans $A$} Il existe exactement $|A|^{|D|}$ applications de $D$ dans $A$, ce qui peut s'écrire 
\[
	|A^D| = |A|^{|D|}
\]

\begin{remark}
	Posons $|D|=p$ et $|A|=n$. Un cas particulier important est celui où $D =\left\{ 1,\cdots,p \right\}$. Or un $p$-uplet $(x_1,\cdots,x_p)$ d'éléments de $A$ peut être vu comme le graphe d'une application de $\left\{ 1,\cdots,p \right\} \to A$. Le nombre de $p$-uplets $(x_1,\cdots,x_p)$ dont les composantes sont des éléments de $A$ est $n^p$ (on retrouve le principe d'indépendance et le nombre de tirage avec remise).
\end{remark}

\begin{exemple}
	Un professeur note chaque étudiant d'une promotion de 300 étudiants avec une note entière entre 0 et 20. Combien de résultats sont possibles ?
\pl{\rep{3cm}}
\sld{ Soit $D$ ensemble des étudiants et $A=\left\{ 0,\cdots,20 \right\}$ l'ensemble des notes. On compte l'ensemble des applications de $D$ dans $A$. Il vient ${21}^{300}$.}
\end{exemple}
\paragraph{Nombre de permutations} Soit $\Omega$ un ensemble fini avec $|\Omega|=n$. Le nombre de permutations de $\Omega$  est 
\[  n !  = n(n-1)\cdots 1 \]

\begin{remark}
	Une permutation  de $\Omega$ est une bijection de $\Omega$ dans lui même.
\end{remark}

\begin{exemple}
	Quel est ne nombre de manières de mélanger un jeu de 32 cartes ?
	\pl{\rep{3cm}}
	\sld{On a $\Omega$ l'ensemble des cartes. On compte en fait le nombre de permutations de $\Omega$. Réponse : $32!$.}

\end{exemple}

\sld{\vfill\pagebreak[5]}%%%%%%%%%%%%%%%

\paragraph{Nombre d'injection de $D$ dans $A$}

On pose $|D| = p$ et $|A| = n$. Il existe une injection de $D$ vers $A$ si et seulement si $p \leq n$. Dans ce cas le nombre d'injection de $D$ dans $A$ est 
\[
	A_n^p = n(n-1) \cdots (n-p+1) = \frac{n!}{(n-p)!}.
\]
\begin{remark}
	Lorsque $n=p$ on a $A_n^n = n!$. En fait un injection entre deux ensembles de même cardinal est une bijection.
\end{remark}

\begin{exemple}
	On a 3500 candidats à un concours. Seules 300 places sont disponibles. Combien de palmarès possible y a-t-il ? (On suppose qu'il n'y a pas d'\textit{ex \ae{}quo})
	\pl{\rep{3cm}}
	\sld{On a $D=\left\{ 1,\cdots,300 \right\} $ l'ensemble des rangs et $A$ est l'ensemble des candidats. On compte en fait le nombre d'applications injectives $D \to A$ car deux candidats ne peuvent avoir le même rang. Réponse: $3500 \times 3499 \times \ldots \times 3201$}
\end{exemple}


\sld{\vfill\pagebreak[5]}%%%%%%%%%%%%%%%

\paragraph{Nombre de parties a $p$ éléments dans $\Omega$} On pose $|\Omega|=n$. 
Le nombre de parties à $p$ éléments dans un ensemble à $n$ éléments est 
\[
	\binom{n}{p} =  \frac{n!}{(n-p)! p!}.
\]

\begin{proof}
	\sld{\small}	On applique le lemme des bergers à 
	\begin{itemize}
		\item $D$ l'ensemble des injections de $\left\{ 1,\cdots,p \right\}$ dans $\Omega$
		\item $A$ est l'ensemble des parties à $p$ éléments dans $\Omega$
		\item $\phi: D\to A$ définie par $\phi(f) = \left\{ f(k) ;  k= 1,\cdots,p\right\}$
	\end{itemize}
	On a vu que $|D| = \frac{n!}{(n-p)!}$. Il n'est pas difficile de voir que $\phi$ est surjective. De plus, étant donnée une partie $\left\{ x_1,\cdots,x_p \right\}$ de $\Omega$, le nombre d'injections $f:\left\{ 1,\cdots,p \right\}\to \Omega$ telles que $\left\{ f(1),\cdots,f(p) \right\} = \left\{ x_1,\cdots,x_p \right\}$ est $p!$ (le nombre de permutation de $\{1,\cdots,p\}$). 
	On applique le lemme des bergers avec $a=p!$. 
\end{proof}

\begin{exemple}
	On a 3500 candidats à un concours. Seules 300 places sont disponibles. Combien de listes alphabétiques de reçus possible y a-t-il ? (On suppose que tous les noms sont différents)
	\pl{\rep{3cm}}
	\sld{On a $\Omega$ l'ensemble des candidats. On compte en fait le nombre de parties à $300$ éléments dans $\Omega$. Réponse: $\binom{3500}{300}$.}
\end{exemple}

\paragraph{Nombre totale de partie} Si $\Omega$ est de cardinal fini avec $|\Omega|= n$ alors $ \card(\Pp(\Omega)) = 2^n$.

\begin{proof}
		\pl{\rep{4cm}} \sld{Remarquer que l'application \[
			\Pp(\Omega) \to \left\{ 0,1 \right\}^\Omega
		\]
	qui a un ensemble $A$ associe son indicatrice $\one_A$ est une bijection. }
\end{proof}
\begin{exemple}
	On a 3500 candidats à un examen.Combien de listes alphabétiques de reçus possible y a-t-il ? (On suppose que tous les noms sont différents)
	\pl{\rep{3cm}}
	\sld{On a $\Omega$ l'ensemble des candidats. On compte en fait le nombre de parties à $\Omega$ car le nombre de reçus n'est pas fixé à l'avance. Réponse: $2^{3500}$.}
\end{exemple}

\sld{\vfill\pagebreak[5]\vfill}%%%%%%%%%%%%%%%
\section{Évènement et probabilités}

\subsection{Expérience aléatoire et événements}

Une \emph{expérience aléatoire} est  une expérience renouvelable (au moins en théorie) dont on ne peut prédire l'issue de manière précise (renouvelée dans des conditions identiques, ne donne pas à chaque essai le même résultat). On connaît cependant l'ensemble des résultats possibles \textit{a priori}.

 \pl{
	 \begin{exemple}
 Voici quelques exemples d'expériences aléatoires:
	 \begin{enumerate}
		 \item lancer deux dés;
		 \item distribuer les cartes au tarot, c'est-à-dire répartir le paquet de 78 cartes entre les 3, 4 ou 5 joueurs;
		 \item observer la formation d'un caractère génétique d'un individu à partir des caractères correspondants de ses parents;
		 \item observer la désintégration d'un noyau atomique radioactif;
		 \item attendre le tram à la station Universités, à partir de 18h.
	 \end{enumerate}
 \end{exemple}}
 \sld{\addtocounter{exemple}{1}}

 \begin{definition}
	 \'Etant donnée une expérience aléatoire, l'\emph{univers} noté $\Omega$ est l'ensemble des résultats possibles (ou \emph{événement élémentaire}). Un \emph{événement} est un sous-ensemble de $\Omega$ (réunion d'événements élémentaires). 
 \end{definition}

 \begin{exemple}
	 Expérience aléatoire: jet d'un dé à 6 faces. 
\sld{
	 \begin{enumerate}
		 \item Univers: $\Omega =\Big \{  \{\text{face du dessus est \drawdie{1}} \}, \cdots , \{\text{face du dessus est \drawdie{6}}\}\Big \}$
		 \item Évènement élémentaires: $\{\text{face du dessus est \drawdie{1} }\}$, \ldots ,$\{\text{face du dessus est \drawdie{6}}\}$. 
		 \item Évènement: $\{\text{face du dessus est \drawdie{4} ou \drawdie{5} ou \drawdie{6} }\} = \Big\{\{\text{face du dessus est,\drawdie{4}}\}, \cdots,\linebreak[5]  \{\text{face du dessus est \drawdie{6}}\Big\}  \} =\{\text{face du dessus est \drawdie{4}}\} \cup  \cdots \cup \{\text{face du dessus est \drawdie{6}}\} $
	 \end{enumerate}
 }\pl{\rep{4cm}}
 \end{exemple}

 
On a le ``dictionnaire'' suivant entre les vocabulaires ensemblistes et probabilistes:
\begin{center}
\begin{tabular}{cc}
\hline Terminologie probabiliste & Terminologie ensembliste \\
\hline
univers, événement certain& $\Omega$ \\
événement impossible  & $\emptyset$ \\
résultat possible, événement élémentaire & $\{\omega\}$ où $\omega$ élément de  $\Omega$ \\
événement & $A$, sous-ensemble de  $\Omega$ \\
$A$  est réalisé & $\omega \in A$ ou $\left\{ \omega \right\} \subset A $\\
$A$  implique  B & $A \subset B$\\
$A$  ou  $B$ & $A \cup B$ \\
$A$  et  $B$ & $A \cap B$ \\
contraire de  $A$ & $A^c$, complémentaire de  $A$ \\
$A$  et  $B$  sont incompatibles & $A \cap B = \emptyset$
\\ \hline
\end{tabular}
\end{center}

\sld{\vfill\pagebreak[5]}%%%%%%%%%%%%%%%

\subsection{Variable aléatoire}

Quand on étudie un phénomène aléatoire, on est amené à étudier des grandeurs numériques (ou vectorielles) liées à celui-ci. En termes vagues, une variable aléatoire est un nombre ou un vecteur (une variable) dont la valeur dépend de l'issue $\omega$ d'une expérience aléatoire. Mathématiquement, il s'agit donc d'une fonction sur l'ensemble $\Omega$. Si cette fonction est à valeurs dans $\R$, on parle de \emph{variable aléatoire réelle}. %Nous donnerons une définition plus formelle par la suite.

\begin{exemple}
	 Expérience aléatoire: jet d'un dé à 6 faces. \sld{On note toujours 
	 \[
		 \Omega =\Big \{  \{\text{face du dessus est \drawdie{1}} \}, \cdots , \{\text{face du dessus est \drawdie{6}}\}\Big \}. 
	 \]
	 La variable aléatoire $ X: \Omega \to \left\{ 1,2,\cdots,6 \right\}$ associe à l'événement $\{\text{face du dessus est \drawdie{1}}\}$ le nombre entier $1$, $\{\text{face du dessus est \drawdie{2}}\}$ le nombre entier $2$, \textit{etc} % \in\left\{ 1,2,\cdots,6 \right\}$.
	 %\[  \begin{array}{c} \{ \text{face du dessus est \drawdie{1}} \} \\ \downarrow\\1\end{array}, \cdots , \{\text{face du dessus est \drawdie{6}}\}\Big \}.\]
 }
 \pl{\rep{4cm}}
\end{exemple}

\sld{\vfill\pagebreak[5]}%%%%%%%%%%%%%%%

\section{Tribus et probabilités}
\begin{definition}
Une famille $\mathcal{A}$ de sous-ensembles d'un ensemble $\Omega$ est une \emph{tribu} ou \emph{$\boldsymbol\sigma$-algèbre} sur $\Omega$ si elle satisfait aux trois axiomes suivants:
\begin{enumerate}[label=$(\roman*)$]
	\item $\Omega \in \mathcal{A}$.
	\item Si $A \in \mathcal{A}$, alors $A^c \in \mathcal{A}$.
	\item Pour toute suite $(A_n)_{n \in \mathbb{N}}$ d'éléments de $\mathcal{A}$, on a $\bigcup_{n \in \mathbb{N}} A_n \in \mathcal{A}$.
\end{enumerate}
\end{definition}

\begin{exemple}
	 \sld{\begin{enumerate}
		\item l'ensemble $\Pp(\Omega)$ est une tribu.
		\item Si A est un sous-ensemble de $\Omega$, la famille $\{A,A^c,\Omega,\emptyset\}$ est une tribu sur $\Omega$ dite \emph{tribu engendrée} par l'événement A.
		\item La tribu $\{\Omega,\emptyset\}$ est appelée \emph{tribu triviale} sur $\Omega$.
	\end{enumerate} }
 \pl{\rep{4cm}}

\end{exemple}


\sld{\vfill\pagebreak[5]}%%%%%%%%%%%%%%%

\begin{definition}
On appelle \emph{espace probabilisable} un couple $(\Omega,\mathcal{A})$ où $\Omega$ est un ensemble et $\mathcal{A}$ une tribu sur l'ensemble $\Omega$. Quand un espace probabilisable est fixé, on dit que $\mathcal{A}$ est \emph{la tribu des événements}.
\end{definition}

\begin{exemple}
	Jet d'un dé: $	 \Omega =\Big \{  \{\text{\drawdie{1}} \}, \cdots , \{\text{\drawdie{6}}\}\Big \} $. On a $\mathcal A = \mathcal P(\Omega)$.
\end{exemple} 

La modélisation d'un phénomène aléatoire et d'une famille d'événements commence par le choix d'un espace probabilisable qui rend compte de l'ensemble des réalisations envisagées et des événements qui peuvent être sujets de l'étude.

\begin{definition}
	Soit $(\Omega,\mathcal{A})$ un espace probabilisable. Une \emph{probabilité} sur cet espace est une application $\mathbb{P}: \mathcal{A} \to [0,1]$ satisfaisant:
\begin{enumerate}[label=$(\roman*)$]
	\item $\mathbb{P} (\Omega) = 1$.
	\item Propriété de \emph{$\boldsymbol\sigma$-additivité}: pour toute suite $(A_n)_{n \in \mathbb{N}}$ d'éléments \emph{disjoints} deux à deux de $\mathcal{A}$,
	$$ \mathbb{P} \Big( \bigcup_{n \in \mathbb{N}} A_n \Big) = \sum_{n=0}^{\infty} \mathbb{P} (A_n) .$$
\end{enumerate}
\end{definition}
\begin{exemple}
	Jet d'un dé: $	 \P(\{\text{face du dessus \drawdie{1}}\}) = \cdots = \P( \{\text{face du dessus \drawdie{6}}\} ) = \frac 1 6 $.
\end{exemple} 

\sld{\vfill\pagebreak[5]}%%%%%%%%%%%%%%%



\begin{proposition}
	\redspace
\begin{enumerate}[label=$(\roman*)$]

	\item Pour tout $A \in \mathcal{A}$, on a: $\mathbb{P} (A^c) = 1 - \mathbb{P} (A)$. En particulier: $\mathbb{P} (\emptyset) = 0$.

	%\item Pour toute suite finie $(A_i)_{1 \leq i \leq n}$ d'éléments disjoints deux à deux de $\mathcal{A}$,

	%$$ \mathbb{P} \Big( \bigcup_{i=1}^n A_i \Big) = \sum_{i=1}^n \mathbb{P} (A_i) .$$

	\item Si $A,B \in \mathcal{A}$ et si $A \subseteq B$, on a
		\[ \mathbb{P} (A) \leq \mathbb{P} (B) \quad \textrm{ et } \quad  \mathbb{P} (B \setminus A) = \mathbb{P} (B) - \mathbb{P} (A).\]

	\item Pour toute suite croissante $(A_n)_{n \in \N}$ d'éléments de $\mathcal{A}$, c'est-à-dire tels que $A_n \subset A_{n+1}$ pour tout $n \in \N$, on a:
	\[ \mathbb{P} \Big( \bigcup_{n \in \mathbb{N}} A_n \Big) = \lim_{n \rightarrow \infty} \mathbb{P} (A_n) .\]

	\item Pour toute suite décroissante $(A_n)_{n \in \N}$ d'éléments de $\mathcal{A}$, c'est-à-dire tels que $A_{n+1} \subset A_n$ pour tout $n \in \N$, on a:
	\[ \mathbb{P} \Big( \bigcap_{n \in \mathbb{N}} A_n \Big) = \lim_{n \rightarrow \infty} \mathbb{P} (A_n) .\]
\end{enumerate}% \sld{ } } 
\end{proposition}
\begin{proof}
	On montre uniquement $(i)$ et $(ii)$
\sld{\begin{enumerate}[label=$(\roman*)$]
	\item $1 = \P(\Omega) = \P(A \cup A^c) = \P(A) + \P(A^c)$
	\item Si $A \subset B$ on a $A\cap B=A$. Cela donne $\P(B) = \P(A \cup (B\cap A^c)) = \P(A) + \P(B \setminus A)$.
\end{enumerate}}\pl{\rep{5cm}}
\end{proof}

\sld{\vfill\pagebreak[5]}%%%%%%%%%%%%%%%


\begin{proposition}
Pour toute suite finie $(A_i)_{1 \leq i \leq n}$ d'éléments disjoints deux à deux de $\mathcal{A}$,
	$$ \mathbb{P} \Big( \bigcup_{i=1}^n A_i \Big) = \sum_{i=1}^n \mathbb{P} (A_i) .$$
\end{proposition}

\begin{proof}
	Découle directement de la $\sigma$-additivité. \end{proof}

\begin{remark}
Cette formule est particulièrement utile pour le calcul des probabilités. Dans le cas $n=2$ on a pour $A,B \subset \Omega$: $\P(A) = \P(A\cap B) + \P(A\cap B^c)$.
\end{remark}
\pl{\pagebreak[4]}

\sld{\vfill\pagebreak[5]}%%%%%%%%%%%%%%%


\begin{proposition}[(Formule de Poincaré ou du crible)]
Si $n \geq 2$, pour toute suite finie $(A_i)_{1 \leq i \leq n}$ d'éléments de $\mathcal{A}$, on a:
\begin{align*}
	\mathbb{P} \Big( \bigcup_{i=1}^n A_i \Big) =  & \sum_{i=1}^n \mathbb{P} (A_i) \\ &\quad -   \sum_{ (i,j) | 1 \leq i < j \leq n} \mathbb{P} (A_i \cap A_j)\\ & \qquad\qquad +  \sum_{(i,j,k) | 1 \leq i < j < k \leq n} \mathbb{P} (A_i \cap A_j \cap A_k) \\ &\qquad \qquad\qquad - \quad \cdots \quad  \\ & \qquad\qquad\qquad\qquad + (-1)^{n-1} \mathbb{P} (A_1 \cap A_2 \cap \cdots \cap A_n).
\end{align*}

\end{proposition}

\begin{proof}
On se contente de montrer et comprendre la formule sur le cas $n=2$ et $n=3$. La formule générale se montre par récurrence.
		\pl{\rep{5cm}}
\end{proof}

\section{Espaces probabilisés discrets}

\subsection{Définition}

\begin{definition}
On appelle \emph{espace probabilisable discret} un espace probabilisable  $(\Omega,\mathcal{A})$ où $\Omega$ est dénombrable et où $\mathcal{A}= \mathcal{P} (\Omega)$, l'ensemble des parties de $\Omega$.  
\end{definition}

Une probabilité $\P$ sur un espace probabilisable discret $\big( \Omega,\mathcal{P} (\Omega) \big)$ est définie par les probabilités des réalisations élémentaires $\{\omega\} \subset \Omega$. On a:
$$ \mathbb{P} (A) = \sum_{\omega \in A} \P (\{\omega\}) $$
pour tout sous-ensemble $A$ de $\Omega$ car les réalisations $\omega$ sont incompatibles deux à deux.

\subsection{Probabilité uniforme sur un ensemble fini}

La probabilité uniforme sur un ensemble fini $\Omega$ attribue la même probabilité à chaque réalisation élémentaire $\{\omega\}$:
\[ \mathbb{P} (\{\omega\}) = \frac{1}{|\Omega|},\]
où $|\Omega|$ désigne le cardinal (nombre d'éléments) de $\Omega$. Par conséquent, on a:
\begin{equation}
	\mathbb{P} (A) = \frac{|A|}{|\Omega|}
	\label{eq.prob_unif}
	\redspace
\end{equation}
pour tout événement $A \subset \Omega$.

\begin{remark}
	Lorsque l'on parle de choix ``au hasard'' sur un ensemble fini, on sous-entend que ce choix est fait au moyen de la probabilité uniforme, c'est-à-dire en donnant à chaque élément de l'ensemble les mêmes chances d'être choisi. On résume souvent la formule \eqref{eq.prob_unif} par  \[\frac{\text{nombre de cas favorables}}{\text{nombre total de cas}}.\] Cela suggère que le calcul des probabilité revient à faire des dénombrements.  On s'aperçoit également qu'une probabilité uniforme sur un ensemble non fini n'est pas définie.
\end{remark}
\sld{\vfill\pagebreak[5]}%%%%%%%%%%%%%%%



\begin{exemple}
On lance $n$ fois un dé à 6 faces équilibré et on cherche la probabilité d'obtenir $k$ fois ($k \leq n$) la face \drawdie{6}.\sld{	 Une réalisation est une succession de $n$ lancers donc $\Omega=\{\drawdie{1},\cdots,\drawdie{6}\}^n$ et $|\Omega|=6^n$. L'événement ``\textit{obtenir $k$ fois $\drawdie{6}$}'' est représenté par le sous-ensemble
\[ A_k=\{(x_1,\cdots,x_n)\in \Omega | x_i=\drawdie{6} \textrm{ pour $k$ indices exactement } \}.\]

Pour obtenir une réalisation appartenant à $A_k$, il nous faut d'abord choisir les $k$ indices pour lesquels les $x_i$ valent $ \drawdie{6}$, puis affecter aux $x_i$ restants une face entre \drawdie{1} et \drawdie{5}. On a donc $|A_k| = \binom{n}{k} 5^{n-k}$ et 
\[\mathbb{P} (A_k) = \binom{n}{k}\frac{5^{n-k}}{6^n}. \]
	%où $\binom{n}{k}=\frac{n !}{k ! (n-k)!} $ est le nombre de combinaisons de $k$ objets parmi $n$. 

	Si $n=3$, on a $\mathbb{P} (A_0)=\frac{5^3}{6^3}=\frac{125}{216}$, $\mathbb{P} (A_1)=3 \times \frac{5^2}{6^3}=\frac{75}{216}$, $\mathbb{P} (A_2)=3 \times \frac{5}{6^3}=\frac{15}{216}$ et $\mathbb{P} (A_3)=\frac{1}{216}$. On vérifie que 
	\[ \mathbb{P} (\Omega) = \mathbb{P} (A_0 \cup A_1 \cup A_2 \cup A_3) = \mathbb{P} (A_0) + \mathbb{P} (A_1) + \mathbb{P} (A_2) + \mathbb{P} (A_3) = 1.\] }
	\pl{\rep{10cm}}
\end{exemple}

