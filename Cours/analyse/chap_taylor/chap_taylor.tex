%\chapter{Développement de Taylor de fonctions à plusieurs variables}
\chapter{Dérivées d'ordres supérieurs et études des extrema}


\sld{\vfill\pagebreak[5]}%%%%%%%%%%%%%%%
\section{Dérivées partielles d'ordre supérieur}

Soit $E$ un \rev{} et $\U$ un ouvert de $E$. On suppose  que $\dim E = n$ et $\mathcal B = \left( e_1,\cdots,e_n \right)$ est une base de $E$.

\subsection{Définitions et propriétés}

\begin{definition}
	%[(Dérivée partielle seconde)]
	Soit $f: \U \to F$ et  $a\in\U$. On suppose que $f$ admet une $j$-ème dérivée partielle $\frac{\partial f}{\partial x_j} (a)$ pour $j=1,\cdots,n$. Si $\frac{\partial f}{\partial x_j}$ admet en $a$ une $k$-ème dérivée partielle $\frac{\partial \left(\frac{\partial f}{\partial x_j}\right)}{\partial x_k} $, $1 \leq k \leq n$, on dit que $f$ admet en $a$ une $(k,j)$-ième \emph{dérivée partielle seconde} que l'on note $\frac{\partial^2 f}{\partial x_k \partial x_j} (a)$.
\end{definition}

\begin{remark}
	En itérant le procédé, on définit les dérivées partielles triples, quadruples\ldots
\end{remark}


\sld{\vfill\pagebreak[5]}%%%%%%%%%%%%%%%
\begin{definition}
	%[(Classe $\Cc^k$)]
	Une fonction $f:\U \to F$ est de \emph{classe $\Cc^k$} sur $\U$ si pour tout $j_1,j_2,\cdots,j_k \in \left\{ 1,\cdots,n \right\}$
la fonction $\frac{\p^k f}{\p x_{j_k} \cdots   \p x_{j_2}\p x_{j_1} }: \U \to F$ est continue sur $\U$.

On dit que $f$ est de classe $\Cc^\infty$ si $f$ est de classe $\Cc^k$ pour tout $k\in\N$.
\end{definition}


\begin{exemple}
Calcul des dérivées partielles secondes de la fonction
$
%f_1(x,y) = x^2; \quad 
f(x,y)= x^2\cos(y)
%; \quad f_3(x,y) = x^y.
$.
\sld{
    On a fait le calcul directement et on remarque que $\frac{\partial^2f}{\\partial x \partial y}(x,y) = \frac{\partial^2f}{\\partial x \partial y} (x,y) = $. 

    La question qui va nous occuper dans la suite de ce chapitre est: quelles sont les conditions sur la fonction $f$ pour que cela soit toujours le cas.
}
\pl{\rep{5cm}}
\end{exemple}

\sld{\vfill\pagebreak[5]}%%%%%%%%%%%%%%%
Le calcul des dérivées secondes de l'exemple précédent semble suggérer que les dérivées secondes $\frac{\p^2 f}{\p x_i \p x_j}$ et $\frac{\p^2 f}{\p x_j \p x_i} $ sont égales. C'est le cas pour les fonctions suffisamment régulières:
\begin{theorem}
	[(Schwarz)]
	On suppose que $f:\U \to F$ est de classe $\Cc^2$ sur $\U$. Alors,  pour tout $i,j \in \left\{ 1,\cdots,n \right\}$ on a sur $\U$  %$\frac{\p^2 f}{\p x_i \p x_j} $ et $\frac{\p^2 f}{\p x_j \p x_i}$ existent sur un ouvert $\U$ de $E$. Si ces fonctions sont continues en $a\in\U$, alors 
	\[\frac{\p^2 f}{\p x_i \p x_j}=\frac{\p^2 f}{\p x_j \p x_i} \]
\end{theorem}

\begin{proof}
    \sld{ 
L'idée de la preuve est de montrer que l'on a l'interversion de limite suivante:

    }
	\pl{\rep{25cm}}	
\end{proof}


\sld{\vfill\pagebreak[5]}%%%%%%%%%%%%%%%
Le résultat s'étend aux dérivées partielles d'ordre supérieur: 
\begin{proposition}
	Si $f:E\to F$ est de classe $\Cc^k$ sur $\U$, alors pour tout $j_1,\cdots,j_k \in \left\{ 1,\cdots,n \right\}$ et toute permutation $\sigma$ de $\left\{ 1,\cdots,k \right\}$, on a:
	\[
		\frac{\p^k f}{ \p x_{j_k} \cdots \p x_{j_1}  }	= \frac{\p^k f}{ \p x_{j_{\sigma(k)}} \cdots \p x_{j_{\sigma(1)}}  }
	\]
\end{proposition}

\begin{proof}
	C'est un corollaire du théorème de Schwarz.
\end{proof}

{\bf\sffamily Notations: }  
	Par exemple, si $f$ est de classe $\Cc^4$ sur $\U$ les calculs de dérivées partielles d'ordres $\leq 4$ peuvent se faire dans un ordre arbitraire et on écrit: $\frac{\p^{4} f}{\p x_1^{2}\p x_2^{2} }$ pour $\frac{\p^{4} f}{\p x_2 \p x_1 \p x_2   \p x_1 }$. 

%Plus généralement, si $f$ est de classe $\Cc^k$, $h = (h_1,\cdots,h_n) \in \R^n$ et $\alpha = (\alpha_1,\cdots,\alpha_n) \in \N^n$, on note $\abs{\alpha} = \alpha_1 + \cdots+ \alpha_n$, $h^\alpha =  h_1^{\alpha_1}h_2^{\alpha_2}\cdots h_n^{\alpha_n}$ et 
%\[
	%\frac{\p^{\abs{\alpha}} f}{\p x^\alpha}  =  \frac{\p^{\alpha_1} f}{\p x_1^{\alpha_1}} \frac{\p^{\alpha_2} f}{\p x_2^{\alpha_2}} \cdots\frac{\p^{\alpha_n} f}{\p x_n^{\alpha_n}} 
%\]

\sld{\vfill\pagebreak[5]}%%%%%%%%%%%%%%%
\subsection{Opérations sur les fonctions de classe $\Cc^k$}

\begin{proposition} Soient $E,F,G$ trois espaces vectoriels normés, $\U \subset E$ et $\V \subset F $ des ouverts~: 
	\begin{enumerate}[label=$(\roman*)$]
		\item Addition: $f,g: \U \to F$ de classe $\Cc^k$ sur $\U$ alors $f+g$ est de classe $\Cc^k$ sur $\U$. 
		\item Multiplication par un scalaire: $f:\U\to F$  de classe $\Cc^k$ sur $\U$ alors $\lambda f$ est  de classe $\Cc^k$ sur $\U$.
		\item Multiplication (cas de $F=\R$): $f,g:\U\to \R$  de classe $\Cc^k$ sur $\U$ alors $fg$ est  de classe $\Cc^k$ sur $\U$.
		\item Inverse (cas de $F=\R$): $f:\U\to \R$  de classe $\Cc^k$ sur $\U$ et $a \in \U$ tel que $f(a) \neq 0$ alors $\frac{1}{f}$ est  de classe $\Cc^k$ sur un voisinage de $a\in \U$.
		\item Composition: $f: \U \to F$ et $g: F \to G $. Si $f$ est  de classe $\Cc^k$ sur $\U$ et $g$ est  de classe $\Cc^k$ sur $\V\supset f( \U)$, alors $g\circ f:E \to G$ est  de classe $\Cc^k$ sur $\U$. 
	\end{enumerate}
\end{proposition}


\sld{\vfill\pagebreak[5]}%%%%%%%%%%%%%%%
\subsection{Formules de Taylor et matrice hessienne}

%\begin{theorem}
	%[(Formule de Taylor-Young)]
	%Soit $f:\U \to F$ une fonction de classe $\Cc^k$ et $ a \in \U$ et $h \in E $ tel que le segment $ [a,a+h] \subset \U$. Alors on a
	%\[
		%f(a+h) = f(a) + d_a f(h) + d_a^2 f (h,h) \cdots + d^k_af(\underbrace{h,\cdots,h}_{k \text{ fois} }) + \underset{h\to 0}{o} (\norm{h}^k)
	%\]
	%où pour tout $\ell =1 ,\cdots,k$  on a noté  
	%\[
	%d^\ell_a f(\underbrace{h,\cdots,h}_{\ell \text{ fois} }) =  \sum_{j_1,\ldots,j_\ell =1}^n \frac{\p^\ell f}{ \p x_{j_1} \cdots \p x_{j_\ell} } (a) h_{j_1}\cdots h_{j_\ell}  \]
%\end{theorem}
\begin{theorem}
	[(Formule de Taylor-Young)]
	Soit $f:\U \to F$ une fonction de classe $\Cc^2$ et $ a \in \U$. Alors il existe une fonction $\omega: E \to F$ définie au voisinage de 0 telle que pour tout $h\in E$ assez petit en norme,
	\[
		f(a+h) = f(a) + d_a f(h) + \frac{1}{2}d_a^2 f (h,h) + \snorm{h }^2 \omega(h) \text{ avec } \snorm{\omega(h) }'\xrightarrow[\snorm{h} \to 0]{} 0
	\]
	où $
	d^2_a f(h,h) =  \sum_{i,j=1}^n \frac{\p^2 f}{ \p x_{i} \p x_{j} } (a) h_{i}h_{j}$.
\end{theorem}

\begin{proof}
%\'Ebauche. %Dans le cas $E = \R^2$ et $F=\R$.
	\pl{\rep{6cm}}
\end{proof}
\begin{exemple}
	Développement  limité en $(0,0)$  et à l'ordre 2 de $(x,y) \mapsto (ye^x,\cos(x+y))$. \plprof{On pose $h = (h_1,h_2) \in \R^2:$
	\begin{enumerate}
		\item $f_1(x,y) = y e^x$ et $f_1(h_1,h_2) = h_2 + h_1h_2 + \snorm{h}^2\omega_1(h)$ où $\snorm{h }^2 \omega_1(h) = o(\norm{h}^2)$.
		\item $f_2(x,y) = \cos(x+y)$ et $f_2(h_1,h_2) = 1 - \frac 1 2  (h_1 + h_2)^2 + \snorm{h}^2\omega_2(h)$ où $\snorm{h }^2 \omega_2(h) = o(\norm{h}^2)$.
		% Dans Maxima: taylor(cos(x+y),[x,y],0,8);
	\end{enumerate}
	Ainsi, on a $f(h_1,h_2) = (0,1) + (h_2,0) + \frac 1 2 (2h_1h_2,(h_1 + h_2)^2 )  + \snorm{h}^2 (\omega_1(h)  ,\omega_2(h)  ) $ }
	\pl{\rep{9cm}}
\end{exemple}

%\begin{exemple}
	%Calcul du développement  limité en $(0,0)$  et à l'ordre 3 de la fonction $(x,y) \mapsto (ye^x,\cos(x+y))$.
	%\begin{enumerate}
		%\item $f_1(x,y) = y e^x$ et $f_1(h_1,h_2) = h_2 + h_1h_2 + \frac 1 2 h_1^2 h_2 + o(\norm{h}^3)$
%%\begin{center}
	%%\tikzexternalenable

	%%\begin{tabular}	{ccc}
		%%\begin{tikzpicture}[scale=.95]
			%%\begin{axis}[z post scale=1.5,zlabel style={rotate=-90},zlabel=$z$,xlabel = $x$,ylabel=$y$,width=.3\textwidth, ,domain=-2:2,zmax=15,zmin=-15,view/h=70] 
				%%\addplot3[surf,opacity=.9,samples=20] gnuplot { y*exp(x)  };
				%%\addplot3[colormap/greenyellow,surf,opacity=.5,samples=20] gnuplot { y};
			%%\end{axis} 
		%%\end{tikzpicture}&
		%%\begin{tikzpicture}[scale=.95]
			%%\begin{axis}[z post scale=1.5,zlabel style={rotate=-90},zlabel=$z$,xlabel = $x$,ylabel=$y$,width=.3\textwidth, ,domain=-2:2,zmax=15,zmin=-15,view/h=70] 
				%%\addplot3[surf,opacity=.9,samples=20] gnuplot { y*exp(x)  };
				%%\addplot3[colormap/greenyellow,surf,opacity=.5,samples=20] gnuplot { y +x*y};
			%%\end{axis} 
		%%\end{tikzpicture} &
		%%\begin{tikzpicture}[scale=.95]
			%%\begin{axis}[z post scale=1.5,zlabel style={rotate=-90},zlabel=$z$,xlabel = $x$,ylabel=$y$,width=.3\textwidth, ,domain=-2:2,zmax=15,zmin=-15,view/h=70] 
				%%\addplot3[surf,opacity=.9,samples=20] gnuplot { y*exp(x)  };
				%%\addplot3[colormap/greenyellow,surf,opacity=.5,samples=20] gnuplot { y+x*y+.5*x**2 *y};
			%%\end{axis} 
																										%%\end{tikzpicture}
%%\\ Ordre 1 & Ordre 2 & Ordre 3																									\end{tabular}
	%%\tikzexternaldisable
%%\end{center}

		%\item $f_2(x,y) = \cos(x+y)$ et $f_2(h_1,h_2) = 1 - \frac 1 2  (h_1 + h_2)^2 + \frac{1}{4!} (h_1 + h_2)^4 + \frac{1}{6!} (h_1 + h_2)^6 + \frac{1}{8!} (h_1 + h_2)^8 + o(\norm{h}^8)$
		%% Dans Maxima: taylor(cos(x+y),[x,y],0,8);
%\begin{center}
	%\tikzexternalenable
	%\begin{tabular}{c}
		%\begin{tabular}	{ccc}
			%\tikzsetnextfilename{cours-dlsin1}	
			%\begin{tikzpicture}[scale=.95]

				%\begin{axis}[z post scale=1.5,zlabel style={rotate=-90},zlabel=$z$,xlabel = $x$,ylabel=$y$,width=.3\textwidth, ,domain=-2:2,zmax=4,zmin=-7,view/h=70,point meta min=-3, point meta max=3] 
					%\addplot3[colormap/greenyellow,surf,opacity=.5,samples=20] gnuplot { 1-.5 *(x+y)**2};
					%\addplot3[surf,opacity=.9,samples=20] gnuplot { cos(x+y)  };
				%\end{axis} 
			%\end{tikzpicture}&
			%\tikzsetnextfilename{cours-dlsin2}	
			%\begin{tikzpicture}[scale=.95]
				%\begin{axis}[z post scale=1.5,zlabel style={rotate=-90},zlabel=$z$,xlabel = $x$,ylabel=$y$,width=.3\textwidth, ,domain=-2:2,zmax=4,zmin=-7,view/h=70,point meta min=-1, point meta max=1] 
					%\addplot3[surf,opacity=.9,samples=20] gnuplot { cos(x+y) };
					%\addplot3[colormap/greenyellow,surf,opacity=.5,samples=20] gnuplot {  1-.5 *(x+y)**2  + (x+y)**4 /24};
				%\end{axis} 
			%\end{tikzpicture} &
			%\tikzsetnextfilename{cours-dlsin3}	
			%\begin{tikzpicture}[scale=.95]
				%\begin{axis}[z post scale=1.5,zlabel style={rotate=-90},zlabel=$z$,xlabel = $x$,ylabel=$y$,width=.3\textwidth, ,domain=-2:2,zmax=4,zmin=-7,view/h=70] 
					%\addplot3[surf,opacity=.9,samples=20] gnuplot { cos(x+y)  };
					%\addplot3[colormap/greenyellow,surf,opacity=.5,samples=20] gnuplot { 1-.5 *(x+y)**2  + (x+y)**4 /24 - (x+y)**6 /720  +(x+y)**8 /40320};
				%\end{axis} 
			%\end{tikzpicture}
			%\\ Ordre 2 & Ordre 4 & Ordre 8																				\end{tabular} \\
		%Illustration de l'approximation de  fonction $(x,y) \mapsto \cos(x+y) $ par des polynômes.
	%\end{tabular}
	%\tikzexternaldisable
%\end{center}
	%\end{enumerate}

%\end{exemple}

%\subsection{Matrice Hessienne}

\sld{\vfill\pagebreak[5]}%%%%%%%%%%%%%%%
\emph{Dans le cas $F=\R$}: Pour une fonction $f: \R^n \to \R$ de classe $\Cc^2$ et $a,h\in\R^n$, la formule de Taylor donne:
	\[
		f(a+h) = f(a) + \sum_{j=1}^n \frac{\p f}{\p x_j}(a) h_j + \frac{1}{2} \sum_{i,j=1}^n \frac{\p^2 f}{\p x_i \p x_j}(a) h_i h_j + \underset{h\to 0}{o} (\norm{h}^2).
	\]

L'application $Q_a f: h \mapsto d_a^2 f(h,h) =  \frac{1}{2} \sum_{i,j=1}^n \frac{\p^2 f}{\p x_i \p x_j}(a) h_i h_j  $ est une forme quadratique sur $\R^n$. On peut donc l'écrire sous forme matricielle:
	\begin{definition}
		Soit $f$ une fonction de classe $\Cc^2$ de $\U \subset \R^n$ dans $\R$ et $a\in\U$. On appelle \emph{matrice Hessienne} de $f$ en $a$ la matrice 
		\[
			\hess_f (a) =\left(\frac{\p^2 f}{\p x_i \p x_j}(a)  \right)_{i,j=1}^n = \begin{pmatrix}
			\frac{\p^2 f}{\p x_1^2}(a) & \frac{\p^2 f}{\p x_1 \p x_2}(a) & \cdots & \frac{\p^2 f}{\p x_1 \p x_n}(a)	\\
			\frac{\p^2 f}{\p x_2 \p x_1}(a)&   \frac{\p^2 f}{\p x_2^2}(a) & \cdots &  \frac{\p^2 f}{\p x_2 \p x_n}(a)	\\
			\vdots & \vdots& & \vdots \\
			\frac{\p^2 f}{\p x_n \p x_1}(a)& \frac{\p^2 f}{\p x_n \p x_2}(a)& \cdots & \frac{\p^2 f}{\p x_n^2}(a) 	\\

			\end{pmatrix}
		\]
	\end{definition}

En notations matricielles on a alors:
\[
f(a+h) = f(a) + [J_f (a)] h + \frac{1}{2} h^t \hess_f (a) h + o(\snorm{h}^2)
\]


	\begin{exemple}
	Soit $f_1(x,y) = y e^x$  \plprof{alors $J_f (0,0) = \begin{pmatrix}
		0	\\ 1
	\end{pmatrix}$ et $ \hess_f (0) = \begin{pmatrix}
		0 & 1 \\ 1& 0
	\end{pmatrix}$. On a:
	\[
		f_1(h_1,h_2) = 0 + (h_1,h_2)\begin{pmatrix}
			0\\1
		\end{pmatrix} + \frac{1}{2}(h_1,h_2) \begin{pmatrix}
		0 & 1 \\ 1& 0
	\end{pmatrix}\begin{pmatrix}
			h_1\\h_2
		\end{pmatrix} + o(\norm{h}^2) = h_2 + h_1h_2 + o(\norm{h}^2)
	\]}
	\pl{\rep{4cm}}
	\end{exemple}


\sld{\vfill\pagebreak[5]}%%%%%%%%%%%%%%%
\section{Étude des extrema locaux}

\subsection{Définitions}

\begin{definition}
	Soit $f:E\to \R$ une fonction définie sur un domaine $\D \subset E$ et $a\in \D$. La fonction $f$ admet en $a$ 
	\begin{enumerate}
		\item un \emph{maximum} (resp. \emph{minimum}) \emph{global} si pour tout $x \in \D $ on a $f(x) \leq f(a)$ (resp. $f(x) \geq f(a)$).
		\item un \emph{maximum} (resp. \emph{minimum}) \emph{strict} si pour tout $x \in \D\setminus \left\{ a \right\} $ on a $f(x) < f(a)$ (resp. $f(x) > f(a)$).
		\item un \emph{maximum} (resp. \emph{minimum}) \emph{local} si il existe un voisinage $\V$ de $a$ tel que pour tout $x \in \D\cap \V $ on a $f(x) \leq f(a)$ (resp. $f(x) \geq f(a)$).
	\end{enumerate}
	On dit que $f$ admet en $a \in \D$ un \emph{extremum global} (resp. \emph{strict} ou \emph{local}) si $f$ admet un maximum ou un minimum global (resp. strict, local).
\end{definition}
%\pl{
\begin{center}
%%%%%%%%%%%%%%%%%%%%% ICI
\begin{tabular}{c}
\begin{tikzpicture}[scale=1]
			\begin{axis}[z post scale=1.5,zlabel=$z$,xlabel = $x$,ylabel=$y$,width=.3\textwidth,domain=-2:2] 
			\addplot3[surf,opacity=.6,samples=40] gnuplot {((x-1)**2-2*y**2)*exp(-2*x**2-y**2)};
			\end{axis} 
		\end{tikzpicture}%
		\hspace{10pt}
		\begin{tikzpicture}[scale=1]
			\begin{axis}[ylabel style={rotate=-90},xlabel = $x$,ylabel=$y$,width=.3\textwidth, ,domain=-2:2,view={0}{90},] 
				\addplot3[samples=60,contour gnuplot={levels={0, 1.2,0.8,1,1.4,0.2,0.4,0.6,-0.01,-0.1,-0.2,-0.3,-0.4},labels=false},thick] { ( (x -1 )^2 - 2*y^2 )* exp(-2*x^2 - y^2)  };
			\end{axis} 
		\end{tikzpicture}
		\\
		Graphe et lignes de niveau de la fonction $(x,y) \mapsto ( (x -1 )^2 - 2y^2 ) e^{-2x^2 - y^2}$. 
		\end{tabular}
\end{center}
%}
%\sld{
%\begin{center}
%	\begin{tabular}{c}
%		\tikzexternalenable
%		\tikzsetnextfilename{cours-extremal}
%		\begin{tikzpicture}[scale=1.5]
%			\begin{axis}[z post scale=1.5,zlabel style={rotate=-90},zlabel=$z$,xlabel = $x$,ylabel=$y$,width=.3\textwidth, ,domain=-2:2,] 
%				\addplot3[surf,opacity=.6,samples=40] gnuplot {((x-1)**2-2*y**2)*exp(-2*x**2-y**2)};
%			\end{axis} 
%		\end{tikzpicture}%
%	\tikzsetnextfilename{cours-extremallc}%
%		\begin{tikzpicture}[scale=1.5]
%			\begin{axis}[ylabel style={rotate=-90},xlabel = $x$,ylabel=$y$,width=.3\textwidth, ,domain=-2:2,view={0}{90},] 
%				\addplot3[samples=60,contour gnuplot={levels={0, 1.2,0.8,1,1.4,0.2,0.4,0.6,-0.01,-0.1,-0.2,-0.3,-0.4},labels=false},thick] { ( (x -1 )^2 - 2*y^2 )* exp(-2*x^2 - y^2)  };
%			\end{axis} 
%		\end{tikzpicture}
%		\tikzexternaldisable \\
%		Graphe et lignes de niveau de la fonction $(x,y) \mapsto ( (x -1 )^2 - 2y^2 ) e^{-2x^2 - y^2}$. \end{tabular}
%\end{center}
%}

\sld{\vfill\pagebreak[5]}%%%%%%%%%%%%%%%
\subsection{Condition nécessaire d'ordre 1}

\begin{definition}
	On suppose que $\dim E = n$. On dit que $f:\U \to F$ de classe $\Cc^1$ sur $\U$ admet un point critique en $a\in\U$ si $\frac{\p f}{\p x_i}(a) = 0 $ pour tout $i=1,\cdots,n$.
\end{definition}

\begin{theorem}
	Soit  $f:\U \to F$ de classe $\Cc^1$ sur $\U$. Si $f$ admet un extremum local en $a\in \U$ alors $a$ est un point critique de $f$.
\end{theorem}
\begin{proof}
	\pl{\rep{4cm}}	
\end{proof}

\begin{exemple}
	La condition n'est pas suffisante: prendre $f(x,y) = xy$ en $(0,0)$.
	\pl{\rep{6cm}}	
\end{exemple}

\sld{\vfill\pagebreak[5]}%%%%%%%%%%%%%%%
\subsection{Condition suffisante d'ordre 2}

On suppose maintenant que $f:E \to \R$ est de classe $\Cc^2$ et que $a$ est un point critique de $f$. D'après la formule de Taylor-Young on a:
\[
	f(a+h) = f(a) + \frac{1}{2} Q_a f(h) + o(\snorm{h}^2)
\]
Le signe de la forme quadratique $Q_a f:h \mapsto d_a^2f(h,h) = h^t \hess_f (a) h$  permet dans certains cas de caractériser les extrema:

\begin{proposition}
	Soit $a \in \D$ un point critique de $f:E \to \R$. Si la forme quadratique $Q_a f $ est~:
	\begin{enumerate}
		\item définie et positive alors $f$ admet un minimum local strict en $a$. 
		\item définie et négative alors $f$ admet un maximum local strict en $a$. 
	\end{enumerate}
\end{proposition}

\begin{proof}
	\pl{\rep{7cm}}	
\end{proof}

\sld{\vfill\pagebreak[5]}%%%%%%%%%%%%%%%
\begin{remark}
	Ne pas oublier la condition ``définie'': soit $f_1:(x,y) \mapsto \frac 1 2 x^2 - y^4$ et $f_2:(x,y) \mapsto \frac 1 2 x^2 + y^4$. On a $J_{f_1} (a) = J_{f_2} (a) = (0,0)$ et $\hess_{f_1}(0,0)=\hess_{f_2}(0,0)= \begin{pmatrix}
		1 & 0 \\ 0 & 0
	\end{pmatrix}$. Ainsi, l'origine est un point critique de $f_1$ et $f_2$, les formes quadratiques $Q_a f_1$ et $Q_a f_2$ sont positives non-définies. 

	\begin{enumerate}
		\item  l'origine n'est pas un extremum de $f_1$: 
			\begin{center}
				\tikzexternalenable%\tikzsetnextfilename{cours-pcdegener}
					\begin{tikzpicture}[scale=.95]
						\begin{axis}[z post scale=1.5,zlabel style={rotate=-90},zlabel=$z$,xlabel = $x$,ylabel=$y$,width=.3\textwidth, ,domain=-2:2,] 
							\addplot3[surf,opacity=.6,samples=40] gnuplot { x**2 - y**4  };
						\end{axis} 
					\end{tikzpicture}				
				\end{center}
\sld{\vfill\pagebreak[5]}%%%%%%%%%%%%%%%
		\item  l'origine est un minimum (global) de $f_2$: 
			\begin{center}
				\tikzsetnextfilename{cours-pcdegener1}
					\begin{tikzpicture}[scale=.95]
						\begin{axis}[z post scale=1.5,zlabel style={rotate=-90},zlabel=$z$,xlabel = $x$,ylabel=$y$,width=.3\textwidth, ,domain=-2:2,] 
							\addplot3[surf,opacity=.6,samples=40] gnuplot { x**2 + y**4  };
						\end{axis} 
					\end{tikzpicture}	%\tikzexternaldisable  
			  \end{center}
	\end{enumerate}
\end{remark}


\sld{\vfill\pagebreak[5]}%%%%%%%%%%%%%%%
\subsection{Fonctions de $\R^2$ dans $\R$}

{\bf\sffamily Notation de Monge:} Si $f:\R^2 \to \R$ est de classe $\Cc^2$ sur $\U$ et $(a,b) \in \U \subset \R^2$. On note $p=\frac{\p f}{\p x}(a,b)$, $q=\frac{\p f}{\p y}(a,b)$, $r=\frac{\p^2 f}{\p x^2}(a,b)$, $s=\frac{\p^2 f}{\p x\p y}(a,b)$, $t=\frac{\p^2 f}{\p y^2}(a,b)$. La formule de Taylor-Young devient:
\[
f(a+h,b+k) = f(a,b) + ph +qk +\frac 1 2 \left( rh^2 + 2shk+tk^2 \right) + o(h^2 + k^2)
\]

\begin{defprop}
	Soit $f:\R^2 \to \R$ de classe $\Cc^2$ sur $\U$ et $(a,b) \in \U$ un point critique de $f$:
	\begin{enumerate}
		\item si $rt - s^2 \neq 0 $ on dit que $(a,b)$ est un \emph{point critique non dégénéré}. Dans ce cas:
			\begin{enumerate}
				\item si $ rt -s^2 > 0$ et $r>0$:  $(a,b)$ est un minimum  local de $f$.
				\item si $ rt -s^2 > 0$ et $r<0$:  $(a,b)$ est un maximum local de $f$.
				\item si $ rt -s^2 < 0$: $(a,b)$ est un \emph{point selle} de $f$.
			\end{enumerate}
		\item si $ rt -s^2 = 0 $ on dit que $(a,b)$ est un \emph{point critique dégénéré}.
	\end{enumerate}
\end{defprop}

\begin{proof}
	\pl{\rep{6cm}}
\end{proof}

\sld{\vfill\pagebreak[5]}%%%%%%%%%%%%%%%
\begin{center}
	\begin{tabular}{ccc}%\tikzexternalenable%\tikzsetnextfilename{cours-ptcrit1}
		\begin{tikzpicture}[scale=.95]
			\begin{axis}[z post scale=1.5,zlabel style={rotate=-90},zlabel=$z$,xlabel = $x$,ylabel=$y$,width=.3\textwidth, ,domain=-2:2,] 
				\addplot3[surf,opacity=.6,samples=40] gnuplot { x**2 + y**2   };
			\end{axis} 
		\end{tikzpicture}
		&%\tikzsetnextfilename{cours-ptcrit2}
		\begin{tikzpicture}[scale=.95]
			\begin{axis}[z post scale=1.5,zlabel style={rotate=-90},zlabel=$z$,xlabel = $x$,ylabel=$y$,width=.3\textwidth, ,domain=-2:2,] 
				\addplot3[surf,opacity=.6,samples=40] gnuplot {-x**2 - y**2   };
			\end{axis} 
		\end{tikzpicture}&%\tikzsetnextfilename{cours-ptcrit3}
		\begin{tikzpicture}[scale=.95]
			\begin{axis}[z post scale=1.5,zlabel style={rotate=-90},zlabel=$z$,xlabel = $x$,ylabel=$y$,width=.3\textwidth, ,domain=-2:2,] 
				\addplot3[surf,opacity=.6,samples=40] gnuplot { x**2 - y**2   };
			\end{axis} 
		\end{tikzpicture} \\
		%\tikzsetnextfilename{cours-ptcritlc1}
		\begin{tikzpicture}[scale=.95]
			\begin{axis}[ylabel style={rotate=-90},xlabel = $x$,ylabel=$y$,width=.3\textwidth, ,domain=-2:2,view={0}{90},] 
				\addplot3[samples=60,contour gnuplot={levels={0, .5,1, 1.5, 2,2.5,3,3.5,4,4.5, 5,5.5,6 },labels=false,},thick] gnuplot { x**2 + y**2  };
			\end{axis} 
		\end{tikzpicture} &%\tikzsetnextfilename{cours-ptcritlc2}
		\begin{tikzpicture}[scale=.95]
			\begin{axis}[ylabel style={rotate=-90},xlabel = $x$,ylabel=$y$,width=.3\textwidth, ,domain=-2:2,view={0}{90},] 
				\addplot3[samples=60,contour gnuplot={levels={0, -.5,-1, -1.5, -2,-2.5,-3,-3.5,-4,-4.5,-5,-5.5,-6 },labels=false
				},thick] gnuplot { -x**2 - y**2  };
			\end{axis} 
		\end{tikzpicture} & %\tikzsetnextfilename{cours-ptcritlc3}
		\begin{tikzpicture}[scale=.95]
			\begin{axis}[ylabel style={rotate=-90},xlabel = $x$,ylabel=$y$,width=.3\textwidth, ,domain=-2:2,view={0}{90},] 
				\addplot3[samples=60,contour gnuplot={levels={0, 1,2,-1,-2},labels=false,
				},thick] gnuplot { x**2 - y**2  };
			\end{axis} 
		\end{tikzpicture} 
		%\tikzexternaldisable
		 \\
Minimum & Maximum & Point selle
\end{tabular}
\end{center}

\begin{exemple}
	$f(x,y) = (x-y)^2 + x^3 + y^3$.
	\pl{\rep{8cm}}
\end{exemple}

\begin{exemple}
	$f(x,y) = (x-y)^2 + x^4 + y^4$.
	\pl{\rep{6cm}}
\end{exemple}



% Nigliot cours

%\section{Opérateurs de l'analyse vectorielles}
