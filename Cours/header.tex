\makeatletter
%--------------------------------------------------------------------------------

%\usepackage{dsfont}
\usepackage[utf8]{inputenc}
\usepackage{lmodern}
\usepackage{dsfont}
\usepackage[T1]{fontenc}
\usepackage[english,french]{babel}

\usepackage{qrcode}

\usepackage{eurosym}
\usepackage{diagbox}
\usepackage{enumitem}\setlist{nosep}\def\labelitemi{--}

\usepackage{graphicx}
\usepackage{float}

\usepackage{array}
\newcolumntype{L}[1]{>{\raggedright\let\newline\\\arraybackslash\hspace{0pt}}m{#1}}
\newcolumntype{C}[1]{>{\centering\let\newline\\\arraybackslash\hspace{0pt}}m{#1}}
\newcolumntype{R}[1]{>{\raggedleft\let\newline\\\arraybackslash\hspace{0pt}}m{#1}}
%----------
%  Version
%-----------

\usepackage{fancyhdr} 

%--------
%  Tkz  
%--------

\usepackage[babel=true,kerning=true]{microtype}
\usepackage[caption=false]{subfig}
\usepackage{xcolor,colortbl}
\usepackage{diagbox,calc,soul,graphicx}

\usepackage{tikz}
\usetikzlibrary{3d,calc,fadings,decorations.pathreplacing,matrix,arrows,decorations.text}
\usetikzlibrary{patterns}
\usetikzlibrary{positioning}
\usetikzlibrary{babel}
\usetikzlibrary{shapes}
\usetikzlibrary{shadings}
\usepackage{tikz-3dplot}
\usepackage{pgfplots}
\usepgfplotslibrary{fillbetween}
\pgfplotsset{compat=newest}
\usepgfplotslibrary{external} 
\tikzexternalize[prefix=output_latex/]
\usepgfplotslibrary{fillbetween}
%\graphicspath{{./output-latex/}}


%\usepackage{etex}
%\reserveinserts{28}
%\usepackage{pstricks}
%\usepackage{pst-solides3d}

\usepackage{gnuplot-lua-tikz}

\newcommand\chideux[1]{#1<=0||(#1!=int(#1))?1/0:x<=0?0.0:exp((0.5*#1-1.0)*log(x)-0.5*x-lgamma(0.5*#1)-#1*0.5*log(2))}
\newcommand\gauss[2]{1/(#2*sqrt(2*pi))*exp(-((x-#1)^2)/(2*#2^2))} 
\newcommand\student[1]{gamma(.5*(#1+1))/(sqrt(#1*pi)*gamma(.5*#1))*((1+x^2/#1)^(-.5*(#1+1)))}

% display dices
\usepackage{xparse}\usetikzlibrary{shapes}
\NewDocumentCommand{\drawdie}{O{}m}{%
\begin{tikzpicture}[x=1em,y=1em,radius=0.1,#1,baseline=0.575ex]
		\draw[rounded corners=0.5] (0,0) rectangle (1,1);
	  \ifodd#2
      \fill[] (0.5,0.5) circle;
        \fi
  \ifnum#2>1
      \fill[] (0.2,0.2) circle;
          \fill[] (0.8,0.8) circle;
     \ifnum#2>3
          \fill[] (0.2,0.8) circle;
       \fill[] (0.8,0.2) circle;
           \ifnum#2>5
         \fill[] (0.8,0.5) circle;
       \fill[] (0.2,0.5) circle;
            \ifnum#2>7
           \fill[] (0.5,0.8) circle;
          \fill[] (0.5,0.2) circle;
        \fi
    \fi
      \fi
      \fi
\end{tikzpicture}%
} 
%------------------
% Math environment
%------------------

\usepackage{latexsym}
\usepackage{amsmath}
\usepackage{amsbsy}
\usepackage{amsfonts}
\usepackage{amssymb}
\usepackage{nicefrac}
\usepackage{amscd}
\usepackage{amsthm}
\usepackage{mathtools}

\newtheoremstyle{definitionSs}{\topsep}{\topsep}%
     {}%         Body font
     {}%         Indent amount (empty = no indent, \parindent = para indent)
     {\sffamily\bfseries}% Thm head font
     {.}%        Punctuation after thm head
     { }%     Space after thm head (\newline = linebreak)
     {\thmname{#1}\thmnumber{~#2}\thmnote{~#3}}%         Thm head spec

\newtheoremstyle{plainSs}{\topsep}{\topsep}%
     {\itshape}%         Body font
     {}%         Indent amount (empty = no indent, \parindent = para indent)
     {\sffamily\bfseries}% Thm head font
     {.}%        Punctuation after thm head
     { }%     Space after thm head (\newline = linebreak)
     {\thmname{#1}\thmnumber{~#2}\thmnote{~#3}}%         Thm head spec

\theoremstyle{definitionSs}
\newtheorem{remark}{Remarque}
%\newtheorem{definition}{Définition}[section]
%\newtheorem{app}{Application}[section]
\newtheorem{exemple}{Exemple}[section]
%\newtheorem{exo}{Exercice}[section]
%\newtheorem{proposition}{Proposition}[section]
%\newtheorem{lemme}{Lemme}[section]
%\newtheorem{theorem}{Théorème}[section]
%\newtheorem{defprop}{D\'efinition-Proposition}[section]


\usepackage[framemethod=tikz]{mdframed}

\newmdtheoremenv[
hidealllines=true,
leftline=true,
skipabove=0pt,
innertopmargin=-5pt,
innerbottommargin=2pt,
linewidth=4pt,
linecolor=gray!90,
innerrightmargin=0pt,
]{definition}{Définition}[section]

\newmdtheoremenv[
hidealllines=true,
leftline=true,
skipabove=0pt,
innertopmargin=-5pt,
innerbottommargin=2pt,
linewidth=4pt,
linecolor=gray!40,
innerrightmargin=0pt,
]{proposition}{Proposition}[section]

\newmdtheoremenv[
hidealllines=true,
leftline=true,
skipabove=0pt,
innertopmargin=-5pt,
innerbottommargin=2pt,
linewidth=4pt,
linecolor=gray!90,
innerrightmargin=0pt,
]{defprop}{Définition - Proposition}[section]

\newmdtheoremenv[
hidealllines=true,
leftline=true,
skipabove=0pt,
innertopmargin=-5pt,
innerbottommargin=2pt,
linewidth=4pt,
linecolor=gray!100,
innerrightmargin=0pt,
]{theorem}{Théorème}[section]

%---------------
% Mise en page
%--------------

\setlength{\parindent}{0pt}

\renewcommand*{\descriptionlabel}[1]{\hspace\labelsep{\itshape #1}}
\renewcommand{\emph}[1]{{\sffamily\bfseries #1}}

%\usepackage{titlesec, blindtext, color}
%\definecolor{gray75}{gray}{0.75}
%\newcommand{\hsp}{\hspace{20pt}}
%\titleformat{\chapter}[hang]{\sffamily\Huge\bfseries}{\thechapter\hsp\textcolor{gray75}{|}\hsp}{0pt}{\Huge\bfseries}
\usepackage{sectsty}
\usepackage{xcolor}
\definecolor{astral}{RGB}{46,116,181}
\allsectionsfont{\color{astral}\normalfont\sffamily\bfseries}

\usepackage[subfigure]{tocloft}
\renewcommand{\cftchapfont}{\sffamily\bfseries\color{astral}}
\renewcommand{\cfttoctitlefont}{\sffamily\bfseries\Huge\color{astral}}
\usepackage{hyperref}

% Poly pour étudiants
%\newif\ifteacherVersion
%\teacherVersiontrue % comment out to hide answers
%\ifteacherversion
%Answer
%\fi
\providecommand{\blanc}[1]{\vspace*{#1}}

%\usepackage{tocloft}
%\renewcommand{\cftchapfont}{\sffamily\bfseries}
%----------------
% Some commands
%----------------

\DeclareMathOperator{\aire}{Aire}
\providecommand{\gf}{g\circ f}
\providecommand{\R}{\ensuremath \mathbb{R}}
\providecommand{\reg}[1]{\mathcal{C}^{#1}}
\providecommand{\1}{\mathbb{1}}
\providecommand{\N}{\mathbb{N}}
\providecommand{\Q}{\mathbb{Q}}
\renewcommand{\L}{\mathcal{L}}
\providecommand{\D}{\mathcal{D}}
\providecommand{\Cc}{\mathcal{C}}
\providecommand{\F}{\mathcal{F}}
\providecommand{\Ee}{\mathcal{E}}
\providecommand{\G}{\mathcal{G}}
\providecommand{\Z}{\mathbb{Z}}
\providecommand{\x}{\ensuremath\boldsymbol{x}}
\providecommand{\y}{\ensuremath\boldsymbol{y}}
\providecommand{\1}{\mathbb{1}}
\providecommand{\p}{\partial}
\providecommand{\Pp}{\mathcal{P}}
\providecommand{\P}{\mathbb{P}}
\providecommand{\E}{\mathbb{E}}
\providecommand{\U}{\mathcal{U}}
\providecommand{\V}{\mathcal{V}}
\providecommand{\ie}{\textit{i.e. }}
\renewcommand{\P}{\mathbb{P}}
\renewcommand{\S}{\mathcal{S}}
\providecommand{\E}{\mathbb{E}}
\providecommand{\one}{\mathds{1}}
\DeclareMathOperator{\card}{Card}
\DeclareMathOperator{\vol}{Vol}
\DeclareMathOperator{\var}{Var}
\DeclareMathOperator{\vect}{Vect}
\DeclareMathOperator{\med}{median}
\DeclareMathOperator{\hess}{Hess}
\DeclareMathOperator{\jac}{Jac}
\DeclareMathOperator{\cov}{cov}

\providecommand{\ncd}{\norm{\cdot}}
\providecommand{\norm}[1]{\left\lVert#1\right\rVert}
\providecommand{\bnorm}[1]{\bigg\lVert#1\bigg\rVert}
\providecommand{\snorm}[1]{\lVert#1\rVert}

\newcommand{\tnorm}[1]{{\left\vert\kern-0.25ex\left\vert\kern-0.25ex\left\vert #1 
    \right\vert\kern-0.25ex\right\vert\kern-0.25ex\right\vert}}

\providecommand{\abs}[1]{\left\lvert#1\right\rvert}
\providecommand{\sabs}[1]{\lvert#1\rvert}
\providecommand{\babs}[1]{\bigg\lvert#1\bigg\rvert}

\providecommand{\prscd}{\prs{\cdot,\cdot}}
\providecommand{\prs}[1]{\left\langle #1\right\rangle}
\providecommand{\sprs}[1]{\langle #1\rangle}
\providecommand{\bprs}[1]{\bigg\langle #1\bigg\rangle}

\providecommand{\rev}{$\R$ espace vectoriel}

\providecommand{\dpar}[2]{\frac{\partial #1}{\partial #2}}

% Multiversioning 


\usepackage{ifthen}

\newcommand{\pl}[1]{%
	\ifthenelse{\equal{\version}{poly}}{#1}{}%
	\ifthenelse{\equal{\version}{polyProf}}{#1}{}%
}
\newcommand{\plprof}[1]{\ifthenelse{\equal{\version}{polyProf}}{#1}{}}
\newcommand{\sld}[1]{\ifthenelse{\equal{\version}{slide}}{#1}{}}


\mdfdefinestyle{response}{
	leftmargin=.01\textwidth,
	rightmargin=.01\textwidth,
	linewidth=1pt
	hidealllines=false,
	leftline=true,
	rightline=true,topline=true,bottomline=true,
        skipabove=\baselineskip,%0pt,
	%innertopmargin=-5pt,
	%innerbottommargin=2pt,
	linecolor=black,
	innerrightmargin=0pt,
	}
%\providecommand{\rep}[1]{$ $ \newline \begin{mdframed}[style=response] \vspace*{#1} \end{mdframed}}

% generate breakable white space allowing students to write notes.
\newcommand*{\DivideLengths}[2]{%
  \strip@pt\dimexpr\number\numexpr\number\dimexpr#1\relax*65536/\number\dimexpr#2\relax\relax sp\relax
}

\providecommand{\rep}[1]{$ $

    \begin{mdframed}[style=response]  
	\vspace*{\DivideLengths{#1}{3cm}cm}
	\pagebreak[1]	
	\vspace*{\DivideLengths{#1}{3cm}cm}
	\pagebreak[1]		
	\vspace*{\DivideLengths{#1}{3cm}cm}
    \end{mdframed}%
}
\providecommand{\repcom}[1]{\begin{mdframed}[style=response] #1 \end{mdframed}}

\def\redspace{\sld{\setlength{\belowdisplayskip}{0pt} \setlength{\belowdisplayshortskip}{0pt}\setlength{\abovedisplayskip}{0pt}\setlength{\abovedisplayshortskip}{0pt}}}
%------------------------------------------------------------------------------
%\DeclareUnicodeCharacter{00A0}{~}

\pdfstringdefDisableCommands{%
    %\renewcommand*{\bm}[1]{#1}%
    \renewcommand*{\R}{R}%
    % any other necessary redefinitions 
}
\makeatother


