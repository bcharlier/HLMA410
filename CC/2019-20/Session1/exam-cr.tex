\documentclass[12pt]{article}
\usepackage{amsmath}
\usepackage{moodle}


\begin{document}

\begin{quiz}[points=3]
  {Regle de la chaine (3pt)}

  \begin{numerical}{Derivee directionnelle}
    On pose $\phi: \mathbb R^2 \to \mathbb R^2$ d\'{e}finie par $\phi(x,y) = (x+2y, y)$. Sachant que la fonction $f:\mathbb R^2 \to \mathbb R$ est diff\'{e}rentiable en $p=(3,1)$ et que l'on a $\frac{\partial f}{\partial x}(p) = 2$ et $\frac{\partial f}{\partial y}(p) = 1$. Calculer $D_v (f \circ \phi)(1,1)$ o\`{u} $v=(1,-1)$.
  \item -3
  \end{numerical}
% On a \nabla f \circ \phi (1,1) = (2*1 + 1 * 0, 2*2+1*1) = (2,5)
% on a  D_v f(p) = d_p f(v) =  \langle \nabla f, v \rangle = 2 - 5 = -3

  \begin{numerical}{Derivee directionnelle}
    On pose $\phi: \mathbb R^2 \to \mathbb R^2$ d\'{e}finie par $\phi(x,y) = (2x, 2x+y)$. Sachant que la fonction $f:\mathbb R^2 \to \mathbb R$ est diff\'{e}rentiable en $p=(2,3)$ et que l'on a $\frac{\partial f}{\partial x}(p) = 1.5$ et  $\frac{\partial f}{\partial y}(p) = 1$. Calculer $D_v (f \circ \phi)(1,1)$ o\`{u} $v=(1,-1)$.
  \item 4
  \end{numerical}
% On a \nabla f \circ \phi (1,1) = (2*1.5 + 2 * 1, 0*1.5 + 1*1) = (5,1)
% on a  D_v f(p) = d_p f(v) =  \langle \nabla f, v \rangle = 5 - 1 = 4

  \begin{numerical}{Derivee directionnelle}
    On pose $\phi: \mathbb R^2 \to \mathbb R^2$ d\'{e}finie par $\phi(x,y) = (x+2, 2y)$. Sachant que la fonction $f:\mathbb R^2 \to \mathbb R$ est diff\'{e}rentiable en $p=(3,2)$ et que l'on a $\frac{\partial f}{\partial x}(p) = 2$ et  $\frac{\partial f}{\partial y}(p) = 1$. Calculer $D_v (f \circ \phi)(1,1)$ o\`{u} $v=(1,-1)$.
  \item 0
  \end{numerical}
% On a \nabla f \circ \phi (1,1) = (1*2 + 1 * 0, 2*1) = (2,2)
% on a  D_v f(p) = d_p f(v) =  \langle \nabla f, v \rangle = 2 - 5 = 0

  \begin{numerical}{Derivee directionnelle}
    On pose $\phi: \mathbb R^2 \to \mathbb R^2$ d\'{e}finie par $\phi(x,y) = (x+y, x-y)$. Sachant que la fonction $f:\mathbb R^2 \to \mathbb R$ est diff\'{e}rentiable en $p=(2,0)$ et que l'on a $\frac{\partial f}{\partial x}(p) = 2$ et  $\frac{\partial f}{\partial y}(p) = 1$. Calculer $D_v (f \circ \phi)(1,1)$ o\`{u} $v=(1,-1)$.
  \item 2
  \end{numerical}
% On a \nabla f \circ \phi (1,1) = (2*1 + 1 * 1, 1*2 -1*1) = (3, 1)
% on a  D_v f(p) = d_p f(v) =  \langle \nabla f, v \rangle = 3 - 1 = 2
\end{quiz}

\end{document}
