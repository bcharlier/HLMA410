\documentclass[12pt]{article}
\usepackage{amsmath}
\usepackage{moodle}


\begin{document}

\begin{quiz}[points=2]
  {Circulation (2pt)}

  \begin{numerical}{Circulation}
    Calculer la circulation du champs de gradient $\nabla f$ o\`{u} $f(x,y) = x^2 y^3 + \exp(41)$ le long du segment de droite entre le point $(7,1)$ et le point $(5,2)$. On donnera un arrondi \`{a} $10^{-2}$ pr\`{e}s.
  \item 151
  \end{numerical}
% 5^2 * 2^3 - 7^2 = 151

  \begin{numerical}{Circulation}
    Calculer la circulation du champs de gradient $\nabla f$ o\`{u} $f(x,y) = x^3 y^2 + \exp(18\ln 2)$ le long du segment de droite entre le point $(3,1)$ et le point $(5,2)$.  On donnera un arrondi \`{a} $10^{-2}$ pr\`{e}s.
  \item 473
  \end{numerical}
% 5^3 * 2^2 - 3^3 = 500 - 27 = 473

  \begin{numerical}{Circulation}
    Calculer la circulation du champs de gradient $\nabla f$ o\`{u} $f(x,y) = 2x^3 y^3 + \exp(20\pi)$ le long du segment de droite entre le point $(2,1)$ et le point $(4,2)$. On donnera un arrondi \`{a} $10^{-2}$ pr\`{e}s.
  \item 1008
  \end{numerical}
% 2*4^3 * 2^3 - 2* 2^3  = 1008

  \begin{numerical}{Circulation}
    Calculer la circulation du champs de gradient $\nabla f$ o\`{u} $f(x,y) = \frac{x^3 y^2}{2} + \exp(40)$ le long du segment de droite entre le point $(2,2)$ et le point $(-1,2)$.  On donnera un arrondi \`{a} $10^{-2}$ pr\`{e}s.
  \item -18
  \end{numerical}
% - 4 /2 - 2^3 * 2^2 /2 = -2 - 16 = - 18
\end{quiz}



%\begin{quiz}[points=3]
%  {Application de la diff\'{e}rentielle (3pt)}

%% ----------------------------------------------------------------
  %\begin{numerical}{Application de la diff\'{e}rentielle}
    %Sachant que la fonction $f:\mathbb R^3 \to \mathbb R$ est de classe $\mathcal C^2(\mathbb R^2)$. Soit $p=(3,2,1)$ et sachant que l'on a $f(p) = 10$,  $\frac{\partial f}{\partial x}(p) = 2$,  $\frac{\partial f}{\partial y}(p) = 1$ et $\frac{\partial f}{\partial z}(p) = 3$ et que $\operatorname{Hess}_f(p) = \begin{pmatrix}
      %1& 2& 0\\ 
      %2& 1& 1\\
      %0& 1& 3
    %\end{pmatrix}$. Calculer une valeur approch\'{e}e de $f(2.9,2.1,1.2)$. On donnera une valeur approch\'{e}e \`{a} $10^{-2}$ pr\`{e}s.
  %\item 10.57
  %\end{numerical}

%\end{quiz}


\end{document}
