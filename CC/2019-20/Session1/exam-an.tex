\documentclass[12pt]{article}
\usepackage{amsmath}
\usepackage{moodle}


\begin{document}

\begin{quiz}[points=4, multiple, shuffle, penalty=.5]
  {Analyse de Fonctions de plusieurs variables (4pt)}


  \begin{multi}{R\'{e}gularit\'{e} des fonctions de plusieurs variables}
    La fonction $f:\mathbb R^2\to \mathbb R$ d\'{e}finie par $f(x,y) = \frac{(x-1) y}{(x-1)^2 + y^2}$.  D\'{e}terminer lesquelles de ces assertions sont vraies.

    \item* $f$ admet des d\'{e}riv\'{e}es partielle en $(1,0)$
    \item* $f$ est de classe $\mathcal C^\infty$ sur $\mathbb R^2 \setminus \left\{(1,0) \right\}$
    \item* $f$ n'est pas diff\'{e}rentiable en $(1,0)$
    \item $f$ est continue en $(1,0)$
    \item Aucune des autres r\'{e}ponses n'est v\'{e}rifi\'{e}e
  \end{multi}

  \begin{multi}{R\'{e}gularit\'{e} des fonctions de plusieurs variables}
    La fonction $f:\mathbb R^2\to \mathbb R$ d\'{e}finie par $f(x,y) = \frac{xy^2}{x^2 + y^2}$. D\'{e}terminer lesquelles de ces assertions sont vraies.

    \item* $f$ est de classe $\mathcal C^\infty$ sur $\mathbb R^2 \setminus \left\{(0,0) \right\}$
    \item* $f$ est continue sur $\mathbb R^2$
    \item* $f$ admet des d\'{e}riv\'{e}es partielle en $(0,0)$
    \item $f$ est de classe $\mathcal C^1$ en $(0,0)$
    \item Aucune des autres r\'{e}ponses n'est v\'{e}rifi\'{e}e
  \end{multi}


  \begin{multi}{Extrema}
    Soit la fonction $f:\mathbb R^2\to \mathbb R$ d\'{e}finie par $f(x,y) = x\exp(-x^2 - 2y^2)$. D\'{e}terminer lesquelles de ces assertions sont vraies.

  \item $f$ poss\`{e}de un point selle en $(0, 0)$
  \item $f$ poss\`{e}de un maximum local en $(-\frac{1}{\sqrt 2}, 0)$
  \item $f$ poss\`{e}de un minimum local en $(\frac{1}{\sqrt2}, 0)$
  \item $f$ poss\`{e}de 3 points critiques: $(0,0)$, $(\frac{1}{\sqrt 2}, 0)$ et $(-\frac{1}{\sqrt 2}, 0)$
  \item* Aucune des autres r\'{e}ponses n'est v\'{e}rifi\'{e}e
  \end{multi}
% points critiques : $ x = \frac{1 \pm \sqrt 3}{2}$ et $y=0$


  \begin{multi}{Extrema}
    Soit la fonction $f:\mathbb R^2\to \mathbb R$ d\'{e}finie par $f(x,y) = x^2 \exp(-x^2y^2)$. 
  \item* $f$ poss\`{e}de un point critique d\'{e}g\'{e}n\'{e}r\'{e} en $(0, 0)$
  \item* $f$ poss\`{e}de un minimum local en $(0, 0)$
  \item $f$ poss\`{e}de un maximum local en $(0, 0)$
  \item $f$ poss\`{e}de un unique point critique en $(0,0)$
  \item $f$ poss\`{e}de une infinit\'{e} de points critiques situ\'{e}s sur les hyperboles  $\{y= \pm \frac{1}{x}, x \neq 0 \}$
  \item Aucune des autres r\'{e}ponses n'est v\'{e}rifi\'{e}e
  \end{multi}
% points critiques : droite x=0. Attention, le point $(0,0)$ est un point critique d\'{e}g\'{e}n\'{e}r\'{e} mais comme $f\geq 0$ il est facile de voir que c'est un min


  %\begin{multi}[points=2]{Extrema}
    %Soit la fonction $f:\mathbb R^2\to \mathbb R$ d\'{e}finie par $f(x,y) = x\exp(-(x-1)^2 - 2y^2)$. \textit{Indication: on donne $\operatorname{Hess}_f(x,y) = \begin{pmatrix}
        %(4x^3 - 8x^2 - 2x + 4 ) & (8x^2 - 8x - 4)y \\
        %(8x^2 - 8x - 4) y & (16xy^2 - 4x)
    %\end{pmatrix} \exp(-(x-1)^2 - 2y^2)$}
  %\item* $f$ poss\`{e}de un maximum local en $(\frac{1 + \sqrt 3}{2}, 0)$
  %\item* $f$ poss\`{e}de un miimum local en $(\frac{1 + \sqrt 3}{2}, 0)$
  %\end{multi}
% points critiques : $ x = \frac{1 \pm \sqrt 3}{2}$ et $y=0$
\end{quiz}

\end{document}
