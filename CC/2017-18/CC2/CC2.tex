\documentclass[a4paper]{tp_um}
\makeatletter
%--------------------------------------------------------------------------------

\usepackage[french]{babel}
\usepackage{amsmath}
\usepackage{amsbsy}
\usepackage{amsfonts}
\usepackage{amssymb}
\usepackage{amscd}
\usepackage{amsthm}
\usepackage{mathtools}
\usepackage{eurosym}
\usepackage{nicefrac}

\usepackage{latexsym}
\usepackage[a4paper,hmargin=20mm,vmargin=25mm]{geometry}
\usepackage{dsfont}
\usepackage[utf8]{inputenc}
\usepackage[T1]{fontenc}
\usepackage{lmodern}

\usepackage{multicol}
\usepackage[inline]{enumitem}
\setlist{nosep}
\setlist[itemize,1]{,label=$-$}


\newenvironment{modenumerate}
  {\enumerate\setupmodenumerate}
  {\endenumerate}

\newif\ifmoditem
\newcommand{\setupmodenumerate}{%
  \global\moditemfalse
  \let\origmakelabel\makelabel
  \def\moditem##1{\global\moditemtrue\def\mesymbol{##1}\item}%
  \def\makelabel##1{%
    \origmakelabel{##1\ifmoditem\rlap{\mesymbol}\fi\enspace}%
    \global\moditemfalse}%
}


\usepackage{sectsty}
%\sectionfont{}
\allsectionsfont{\color{astral}\normalfont\sffamily\bfseries\normalsize}

\usepackage{graphicx}
\usepackage{tikz}
\usetikzlibrary{babel}
\usepackage{tikz,tkz-tab}

\usepackage[babel=true, kerning=true]{microtype}


\usepackage{pgfplots}
\usepgfplotslibrary{fillbetween}
\pgfplotsset{compat=newest}
\usepgfplotslibrary{external} 
\tikzexternalize[prefix=./output_latex/]
%\DeclareSymbolFont{RalphSmithFonts}{U}{rsfs}{m}{n}
%\DeclareSymbolFontAlphabet{\mathscr}{RalphSmithFonts}
%\def\mathcal#1{{\mathscr #1}}



\providecommand{\abs}[1]{\left|#1\right|}
\providecommand{\norm}[1]{\left\Vert#1\right\Vert}
\providecommand{\U}{\mathcal{U}}
\providecommand{\R}{\mathbb{R}}
\providecommand{\Cc}{\mathcal{C}}
\providecommand{\reg}[1]{\mathcal{C}^{#1}}
\providecommand{\1}{\mathds{1}}
\providecommand{\N}{\mathbb{N}}
\providecommand{\Z}{\mathbb{Z}}
\providecommand{\p}{\partial}
\providecommand{\one}{\mathds{1}}
\providecommand{\E}{\mathbb{E}}\providecommand{\V}{\mathbb{V}}
\renewcommand{\P}{\mathbb{P}}


%Operateur
\providecommand{\abs}[1]{\left\lvert#1\right\rvert}
\providecommand{\sabs}[1]{\lvert#1\rvert}
\providecommand{\babs}[1]{\bigg\lvert#1\bigg\rvert}
\providecommand{\norm}[1]{\left\lVert#1\right\rVert}
\providecommand{\bnorm}[1]{\bigg\lVert#1\bigg\rVert}
\providecommand{\snorm}[1]{\lVert#1\rVert}
\providecommand{\prs}[1]{\left\langle #1\right\rangle}
\providecommand{\sprs}[1]{\langle #1\rangle}
\providecommand{\bprs}[1]{\bigg\langle #1\bigg\rangle}

\DeclareMathOperator{\deet}{Det}
\DeclareMathOperator{\hess}{Hess}
\DeclareMathOperator{\jac}{Jac}


\newcommand\rst[2]{{#1}_{\restriction_{#2}}}



% generate breakable white space allowing students to write notes.

\usepackage[framemethod=tikz]{mdframed}

\mdfdefinestyle{response}{
	leftmargin=.01\textwidth,
	rightmargin=.01\textwidth,
	linewidth=1pt
	hidealllines=false,
	leftline=true,
	rightline=true,topline=true,bottomline=true,
	skipabove=0pt,
	%innertopmargin=-5pt,
	%innerbottommargin=2pt,
	linecolor=black,
	innerrightmargin=0pt,
	}



\newcommand*{\DivideLengths}[2]{%
  \strip@pt\dimexpr\number\numexpr\number\dimexpr#1\relax*65536/\number\dimexpr#2\relax\relax sp\relax
}

\providecommand{\rep}[1]{$ $ \newline \begin{mdframed}[style=response]  
	
	\vspace*{\DivideLengths{#1}{3cm}cm}
	\pagebreak[1]	
	\vspace*{\DivideLengths{#1}{3cm}cm}
	\pagebreak[1]		
	\vspace*{\DivideLengths{#1}{3cm}cm}   \end{mdframed}}

\providecommand{\blanc}[1]{$ $ \newline 
	
	\vspace*{\DivideLengths{#1}{3cm}cm}
	\pagebreak[1]	
	\vspace*{\DivideLengths{#1}{3cm}cm}
	\pagebreak[3]		
	\vspace*{\DivideLengths{#1}{3cm}cm}}

\usepackage{ifthen}

\newcommand{\eno}[1]{%
	\ifthenelse{\equal{\version}{eno}}{#1}{}%
}
\newcommand{\cor}[1]{%
        \ifthenelse{\equal{\version}{cor}}{
\medskip 

{\small \color{gray} #1}

\medskip 
}{}
}

%------------------------------------------------------------------------------
%\DeclareUnicodeCharacter{00A0}{~}
\makeatother


%\makeatletter
%--------------------------------------------------------------------------------

\usepackage[frenchb]{babel}

\usepackage{amsmath}
\usepackage{amsbsy}
\usepackage{amsfonts}
\usepackage{amssymb}
\usepackage{amscd}
\usepackage{amsthm}
\usepackage{mathtools}
\usepackage{eurosym}
\usepackage{nicefrac}

\usepackage{latexsym}
\usepackage[a4paper,hmargin=20mm,vmargin=25mm]{geometry}
\usepackage{dsfont}
\usepackage[utf8]{inputenc}
\usepackage[T1]{fontenc}

\usepackage{multicol}
\usepackage[inline]{enumitem}
%\setlist{nosep}
\setlist[itemize,1]{,label=$-$}

\usepackage{sectsty}
%\sectionfont{}
\allsectionsfont{\normalfont\sffamily\bfseries\normalsize}

\usepackage{graphicx}
\usepackage{tikz}

\usepackage{pgfplots}
\usepgfplotslibrary{fillbetween}
\pgfplotsset{compat=newest}
%\usepgfplotslibrary{external} 
%\tikzexternalize[prefix=./output_latex/]
%\DeclareSymbolFont{RalphSmithFonts}{U}{rsfs}{m}{n}
%\DeclareSymbolFontAlphabet{\mathscr}{RalphSmithFonts}
%\def\mathcal#1{{\mathscr #1}}

\newcounter{zut}
\setcounter{zut}{1}
\newcommand{\exo}[1]{\noindent {\sffamily\bfseries Exercice~\thezut. #1} \
		   \addtocounter{zut}{1}}



\providecommand{\abs}[1]{\left|#1\right|}
\providecommand{\norm}[1]{\left\Vert#1\right\Vert}
\providecommand{\U}{\mathcal{U}}
\providecommand{\R}{\mathbb{R}}
\providecommand{\Cc}{\mathcal{C}}
\providecommand{\reg}[1]{\mathcal{C}^{#1}}
\providecommand{\1}{\mathds{1}}
\providecommand{\N}{\mathbb{N}}
\providecommand{\Z}{\mathbb{Z}}
\providecommand{\E}{\mathbb{E}}
\providecommand{\p}{\partial}
\providecommand{\one}{\mathds{1}}
\renewcommand{\P}{\mathbb{P}}


%Operateur
\providecommand{\abs}[1]{\left\lvert#1\right\rvert}
\providecommand{\sabs}[1]{\lvert#1\rvert}
\providecommand{\babs}[1]{\bigg\lvert#1\bigg\rvert}
\providecommand{\norm}[1]{\left\lVert#1\right\rVert}
\providecommand{\bnorm}[1]{\bigg\lVert#1\bigg\rVert}
\providecommand{\snorm}[1]{\lVert#1\rVert}
\providecommand{\prs}[1]{\left\langle #1\right\rangle}
\providecommand{\sprs}[1]{\langle #1\rangle}
\providecommand{\bprs}[1]{\bigg\langle #1\bigg\rangle}

\DeclareMathOperator{\deet}{Det}
\DeclareMathOperator{\vol}{Vol}
\DeclareMathOperator{\aire}{Aire}
\DeclareMathOperator{\hess}{Hess}
\DeclareMathOperator{\var}{Var}

%------------------------------------------------------------------------------
\DeclareUnicodeCharacter{00A0}{~}
\makeatother


\ue{HLMA410}

%-----------------------------------------------------------------------------

\title{\large \sffamily\bfseries Contrôle continu 2}

\begin{document}

\maketitle
\textit{Durée 1h10. Les documents, la calculatrice, les téléphones portables, tablettes, ordinateurs ne sont pas autorisés. La qualité de la rédaction sera prise en compte.} 

\bigskip
\bigskip

\exo{} Soit $u = (2,1)$ et $D$ la droite vectorielle dirigee par $u$ (\textit{i.e.} $D = \R u$). Soit $x = (x_1,x_2) \in \mathbb R^2$:

\begin{enumerate}
    \item Calculer $d_1(x)$ la distance de $x$ a $D$ en fonction de $x_1$ et $x_2$. 
        \blanc{4cm}
        
    \item Calculer $d_2(x) = \prs{x,u}$ en fonction de $x_1$ et $x_2$.

        \blanc{4cm}

    \item Montrer que $N(x) = \sqrt{5}|d_1(x)| + |d_2(x)| $ définit une norme sur $\R^2$.

        \blanc{8cm} \pagebreak
    \item  Dessiner la boule unité pour la norme $N$.
        \begin{center}
            \begin{tikzpicture}[scale=2]
                \def\xone{-1.2}
                \def\xtwo{1.2}
                \def\yone{-1.2}
                \def\ytwo{1.2}
% grid
                \draw[step=.2cm,help lines,gray!50] (\xone,\yone) grid (\xtwo,\ytwo);
                \draw[thin,->] (\xone-.1, 0) -- (\xtwo+.1, 0) node[right] {$x$};
                \draw[thin,->] (0, \yone-.1) -- (0, \ytwo+.1) node[above] {$y$};
		
			   \draw (1,-0.01) node {$\bullet$} ;
			   \draw (1,0) node [below]{$1$} ;  

			   \draw (0,1-0.01) node {$\bullet$} ;
			   \draw (0,1) node [right]{$1$} ;  

% 	          \draw [red, very thick]  (\xone,\xone/2) -- (\xtwo,\xtwo/2) ;
% 	          \draw [red, very thick]  (-\yone/2,\yone) -- (-\ytwo/2,\ytwo) ;
%				\draw [green] (-2/5,-1/5) -- (1/5,-2/5) -- (2/5,1/5)--  (-1/5,2/5) -- cycle ; 
					          
            \end{tikzpicture}
        \end{center}
        
        \item Montrer que $N$ et $\|\cdot\|_2$ sont \'equivalentes.
         \blanc{6cm}
       
\end{enumerate}

%\exo[]{(Distance SNCF)} On note $\norm{\cdot}_2$ la norme euclidienne usuelle sur $\R^2$. On
%note $O=(0,0)$ l'origine du plan et on définit
%\[
	%d(A,B) = \begin{cases}
		%%\snorm{u-v}_2, &\text{si les points $u$, $v$ et $O$ sont alignés  } \\
		%%\snorm{u}_2 + \snorm{v}_2, &\text{sinon.}
		%\snorm{\overrightarrow{AB}}_2, &\text{si les points $A$, $B$ et $O$ sont alignés  } \\
		%\snorm{\overrightarrow{OA}}_2 + \snorm{\overrightarrow{OB}}_2, &\text{sinon.}
	%\end{cases}
%\]
%\begin{enumerate}
    %\item Donner la définition d'une distance. 
        %\blanc{6cm}        
%On admettra dans la suite que $d$ est une distance sur $\R^2$.
	%\item Dessiner l'ensemble des points situés à distance inférieure à 2 du point $C = (2,0)$.  Puis dessiner l'ensemble des points situés à distance inférieure à 3 du point $C = (2,0)$.
				%\begin{center}
				%\begin{tikzpicture}[scale=.5]
					%\def\xone{-5}
					%\def\xtwo{5}
					%\def\yone{-5}
                    %\def\ytwo{5}
%% grid
					%\draw[step=1cm,help lines,gray!50] (\xone-.2,\yone-.2) grid (\xtwo+.2,\ytwo+.2);
					%\draw[thin,->] (\xone-.3, 0) -- (\xtwo+.3, 0) node[right] {$x$};
					%\draw[thin,->] (0, \yone-.3) -- (0, \ytwo+.3) node[above] {$y$};
                %\end{tikzpicture}\hspace{2cm}
				%\begin{tikzpicture}[scale=.5]
					%\def\xone{-5}
					%\def\xtwo{5}
					%\def\yone{-5}
                    %\def\ytwo{5}
%% grid
					%\draw[step=1cm,help lines,gray!50] (\xone-.2,\yone-.2) grid (\xtwo+.2,\ytwo+.2);
					%\draw[thin,->] (\xone-.3, 0) -- (\xtwo+.3, 0) node[right] {$x$};
					%\draw[thin,->] (0, \yone-.3) -- (0, \ytwo+.3) node[above] {$y$};
				%\end{tikzpicture}
			%\end{center}
	%\item On considère la suite $u = ( 1, \frac{1}{m+1})_{m\in\N}$ dans $\R^2$. Montrer que $u$ converge vers $\ell=(1,0)$ pour la norme $2$ mais que la suite $d(u_m,\ell)$ ne tend pas vers 0 quand $m \to +\infty$.
        %\blanc{5cm}
	%\item Existe-t-il une norme $\snorm{\cdot}$ sur $\R^2$ telle que $d(A,B) = \snorm{\overrightarrow{AB}}$ pour tout $A,B\in\R^2$? 
        %\blanc{2cm}
%\end{enumerate}
		
		
		
		
		
\exo{}  Soit la courbe paramétrée $\Gamma=\left( \R, \phi \right)$ définie par $ \phi(t) = \begin{cases}x(t)= t - \tanh t \\ y(t) = \frac{1}{\cosh t} \end{cases}$ pour $t\in\R$

	\begin{enumerate}
        \item \'Etudier la parité des fonctions $x(\cdot)$ et $y(\cdot)$. Quelle(s) symétrie(s) cela implique-t-il sur le support de la courbe $\Gamma$? Peut on réduire le domaine d'étude ?
			\blanc{5cm}


                    \item Calculer $\phi', \phi''$ (on donne $\phi'''(t)=\begin{pmatrix}
                                2(1- 2\sinh^2 t) /\cosh^4t \\ (5\tanh t - 6 \tanh^3 t ) / \cosh t  
                    \end{pmatrix}$) et déterminer si $\Gamma$ à un/des point(s) stationnaire(s).
			\blanc{9cm}


        \item On se place en $t=0$: donner la nature du point $\Gamma(0)$ ainsi que le comportement local de la courbe (faire un petit dessin).
			\blanc{13cm}
	%	\item Y a t il des point de la courbe avec une tangente verticale ? horizontal ? de direction $(1,1)$ ?
			%\blanc{6cm}
	
        \item On se place au voisinage de $t=+\infty$. Étudier la branche infinie (asymptote et position relative).

			\blanc{7cm}
		
		\item Compléter le tableau de variations suivant:
			\begin{center}
				\begin{tabular}{|c|ccccc|}
					\hline    $t$       & $-\infty$ & \hspace{5cm}   &  0 &  \hspace{5cm} & $\infty$ \\[0.3cm]\hline\hline
					signe de $x'(t)$    &           &                &    &                 &          \\[0.4cm]\hline
       			 variation de $x(t)$    &           &                &    &                 &     	 \\[0.9cm]\hline\hline
					signe de $y'(t)$    &           &                &    &                 &          \\[0.4cm]\hline
					variation de $y(t)$ &           &                &    &                 &          \\[0.9cm]\hline
				\end{tabular}
			\end{center}
                        \vfill
		\item  Sur le graphique suivant, tracer la courbe $\Gamma$ ainsi que les tangentes et asymptotes étudiées aux questions précédentes. 
				\begin{center}
				\begin{tikzpicture}[scale=.5]
					\def\xone{-12}
					\def\xtwo{12}
					\def\yone{-1}
                    \def\ytwo{5}
% grid
					\draw[step=1cm,help lines,gray!50] (\xone-.2,\yone-.2) grid (\xtwo+.2,\ytwo+.2);
					\draw[thin,->] (\xone-.3, 0) -- (\xtwo+.3, 0) node[right] {$x$};
					\draw[thin,->] (0, \yone-.3) -- (0, \ytwo+.3) node[above] {$y$};
				\end{tikzpicture}
			\end{center}
			%\blanc{8cm}



	\end{enumerate}
 
\end{document}
