\documentclass[a4paper]{tp_um}
\makeatletter
%--------------------------------------------------------------------------------

\usepackage[french]{babel}
\usepackage{amsmath}
\usepackage{amsbsy}
\usepackage{amsfonts}
\usepackage{amssymb}
\usepackage{amscd}
\usepackage{amsthm}
\usepackage{mathtools}
\usepackage{eurosym}
\usepackage{nicefrac}

\usepackage{latexsym}
\usepackage[a4paper,hmargin=20mm,vmargin=25mm]{geometry}
\usepackage{dsfont}
\usepackage[utf8]{inputenc}
\usepackage[T1]{fontenc}
\usepackage{lmodern}

\usepackage{multicol}
\usepackage[inline]{enumitem}
\setlist{nosep}
\setlist[itemize,1]{,label=$-$}


\newenvironment{modenumerate}
  {\enumerate\setupmodenumerate}
  {\endenumerate}

\newif\ifmoditem
\newcommand{\setupmodenumerate}{%
  \global\moditemfalse
  \let\origmakelabel\makelabel
  \def\moditem##1{\global\moditemtrue\def\mesymbol{##1}\item}%
  \def\makelabel##1{%
    \origmakelabel{##1\ifmoditem\rlap{\mesymbol}\fi\enspace}%
    \global\moditemfalse}%
}


\usepackage{sectsty}
%\sectionfont{}
\allsectionsfont{\color{astral}\normalfont\sffamily\bfseries\normalsize}

\usepackage{graphicx}
\usepackage{tikz}
\usetikzlibrary{babel}
\usepackage{tikz,tkz-tab}

\usepackage[babel=true, kerning=true]{microtype}


\usepackage{pgfplots}
\usepgfplotslibrary{fillbetween}
\pgfplotsset{compat=newest}
\usepgfplotslibrary{external} 
\tikzexternalize[prefix=./output_latex/]
%\DeclareSymbolFont{RalphSmithFonts}{U}{rsfs}{m}{n}
%\DeclareSymbolFontAlphabet{\mathscr}{RalphSmithFonts}
%\def\mathcal#1{{\mathscr #1}}



\providecommand{\abs}[1]{\left|#1\right|}
\providecommand{\norm}[1]{\left\Vert#1\right\Vert}
\providecommand{\U}{\mathcal{U}}
\providecommand{\R}{\mathbb{R}}
\providecommand{\Cc}{\mathcal{C}}
\providecommand{\reg}[1]{\mathcal{C}^{#1}}
\providecommand{\1}{\mathds{1}}
\providecommand{\N}{\mathbb{N}}
\providecommand{\Z}{\mathbb{Z}}
\providecommand{\p}{\partial}
\providecommand{\one}{\mathds{1}}
\providecommand{\E}{\mathbb{E}}\providecommand{\V}{\mathbb{V}}
\renewcommand{\P}{\mathbb{P}}


%Operateur
\providecommand{\abs}[1]{\left\lvert#1\right\rvert}
\providecommand{\sabs}[1]{\lvert#1\rvert}
\providecommand{\babs}[1]{\bigg\lvert#1\bigg\rvert}
\providecommand{\norm}[1]{\left\lVert#1\right\rVert}
\providecommand{\bnorm}[1]{\bigg\lVert#1\bigg\rVert}
\providecommand{\snorm}[1]{\lVert#1\rVert}
\providecommand{\prs}[1]{\left\langle #1\right\rangle}
\providecommand{\sprs}[1]{\langle #1\rangle}
\providecommand{\bprs}[1]{\bigg\langle #1\bigg\rangle}

\DeclareMathOperator{\deet}{Det}
\DeclareMathOperator{\hess}{Hess}
\DeclareMathOperator{\jac}{Jac}


\newcommand\rst[2]{{#1}_{\restriction_{#2}}}



% generate breakable white space allowing students to write notes.

\usepackage[framemethod=tikz]{mdframed}

\mdfdefinestyle{response}{
	leftmargin=.01\textwidth,
	rightmargin=.01\textwidth,
	linewidth=1pt
	hidealllines=false,
	leftline=true,
	rightline=true,topline=true,bottomline=true,
	skipabove=0pt,
	%innertopmargin=-5pt,
	%innerbottommargin=2pt,
	linecolor=black,
	innerrightmargin=0pt,
	}



\newcommand*{\DivideLengths}[2]{%
  \strip@pt\dimexpr\number\numexpr\number\dimexpr#1\relax*65536/\number\dimexpr#2\relax\relax sp\relax
}

\providecommand{\rep}[1]{$ $ \newline \begin{mdframed}[style=response]  
	
	\vspace*{\DivideLengths{#1}{3cm}cm}
	\pagebreak[1]	
	\vspace*{\DivideLengths{#1}{3cm}cm}
	\pagebreak[1]		
	\vspace*{\DivideLengths{#1}{3cm}cm}   \end{mdframed}}

\providecommand{\blanc}[1]{$ $ \newline 
	
	\vspace*{\DivideLengths{#1}{3cm}cm}
	\pagebreak[1]	
	\vspace*{\DivideLengths{#1}{3cm}cm}
	\pagebreak[3]		
	\vspace*{\DivideLengths{#1}{3cm}cm}}

\usepackage{ifthen}

\newcommand{\eno}[1]{%
	\ifthenelse{\equal{\version}{eno}}{#1}{}%
}
\newcommand{\cor}[1]{%
        \ifthenelse{\equal{\version}{cor}}{
\medskip 

{\small \color{gray} #1}

\medskip 
}{}
}

%------------------------------------------------------------------------------
%\DeclareUnicodeCharacter{00A0}{~}
\makeatother


%\makeatletter
%--------------------------------------------------------------------------------

\usepackage[frenchb]{babel}

\usepackage{amsmath}
\usepackage{amsbsy}
\usepackage{amsfonts}
\usepackage{amssymb}
\usepackage{amscd}
\usepackage{amsthm}
\usepackage{mathtools}
\usepackage{eurosym}
\usepackage{nicefrac}

\usepackage{latexsym}
\usepackage[a4paper,hmargin=20mm,vmargin=25mm]{geometry}
\usepackage{dsfont}
\usepackage[utf8]{inputenc}
\usepackage[T1]{fontenc}

\usepackage{multicol}
\usepackage[inline]{enumitem}
%\setlist{nosep}
\setlist[itemize,1]{,label=$-$}

\usepackage{sectsty}
%\sectionfont{}
\allsectionsfont{\normalfont\sffamily\bfseries\normalsize}

\usepackage{graphicx}
\usepackage{tikz}

\usepackage{pgfplots}
\usepgfplotslibrary{fillbetween}
\pgfplotsset{compat=newest}
%\usepgfplotslibrary{external} 
%\tikzexternalize[prefix=./output_latex/]
%\DeclareSymbolFont{RalphSmithFonts}{U}{rsfs}{m}{n}
%\DeclareSymbolFontAlphabet{\mathscr}{RalphSmithFonts}
%\def\mathcal#1{{\mathscr #1}}

\newcounter{zut}
\setcounter{zut}{1}
\newcommand{\exo}[1]{\noindent {\sffamily\bfseries Exercice~\thezut. #1} \
		   \addtocounter{zut}{1}}



\providecommand{\abs}[1]{\left|#1\right|}
\providecommand{\norm}[1]{\left\Vert#1\right\Vert}
\providecommand{\U}{\mathcal{U}}
\providecommand{\R}{\mathbb{R}}
\providecommand{\Cc}{\mathcal{C}}
\providecommand{\reg}[1]{\mathcal{C}^{#1}}
\providecommand{\1}{\mathds{1}}
\providecommand{\N}{\mathbb{N}}
\providecommand{\Z}{\mathbb{Z}}
\providecommand{\E}{\mathbb{E}}
\providecommand{\p}{\partial}
\providecommand{\one}{\mathds{1}}
\renewcommand{\P}{\mathbb{P}}


%Operateur
\providecommand{\abs}[1]{\left\lvert#1\right\rvert}
\providecommand{\sabs}[1]{\lvert#1\rvert}
\providecommand{\babs}[1]{\bigg\lvert#1\bigg\rvert}
\providecommand{\norm}[1]{\left\lVert#1\right\rVert}
\providecommand{\bnorm}[1]{\bigg\lVert#1\bigg\rVert}
\providecommand{\snorm}[1]{\lVert#1\rVert}
\providecommand{\prs}[1]{\left\langle #1\right\rangle}
\providecommand{\sprs}[1]{\langle #1\rangle}
\providecommand{\bprs}[1]{\bigg\langle #1\bigg\rangle}

\DeclareMathOperator{\deet}{Det}
\DeclareMathOperator{\vol}{Vol}
\DeclareMathOperator{\aire}{Aire}
\DeclareMathOperator{\hess}{Hess}
\DeclareMathOperator{\var}{Var}

%------------------------------------------------------------------------------
\DeclareUnicodeCharacter{00A0}{~}
\makeatother


\ue{HLMA410}

%-----------------------------------------------------------------------------

\title{\large \sffamily\bfseries Examen}

\begin{document}

\maketitle
\textit{Durée 3h00. Les documents, la calculatrice, les téléphones portables, tablettes, ordinateurs ne sont pas autorisés. Les exercices sont indépendants. La qualité de la rédaction sera prise en compte.} 

\bigskip
\bigskip

\exo{}
Un livre contient des erreurs de rédaction. À chaque relecture, une faute non corrigée est corrigée avec une probabilité de $1/3$. Les corrections des différentes fautes sont indépendantes les unes des autres; les relectures successives aussi.
\begin{enumerate}
    \item On suppose que le livre contient exactement 4 erreurs. Soit $n\in\N$, calculer la probabilité que toutes les fautes ait été corrigées en $n$ relectures. %Combien faut il faire de relectures pour que la probabilité qu'il ne subsiste aucune erreur soit suppérieure à $0.9$?

        %\medskip

    %On note $C_i$ la variable aléatoire correspondant au nombre de relectures nécessaires pour corriger la faute $i=1,2,3,4$. On cherche donc la probabilité de l'évènement 
    %\[
    %A_n  = \{C_1 \leq n\} \cap \cdots \cap \left\{ C_4 \leq n \right\}.
    %\]
    %A chaque relecture, la probabilité de succès est $1/3$ et on reconnaît un schéma de Bernoulli répété de manière i.i.d. Autrement dit, $C_i$ suit une loi géométrique de paramètre $1/3$ et on  a 
        %\[
            %\P(C_i \leq n) = \sum_{\ell=1}^n \P(C_i =\ell) = \frac 13 \sum_{\ell=1}^n \left( \frac 23 \right)^{\ell-1} = 1 - \left( \frac 23 \right)^n
    %\]
    %Ainsi, la probabilité que toutes les fautes soient corrigées en $n$ relectures est 
    %\begin{equation}\label{eq.1}
        %\P\left( \max_{i=1,\cdots,4} \left\{ C_i \right\} \leq n \right) = \left(1 - \left( \frac 23 \right)^n \right)^4.
    %\end{equation}

        %\medskip
    
\item On suppose maintenant que le livre contient un nombre aléatoire d'erreurs qui suit une loi uniforme sur $\left\{ 0,1,2,3,4 \right\}$. Soit $n\in\N$, calculer la probabilité que toutes les fautes ait été corrigées en $n$ relectures.%Combien faut il faire de relectures pour que la probabilité qu'il ne subsiste aucune erreur soit supérieure à $0.9$?

        %\medskip

         %Pour faire le calcul, il faut conditionner par le nombre aléatoire $E$ de fautes dans le livre. On a, avec les notations de la question précédente,
        %\[
            %\P\left( \{C_1 \leq n\} \cap \cdots \cap \left\{ C_e \leq n \right\} \big| \{E=e\} \right) = \left(1 - \left( \frac 23 \right)^n \right)^e
        %\]
        %En utilisant la formule des probabilités totale (principe de partition), il vient,
    %\begin{align}
            %\P\left( \max_{i=0,\ldots,E} \left\{ C_i \right\} \leq n \right) &= \sum_{e=0}^4  \P\left( \{C_1 \leq n\} \cap \cdots \cap \left\{ C_e \leq n\right\} \big| \{E=e\} \right) \P(E=e) \nonumber\\
            %&= \frac 15 \frac{ 1 -\left(1 - \left( \frac 23 \right)^n \right)^5}{ \left( \frac 23 \right)^n }. \label{eq.2}
        %\end{align}

    \item Dans lequel des 2 cas, faudra-t-il faire le moins de relectures pour que la probabilité qu'il ne subsiste aucune erreur soit supérieure à $0.9$? 

%\medskip
%Dans le deuxième cas, car le nombre de fautes est au plus 4. On peut faire l'application numérique:  le membre de droite de \ref{eq.1} est plus grand que 0.9 dès lors que $n\geq 10$ tandis que le membre de droite de l'équation \eqref{eq.2} le sera quand  $n \geq 8$. 

\end{enumerate}




\exo{}
Soit $\varphi : [0,\infty[ \to \mathbb R$ d\'erivable et de d\'eriv\'ee continue sur $[0,\infty[$. 
On pose:
\[
\begin{array}{rrcl}
f  : & \mathbb R^2 & \longrightarrow & \mathbb R  \\
		& (x,y) & \longmapsto & \varphi(\sqrt{x^2+y^2})
\end{array}
\]
\begin{enumerate}
\item Montrer que $f \in \mathcal C(\mathbb R^2)$.

%\medskip

%La fonction $f$ est la composée d'une fonction continue $(x,y) \mapsto \sqrt{x^2 + y^2}$ et d'une fonction $\mathcal C^1(\R)$. Elle est donc bien continue sur $\R^2$.

%\medskip

\item Montrer que $f \in \mathcal C^1(\mathbb R^2 \setminus \{(0,0)\})$ et calculer $\nabla f(x,y)$
pour tout $(x,y) \in \mathbb R^2 \setminus \{(0,0)\}$.

%\medskip
%La fonction $f$ est la composée d'une fonction $\mathcal C^1(\mathbb R^2 \setminus \{(0,0)\})$ (toujours l'application $(x,y) \mapsto \sqrt{x^2 + y^2}$) et d'une fonction $\mathcal C^1(\R)$. Elle est donc bien $\mathcal C^1(\mathbb R^2 \setminus \{(0,0)\})$. On a
%\[
    %\frac{\partial f}{\partial x} (x,y) = \frac{x}{\sqrt{x^2+y^2}} \varphi'(\sqrt{x^2 + y^2}) \quad \text{ et } \quad \frac{\partial f}{\partial y} (x,y) = \frac{y}{\sqrt{x^2+y^2}} \varphi'(\sqrt{x^2 + y^2})
%\]

%\medskip

\item En d\'eduire que $f \in \mathcal C^{1}(\mathbb R^2)$  si et seulement si $\varphi'(0) = 0.$
On supposera cette condition satisfaite par la suite.

%\medskip

%\begin{itemize}
    %\item[$\Rightarrow$] Comme $f\in \mathcal C^{1}(\mathbb R^2)$ on a $\lim_{(x,y) \to (0,0)} \frac{\partial f}{\partial x} (x,y) =\lim_{x\to 0^+}\frac{\partial f}{\partial x} (x,0) = \lim_{x\to 0^-}\frac{\partial f}{\partial x} (x,0)$. De plus, on a $\lim_{x\to 0^+}\frac{\partial f}{\partial x} (x,0) = \varphi'(0)$ et $\lim_{x\to 0^-}\frac{\partial f}{\partial x} (x,0) = -\varphi'(0)$. Ce qui donne $\varphi'(0) = 0$ (car ``0 est le seul nombre égal à son opposé'').
    %\item [$\Leftarrow$]  Le taux d'accroissement de la première fonction partielle satisfait:
%\[
    %\lim_{h\to 0} \frac{f(h,0) - f(0,0)}{h} = \lim_{h\to 0} \frac{\varphi(\abs{h})}{h} = \varphi'(0) = 0
%\]
%et on a $ \frac{\partial f}{\partial x}(0,0) =0$. De même $ \frac{\partial f}{\partial y} (0,0) =0$.  Reste à montrer que les dérivées partielles sont continues en 0. C'est bien le cas car 
%\[
    %\abs{\frac{\partial f}{\partial x} (x,y) } \leq \frac{r\abs{\cos\theta} }{r} \varphi'(r) \leq \varphi'(r) \xrightarrow[r\to 0]{} 0.
%\]
%Le même raisonnement permet de voir que $\frac{\partial f}{\partial y}$ est aussi continue en l'origine.
%\end{itemize}
%\medskip

\item On suppose de plus $\varphi'$ d\'erivable et $\varphi''$ continue sur $[0,\infty[.$
\begin{enumerate}
\item Montrer que $f \in \mathcal C^2(\mathbb R^2 \setminus \{(0,0)\})$ et, pour $(x,y) \in \mathbb R^2 \setminus \{(0,0)\},$ 
calculer 
\[
\Delta f(x,y) = \dfrac{\partial^2 f}{\partial x^2} (x,y) + \dfrac{\partial^2 f}{\partial y^2} (x,y) \,, \
\]
en fonction de $\sqrt{x^2+y^2},$ $\varphi'(\sqrt{x^2+y^2})$ et $\varphi''(\sqrt{x^2+y^2}).$

%\medskip

%La fonction $f$ est $\mathcal C^2$ sur $\mathbb R^2 \setminus \{(0,0)\}$ car elle c'est la composée de $(x,y) \mapsto \sqrt{x+y}$ qui est $\mathcal C^2(\mathbb R^2 \setminus \{(0,0)\}, ]0,+\infty[)$  et de  $\varphi$ qui est $\mathcal C^2(]0,+\infty[, \R)$. 

%On a $\frac{\partial^2 f}{\partial x^2} (x,y) = \frac{y^2}{ (x^2+y^2)^{3/2}}  \varphi'(\sqrt{x^2+y^2}) +\frac{x^2}{{x^2+y^2}} \varphi''(\sqrt{x^2 + y^2}) $ et 
%$\frac{\partial^2 f}{\partial y^2} (x,y) = \frac{x^2}{ (x^2+y^2)^{3/2}} \linebreak[4] \varphi'(\sqrt{x^2+y^2}) +\frac{y^2}{{x^2+y^2}} \varphi''(\sqrt{x^2 + y^2}) $. Ainsi,
%\[
    %\Delta f(x,y) = \frac{1}{\sqrt{x^2 + y^2} }\varphi'( \sqrt{x^2 + y^2}) + \varphi''( \sqrt{x^2 + y^2}).
%\]

%\medskip

\item Montrer que $\Delta f \in \mathcal C(\mathbb R^2 \setminus \{(0,0)\})$  et admet un prolongement
par continuit\'e sur $\mathbb R^2.$

%\medskip

%La fonction $\varphi$ est $\mathcal C^2$ sur $\R$ et $\Delta f$ est donc $\mathcal C(\mathbb R^2 \setminus \{(0,0)\}) $ par composition. On rappelle au passage que $\varphi'(0) = 0$ et que $\varphi''$ est continue en 0. Reste à voir la limite en l'origine :
%\[
    %\lim_{(x,y) \to (0,0) }  \Delta f(x,y)  =  \underbrace{\lim_{r \to 0 } \frac{\varphi'(r)}{r}}_{\lim_{r \to 0^+ }\limits \frac{\varphi'(r) - \varphi'(0)}{r - 0} = \varphi''(0)} + \varphi''(0) = 2 \varphi''(0). 
%\]
%On peut donc prolonger par continuité $\Delta f$ en l'origine avec la valeur $2\varphi''(0)$. 

\end{enumerate}

\end{enumerate}


\exo{} 
On définit les  applications $\R^2\to\R$ suivantes:
\begin{align*}
    N_1(x,y) &= \abs x + \abs y + \max\left\{ \abs x, \abs y \right\},\\
    N_2(x,y) &= \abs x + \abs y + \min\left\{ \abs x, \abs y \right\},\\
    N_3(x,y) &= N_1(x,y) + N_2(x,y).
\end{align*}
        \begin{enumerate}
            \item  Tracer la courbe de niveau 1 de chacune de ces applications.
            \item Au vu des dessins, justifier dans quels cas ces applications définissent une norme sur $\R^2$.
        \end{enumerate}


\bigskip 


\exo{} On pose pour tout $x,y \in\R$, \[f(x,y) = \abs{4(x - 1)^2 + 9(y + 2)^2 - 1}.\]% Le but de cet exerci 
\begin{enumerate}
	%\item Déterminer les réels $a_1,a_2,c_1,c_2$ et $b$ tels que $f(x , y) = \abs{a_1(x-c_1)^2 + a_2 (y-c_2)^2 -b}$. % Appliquer l'algorithme de la décomposition de Gauss et en déduire le signe de $q$
    \item On pose $ N = \{(x,y) \in \R^2,  4(x - 1)^2 + 9(y + 2)^2 - 1 \leq 0\}$. C'est l'intérieur de l'ellipse suivante: 
	\begin{center}
			\begin{tikzpicture}[scale=1]
                            \begin{axis}[xlabel=$x$,ylabel=$y$, view={0}{90}, axis equal, xmin=0, xmax=3, ymin=-3, ymax=0, gris]%,xtick=\empty,ytick=\empty,ztick=\empty ]
                                   \addplot3[fill,samples=200,contour gnuplot={levels={.5},labels=false}] gnuplot {(-4*(x - 1)**2 - 9*(y + 2)**2  +1)};
                                   % \draw (axis cs:1,-2) ellipse (.25,.23);
				\end{axis}
			\end{tikzpicture}
		\end{center}

	
       % \begin{enumerate}
        %    \item Dessiner $N$. 

         %       \medskip
                
          %      On a $N = \{(x,y) \in \R^2,  4(x - 1)^2 + 9(y + 2)^2  <1 \}$. C'est l'intérieur d'une ellipse centrée en $(1,-2)$.

           %     \medskip
                
            %\item
                Calculer $ I = \iint_N f(x,y) dxdy$ en utilisant un changement de variable. 

%\medskip

                %Posons le changement de variable affine $(x,y) \mapsto (X = 2(x-1), Y = 3(y+2))$. On a alors: 
                %\begin{align*}
                    %I = \iint_N f(x,y) dxdy = 6 \iint_M (1 - X^2 - Y^2) dXdY 
                %\end{align*}
                %avec $M = \left\{ (X,Y) \in \R^2 | X^2 + Y^2 <1 \right\}$. On peut passer en coordonnées polaire  pour simplifier et on a 
                %\[
                    %I = 6\int_0^{2\pi} \int_0^1 (1- r^2)r drd\theta = 12\pi \left( \frac 12 - \frac 13 \right) = 2 \pi.
                %\]
                
        %%\end{enumerate}
		%\bigskip
	\item Étudier la continuité $f$ et donner l'ensemble image de $f$. 

%\medskip
		%La fonction $f$ est définie et continue sur $\R^2$ (composée d'un polynome et de la valeur absolue). Elle est à valeurs dans $[0,+\infty[$ car le polynôme s'annule (cf question précédente) et n'est pas borné sur $\R^2$.
%\medskip

	\item Dessiner l'ensemble $L = \{(x,y) \in \R^2 | f(x,y)> 1/2\}$.
%\medskip

%Il faut séparer les cas $N$ et $N^c$. Ainsi, dans $N$ 
%\[
	%f(x,y) = -4(x - 1)^2 - 9(y + 2)^2  +1
%\]
%et $L_N = \{(x,y) \in \R^2 | 4(x - 1)^2 + 9(y - 2)^2  <1/2 \}$. Et dans $N^c$ on a 
%\[
	%f(x,y) = 4(x - 1)^2 + 9(y + 2)^2  -1
%\]
%et $L_{N^c}= \{(x,y) \in \R^2 | 4(x - 1)^2 + 9(y - 2)^2  > 3/2 \}$. L'ensemble recherché est $L = L_N \cup L_{N^c}$ et est le complémentaire d'une couronne ellipsoïdale centrée en $(1,-2)$.
	%\begin{center}
			%\begin{tikzpicture}[scale=1]
				%\begin{axis}[xlabel=$x$,ylabel=$y$, view={0}{90}, axis equal]%,xtick=\empty,ytick=\empty,ztick=\empty ]
					%\addplot3[name path=poly,samples=100,contour gnuplot={levels={.5},labels=false}] gnuplot {(-4*(x - 1)**2 - 9*(y + 2)**2  +1)};
					%\addplot3[name path=poly2,samples=100,contour gnuplot={levels={.5},labels=false}] gnuplot {-(-4*(x - 1)**2 - 9*(y + 2)**2  +1)};
					 %%\fill[blue,use path=poly] (axis cs:.5,-2.5) rectangle (axis cs:1,1);
				%\end{axis}
			%\end{tikzpicture}
		%\end{center}

%\bigskip

	\item Sur quel ensemble $f$ est-elle $\mathcal C^\infty$ ? Justifier la réponse.
	
		%\bigskip
		%Soit  $Z =\{ (x,y) , 4x^2 + 9y^2 - 8x + 36y + 39= 0 \}$. La fonction $f$ est $\mathcal C^\infty$ sur $Z$ (car c'est la composée de 2 fonctions régulières). Sur $Z$, la fonction $f$ n'est pas différentiable (point anguleux dû à la non dérivabilité de la fonction valeur absolue en $0$). Cette dernière affirmation mériterait une étude un peu plus poussée\ldots
		%\bigskip

	\item Donner les points de minimum de $f$ sur $\R^2$. Indiquer, en justifiant, la nature de ces points (minimum global ou local).
		\textit{Indication: cette question se traitera sans calcul}
	
		%\bigskip
	%On a vu à la question 2 que $f$ est à valeurs positives. Sur $Z$ elle s'annule. L'ensemble des points de $Z$ sont des minima globaux.
		%\bigskip
	\item Calculer le gradient et la Hessienne de $f$ en les points de $\R^2$ pour lesquels ces quantités sont bien définies.
		%\bigskip

	%Attention: bien séparer les cas $N$ et $N^c\setminus Z$ (car $f$ n'est pas différentiable en $Z$):
	%\[
		%\nabla f(x,y) = \begin{cases}
		%(-8x+8,-18y-36)  \text{ si $(x,y) \in N$}\\
		%(8x-8,18y +36)	 \text{ si $(x,y) \in N^c\setminus Z$}
		%\end{cases}
	%\]
	%et
	%\[
		%\operatorname{Hess}_f (x,y) = \begin{cases}
			%\begin{pmatrix}
				 %-8 & 0 \\ 0 & -18
			 %\end{pmatrix} \text{ si $(x,y) \in N$}\\
			 %\begin{pmatrix}
				 %8 & 0 \\ 0 & 18
			 %\end{pmatrix}\text{ si $(x,y) \in N^c\setminus Z$}
		%\end{cases}
	%\]

		%\bigskip

	\item Déterminer alors le(s) point(s) critique(s) de $f$ donner leur nature (minimum/maximum, local/global, point selle,\ldots).

		%\bigskip
%On a $\nabla f (x,y) = (0,0)$ si et seulement si $(x,y) =(1,-2)$. La Hessienne en $(1,-2)\in N$ est definie négative et $(1,-2) $ est un maximum local.
		%\bigskip
	%\item Tracer qualitativement le graphe de $f$.

	%\begin{center}
			%\begin{tikzpicture}[scale=1]
				%\begin{axis}[xlabel=$x$,ylabel=$y$, view={10}{35}]%,xtick=\empty,ytick=\empty,ztick=\empty ]
					%\addplot3[surf,opacity=.7,samples=50,domain=0:2,y domain=-2.5:-1.5] gnuplot {abs(-4*(x - 1)**2 - 9*(y + 2)**2  +1)};
				%\end{axis}
			%\end{tikzpicture}
		%\end{center}
%\end{enumerate}





%\bigskip


\end{document}
