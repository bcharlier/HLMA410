\documentclass[a4paper]{tp_um}
\makeatletter
%--------------------------------------------------------------------------------

\usepackage[french]{babel}
\usepackage{amsmath}
\usepackage{amsbsy}
\usepackage{amsfonts}
\usepackage{amssymb}
\usepackage{amscd}
\usepackage{amsthm}
\usepackage{mathtools}
\usepackage{eurosym}
\usepackage{nicefrac}

\usepackage{latexsym}
\usepackage[a4paper,hmargin=20mm,vmargin=25mm]{geometry}
\usepackage{dsfont}
\usepackage[utf8]{inputenc}
\usepackage[T1]{fontenc}
\usepackage{lmodern}

\usepackage{multicol}
\usepackage[inline]{enumitem}
\setlist{nosep}
\setlist[itemize,1]{,label=$-$}


\newenvironment{modenumerate}
  {\enumerate\setupmodenumerate}
  {\endenumerate}

\newif\ifmoditem
\newcommand{\setupmodenumerate}{%
  \global\moditemfalse
  \let\origmakelabel\makelabel
  \def\moditem##1{\global\moditemtrue\def\mesymbol{##1}\item}%
  \def\makelabel##1{%
    \origmakelabel{##1\ifmoditem\rlap{\mesymbol}\fi\enspace}%
    \global\moditemfalse}%
}


\usepackage{sectsty}
%\sectionfont{}
\allsectionsfont{\color{astral}\normalfont\sffamily\bfseries\normalsize}

\usepackage{graphicx}
\usepackage{tikz}
\usetikzlibrary{babel}
\usepackage{tikz,tkz-tab}

\usepackage[babel=true, kerning=true]{microtype}


\usepackage{pgfplots}
\usepgfplotslibrary{fillbetween}
\pgfplotsset{compat=newest}
\usepgfplotslibrary{external} 
\tikzexternalize[prefix=./output_latex/]
%\DeclareSymbolFont{RalphSmithFonts}{U}{rsfs}{m}{n}
%\DeclareSymbolFontAlphabet{\mathscr}{RalphSmithFonts}
%\def\mathcal#1{{\mathscr #1}}



\providecommand{\abs}[1]{\left|#1\right|}
\providecommand{\norm}[1]{\left\Vert#1\right\Vert}
\providecommand{\U}{\mathcal{U}}
\providecommand{\R}{\mathbb{R}}
\providecommand{\Cc}{\mathcal{C}}
\providecommand{\reg}[1]{\mathcal{C}^{#1}}
\providecommand{\1}{\mathds{1}}
\providecommand{\N}{\mathbb{N}}
\providecommand{\Z}{\mathbb{Z}}
\providecommand{\p}{\partial}
\providecommand{\one}{\mathds{1}}
\providecommand{\E}{\mathbb{E}}\providecommand{\V}{\mathbb{V}}
\renewcommand{\P}{\mathbb{P}}


%Operateur
\providecommand{\abs}[1]{\left\lvert#1\right\rvert}
\providecommand{\sabs}[1]{\lvert#1\rvert}
\providecommand{\babs}[1]{\bigg\lvert#1\bigg\rvert}
\providecommand{\norm}[1]{\left\lVert#1\right\rVert}
\providecommand{\bnorm}[1]{\bigg\lVert#1\bigg\rVert}
\providecommand{\snorm}[1]{\lVert#1\rVert}
\providecommand{\prs}[1]{\left\langle #1\right\rangle}
\providecommand{\sprs}[1]{\langle #1\rangle}
\providecommand{\bprs}[1]{\bigg\langle #1\bigg\rangle}

\DeclareMathOperator{\deet}{Det}
\DeclareMathOperator{\hess}{Hess}
\DeclareMathOperator{\jac}{Jac}


\newcommand\rst[2]{{#1}_{\restriction_{#2}}}



% generate breakable white space allowing students to write notes.

\usepackage[framemethod=tikz]{mdframed}

\mdfdefinestyle{response}{
	leftmargin=.01\textwidth,
	rightmargin=.01\textwidth,
	linewidth=1pt
	hidealllines=false,
	leftline=true,
	rightline=true,topline=true,bottomline=true,
	skipabove=0pt,
	%innertopmargin=-5pt,
	%innerbottommargin=2pt,
	linecolor=black,
	innerrightmargin=0pt,
	}



\newcommand*{\DivideLengths}[2]{%
  \strip@pt\dimexpr\number\numexpr\number\dimexpr#1\relax*65536/\number\dimexpr#2\relax\relax sp\relax
}

\providecommand{\rep}[1]{$ $ \newline \begin{mdframed}[style=response]  
	
	\vspace*{\DivideLengths{#1}{3cm}cm}
	\pagebreak[1]	
	\vspace*{\DivideLengths{#1}{3cm}cm}
	\pagebreak[1]		
	\vspace*{\DivideLengths{#1}{3cm}cm}   \end{mdframed}}

\providecommand{\blanc}[1]{$ $ \newline 
	
	\vspace*{\DivideLengths{#1}{3cm}cm}
	\pagebreak[1]	
	\vspace*{\DivideLengths{#1}{3cm}cm}
	\pagebreak[3]		
	\vspace*{\DivideLengths{#1}{3cm}cm}}

\usepackage{ifthen}

\newcommand{\eno}[1]{%
	\ifthenelse{\equal{\version}{eno}}{#1}{}%
}
\newcommand{\cor}[1]{%
        \ifthenelse{\equal{\version}{cor}}{
\medskip 

{\small \color{gray} #1}

\medskip 
}{}
}

%------------------------------------------------------------------------------
%\DeclareUnicodeCharacter{00A0}{~}
\makeatother


%\makeatletter
%--------------------------------------------------------------------------------

\usepackage[frenchb]{babel}

\usepackage{amsmath}
\usepackage{amsbsy}
\usepackage{amsfonts}
\usepackage{amssymb}
\usepackage{amscd}
\usepackage{amsthm}
\usepackage{mathtools}
\usepackage{eurosym}
\usepackage{nicefrac}

\usepackage{latexsym}
\usepackage[a4paper,hmargin=20mm,vmargin=25mm]{geometry}
\usepackage{dsfont}
\usepackage[utf8]{inputenc}
\usepackage[T1]{fontenc}

\usepackage{multicol}
\usepackage[inline]{enumitem}
%\setlist{nosep}
\setlist[itemize,1]{,label=$-$}

\usepackage{sectsty}
%\sectionfont{}
\allsectionsfont{\normalfont\sffamily\bfseries\normalsize}

\usepackage{graphicx}
\usepackage{tikz}

\usepackage{pgfplots}
\usepgfplotslibrary{fillbetween}
\pgfplotsset{compat=newest}
%\usepgfplotslibrary{external} 
%\tikzexternalize[prefix=./output_latex/]
%\DeclareSymbolFont{RalphSmithFonts}{U}{rsfs}{m}{n}
%\DeclareSymbolFontAlphabet{\mathscr}{RalphSmithFonts}
%\def\mathcal#1{{\mathscr #1}}

\newcounter{zut}
\setcounter{zut}{1}
\newcommand{\exo}[1]{\noindent {\sffamily\bfseries Exercice~\thezut. #1} \
		   \addtocounter{zut}{1}}



\providecommand{\abs}[1]{\left|#1\right|}
\providecommand{\norm}[1]{\left\Vert#1\right\Vert}
\providecommand{\U}{\mathcal{U}}
\providecommand{\R}{\mathbb{R}}
\providecommand{\Cc}{\mathcal{C}}
\providecommand{\reg}[1]{\mathcal{C}^{#1}}
\providecommand{\1}{\mathds{1}}
\providecommand{\N}{\mathbb{N}}
\providecommand{\Z}{\mathbb{Z}}
\providecommand{\E}{\mathbb{E}}
\providecommand{\p}{\partial}
\providecommand{\one}{\mathds{1}}
\renewcommand{\P}{\mathbb{P}}


%Operateur
\providecommand{\abs}[1]{\left\lvert#1\right\rvert}
\providecommand{\sabs}[1]{\lvert#1\rvert}
\providecommand{\babs}[1]{\bigg\lvert#1\bigg\rvert}
\providecommand{\norm}[1]{\left\lVert#1\right\rVert}
\providecommand{\bnorm}[1]{\bigg\lVert#1\bigg\rVert}
\providecommand{\snorm}[1]{\lVert#1\rVert}
\providecommand{\prs}[1]{\left\langle #1\right\rangle}
\providecommand{\sprs}[1]{\langle #1\rangle}
\providecommand{\bprs}[1]{\bigg\langle #1\bigg\rangle}

\DeclareMathOperator{\deet}{Det}
\DeclareMathOperator{\vol}{Vol}
\DeclareMathOperator{\aire}{Aire}
\DeclareMathOperator{\hess}{Hess}
\DeclareMathOperator{\var}{Var}

%------------------------------------------------------------------------------
\DeclareUnicodeCharacter{00A0}{~}
\makeatother


\ue{HLMA410}

%-----------------------------------------------------------------------------

\title{\large \sffamily\bfseries Contrôle continu 3}

\begin{document}

\maketitle
\textit{Durée 1h10. Les documents, la calculatrice, les téléphones portables, tablettes, ordinateurs ne sont pas autorisés. La qualité de la rédaction sera prise en compte.} 

\bigskip
\bigskip

\exo{} On définit $f:\R^2\setminus\left\{ (0,0) \right\} \to \R$ par 
\[
    f(x,y) = \frac{x^2}{(x^2 + y^2)^{3/4} }
\]
\begin{enumerate}
    \item Déterminer la limite de $f$ en $(0,0)$.
        %\blanc{4cm}
        
        \medskip

        On a $\abs{f(x,y)} \leq (x^2 + y^2) /(x^2 + y^2)^{3/4} = (x^2 + y^2)^{1/4} \xrightarrow[(x,y)\to(0,0)]{} 0$. Ainsi $\lim_{(x,y)\to(0,0)}f(x,y) = 0$.
        
        \medskip

    \item En déduire que l'on peut prolonger $f$ en une fonction $\tilde f$ continue sur tout $\R^2$.
        %\blanc{4cm}
        
        \medskip

        Comme $f$ est clairement continue (c'est un quotient de fonctions continues dont le dénominateur ne s'annule pas) sur $\R^2\setminus\left\{ (0,0) \right\}$, il suffit de poser $\tilde f(x,y)= \begin{cases}
            f(x,y) &\text{ si } (y,x) \neq (0,0) \\
            0 &\text{ si } (x,y) = (0,0)
        \end{cases}$.
        
        \medskip

    \item \'Etudier l'existence des dérivées partielles de $\tilde f$. Les calculer lorsqu'elles existent.
        %\blanc{13cm}
        
        \medskip

        Sur $\R^2\setminus\left\{ (0,0) \right\} $ la fonction $\tilde f$ admet les dérivées partielles suivantes : 
        \[
            \frac{\partial \tilde f}{\partial x}(x,y) = {{2\,x}\over{\left(y^2+x^2\right)^{{{3}\over{4}}}}}-{{3\,x^3}\over{
 2\,\left(y^2+x^2\right)^{{{7}\over{4}}}}}   \quad \text{ et }  \frac{\partial \tilde f}{\partial y}(x,y)  = -{{3\,x^2\,y}\over{2\,\left(y^2+x^2\right)^{{{7}\over{4}}}}}
        \]
        En $(x,y) = (0,0)$ la fonction partielle $y \mapsto f(0,y) = 0$  et il vient $\frac{\partial \tilde f}{\partial y}(0,0) = 0$. De plus 
        \[
            \frac{ \tilde f(x,0) - \tilde f(0,0) }{x - 0} = \abs{x}^{-1/2}, 
        \]
ce taux d'accroissement tend vers $+\infty$ si $x$ tend vers $0$, et donc $\tilde f$ n'admet pas de dérivée partielle par rapport à la première variable en $(0,0)$.
        
        \medskip

    \item Sur quel domaine la fonction $\tilde f$ est-elle différentiable ?
        %\blanc{5cm}

        \medskip
        
        Sur $\R^2\setminus\left\{ (0,0) \right\} $ la fonction $\tilde f$ est $\mathcal C^1$ et est donc différentiable. En l'origine, on a un point singulier: la fonction partielle $x \mapsto \tilde f(x,0) = \abs{x}^{2-3/2} = \abs{x}^{1/2}$ admet un point anguleux en $x=0$:
        \begin{center}
            \begin{tikzpicture}[scale=.5]
                \begin{axis}[,xlabel=$x$,ylabel=$y$]%,xtick=\empty,ytick=\empty,ztick=\empty ]
                    \addplot3[surf,opacity=.7,samples=50] gnuplot {x**2 /(x**2 + y**2)**(0.75) };
                \end{axis}
            \end{tikzpicture}
        \end{center}


\end{enumerate}


\exo{} Soit $a\in\R$, on cherche toutes les fonctions $g:\R^2\to\R$ de classe $\mathcal C^1$ sur $\R^2$ vérifiant 
\begin{equation}\label{eq.par}
    \frac{\partial g}{\partial x} (x,y) - \frac{\partial g}{\partial y}(x,y) = a.
\end{equation}

\begin{enumerate}
    \item On pose $\phi : (u,v)  \to ((u+v)/2,(v-u)/2)$. Montrer que $\phi$ est un $\mathcal C^1$-diffeomorphisme sur $\R^2$ et preciser son inverse. 
        %\blanc{6cm}
        
        \medskip

        La fonction $\phi$ est une application linéaire inversible (d'inverse $\phi^{-1}(x,y) = \left( x-y, x+y \right)$). Les applications $\phi$ et $\phi^{-1}$  sont donc $\mathcal C^1(\R^2)$ et donnent bien des difféomorphismes du plan.

        
        \medskip

    \item Étant donnée une fonction $g:\R^2\to\R$ solution de \eqref{eq.par}, on pose $f = g \circ \phi %\left( \frac{u+v}{2}, \frac{v-u}{2} \right)
        $. Démontrer alors que $\frac{\partial f}{\partial u}(u,v) = \frac a 2$.
        %\blanc{8cm}
        
        \medskip
        On pose $\phi(u,v) = (x(u,v),y(u,v))$ et il faut appliquer la règle de la chaîne: 
        \begin{align*}
            \frac{\partial f}{\partial u}(u,v) &= \frac{\partial g}{\partial x}(x,y) \frac{\partial x}{\partial u} (u,v)  +\frac{\partial g}{\partial y}(x,y) \frac{\partial y}{\partial u} (u,v) \\
            & = \frac{\partial g}{\partial x}(x,y) \frac 12 -\frac{\partial g}{\partial y}(x,y)  \frac 12 \\
            & = \frac 12 \underbrace{\left( \frac{\partial g}{\partial x}(x,y) -\frac{\partial g}{\partial y}(x,y)  \right)}_{\text{utiliser \eqref{eq.par}}} = \frac a2
        \end{align*}

        \medskip

    \item Intégrer l'expression de la question précédente pour en déduire une expression générique de $f$.
        %\blanc{6cm}

        \medskip

        La dérivée partielle de $f$ par rapport à $u$ est une constante: on a donc \[f(u,v) = \frac a2 u + h(v) + b\] où $h:\R\to\R$ est une fonction $\mathcal C^1$ quelconque et $b\in \R$ (on aurait pu ``rentrer'' la constante $b$ dans la fonction $h$\ldots).

        \medskip

    \item En déduire les solutions de \eqref{eq.par}.
        %\blanc{5cm}

\medskip

        On utilise le changement de variable inverse. On a  
        \[
            g(x,y) = f \circ \phi^{-1} (x,y) = \frac a2 (x-y) + h(x+y) + b.
        \]
        qui est bien solution de \eqref{eq.par}. Voici un exemple avec $a=2$, $b=0$ et $h(v) = 3\sin\left( - v^2 /10 \right)$:
        \begin{center}
            \begin{tikzpicture}[scale=.5]
                \begin{axis}[,xlabel=$x$,ylabel=$y$,view={-110}{25}]%,xtick=\empty,ytick=\empty,ztick=\empty ]
                    \addplot3[surf,opacity=.7,samples=50] gnuplot { (x-y) + 3*sin(- .1*( x+y)**2 ) };
                \end{axis}
            \end{tikzpicture}
        \end{center}

\end{enumerate}

\exo{}
Soit $f : \mathbb R^2 \to \mathbb R$, définie par $f(x,y) = \tfrac13 x^3 + 4 xy^2 + x^2 +4y^2$.
\begin{enumerate}
    \item Démontrer que les 4 points critiques de la fonction $f$ sont $(0,0)$, $(-2,0)$, $(-1,-1/2)$ et $(-1,1/2)$.
        %\blanc{8cm}

        \medskip

La fonction $f$ est clairement de classe $\mathcal C^2$ sur $\mathbb R^2$ car polyn\^ome. Pour tout $(x,y) \in \mathbb R^2$ on a
$$
\frac{\partial f}{ \partial x}(x,y) = x^2 + 4y^2 + 2x , \quad
\frac{\partial f}{ \partial y}(x,y) = 8xy + 8y,
$$
et
$$
\frac{\partial^2 f}{ \partial x^2}(x,y) = 2x+2 , \quad
\frac{\partial^2 f}{ \partial x \partial y}(x,y) = 8y , \quad
\frac{\partial^2 f}{ \partial y^2}(x,y) = 8x+8.
$$
Un point $(x,y) \in \mathbb R^2$ est un point critique si et seulement si
$$
\left\{
\begin{aligned}
& x^2 + 4y^2 + 2x = 0 \\
&  y(x + 1)=0
\end{aligned}
\right.
\quad \Leftrightarrow \quad 
\left\{
\begin{aligned}
& x^2 + 4y^2 + 2x = 0 \\
& y=0 \text{ ou } x=-1
\end{aligned}
\right. .
$$
Ainsi si $y=0$ on obtient de la premi\`ere \'equation $x = 0$ ou $x=-2$ ; et si $x=-1$ on obtient $y=-1/2$ ou $y=1/2$. Il y a donc quatre points critiques: $(0,0)$, $(-2,0)$, $(-1,-1/2)$ et $(-1,1/2)$.

        \medskip

    \item Déterminer la nature (maximum, minimum, point selle) de chacun de ces points critiques.
        %\blanc{10cm}

        \medskip

        \begin{itemize}
            \item
                Point $(0,0)$: on a
                $$
                \mathrm{Hess}_f (0,0) = \left(\begin{matrix} 
                        2 & 0 \\
                        0 & 8  
                \end{matrix} \right),
                $$
                et avec les notations du cours on obtient $rt-s^2 = 16 > 0$ et $r=2>0$, donc $(0,0)$ est un minimum local de $f$. 

            \item
                Point $(-2,0)$: on a
                $$
                \mathrm{Hess}_f (-2,0) = \left(\begin{matrix} 
                        -2 & 0 \\
                        0 & -8  
                \end{matrix} \right),
                $$
                ce qui implique $rt-s^2 = 16  > 0$ et $r=-2<0$, donc $(-2,0)$ est un maximum local de $f$. 

            \item
                Point $(-1,-1/2)$: on a
                $$
                \mathrm{Hess}_f (-1,-1/2) = \left(\begin{matrix} 
                        0 & -4 \\
                        -4 & 0  
                \end{matrix} \right),
                $$
                ce qui implique $rt-s^2 = -16  < 0$, donc $(-1,-1/2)$ est un point selle de $f$. 

            \item
                Point $(-1,1/2)$: on a
                $$
                \mathrm{Hess}_f (-1,1/2) = \left(\begin{matrix} 
                        0 & 4 \\
                        4 & 0  
                \end{matrix} \right),
                $$
                ce qui implique $rt-s^2 = -16  < 0$, donc $(-1,1/2)$ est un point selle de $f$. 
        \end{itemize}

\medskip

    \item La fonction $f$ possède-t-elle un maximum global?
        %\blanc{1cm} 
        \medskip

        %C'est une fonction polynomiale (non constante) et elle n'est donc pas bornée. Il n'y a pas d'extrema globaux finis.
        On peut considérer la restriction $y \mapsto f(0,y)  = 4y^2$ qui n'est pas majorée par un nombre fini. La fonction $f$ n'admet donc pas de maximum global.
\end{enumerate}





\end{document}
