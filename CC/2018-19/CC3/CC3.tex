\documentclass[a4paper]{tp_um}
\makeatletter
%--------------------------------------------------------------------------------

\usepackage[french]{babel}
\usepackage{amsmath}
\usepackage{amsbsy}
\usepackage{amsfonts}
\usepackage{amssymb}
\usepackage{amscd}
\usepackage{amsthm}
\usepackage{mathtools}
\usepackage{eurosym}
\usepackage{nicefrac}

\usepackage{latexsym}
\usepackage[a4paper,hmargin=20mm,vmargin=25mm]{geometry}
\usepackage{dsfont}
\usepackage[utf8]{inputenc}
\usepackage[T1]{fontenc}
\usepackage{lmodern}

\usepackage{multicol}
\usepackage[inline]{enumitem}
\setlist{nosep}
\setlist[itemize,1]{,label=$-$}


\newenvironment{modenumerate}
  {\enumerate\setupmodenumerate}
  {\endenumerate}

\newif\ifmoditem
\newcommand{\setupmodenumerate}{%
  \global\moditemfalse
  \let\origmakelabel\makelabel
  \def\moditem##1{\global\moditemtrue\def\mesymbol{##1}\item}%
  \def\makelabel##1{%
    \origmakelabel{##1\ifmoditem\rlap{\mesymbol}\fi\enspace}%
    \global\moditemfalse}%
}


\usepackage{sectsty}
%\sectionfont{}
\allsectionsfont{\color{astral}\normalfont\sffamily\bfseries\normalsize}

\usepackage{graphicx}
\usepackage{tikz}
\usetikzlibrary{babel}
\usepackage{tikz,tkz-tab}

\usepackage[babel=true, kerning=true]{microtype}


\usepackage{pgfplots}
\usepgfplotslibrary{fillbetween}
\pgfplotsset{compat=newest}
\usepgfplotslibrary{external} 
\tikzexternalize[prefix=./output_latex/]
%\DeclareSymbolFont{RalphSmithFonts}{U}{rsfs}{m}{n}
%\DeclareSymbolFontAlphabet{\mathscr}{RalphSmithFonts}
%\def\mathcal#1{{\mathscr #1}}



\providecommand{\abs}[1]{\left|#1\right|}
\providecommand{\norm}[1]{\left\Vert#1\right\Vert}
\providecommand{\U}{\mathcal{U}}
\providecommand{\R}{\mathbb{R}}
\providecommand{\Cc}{\mathcal{C}}
\providecommand{\reg}[1]{\mathcal{C}^{#1}}
\providecommand{\1}{\mathds{1}}
\providecommand{\N}{\mathbb{N}}
\providecommand{\Z}{\mathbb{Z}}
\providecommand{\p}{\partial}
\providecommand{\one}{\mathds{1}}
\providecommand{\E}{\mathbb{E}}\providecommand{\V}{\mathbb{V}}
\renewcommand{\P}{\mathbb{P}}


%Operateur
\providecommand{\abs}[1]{\left\lvert#1\right\rvert}
\providecommand{\sabs}[1]{\lvert#1\rvert}
\providecommand{\babs}[1]{\bigg\lvert#1\bigg\rvert}
\providecommand{\norm}[1]{\left\lVert#1\right\rVert}
\providecommand{\bnorm}[1]{\bigg\lVert#1\bigg\rVert}
\providecommand{\snorm}[1]{\lVert#1\rVert}
\providecommand{\prs}[1]{\left\langle #1\right\rangle}
\providecommand{\sprs}[1]{\langle #1\rangle}
\providecommand{\bprs}[1]{\bigg\langle #1\bigg\rangle}

\DeclareMathOperator{\deet}{Det}
\DeclareMathOperator{\hess}{Hess}
\DeclareMathOperator{\jac}{Jac}


\newcommand\rst[2]{{#1}_{\restriction_{#2}}}



% generate breakable white space allowing students to write notes.

\usepackage[framemethod=tikz]{mdframed}

\mdfdefinestyle{response}{
	leftmargin=.01\textwidth,
	rightmargin=.01\textwidth,
	linewidth=1pt
	hidealllines=false,
	leftline=true,
	rightline=true,topline=true,bottomline=true,
	skipabove=0pt,
	%innertopmargin=-5pt,
	%innerbottommargin=2pt,
	linecolor=black,
	innerrightmargin=0pt,
	}



\newcommand*{\DivideLengths}[2]{%
  \strip@pt\dimexpr\number\numexpr\number\dimexpr#1\relax*65536/\number\dimexpr#2\relax\relax sp\relax
}

\providecommand{\rep}[1]{$ $ \newline \begin{mdframed}[style=response]  
	
	\vspace*{\DivideLengths{#1}{3cm}cm}
	\pagebreak[1]	
	\vspace*{\DivideLengths{#1}{3cm}cm}
	\pagebreak[1]		
	\vspace*{\DivideLengths{#1}{3cm}cm}   \end{mdframed}}

\providecommand{\blanc}[1]{$ $ \newline 
	
	\vspace*{\DivideLengths{#1}{3cm}cm}
	\pagebreak[1]	
	\vspace*{\DivideLengths{#1}{3cm}cm}
	\pagebreak[3]		
	\vspace*{\DivideLengths{#1}{3cm}cm}}

\usepackage{ifthen}

\newcommand{\eno}[1]{%
	\ifthenelse{\equal{\version}{eno}}{#1}{}%
}
\newcommand{\cor}[1]{%
        \ifthenelse{\equal{\version}{cor}}{
\medskip 

{\small \color{gray} #1}

\medskip 
}{}
}

%------------------------------------------------------------------------------
%\DeclareUnicodeCharacter{00A0}{~}
\makeatother


%\makeatletter
%--------------------------------------------------------------------------------

\usepackage[frenchb]{babel}

\usepackage{amsmath}
\usepackage{amsbsy}
\usepackage{amsfonts}
\usepackage{amssymb}
\usepackage{amscd}
\usepackage{amsthm}
\usepackage{mathtools}
\usepackage{eurosym}
\usepackage{nicefrac}

\usepackage{latexsym}
\usepackage[a4paper,hmargin=20mm,vmargin=25mm]{geometry}
\usepackage{dsfont}
\usepackage[utf8]{inputenc}
\usepackage[T1]{fontenc}

\usepackage{multicol}
\usepackage[inline]{enumitem}
%\setlist{nosep}
\setlist[itemize,1]{,label=$-$}

\usepackage{sectsty}
%\sectionfont{}
\allsectionsfont{\normalfont\sffamily\bfseries\normalsize}

\usepackage{graphicx}
\usepackage{tikz}

\usepackage{pgfplots}
\usepgfplotslibrary{fillbetween}
\pgfplotsset{compat=newest}
%\usepgfplotslibrary{external} 
%\tikzexternalize[prefix=./output_latex/]
%\DeclareSymbolFont{RalphSmithFonts}{U}{rsfs}{m}{n}
%\DeclareSymbolFontAlphabet{\mathscr}{RalphSmithFonts}
%\def\mathcal#1{{\mathscr #1}}

\newcounter{zut}
\setcounter{zut}{1}
\newcommand{\exo}[1]{\noindent {\sffamily\bfseries Exercice~\thezut. #1} \
		   \addtocounter{zut}{1}}



\providecommand{\abs}[1]{\left|#1\right|}
\providecommand{\norm}[1]{\left\Vert#1\right\Vert}
\providecommand{\U}{\mathcal{U}}
\providecommand{\R}{\mathbb{R}}
\providecommand{\Cc}{\mathcal{C}}
\providecommand{\reg}[1]{\mathcal{C}^{#1}}
\providecommand{\1}{\mathds{1}}
\providecommand{\N}{\mathbb{N}}
\providecommand{\Z}{\mathbb{Z}}
\providecommand{\E}{\mathbb{E}}
\providecommand{\p}{\partial}
\providecommand{\one}{\mathds{1}}
\renewcommand{\P}{\mathbb{P}}


%Operateur
\providecommand{\abs}[1]{\left\lvert#1\right\rvert}
\providecommand{\sabs}[1]{\lvert#1\rvert}
\providecommand{\babs}[1]{\bigg\lvert#1\bigg\rvert}
\providecommand{\norm}[1]{\left\lVert#1\right\rVert}
\providecommand{\bnorm}[1]{\bigg\lVert#1\bigg\rVert}
\providecommand{\snorm}[1]{\lVert#1\rVert}
\providecommand{\prs}[1]{\left\langle #1\right\rangle}
\providecommand{\sprs}[1]{\langle #1\rangle}
\providecommand{\bprs}[1]{\bigg\langle #1\bigg\rangle}

\DeclareMathOperator{\deet}{Det}
\DeclareMathOperator{\vol}{Vol}
\DeclareMathOperator{\aire}{Aire}
\DeclareMathOperator{\hess}{Hess}
\DeclareMathOperator{\var}{Var}

%------------------------------------------------------------------------------
\DeclareUnicodeCharacter{00A0}{~}
\makeatother



%\def\version{eno}
\def\version{cor}


\ue{HLMA410}

%-----------------------------------------------------------------------------

\title{\large \sffamily\bfseries Contrôle continu 3}

\tikzexternaldisable
\begin{document}

\maketitle
\textit{Durée 1h10. Les documents, la calculatrice, les téléphones portables, tablettes, ordinateurs ne sont pas autorisés. La qualité de la rédaction sera prise en compte.} 

\bigskip
\bigskip


\exo{} Soit la fonction définie par $f(x,y) = \sin x + y^2 -2y +1$.
\begin{enumerate}
    \item Déterminer le domaine de définition de $f$ et %la régularité  de $f$ (est-elle continue ou différentiable ou \ldots?) 
        calculer la différentielle de $f$ en tout point $(x,y)$ de son domaine de définition.

        \cor{
            La fonction $f$ est définie sur $\R^2$. Sa différentielle en $(x,y)$ est l'application linéaire de $\R^2$ dans $\R$ définie par $d_{(x,y)} f (h_1,h_2) =  h_1\cos x +  h_2(2y-2)$. 
        }
        \eno{
            \vspace{5cm}
        }

    \item Calculer les points critiques de $f$.

        \cor{
            Les points critiques sont $(x,y) \in \R^2$ tels que  $\begin{cases}\cos x = 0 \\ 2y-2 =0\end{cases}$. Ils sont donc de la forme $( (k+1/2) \pi, 1)$ et sont donc situés sur la droite verticale $y=1$. 
        }
        \eno{
            \vspace{3cm}
        }
    \item Déterminer la nature des points critiques (minimum, maximum ou point selle).

        \cor{ 
            La Hessienne de $f$ est $\operatorname{Hess} (x,y) = \begin{pmatrix}
                -\sin x & 0 \\ 0 & 2
            \end{pmatrix}$ et $\operatorname{Hess}( (k+1/2) \pi, 1)  = \begin{pmatrix}
            (-1)^{k+1} & 0 \\ 0 & 2
        \end{pmatrix}$.  Par conséquent, si $k$ est impair, le point $((k+1/2)\pi ,1)$ présente un minimum local et, si $k$ est pair, le point $((k+1/2)\pi,1)$ présente un point selle.
        }

        \eno{
            \vspace{8cm}
        }


\end{enumerate}

\exo{}
Soit $ U = ]0,\pi/2[ \times ]0,\pi/2[$ et $ \begin{cases}
        \Phi : U  &\to \R^2\\
        (x,y) &\mapsto (\cos(x+y),\sin(x-y))
    \end{cases}$.

          \begin{enumerate}
              \item Montrer que $U$ est un ouvert de $\R^2$
                  \cor{
                      L'ensemble $U$ est un ouvert car tous ses points sont contenus dans une boule ouverte incluse dans $U$. En effet, soit $a = (a_1, a_2) \in U$ et $r = \min\{ a_1 , a_2, |a_1 -\frac \pi 2|, |a_2 - \frac \pi 2| \}$. On a bien $B_{\|\cdot\|_2}(a, r/2) \subset U$. 
                  }

                  \eno{
                  \vspace{6cm}

                  \vspace*{7cm}
              }

              \item Soit $(c,s)  \in \mathbb R^2$ et $(x,y) \in U$. Montrer que $(c,s) = \Phi(x,y)$ si et seulement si $x = (\arccos(c) + \arcsin(s))/2$ et   $y = (\arccos(c) - \arcsin(s))/2$.

                  \cor{
                  
                  }

        \eno{
            \vspace*{6cm}
        }

              \item Montrer que $V=\Phi(U)$ est un ouvert.

                  \cor{
                      C'est l'image réciproque d'un ouvert par l'application continue $\Phi^{-1}$.

                  }
        \eno{
            \vspace{3cm}
        }
              \item Calculer la jacobienne de $\Phi$ et montrer qu'elle est inversible.
                  \cor{
On a 
                      \[
                          \operatorname{Jac}_\Phi (x,y) = \begin{pmatrix}
                              -\sin(x+y) & -\sin(x+y) \\
                              \cos(x-y)  &  -\cos(x-y)
                          \end{pmatrix}
                      \]
                      Ce qui donne $\det(\operatorname{Jac}_\Phi(x,y)) = 2\sin(x+y) \cos(x-y)$ qui s'annule si et seulement si $x+y=k\pi$ ou $x-y = \frac \pi 2 + k\pi$ pour un $k\in\Z$. Or ces deux dernières équations n'ont pas de solution sur $U$ car $ 0 <x+y<\pi$ et $-\frac \pi 2 <x -y < \frac \pi 2$. 
                  }
        \eno{
            \vspace{6cm}
        }

    \item $\Phi$ realise-t-il un $\mathcal C^1$-diffeomorphisme de $U$ sur $V$ ? Justifier.
                  \cor{
          Théorème d'inversion globale. %Pour l'appliquer il reste à vérifier que $\Phi$ est injective (les questions précédentes sont préciséménent le reste des hypothèses). 
      }
  
        \eno{
            \vspace*{4cm}
        }

  \item Calculer la matrice jacobienne de $\Phi^{-1}$ en $(c,s) = (0,0)$. %(\sqrt 2/2,\sqrt 2/2)$.

          \cor{
              On a $\Phi(\pi/4,\pi/4) = ( 0, 0) $ et  $\operatorname{Jac}_\Phi (\pi/4,\pi/4) = \begin{pmatrix}
                              -1 & -1 \\
                                1  &  - 1
                            \end{pmatrix}$. Ainsi $\operatorname{Jac}_\Phi (\pi/4,\pi/4) \circ \operatorname{Jac^{-1}}_\Phi (0, 0) = Id_2$. Ce qui donne $\operatorname{Jac^{-1}}_\Phi (0, 0) =\begin{pmatrix}
                              -1 & -1 \\
                                1  &  - 1
                            \end{pmatrix}^{-1}  = \frac 1 2 \begin{pmatrix}
                              -1 & 1 \\
                              -1 & -1
                          \end{pmatrix}$.

          }
        \eno{
            \vspace{5cm}
        }
  \end{enumerate}

\exo{}
Soit $B= \{ a \in \R^2,\,  \|a\|_2 \leq 1 \}$. Calculer alors
\[ I  = \iint_{B} \frac{1}{(1+x^2+y^2)} dx dy.\]
        \eno{
            \vfill

            \vspace{6cm}

            \vspace*{6cm}
        }

%\exo{}	
%On considère la fonction $f:\R^2 \to \R$ dont l'expression est 
%\[
%f(x,y) = \sqrt{\frac{x+y}{x-y}}.
%\]

%\begin{enumerate}
    %\item  Déterminer et dessiner le domaine de définition $\mathcal D$ de $f$ ainsi que sa nature (ouvert, fermé).

        %\begin{center}
            %\begin{tikzpicture}\pgfplotsset{compat=1.8}
                %\begin{axis}[height=6cm,width=6cm,enlargelimits=true, axis lines=center, axis on top, xlabel={$x$}, ylabel={$y$}, axis equal, ymin=-10.2,ymax=10.2,xmin=-10.2,xmax=10.2,grid=both, minor tick num=2,]

                    %\cor{

                        %\addplot[samples=5, very thick,red, dashed,  domain = -10:10, samples y =0] ({x},{x});
                        %\addplot[samples=5, very thick,red, domain = -10:10, samples y =0] ({x},{-x});
                    %}

                %\end{axis}
            %\end{tikzpicture}
        %\end{center}
        
    %\cor{
        %La fonction $f$ est bien définie dès lors que $x\neq y$ et $\frac{x+y}{x-y}\geq 0$. Ce qui donne, $x\geq-y$ et $-y >-x$ ou $x \leq -y $ et $ -y <-x$. C'est à dire entre les droites $y = -x$ (incluse) et $y= x$ (excluse). Ainsi, l'origine d'est pas dans le domaine de définition.
    
    %}

%\eno{

    %\vspace{4cm}
%}

    %\item Déterminer l'ensemble de points adhérents $\bar{\mathcal D}$ de $\mathcal D$. 

        %\cor{
            %On a $\bar{\mathcal D} = \mathcal D \cup \left\{ y=x \right\}$.
        %}
%\eno{

    %\vspace{4cm}
%}


    %\item \'Etudier la limite de $f$ en $(0,0)$. %$A = \{ (x,y) \in \mathbb R^2 \mid  x \ge 0 \text{ et } -x \le y \le x \}$.

        %\cor{%La fonction $f$ est clairement continue sur son domaine de définition. On se concentre donc sur la droite $x=y$ qui est sur le bord du domaine $A$.
            %La fonction $f$ n'admet pas de limite en l'origine. Prendre par exemple, $u_n=(n^{-1}, n^{-1} + n^{-2})$ qui donne  $f(u_n) = n \sqrt{2n^{-1} + n^{-2}} \to +\infty$ et $v_n=(n^{-1}, -n^{-1})$ qui donne $f(v_n) = 0$. 
        
        %}

%\exo{(Différentielle)} Pour $(x,y,z)\in \R^3$ on note $f(x,y,z) = (\cos(xy) \cos(z),(1+x^2)^{yz} )$. Montrer que la fonction $f$ est différentiable au point $(a,1,0)$ pour tout $a$ de $\R$ puis calculer sa différentielle en ce point.

%\cor{
%}



\exo{} Calculer $\iint_D y dxdy$ où $D$ est le domaine de $\R^2$ dessiné ci-contre
\begin{center}
    \begin{tikzpicture}\pgfplotsset{compat=1.8}
        \draw[red, fill=red!50, thick] (-3,2) -- (4,2) -- (1,0) node[below, black] {$1$} -- (-2,0) -- (-3,2) -- cycle ;
        \draw[color=gray, style=dotted] (-5,0) grid[xstep=1cm, ystep=1cm] (6cm,3cm);
        \draw[->] (0,0) -- node[below, black, pos=.3333, left] {$1$} (0,3) node [above] {$y$};
        \draw[->] (-5,0) -- (6,0) node [right] {$x$};
    \end{tikzpicture}
\end{center}

\cor{

    On a 
    $D=\{(x,y)\in\R^2 | -3\leq x \leq -2,-2x\leq 4\leq y\leq 2\} \cup \{ (x,y)\in \R^2 ∣-2\leq x\leq 1, 0\leq y\leq 2\}\cup \{(x,y)\in \R^2 | 1\leq x\leq 4, \frac 2 3x - \frac 2 3 \leq y \leq 2\}$
    ou encore, ce qui est plus simple, $D=\{(x,y)\in\R^2 ∣ 0\leq y\leq 2,-\frac 1 2y-2\leq x\leq \frac 3 2 y+1\}$. Cela donne,
    \[
        \iint_D y dxdy = \int_0^2 \left( \int_{-\frac y 2 - 2 }^{\frac 3 2 y +1} y dy \right) dx = \frac{34}{3}.
    \]
}


\end{enumerate}

\end{document}
