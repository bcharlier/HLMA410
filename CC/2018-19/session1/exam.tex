\documentclass[a4paper]{tp_um}
\makeatletter
%--------------------------------------------------------------------------------

\usepackage[french]{babel}
\usepackage{amsmath}
\usepackage{amsbsy}
\usepackage{amsfonts}
\usepackage{amssymb}
\usepackage{amscd}
\usepackage{amsthm}
\usepackage{mathtools}
\usepackage{eurosym}
\usepackage{nicefrac}

\usepackage{latexsym}
\usepackage[a4paper,hmargin=20mm,vmargin=25mm]{geometry}
\usepackage{dsfont}
\usepackage[utf8]{inputenc}
\usepackage[T1]{fontenc}
\usepackage{lmodern}

\usepackage{multicol}
\usepackage[inline]{enumitem}
\setlist{nosep}
\setlist[itemize,1]{,label=$-$}


\newenvironment{modenumerate}
  {\enumerate\setupmodenumerate}
  {\endenumerate}

\newif\ifmoditem
\newcommand{\setupmodenumerate}{%
  \global\moditemfalse
  \let\origmakelabel\makelabel
  \def\moditem##1{\global\moditemtrue\def\mesymbol{##1}\item}%
  \def\makelabel##1{%
    \origmakelabel{##1\ifmoditem\rlap{\mesymbol}\fi\enspace}%
    \global\moditemfalse}%
}


\usepackage{sectsty}
%\sectionfont{}
\allsectionsfont{\color{astral}\normalfont\sffamily\bfseries\normalsize}

\usepackage{graphicx}
\usepackage{tikz}
\usetikzlibrary{babel}
\usepackage{tikz,tkz-tab}

\usepackage[babel=true, kerning=true]{microtype}


\usepackage{pgfplots}
\usepgfplotslibrary{fillbetween}
\pgfplotsset{compat=newest}
\usepgfplotslibrary{external} 
\tikzexternalize[prefix=./output_latex/]
%\DeclareSymbolFont{RalphSmithFonts}{U}{rsfs}{m}{n}
%\DeclareSymbolFontAlphabet{\mathscr}{RalphSmithFonts}
%\def\mathcal#1{{\mathscr #1}}



\providecommand{\abs}[1]{\left|#1\right|}
\providecommand{\norm}[1]{\left\Vert#1\right\Vert}
\providecommand{\U}{\mathcal{U}}
\providecommand{\R}{\mathbb{R}}
\providecommand{\Cc}{\mathcal{C}}
\providecommand{\reg}[1]{\mathcal{C}^{#1}}
\providecommand{\1}{\mathds{1}}
\providecommand{\N}{\mathbb{N}}
\providecommand{\Z}{\mathbb{Z}}
\providecommand{\p}{\partial}
\providecommand{\one}{\mathds{1}}
\providecommand{\E}{\mathbb{E}}\providecommand{\V}{\mathbb{V}}
\renewcommand{\P}{\mathbb{P}}


%Operateur
\providecommand{\abs}[1]{\left\lvert#1\right\rvert}
\providecommand{\sabs}[1]{\lvert#1\rvert}
\providecommand{\babs}[1]{\bigg\lvert#1\bigg\rvert}
\providecommand{\norm}[1]{\left\lVert#1\right\rVert}
\providecommand{\bnorm}[1]{\bigg\lVert#1\bigg\rVert}
\providecommand{\snorm}[1]{\lVert#1\rVert}
\providecommand{\prs}[1]{\left\langle #1\right\rangle}
\providecommand{\sprs}[1]{\langle #1\rangle}
\providecommand{\bprs}[1]{\bigg\langle #1\bigg\rangle}

\DeclareMathOperator{\deet}{Det}
\DeclareMathOperator{\hess}{Hess}
\DeclareMathOperator{\jac}{Jac}


\newcommand\rst[2]{{#1}_{\restriction_{#2}}}



% generate breakable white space allowing students to write notes.

\usepackage[framemethod=tikz]{mdframed}

\mdfdefinestyle{response}{
	leftmargin=.01\textwidth,
	rightmargin=.01\textwidth,
	linewidth=1pt
	hidealllines=false,
	leftline=true,
	rightline=true,topline=true,bottomline=true,
	skipabove=0pt,
	%innertopmargin=-5pt,
	%innerbottommargin=2pt,
	linecolor=black,
	innerrightmargin=0pt,
	}



\newcommand*{\DivideLengths}[2]{%
  \strip@pt\dimexpr\number\numexpr\number\dimexpr#1\relax*65536/\number\dimexpr#2\relax\relax sp\relax
}

\providecommand{\rep}[1]{$ $ \newline \begin{mdframed}[style=response]  
	
	\vspace*{\DivideLengths{#1}{3cm}cm}
	\pagebreak[1]	
	\vspace*{\DivideLengths{#1}{3cm}cm}
	\pagebreak[1]		
	\vspace*{\DivideLengths{#1}{3cm}cm}   \end{mdframed}}

\providecommand{\blanc}[1]{$ $ \newline 
	
	\vspace*{\DivideLengths{#1}{3cm}cm}
	\pagebreak[1]	
	\vspace*{\DivideLengths{#1}{3cm}cm}
	\pagebreak[3]		
	\vspace*{\DivideLengths{#1}{3cm}cm}}

\usepackage{ifthen}

\newcommand{\eno}[1]{%
	\ifthenelse{\equal{\version}{eno}}{#1}{}%
}
\newcommand{\cor}[1]{%
        \ifthenelse{\equal{\version}{cor}}{
\medskip 

{\small \color{gray} #1}

\medskip 
}{}
}

%------------------------------------------------------------------------------
%\DeclareUnicodeCharacter{00A0}{~}
\makeatother



\usepackage{lastpage}
\cfoot{\thepage\ sur \pageref{LastPage}}

%\def\version{eno}
\def\version{cor}
\ue{HLMA410}

%-----------------------------------------------------------------------------

\title{\large \sffamily\bfseries Examen}

\begin{document}

\maketitle
\textit{Durée 3h00. Les documents, la calculatrice, les téléphones portables, tablettes, ordinateurs ne sont pas autorisés. Les exercices sont indépendants. La qualité de la rédaction sera prise en compte.} 

\bigskip
\bigskip


\section{Topologie et formes quadratiques}

\exo{} Soient $a, b \in \R$. On pose $N_{a,b}: \R^2 \to \R$ définie par \[N_{a,b}(x,y) = a \abs x + b\abs y.\]
\begin{enumerate}
    \item Démontrer que si $N_{a,b}$ est une norme sur $\R^2$ alors $a,b >0$. 
    \item Réciproquement, démontrer que si $a,b >0$ alors $N_{a,b}$ est une norme.
    \item Dessiner les lignes de niveau $0.5$, $1$ et $2$ de $N_{2,1}$.
    %\item L'application $N_{2,1}$ est-elle différentiable sur son domaine de définition?
\end{enumerate}

\exo{} Soit $q:(x,y,z) \mapsto x^2 -2xy + 2y^2 + 4yz + 8 z^2$ une forme quadratique définie sur $\R^3$.
\begin{enumerate}
    \item Écrire $q$ sous forme de somme de carrés d'applications linéaires indépendantes.
\cor{
    On a $q(x,y,z) = (x-y)^2 + (y+2z)^2 + 4 z^2 $.
}
    \item Donner la forme bilinéaire symétrique $\varphi$ associée à $q$.

        \cor{
            La fonction $\varphi: \R^3 \times\R^3 \to \R$ est définie par $\varphi( (x_1, y_1, z_1), (x_2,y_2,z_2)) = x_1x_2 - x_1y_2 - y_1x_2 + 2 y_1y_2 + 2y_1z_2 + 2y_2z_1 + 8 z_1z_2= (x_1-y_1)(x_2-y_2) + (y_1+2z_1)(y_2+2z_2) + 4 z_1z_2$ (on a utilisé la forme factorisée de la question précédente pour la seconde égalité).
        }
    \item La fonction $\varphi$ de la question précédente définit-elle un produit scalaire sur $\R^3$? Justifier.
        \cor{
        Oui, $\varphi$ est positive (d'après la question 1, c'est bien une somme de carrée) et est définie car $q(x,y,z) = 0$ si et seulement si $x=y=z=0$ (les formes linéaires étant linéairement indépendantes par construction).}
\end{enumerate}

\section{Courbes paramétrées}

\exo{} Soit $\Gamma(t) = (\cos t, \sin (t/3))$ pour $t\in \left[0, 3\pi/2  \right]$. 

\begin{enumerate}
    \item Donner le tableau de variation de $\Gamma$.
        \cor{
            On a de manière immédiate $x'(t) =-\sin t$ et $y'(t) =\frac 1 3 \cos t/3$. Ce qui donne

            \begin{center}
                \begin{tikzpicture}
                    \tkzTabInit{$t$ / 1 , $x'(t)$ / 1, $x(t)$ / 1.5, $y'(t)$ / 1, $y(t)$ / 1.5}{$0$, $\pi$, $3\pi/2$}
                    \tkzTabLine{0, -, 0, +, }
                    \tkzTabVar{+/ 1, -/ $-1$, +/ 0}
                    \tkzTabLine{,+ ,  ,+,0 }
                    \tkzTabVar{-/ $0$, R/ , +/ $1$}
                    \tkzTabIma{1}{3}{2}{$\frac{\sqrt 3}{2}$}
                \end{tikzpicture}
            \end{center}
            
        }

    \item Déterminer les points critiques de  $\Gamma$.

        \cor{
         Il n'y en a pas.
        }

    \item Déterminer les points de  $\Gamma$ qui admettent une tangente horizontale ou une tangente verticale.
        \cor{
            Les tangentes verticales sont données par les solutions de $x'(t) = 0$. On est alors en $\Gamma(0) = (1,0) $ et $\Gamma(\pi) = (-1, 3\pi/2)$. 

            Les tangentes horizontales sont données par les solutions de $y'(t) = 0$. On est alors en $\Gamma(3\pi/2) = (0,1)$.
        }


    \item Déterminer le(s) point(s) de $\Gamma$ qui intersecte(nt) l'axe $Oy$.
        \cor{
            En posant $\Gamma(t) = (x(t), y(t))$ on cherche les solutions éventuels de $x(t) = 0$. Il y en a deux dans l'intervalle de définition: pour $t=3\pi/2$ on a $\Gamma(t)=(0,1)$  et $t=\pi/2$ on a $\Gamma=(0,1/2)$.
        }
    \item Tracer la courbe.

\cor{

        \begin{center}
            \begin{tikzpicture}\pgfplotsset{compat=1.8}
                \begin{axis}[height=6cm,width=6cm, axis lines=center, axis on top, xlabel={$x$}, ylabel={$y$}, axis equal,grid=both, minor tick num=2,]
                    \addplot[samples=200, very thick,red,  domain = 0:3*pi/2] ({cos(deg(x))},{sin(deg(x)/3)});
                \end{axis}
            \end{tikzpicture}
        \end{center}
    }
\end{enumerate}

\section{Calcul différentiel}


%\exo[]

\exo{} On considère la fonction $f$ définie sur $\R^2$ par 
\[
f(x,y) =
    \begin{cases}
           0 & \text{ si } x=y=0 \\
           xy \frac{x^2 -y^2}{x^2 + y^2}  & \text{sinon}
    \end{cases}
\]
\begin{enumerate}
    \item La fonction $f$ est-elle continue sur $\R^2$?

        \cor{La fonction $f$ est clairement continue en dehors de l'origine comme quotient de polynôme. Il s'agit donc de vérifier que $\lim_{(x,y)\to (0,0)} f(x,y) = 0$. En effet, on a
        \[
            \abs{f(x,y)} \leq \frac{x^2 + y^2 }{x^2 + y^2}\abs{xy} \leq \frac 1 2 (x^2 + y^2) \xrightarrow[(x,y) \to (0,0)]{} 0.
    \]
    Et $f$ est bien continue sur $\R^2$.
        }
    \item La fonction $f$ est-elle $\mathcal C^1(\R^2)$?
        \cor{La fonction $f$ admet des dérivées partielles en dehors de l'origine comme quotient de polynôme. On calcul les dérivées partielles pour $(x,y) \neq (0,0)$:
            \begin{align*}
                \frac{\partial f}{\partial x} (x,y) = y \frac{x^4 - y^4 + 4 x^2y^2}{(x^2+y^2)^2} \quad \text{ et } \quad 
                \frac{\partial f}{\partial y} (x,y) = x \frac{x^4 - y^4 - 4 x^2y^2}{(x^2+y^2)^2}
            \end{align*}
            En $(0,0)$, on remarque que les fonctions partielles $f(x,0) = f(0,y) = 0$ et on a immédiatement que  $\frac{\partial f}{\partial x} (0,0) = \frac{\partial f}{\partial x} (0,0) = 0$.

            Reste donc à voir si les dérivées partiels sont continues en l'origine. C'est bien le cas car si on passe en coordonnées polaire on a:
            \[
                \abs{\frac{\partial f}{\partial x} (x,y) } \leq 6r
            \]
            qui tends vers $0$ avec $r$. De même pour $\frac{\partial f}{\partial x}$  est aussi continue en l'origine. 
        }

    \item La fonction $f$ est-elle différentiable sur $\R^2$?

        \cor{
            Oui, car elle est $\mathcal C^1$.
    }

    \item Calculer $\frac{\partial^2 f}{\partial x \partial y} (0,0)$ et $\frac{\partial^2 f}{\partial y \partial x} (0,0) $. Quelle conclusion sur $f$ peut-on en tirer?

\cor{
Calculons  les dérivées secondes croisées à l'origine, en revenant à la définition
\[
    \frac{\partial^2 f}{\partial x \partial y} (0,0) = \lim_{x\to 0} \frac{\frac{\partial f}{\partial y} (x,0) -\frac{\partial f}{\partial y} (0,0)}{x} 
    \quad \text{ et } \quad
    \frac{\partial^2 f}{\partial y \partial x} (0,0) = \lim_{y\to 0} \frac{\frac{\partial f}{\partial x} (0,y) -\frac{\partial f}{\partial x} (0,0)}{y} 
\]
Ce qui donne avec $\frac{\partial f}{\partial y} (x,0) = x $ et $\frac{\partial f}{\partial x} (0,y) = -y $ 
\[
    \frac{\partial^2 f}{\partial x \partial y} (0,0) = \lim_{x\to 0} \frac{x}{x} =1
    \quad \text{ et } \quad
    \frac{\partial^2 f}{\partial y \partial x} (0,0) = \lim_{y\to 0} \frac{-y}{y} =-1
\]
les dérivées secondes croisées à l'origine ne sont pas égales et $f'$ n'est donc pas de classe $C^2$ (cf.théorème de Schwarz).
}

\end{enumerate}

\exo{} Calculer les points critiques de la fonction $(x,y) \mapsto x e^{-x^2 - y^2}$ définie sur $\R^2$. En donner leur nature.
\cor{
    On a $\frac{\partial f }{\partial x} (x,y) =(1 - 2x^2) e^{-x^2 - y^2}$ et $\frac{\partial f }{\partial y} (x,y) = -2xy e^{-x^2 - y^2}$ et $Hess_f(x,y) =\begin{pmatrix}
        -2x(3 - 2x^2) & -2y(1-2x^2) \\ -2y(1-2x^2)&  -2x(1-2y^2 )
    \end{pmatrix} e^{-x^2 -y^2}$. Les points critiques sont $(1/\sqrt 2,0)$ (maximum local) et $(-1/\sqrt 2,0)$ (minimum local). %Après le calcule des dérivées partielles secondes on a :

}


\section{Intégration}

\exo{} Calculer l'intégrale $\iint_D xy dxdy$ où $D$ est le domaine du plan qui est l'intersection du disque de centre $(0,1)$ et de rayon 1 et du disque de centre $(1,0)$ et de rayon 1.
% exo 26 de MULTIPLE.pdf 
\cor{
    Le domaine $D$ est symétrique par rapport à la première bissectrice. Sur $D$, on a $f(y, x) =f(x, y)$. On a donc 
    \[
    I=\iint_D f(x, y)dx dy = 2\iint_{D_1} f(x, y)dx dy
\]
où $D_1$ est la partie du domaine $D$ située sous la première bissectrice. L'équation du cercle de centre $(0,1)$ est $x^2+ (y-1)^2= 1$ ou encore $x^2+y^2-2y= 0$. La partie inférieure du cercle a donc pour équation $x=\sqrt{2y-y^2}$. Lorsque $y$ est fixé entre $0$ et $1$, le nombre $x$ varie de $y$ à $\sqrt{2y-y^2}$ et donc
    \[
        (I_x)_1(y) = \int^{\sqrt{ 2y-y^2}}_y xy dx= \left[y\frac{x^2}{2}\right]^{x=\sqrt{2y-y2}}_{x=y} = y^2-y^3
    \]
Puis
\[
    I= 2\int_0^1 (I_x)_1(y)dy= 2\int_0^1 (y^2-y^3)dy= 2\left[\frac{y^3}{3} - \frac{y^4}{4} \right]_{y=0}^1 = 2(\frac 1 3 - \frac 1 4)= \frac 1 6
\]
}


\exo{} Soit $h_1, h_2 \in \R$ tels que $h_1 < h_2 $ et 
\[
    V(h_1,h_2) = \left\{ (x,y,z) \in \R^3 \, |\, h_1 \leq z \leq h_2 \text{ et } x^2 + y^2 \leq 2 z^2 \right\}.
\]
\begin{enumerate}
    \item Dessiner l'intersection de $V(1,2)$ avec le plan $Oxz$. 
    \item Quel objet géométrique simple représente $V(h_1,h_2)$?
        \cor{
        C'est un cône tronqué.
    }
    \item Calculer le volume de $V(h_1,h_2)$ en fonction de $h_1$ et $h_2$.

        \cor{ En passant en coordonnées cylindrique le cône a pour équation cylindrique $r=\sqrt 2z$. On intègre donc sur le domaine  $D = \{(r, \theta, z) | 0 \leq r\leq \sqrt 2z ,-\pi\leq \theta\leq \pi , h_1\leq z\leq h_2\}$ et le volume demandé est 
            \[
          V(h_1,h_2)=      \iint_D rdrd\theta dz.
            \]

Si $(\theta, z)$ appartient à $D_1= [-\pi, \pi] \times [h_1, h_2]$ on calcule tout d'abord
\[
    I_r(\theta,z) = \int_0^{\sqrt 2z} rdr = \left[\frac{z^2}{2} \right]_0^{\sqrt 2z} =  z^2.
\]
Par suite (les variables se séparent)
\[
    V(h_1,h_2) = \iint_{D_1} I_r(\theta,z) d\theta dz = \int_0^{2\pi} d\theta \int_{h_1}^{h_2} z^2 dz = \frac 2 3 \pi (h_2^3 - h_1^3).
\]
        }
\end{enumerate}




%\exo{}
%Soit $x, y \in\R$.

%\begin{enumerate}
    %\item Rappeler la définition d'une norme.
    %\item On suppose dans cette question que $(E, \snorm{\cdot})$ est un espace vectoriel normé quelconque.
        %\begin{enumerate}
            %\item  Démontrer que, pour tous $x,y \in \R$, on a $\snorm{x} + \snorm{y} \leq \snorm{x+y} + \snorm{x-y}$. 
            %\item  En déduire que $\snorm{x} + \snorm{y} \leq 2 \max\{ \snorm{x+y},  \snorm{x-y} \}$. 
            %\item  Dans le cas où  $E = (\R, \abs{ \cdot})$, trouver un exemple où $\snorm{x} + \snorm{y} = 2 \max\{ \snorm{x+y},  \snorm{x-y} \} $
        %\end{enumerate}
    %\item 
        %\begin{enumerate}
            %\item On suppose désormais que la norme est issue d'un produit scalaire.
            %\item 
        %\end{enumerate}

%\end{enumerate}
%% http://www.bibmath.net/ressources/index.php?action=affiche&quoi=mathspe/feuillesexo/evn&type=fexo


\end{document}
