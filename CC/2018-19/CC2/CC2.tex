\documentclass[a4paper]{tp_um}
\makeatletter
%--------------------------------------------------------------------------------

\usepackage[french]{babel}
\usepackage{amsmath}
\usepackage{amsbsy}
\usepackage{amsfonts}
\usepackage{amssymb}
\usepackage{amscd}
\usepackage{amsthm}
\usepackage{mathtools}
\usepackage{eurosym}
\usepackage{nicefrac}

\usepackage{latexsym}
\usepackage[a4paper,hmargin=20mm,vmargin=25mm]{geometry}
\usepackage{dsfont}
\usepackage[utf8]{inputenc}
\usepackage[T1]{fontenc}
\usepackage{lmodern}

\usepackage{multicol}
\usepackage[inline]{enumitem}
\setlist{nosep}
\setlist[itemize,1]{,label=$-$}


\newenvironment{modenumerate}
  {\enumerate\setupmodenumerate}
  {\endenumerate}

\newif\ifmoditem
\newcommand{\setupmodenumerate}{%
  \global\moditemfalse
  \let\origmakelabel\makelabel
  \def\moditem##1{\global\moditemtrue\def\mesymbol{##1}\item}%
  \def\makelabel##1{%
    \origmakelabel{##1\ifmoditem\rlap{\mesymbol}\fi\enspace}%
    \global\moditemfalse}%
}


\usepackage{sectsty}
%\sectionfont{}
\allsectionsfont{\color{astral}\normalfont\sffamily\bfseries\normalsize}

\usepackage{graphicx}
\usepackage{tikz}
\usetikzlibrary{babel}
\usepackage{tikz,tkz-tab}

\usepackage[babel=true, kerning=true]{microtype}


\usepackage{pgfplots}
\usepgfplotslibrary{fillbetween}
\pgfplotsset{compat=newest}
\usepgfplotslibrary{external} 
\tikzexternalize[prefix=./output_latex/]
%\DeclareSymbolFont{RalphSmithFonts}{U}{rsfs}{m}{n}
%\DeclareSymbolFontAlphabet{\mathscr}{RalphSmithFonts}
%\def\mathcal#1{{\mathscr #1}}



\providecommand{\abs}[1]{\left|#1\right|}
\providecommand{\norm}[1]{\left\Vert#1\right\Vert}
\providecommand{\U}{\mathcal{U}}
\providecommand{\R}{\mathbb{R}}
\providecommand{\Cc}{\mathcal{C}}
\providecommand{\reg}[1]{\mathcal{C}^{#1}}
\providecommand{\1}{\mathds{1}}
\providecommand{\N}{\mathbb{N}}
\providecommand{\Z}{\mathbb{Z}}
\providecommand{\p}{\partial}
\providecommand{\one}{\mathds{1}}
\providecommand{\E}{\mathbb{E}}\providecommand{\V}{\mathbb{V}}
\renewcommand{\P}{\mathbb{P}}


%Operateur
\providecommand{\abs}[1]{\left\lvert#1\right\rvert}
\providecommand{\sabs}[1]{\lvert#1\rvert}
\providecommand{\babs}[1]{\bigg\lvert#1\bigg\rvert}
\providecommand{\norm}[1]{\left\lVert#1\right\rVert}
\providecommand{\bnorm}[1]{\bigg\lVert#1\bigg\rVert}
\providecommand{\snorm}[1]{\lVert#1\rVert}
\providecommand{\prs}[1]{\left\langle #1\right\rangle}
\providecommand{\sprs}[1]{\langle #1\rangle}
\providecommand{\bprs}[1]{\bigg\langle #1\bigg\rangle}

\DeclareMathOperator{\deet}{Det}
\DeclareMathOperator{\hess}{Hess}
\DeclareMathOperator{\jac}{Jac}


\newcommand\rst[2]{{#1}_{\restriction_{#2}}}



% generate breakable white space allowing students to write notes.

\usepackage[framemethod=tikz]{mdframed}

\mdfdefinestyle{response}{
	leftmargin=.01\textwidth,
	rightmargin=.01\textwidth,
	linewidth=1pt
	hidealllines=false,
	leftline=true,
	rightline=true,topline=true,bottomline=true,
	skipabove=0pt,
	%innertopmargin=-5pt,
	%innerbottommargin=2pt,
	linecolor=black,
	innerrightmargin=0pt,
	}



\newcommand*{\DivideLengths}[2]{%
  \strip@pt\dimexpr\number\numexpr\number\dimexpr#1\relax*65536/\number\dimexpr#2\relax\relax sp\relax
}

\providecommand{\rep}[1]{$ $ \newline \begin{mdframed}[style=response]  
	
	\vspace*{\DivideLengths{#1}{3cm}cm}
	\pagebreak[1]	
	\vspace*{\DivideLengths{#1}{3cm}cm}
	\pagebreak[1]		
	\vspace*{\DivideLengths{#1}{3cm}cm}   \end{mdframed}}

\providecommand{\blanc}[1]{$ $ \newline 
	
	\vspace*{\DivideLengths{#1}{3cm}cm}
	\pagebreak[1]	
	\vspace*{\DivideLengths{#1}{3cm}cm}
	\pagebreak[3]		
	\vspace*{\DivideLengths{#1}{3cm}cm}}

\usepackage{ifthen}

\newcommand{\eno}[1]{%
	\ifthenelse{\equal{\version}{eno}}{#1}{}%
}
\newcommand{\cor}[1]{%
        \ifthenelse{\equal{\version}{cor}}{
\medskip 

{\small \color{gray} #1}

\medskip 
}{}
}

%------------------------------------------------------------------------------
%\DeclareUnicodeCharacter{00A0}{~}
\makeatother


%\makeatletter
%--------------------------------------------------------------------------------

\usepackage[frenchb]{babel}

\usepackage{amsmath}
\usepackage{amsbsy}
\usepackage{amsfonts}
\usepackage{amssymb}
\usepackage{amscd}
\usepackage{amsthm}
\usepackage{mathtools}
\usepackage{eurosym}
\usepackage{nicefrac}

\usepackage{latexsym}
\usepackage[a4paper,hmargin=20mm,vmargin=25mm]{geometry}
\usepackage{dsfont}
\usepackage[utf8]{inputenc}
\usepackage[T1]{fontenc}

\usepackage{multicol}
\usepackage[inline]{enumitem}
%\setlist{nosep}
\setlist[itemize,1]{,label=$-$}

\usepackage{sectsty}
%\sectionfont{}
\allsectionsfont{\normalfont\sffamily\bfseries\normalsize}

\usepackage{graphicx}
\usepackage{tikz}

\usepackage{pgfplots}
\usepgfplotslibrary{fillbetween}
\pgfplotsset{compat=newest}
%\usepgfplotslibrary{external} 
%\tikzexternalize[prefix=./output_latex/]
%\DeclareSymbolFont{RalphSmithFonts}{U}{rsfs}{m}{n}
%\DeclareSymbolFontAlphabet{\mathscr}{RalphSmithFonts}
%\def\mathcal#1{{\mathscr #1}}

\newcounter{zut}
\setcounter{zut}{1}
\newcommand{\exo}[1]{\noindent {\sffamily\bfseries Exercice~\thezut. #1} \
		   \addtocounter{zut}{1}}



\providecommand{\abs}[1]{\left|#1\right|}
\providecommand{\norm}[1]{\left\Vert#1\right\Vert}
\providecommand{\U}{\mathcal{U}}
\providecommand{\R}{\mathbb{R}}
\providecommand{\Cc}{\mathcal{C}}
\providecommand{\reg}[1]{\mathcal{C}^{#1}}
\providecommand{\1}{\mathds{1}}
\providecommand{\N}{\mathbb{N}}
\providecommand{\Z}{\mathbb{Z}}
\providecommand{\E}{\mathbb{E}}
\providecommand{\p}{\partial}
\providecommand{\one}{\mathds{1}}
\renewcommand{\P}{\mathbb{P}}


%Operateur
\providecommand{\abs}[1]{\left\lvert#1\right\rvert}
\providecommand{\sabs}[1]{\lvert#1\rvert}
\providecommand{\babs}[1]{\bigg\lvert#1\bigg\rvert}
\providecommand{\norm}[1]{\left\lVert#1\right\rVert}
\providecommand{\bnorm}[1]{\bigg\lVert#1\bigg\rVert}
\providecommand{\snorm}[1]{\lVert#1\rVert}
\providecommand{\prs}[1]{\left\langle #1\right\rangle}
\providecommand{\sprs}[1]{\langle #1\rangle}
\providecommand{\bprs}[1]{\bigg\langle #1\bigg\rangle}

\DeclareMathOperator{\deet}{Det}
\DeclareMathOperator{\vol}{Vol}
\DeclareMathOperator{\aire}{Aire}
\DeclareMathOperator{\hess}{Hess}
\DeclareMathOperator{\var}{Var}

%------------------------------------------------------------------------------
\DeclareUnicodeCharacter{00A0}{~}
\makeatother



%\def\version{eno}
\def\version{cor}


\ue{HLMA410}

%-----------------------------------------------------------------------------

\title{\large \sffamily\bfseries Contrôle continu 2}

\begin{document}

\maketitle
\textit{Durée 1h10. Les documents, la calculatrice, les téléphones portables, tablettes, ordinateurs ne sont pas autorisés. La qualité de la rédaction sera prise en compte.} 

\bigskip
\bigskip

\exo{Formes quadratique} Déterminer la matrice et la  signature des formes quadratiques suivantes:
\begin{enumerate}
    \item $q_1(x,y,z,t)=x^2+3y^2+4z^2+t^2+2xy+xt+yt$

        \cor{On a 
            $q(x,y,z,t)=(x+y+t/2)^2+2y^2+4z^2+3/4t^2$. Cette forme quadratique est positive (et même définie positive, sa signature est $(4,0)$). 
        }
        \eno{
            \vspace*{8cm} \vfill
        }
    \item $q_2(x,y,z) = x^2 + y^2 + 2z(x\cos\alpha+y\sin\alpha)$ où $\alpha\in\R$.

\cor{ 
    
    On trouve 
$q_2(x,y,z) = (x+z\cos\alpha)^2 + (y+z\sin\alpha )^2-z^2$. Cette  forme quadratique n'est pas positive (sa signature est $(2,1)$). 
}

        \eno{
            \vspace*{9cm} \vfill
        }
\end{enumerate}


\exo{} Soit $E$ un espace vectoriel euclidien et $x,y$ deux éléments de $E$. Montrer que $x$ et $y$ sont orthogonaux si et seulement si $\snorm{x+ \lambda y } \geq \snorm{x}$ pour tout $\lambda\in\R$.

        \eno{
            \vspace*{10cm} \vfill
        }

        \cor{Supposons que $x$ et $y$ sont orthogonaux. On choisit $\lambda\in \R$ quelconque et on a $\snorm{x + \lambda y}^2 = \snorm{x}^2 + \lambda^2\snorm{y}^2 + 2\lambda\prs{x,y} = \snorm{x}^2 + \lambda^2\snorm{y}^2 \geq \snorm{x}^2$.
            
        Réciproquement, si $\snorm{x+ \lambda y } - \snorm{x} \geq 0$ pour tout $\lambda\in\R$, il vient que 
        \[
            \lambda(\lambda \snorm{y}^2 +2\prs{x,y}) \geq 0 
        \]
        Dressant le tableau de signes de ce produit, il ne peut être toujours positif que si:
       \begin{itemize}
           \item $2\prs{x,y}+\lambda\snorm{y}$  est toujours nul, c'est-à-dire si $y=0$
           \item $2\prs{x,y}+\lambda\snorm{y}$ ne s'annule qu'en $\lambda=0$, c'est-à-dire si $\prs{x,y}=0$. 
       \end{itemize}
            Dans les deux cas, on trouve bien que $x$ et $y$ sont orthogonaux.
        %Si $\lambda=-1$, on a $\snorm{y}^2 \geq 2\prs{x,y}$ et si $\lambda = 1$
    }


\exo{Lignes de niveau} 
\begin{enumerate}
    \item Donner le domaine de définition de la fonction $f_1(x,y)=(y+x)^2$ et dessiner les lignes de niveau $k$ avec $k=-1$ et $k=1$ 


        \begin{minipage}{.5\linewidth}
            \begin{tikzpicture}\pgfplotsset{compat=1.8}
                \begin{axis}[height=8cm,width=8cm,enlargelimits=true, axis lines=center, axis on top, xlabel={$x$}, ylabel={$y$}, axis equal,
                    ymin=-2.2,ymax=2.2,xmin=-2.2,xmax=2.2,grid=both, minor tick num=2,]

                    \cor{
                        \addplot[samples=5, very thick,red, domain = -3:3, samples y =0] ({x},{-x+1});
                        \addplot[samples=5, very thick,red, domain = -3:3, samples y =0] ({x},{-x-1});
                    }
                \end{axis}  
            \end{tikzpicture}
        \end{minipage}
        \begin{minipage}[]{.49\linewidth}
           \cor{ La fonction étant positive l'ensemble de niveau $-1$ est vide. Pour l'ensemble de niveau 1 on a:
            \[
                (x+y)^2 = 1 \Leftrightarrow x+y = \pm 1
            \]
        C'est donc l'union des 2 droites parallèles dessinées ci contre.}
        \end{minipage}
      %  
        \eno{ 
        \newpage
    }%

    \item Soit $f_2(x,y)=\frac{x^4+y^4}{8-x^2y^2}$.
        \begin{enumerate}
            \item Dessiner le domaine de définition $D$  de $f_2$

                \begin{minipage}{.5\linewidth}
                    \begin{tikzpicture}\pgfplotsset{compat=1.8}
                        \begin{axis}[height=10cm,width=10cm,enlargelimits=true, axis lines=center, axis on top, xlabel={$x$}, ylabel={$y$}, axis equal,
                            ymin=-10.2,ymax=10.2,xmin=-10.2,xmax=10.2,grid=both, minor tick num=1,]

                            \cor{
                                \addplot[samples=500, very thick,red, domain = -10:10, samples y =0] ({x},{sqrt(8)/abs(x)});
                                \addplot[samples=500, very thick,red, domain = -10:10, samples y =0] ({x},{-sqrt(8)/abs(x)});
                            }
                        \end{axis}  
                    \end{tikzpicture}
                \end{minipage}\hfill
                \begin{minipage}[]{.3\linewidth}
                    \cor{
                        Il s'agit de déterminer l'ensemble des points du plan vérifiant $8 - x^2y^2$. 
                        
                        Attention, il faut distinguer 4 cas correspondant aux 4 quadrants (ie supposer succéssivement que $x>0$, $x<0$, $y>0$, $y<0$). Dans tous les cas on trouve une hyperbole. Le domaine de définition de $f_2$ est tout le plan privé des 4 courbes en rouge ci contre.
                    }
                \end{minipage}
        \item Déterminer la nature de $D$ (ouvert, fermé, ni l'un ni l'autre)? Démontrer. 

            \eno{\vspace{8cm}}
            \cor{
                Le domaine de définition de $f_2$ est ouvert. En effet, c'est le complémentaire d'un fermé car $F =\{ (x,y) \in \R^2 | 8 - x^2y^2\}$ est l'image réciproque du singleton $\{8\}$ (qui est un fermé) par l'application continue $(x,y) \mapsto x^2y^2$. On peut aussi le montrer directement en prenant une suite de $F$ qui converge et démonter que la limite est encore dans $F$.
            }
        \item Calculer la ligne de niveau $k=2$. Quel objet géométrique est-ce?

            \cor{La ligne de niveau est l'ensemble des $(x,y)\in\R^2$ tels que $x^4+y^4 + 2x^2y^2 = 16$. Ce qui donne $(x^2 + y^2)^2 = 16$ et implique que $x^2 + y^2 =4$. C'est donc le cercle $C$ de centre $(0,0)$ et de rayon 2 auquel  il faut retirer les points pour lesquels $x^2y^2=8$, points pour lesquels la fonction $f$ n'est pas définie. Les points du cercle vérifiant cette relation vérifient aussi \[x^2+8/x^2=4 \Rightarrow x^4-4x^2+8=0.\] 
            Posons $X=x^2$, alors $X$ vérifie $X^2-4X+8=0$. Le discriminant de cette équation est $16-4\times 8=-16<0$. Ainsi, l'équation n'a pas de solutions dans $\R$. La courbe de niveau recherché est bien le cercle de centre l'origine et de rayon $2$.  }
            
            \eno{
            
            \vspace{8cm}

            \newpage

            \vspace*{5cm}
        }
        \end{enumerate}
\end{enumerate}


\exo{Limites}	
\'Etudier la limite en l'origine de la fonction
\[
f(x,y) = \frac{x \sin(y) - y \sin(x)}{x^2 + y^2}.
\]
        \eno{
            \newpage
        }

        \cor{
            On a, au voisinage de $(0,0)$,
\begin{align*}
    \abs{f(x,y)} & \leq 2\frac{\abs{x (y +o(y^2) ) - y(x + o(x^2)) }}{ x^2 + y^2} \\
    & \leq \frac{ o(x^2) + o(y^2)}{x^2 + y^2} 
     = \frac{ x^2 \varepsilon_1(x) + y^2 \varepsilon_2(y)}{x^2 + y^2} \qquad \text{avec } \lim_{x\to 0}\varepsilon_1(x) = \lim_{y\to 0} \varepsilon_2(y) = 0, \\
    & \leq \frac{(x^2 + y^2) \max\{\varepsilon_1(x) , \varepsilon_2(y)\}  }{x^2 + y^2}\\
& \leq  \max\{\varepsilon_1(x) , \varepsilon_2(y)\} = \snorm{ (\varepsilon_1(x), \varepsilon_2(y)) }_\infty \xrightarrow[(x,y)\to(0,0)]{} 0  \end{align*}

Ainsi, $\lim_{(x,y) \to (0,0)} f(x,y) = 0$.
        }

%\exo{}	
%On considère la fonction $f:\R^2 \to \R$ dont l'expression est 
%\[
%f(x,y) = \sqrt{\frac{x+y}{x-y}}.
%\]

%\begin{enumerate}
    %\item  Déterminer et dessiner le domaine de définition de $f$ ainsi que sa nature (ouvert, fermé).

        %\begin{tikzpicture}\pgfplotsset{compat=1.8}
            %\begin{axis}[height=6cm,width=6cm,enlargelimits=true, axis lines=center, axis on top, xlabel={$x$}, ylabel={$y$}, axis equal, ymin=-10.2,ymax=10.2,xmin=-10.2,xmax=10.2,grid=both, minor tick num=2,]

            %\cor{

                %\addplot[samples=5, very thick,red, dashed,  domain = -10:10, samples y =0] ({x},{x});
                %\addplot[samples=5, very thick,red, domain = -10:10, samples y =0] ({x},{-x});
             %}

            %\end{axis}
        %\end{tikzpicture}
        
    %\cor{
        %La fonction $f$ est bien définie dès lors que $x\neq y$ et $\frac{x+y}{x-y}\geq 0$. Ce qui donne, $x\geq-y$ et $-y >-x$ ou $x \leq -y $ et $ -y <-x$. C'est à dire entre les droites $y = -x$ (incluse) et $y= x$ (excluse). Ainsi, l'origine d'est pas dans le domaine de définition.
    
    %}

%\eno{

    %\vspace{4cm}
%}
    %\item \'Etudier la limite de $f$ en les points de l'ensemble $A = \{ (x,y) \in \mathbb R^2 \mid  x \ge 0 \text{ et } -x \le y \le x \}$.

        %\cor{La fonction $f$ est clairement continue sur son domaine de défintion. On se concentre donc sur la droite $x=y$ qui est sur le bord du domaine $A$.
            %Prendre $u_n=(n^{-1}, n^{-1} + n^{-2})$ et on a  $f(u_n) = n (2n^{-1} + n^{-2} ) \to 2$ et 
        
        %}
%\end{enumerate}

\end{document}
