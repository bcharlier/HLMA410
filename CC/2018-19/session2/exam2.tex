\documentclass[a4paper]{tp_um}
\makeatletter
%--------------------------------------------------------------------------------

\usepackage[french]{babel}
\usepackage{amsmath}
\usepackage{amsbsy}
\usepackage{amsfonts}
\usepackage{amssymb}
\usepackage{amscd}
\usepackage{amsthm}
\usepackage{mathtools}
\usepackage{eurosym}
\usepackage{nicefrac}

\usepackage{latexsym}
\usepackage[a4paper,hmargin=20mm,vmargin=25mm]{geometry}
\usepackage{dsfont}
\usepackage[utf8]{inputenc}
\usepackage[T1]{fontenc}
\usepackage{lmodern}

\usepackage{multicol}
\usepackage[inline]{enumitem}
\setlist{nosep}
\setlist[itemize,1]{,label=$-$}


\newenvironment{modenumerate}
  {\enumerate\setupmodenumerate}
  {\endenumerate}

\newif\ifmoditem
\newcommand{\setupmodenumerate}{%
  \global\moditemfalse
  \let\origmakelabel\makelabel
  \def\moditem##1{\global\moditemtrue\def\mesymbol{##1}\item}%
  \def\makelabel##1{%
    \origmakelabel{##1\ifmoditem\rlap{\mesymbol}\fi\enspace}%
    \global\moditemfalse}%
}


\usepackage{sectsty}
%\sectionfont{}
\allsectionsfont{\color{astral}\normalfont\sffamily\bfseries\normalsize}

\usepackage{graphicx}
\usepackage{tikz}
\usetikzlibrary{babel}
\usepackage{tikz,tkz-tab}

\usepackage[babel=true, kerning=true]{microtype}


\usepackage{pgfplots}
\usepgfplotslibrary{fillbetween}
\pgfplotsset{compat=newest}
\usepgfplotslibrary{external} 
\tikzexternalize[prefix=./output_latex/]
%\DeclareSymbolFont{RalphSmithFonts}{U}{rsfs}{m}{n}
%\DeclareSymbolFontAlphabet{\mathscr}{RalphSmithFonts}
%\def\mathcal#1{{\mathscr #1}}



\providecommand{\abs}[1]{\left|#1\right|}
\providecommand{\norm}[1]{\left\Vert#1\right\Vert}
\providecommand{\U}{\mathcal{U}}
\providecommand{\R}{\mathbb{R}}
\providecommand{\Cc}{\mathcal{C}}
\providecommand{\reg}[1]{\mathcal{C}^{#1}}
\providecommand{\1}{\mathds{1}}
\providecommand{\N}{\mathbb{N}}
\providecommand{\Z}{\mathbb{Z}}
\providecommand{\p}{\partial}
\providecommand{\one}{\mathds{1}}
\providecommand{\E}{\mathbb{E}}\providecommand{\V}{\mathbb{V}}
\renewcommand{\P}{\mathbb{P}}


%Operateur
\providecommand{\abs}[1]{\left\lvert#1\right\rvert}
\providecommand{\sabs}[1]{\lvert#1\rvert}
\providecommand{\babs}[1]{\bigg\lvert#1\bigg\rvert}
\providecommand{\norm}[1]{\left\lVert#1\right\rVert}
\providecommand{\bnorm}[1]{\bigg\lVert#1\bigg\rVert}
\providecommand{\snorm}[1]{\lVert#1\rVert}
\providecommand{\prs}[1]{\left\langle #1\right\rangle}
\providecommand{\sprs}[1]{\langle #1\rangle}
\providecommand{\bprs}[1]{\bigg\langle #1\bigg\rangle}

\DeclareMathOperator{\deet}{Det}
\DeclareMathOperator{\hess}{Hess}
\DeclareMathOperator{\jac}{Jac}


\newcommand\rst[2]{{#1}_{\restriction_{#2}}}



% generate breakable white space allowing students to write notes.

\usepackage[framemethod=tikz]{mdframed}

\mdfdefinestyle{response}{
	leftmargin=.01\textwidth,
	rightmargin=.01\textwidth,
	linewidth=1pt
	hidealllines=false,
	leftline=true,
	rightline=true,topline=true,bottomline=true,
	skipabove=0pt,
	%innertopmargin=-5pt,
	%innerbottommargin=2pt,
	linecolor=black,
	innerrightmargin=0pt,
	}



\newcommand*{\DivideLengths}[2]{%
  \strip@pt\dimexpr\number\numexpr\number\dimexpr#1\relax*65536/\number\dimexpr#2\relax\relax sp\relax
}

\providecommand{\rep}[1]{$ $ \newline \begin{mdframed}[style=response]  
	
	\vspace*{\DivideLengths{#1}{3cm}cm}
	\pagebreak[1]	
	\vspace*{\DivideLengths{#1}{3cm}cm}
	\pagebreak[1]		
	\vspace*{\DivideLengths{#1}{3cm}cm}   \end{mdframed}}

\providecommand{\blanc}[1]{$ $ \newline 
	
	\vspace*{\DivideLengths{#1}{3cm}cm}
	\pagebreak[1]	
	\vspace*{\DivideLengths{#1}{3cm}cm}
	\pagebreak[3]		
	\vspace*{\DivideLengths{#1}{3cm}cm}}

\usepackage{ifthen}

\newcommand{\eno}[1]{%
	\ifthenelse{\equal{\version}{eno}}{#1}{}%
}
\newcommand{\cor}[1]{%
        \ifthenelse{\equal{\version}{cor}}{
\medskip 

{\small \color{gray} #1}

\medskip 
}{}
}

%------------------------------------------------------------------------------
%\DeclareUnicodeCharacter{00A0}{~}
\makeatother



\usepackage{lastpage}
\cfoot{\thepage\ sur \pageref{LastPage}}

%\def\version{cor}
\def\version{eno}
\ue{HLMA410}
\usepackage{tikz,tkz-tab}
%-----------------------------------------------------------------------------

\title{\large \sffamily\bfseries Examen}

\begin{document}

\maketitle
\textit{Durée 3h00. Les documents, la calculatrice, les téléphones portables, tablettes, ordinateurs ne sont pas autorisés. Les exercices sont indépendants. La qualité de la rédaction sera prise en compte.} 

\bigskip
\bigskip


\section{Topologie et formes quadratiques}

\exo{} Soit $(a,b) \in \mathbb R^2$ et $\phi : \mathbb R^2 \times \mathbb R^2 \to \mathbb R$
d\'efinie par
\[
\phi((x_1,x_2), (y_1,y_2)) = x_1y_1 + 4x_1y_2 + b x_2y_1 + a x_2y_2 \quad 
\forall \, ((x_1,x_2),(y_1,y_2)) \in \mathbb R^2 \times \mathbb R^2.
\]
\begin{enumerate}
\item Justifier que $\phi$ est bilin\'eaire. Montrer qu'elle est sym\'etrique si et seulement si $b=4.$
Exprimer alors la matrice associ\'ee. 
\item On suppose $b=4.$ Exprimer une condition sur $a$ pour que $\phi$ soit d\'efinie positive.
\item A quelle condition sur la paire $(a,b)$  l'application $\phi$ d\'efinit-elle un produit scalaire sur $\mathbb R^2.$
\item On pose $b=4$ et $a=16.$ Montrer que l'ensemble 
    \[
        C =\{(x_1,x_2) \in \mathbb R^2 \, | \, \phi((x_1,x_2),(x_1,x_2)) = 1\}
    \]
est un ferm\'e et le dessiner dans un rep\`ere orthonorm\'e. $C$ est-il compact ? 
\end{enumerate}

\section{Courbes paramétrées}

\exo{} Soit $\Gamma(t) = (\sin t, \frac{\sin t}{2 + \cos t})$ pour $t\in \left[0, \pi  \right]$. 

\begin{enumerate}
    \item Donner le tableau de variation de $\Gamma$.
        \cor{
            On a de manière immédiate $x'(t) =\cos t$ et $y'(t) =\frac{2\cos t +1}{(2+ \cos t)^2}$. Ce qui donne

            \begin{center}
                \begin{tikzpicture}
                    \tkzTabInit{$t$ / 1 , $x'(t)$ / 1, $x(t)$ / 1.5, $y'(t)$ / 1, $y(t)$ / 1.5}{$0$, $\pi/2$, $2\pi/3$, $\pi$}
                    \tkzTabLine{, +, 0, -,,-, }
                    \tkzTabVar{-/ 0, +/ $1$, R/,-/ 0}
                    \tkzTabIma{2}{4}{3}{$\frac{\sqrt 3}{2}$}
                    \tkzTabLine{,+ ,,+,0,-, }
                    \tkzTabVar{-/ $0$, R/ , +/ $\frac{\sqrt 3}{2}$, -/ $0$}
                    \tkzTabIma{1}{3}{2}{$\frac{1}{2}$}
                \end{tikzpicture}
            \end{center}
            
        }

    \item Déterminer le point double de  $\Gamma$. Calculer les tangentes en ce point double

        \cor{
            Le point double est l'origine atteint en $t=0$ et $t=\pi$. Les tengentes passent donc par l'origine. La pente de la droite est alors $\frac{x'(0)}{y'(0)} = \frac 1 3$ et $\frac{x'(\pi)}{y'(\pi)} = 1$ respectivement.
        }

    \item Déterminer points de  $\Gamma$ qui admettent une tangente horizontale ou une tangente verticale.
        \cor{
            Les tangentes verticales sont données par les solutions de $x'(t) = 0$. On est alors en $\Gamma(\pi/2) = (1,1/2) $. 

            Les tangentes horizontales sont données par les solutions de $y'(t) = 0$. On est alors en $\Gamma(2\pi/3) = \sqrt 3 (1/2,1/3)$.
        }

    \item Tracer la courbe $t\mapsto \Gamma(t)$  pour $t\in [-\pi,\pi]$.

\cor{
    Attention, ici le domaine pour le tracé est différent du domaine d'étude. Mais comme $\Gamma(t) = - \Gamma(-t)$ on a une symétrie centrale par rapport à l'origine. Cela donne:
    
        \begin{center}
            \begin{tikzpicture}\pgfplotsset{compat=1.8}
                \begin{axis}[height=6cm,width=6cm, axis lines=center, axis on top, xlabel={$x$}, ylabel={$y$}, axis equal,grid=both, minor tick num=2,]
                    \addplot[samples=200, very thick,blue,  domain = 0:pi] ({sin(deg(x))},{sin(deg(x))/(2+cos(deg(x)))});
                    \addplot[samples=200, very thick,red,  domain = -pi:0] ({sin(deg(x))},{sin(deg(x))/(2+cos(deg(x)))});
                \end{axis}
            \end{tikzpicture}
        \end{center}
    }
\end{enumerate}

\section{Calcul différentiel}

\exo{} Soit $f : \mathbb R^2 \to \mathbb R$ définie par
$$
f(x,y)=\begin{cases}
\sin(y-x) & \text{ si }  y>|x|  \\
0 & \text{ si }  y=|x| \\  
\frac{x-y}{\sqrt{x^2 + y^2}} & \text{si }  y<|x| 
\end{cases}
$$
\begin{enumerate}
\item La fonction $f$ est-elle continue sur $\mathbb R^2$ ?
\item La fonction $f$ est-elle différentiable sur $\mathbb R^2$ ?
\item La fonction $f$ est-elle $C^1$ sur $\mathbb R^2$ ?
\end{enumerate}


\exo{} Soit $f : \mathbb R^2 \to \mathbb R$ définie par
$$
f(x,y)=\begin{cases}
\frac{x^2y}{x^4+y^2} & \text{ si } (x,y) \neq (0,0)  \\
0 & \text{ si } (x,y)=(0,0) 
\end{cases}
$$
\begin{enumerate}
\item La fonction $f$ est-elle continue sur $\mathbb R^2$ ?
\item Soit $v \in \mathbb R^2$ un vecteur non nul. La fonction $f$ admet-elle une dérivée en suivant $v$ sur $\mathbb R^2$ ?
\item La fonction $f$ est-elle différentiable sur $\mathbb R^2$ ?
\end{enumerate}

%\exo[]


\exo{} Soit la fonction $f:(x,y) \mapsto \abs{x} e^{-x^2 - y^2}$ définie sur $\R^2$. 
 
\begin{enumerate}
    \item Déterminer l'ensemble $D$ de $\R^2$ sur lequel $f$ est $\mathcal C^2$.

        \cor{
            La fonction $f$ est $\mathcal C^\infty$ sur $D = \R^2 \setminus \{x=0\}$ à cause du point anguleux de la valeur absolue. Reste à voir que $f$ n'est pas  $\mathcal C^2$ sur $\{x=0\}$. En effet $\frac{\partial f}{\partial x}(0,y) = \begin{cases}
                (1 -2x^2) e^{-x^2 - y^2} \text{ si } x \geq 0 \\
                (-1 + 2x^2)e^{-x^2 - y^2} \text{ si } x <0
            \end{cases}$ qui admet $1$ et $-1$ pour limite à gauche et à droite en $0$.
        }
    \item Déterminer les valeurs de $f$ sur $\R^2 \setminus D$. En déduire la nature des extremum de $f$ sur cet ensemble.

        \cor{

            La fonction $f$ est identiquement nulle sur cette ensemble. C'est clairement le minimum globale de $f$.
        }

    \item Calculer les points critiques de la $f$. En donner leur nature. 
        \cor{
            On a $\frac{\partial f }{\partial x} (x,y) =(1 - 2x^2) e^{-x^2 - y^2}$ et $\frac{\partial f }{\partial y} (x,y) = -2xy e^{-x^2 - y^2}$. Les points critiques sont $(1/\sqrt 2,0)$ (maximum local) et $(-1/\sqrt 2,0)$ (maximum local). %Après le calcule des dérivées partielles secondes on a :
        }
    \item Faire le bilan: lister les extrema de $f$.

        \cor{
            Minimum globale: 0 sur la droite $x=0$ et deux maximum globaux en $(1/\sqrt 2,0)$  et $(-1/\sqrt 2,0)$ (l'argument n'est pas facile ici\ldots)
        }
\end{enumerate}

\section{Intégration}

\exo{Centre de gravité} 


exo{} 
Pour tout entier pair $n\geq 2$ on consid\`ere $\Delta_n$, le domaine du plan d\'elimit\'e par les courbes d'\'equations $y=x^n$ et $x=y^n$.
\begin{enumerate}
\item Calculer l'aire $S_n$ de $\Delta_n$. 
\item V\'erifier que $\lim\limits_{n \to +\infty} S_n =1$ et expliquer pourquoi cette valeur \'etait attendue.
\end{enumerate}


\exo{} 
Pour $R>0$, on consid\`ere le domaine $\Delta=\{(x,y,z)\in \R^3\;;\; x^2+y^2+z^2\leq R^2\;,\; z\ge 0\}$.
\begin{enumerate}
\item  Dessiner et caract\'eriser g\'eom\'etriquement $\Delta$.
\item  Donner son volume $V$ sans faire obligatoirement de calcul.
\item Calculer la hauteur $z_G$ de son centre de gravit\'e.%, donn\'ee par $z_G=\frac{1}{V}\intt_\Delta z\,dx\, dy\, dz$.
\end{enumerate}
%On rappelle que l'\'el\'ement d'int\'egration en coordonn\'ees sph\'eriques (avec latitude) est $r^2\cos(\varphi)\,dr d\theta d\varphi$.




\end{document}

