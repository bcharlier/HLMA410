\documentclass{article}
\makeatletter
%--------------------------------------------------------------------------------

\usepackage[frenchb]{babel}

\usepackage{amsmath}
\usepackage{amsbsy}
\usepackage{amsfonts}
\usepackage{amssymb}
\usepackage{amscd}
\usepackage{amsthm}
\usepackage{mathtools}
\usepackage{eurosym}
\usepackage{nicefrac}

\usepackage{latexsym}
\usepackage[a4paper,hmargin=20mm,vmargin=25mm]{geometry}
\usepackage{dsfont}
\usepackage[utf8]{inputenc}
\usepackage[T1]{fontenc}

\usepackage{multicol}
\usepackage[inline]{enumitem}
%\setlist{nosep}
\setlist[itemize,1]{,label=$-$}

\usepackage{sectsty}
%\sectionfont{}
\allsectionsfont{\normalfont\sffamily\bfseries\normalsize}

\usepackage{graphicx}
\usepackage{tikz}

\usepackage{pgfplots}
\usepgfplotslibrary{fillbetween}
\pgfplotsset{compat=newest}
%\usepgfplotslibrary{external} 
%\tikzexternalize[prefix=./output_latex/]
%\DeclareSymbolFont{RalphSmithFonts}{U}{rsfs}{m}{n}
%\DeclareSymbolFontAlphabet{\mathscr}{RalphSmithFonts}
%\def\mathcal#1{{\mathscr #1}}

\newcounter{zut}
\setcounter{zut}{1}
\newcommand{\exo}[1]{\noindent {\sffamily\bfseries Exercice~\thezut. #1} \
		   \addtocounter{zut}{1}}



\providecommand{\abs}[1]{\left|#1\right|}
\providecommand{\norm}[1]{\left\Vert#1\right\Vert}
\providecommand{\U}{\mathcal{U}}
\providecommand{\R}{\mathbb{R}}
\providecommand{\Cc}{\mathcal{C}}
\providecommand{\reg}[1]{\mathcal{C}^{#1}}
\providecommand{\1}{\mathds{1}}
\providecommand{\N}{\mathbb{N}}
\providecommand{\Z}{\mathbb{Z}}
\providecommand{\E}{\mathbb{E}}
\providecommand{\p}{\partial}
\providecommand{\one}{\mathds{1}}
\renewcommand{\P}{\mathbb{P}}


%Operateur
\providecommand{\abs}[1]{\left\lvert#1\right\rvert}
\providecommand{\sabs}[1]{\lvert#1\rvert}
\providecommand{\babs}[1]{\bigg\lvert#1\bigg\rvert}
\providecommand{\norm}[1]{\left\lVert#1\right\rVert}
\providecommand{\bnorm}[1]{\bigg\lVert#1\bigg\rVert}
\providecommand{\snorm}[1]{\lVert#1\rVert}
\providecommand{\prs}[1]{\left\langle #1\right\rangle}
\providecommand{\sprs}[1]{\langle #1\rangle}
\providecommand{\bprs}[1]{\bigg\langle #1\bigg\rangle}

\DeclareMathOperator{\deet}{Det}
\DeclareMathOperator{\vol}{Vol}
\DeclareMathOperator{\aire}{Aire}
\DeclareMathOperator{\hess}{Hess}
\DeclareMathOperator{\var}{Var}

%------------------------------------------------------------------------------
\DeclareUnicodeCharacter{00A0}{~}
\makeatother



%-----------------------------------------------------------------------------
\begin{document}

\begin{center}
{\large \sffamily\bfseries RENDRE LE SUJET AVEC LA COPIE. }
\end{center}



\noindent Université Montpellier 2 \hfill Année 2014-2015

\noindent HLMA 410
 


\bigskip

\begin{center}
{\large \sffamily\bfseries Examen - session 2}
\end{center}

\textit{Durée 2h00. Les documents, la calculatrice, les téléphones portables, tablettes, ordinateurs ne sont pas autorisés. La qualité de la rédaction sera prise en compte.} 

\bigskip
\bigskip

\section{Analyse}

\exo{Question de cours}
Soient $u=(u_1,u_2,u_3)$ et $v = (v_1,v_2,v_3)$ deux vecteurs de $\R^3$.  On note également $\prs{\cdot,\cdot}$ le produit scalaire canonique de $\R^3$.
\begin{enumerate}
	\item Donner la définition du produit vectoriel $u\wedge v$.
%	\item Donner l'interprétation géométrique (faire un dessin!) et rappeler les règles de calculs élémentaires.
	\item Montrer que $\prs{u, u\wedge v}  = 0$.
	\item Soient $(e_1,e_2,e_3)$ une base orthonormale directe de $\R^3$. Calculer : \\
		\begin{enumerate*}
			\item $e_1\wedge e_2 \qquad$
			\item $e_3 \wedge e_2 \qquad$
			\item $e_2 \wedge e_2 $
		\end{enumerate*}
		\item L'application $ v \mapsto (1,2,3) \wedge v$ est-elle continue ? Est-elle différentiable ?
\end{enumerate}

\bigskip

\exo{}
On considère l'application 
\begin{align*}
	f : \R^3 & \to \R^2 \\
(x,y,z) & \mapsto (xye^z,\cos(yz))
\end{align*}
\begin{enumerate}
	\item Justifier que $f$ est de classe $\mathcal C^\infty$ sur $\R^3$.
	\item Donner la différentielle de $f$ au point $(1,2,3)$.
\end{enumerate}

\bigskip

\bigskip

\exo{}
Pour tout $(x,y) \in \R^2 \setminus \left\{ (0,0) \right\}$ on note $f(x,y) = \frac{xy^2}{x^2 + y^2}$.
\begin{enumerate}
	\item Montrer que $f$ se prolonge en une fonction continue $\tilde f$ définie sur $\R^2$.
	\item En quels points de $\R^2$ la fonction $\tilde f$ est elle différentiable ?
\end{enumerate}

\bigskip

%\exo{Question de cours}
%Soit $E$ un espace vectoriel de dimension finie et $q:E\to \R$ une forme quadratique.
%\begin{enumerate}
	%\item Comment peut-on calculer la forme bilinéaire symétrique $\varphi:E\times E \to \R$ associée à $q$?
	%\item Application : on pose $E =\R^2$, et $q(x,y) = (x+y)^2 + 2 y^2$ 
		%\begin{enumerate}
			%\item Calculer la forme bilinéaire $\varphi $ associée à $q$.
			%\item Montrer que $\varphi$ est un produit scalaire.
		%\end{enumerate}
%\end{enumerate}

%\bigskip



\exo{} Soit $f(x,y) = \exp \left ( -9x^2 - \frac{y^2}{4} +36 x -y -37 \right)$.% 4*(x-2)^2 + (y+2)^2/4
\begin{enumerate}
	\item \'Etudier la continuité $f$. Sur quel ensemble $f$ est-elle $\mathcal C^\infty$  ?
	%\item Donner les points de minimums de $f$ sur $\R^2$. Indiquer aussi la nature de ces points (minimum global ou local).
	\item Calculer le gradient et la Hessienne de $f$ en les points de $\R^2$ pour lesquels ces quantités sont bien définies.
	\item Déterminer alors le(s) point(s) critique(s) de $f$ et donner leur nature (minimum/maximum, local/global, point selle,\ldots).
	\item On pose $D_1=\left\{ (X,Y)\in\R^2 | X^2 + Y^2 < 1 \right\}$ et on donne $\iint_{D_1} \exp(-X^2 -Y^2 )dXdY  = \pi(1 - e^{-1})$.
	%\item On rappelle que $\iint_{\R^2} \exp(-X^2 -Y^2 )dXdY  = \pi$.
		%\begin{enumerate}
			%\item Déterminer les réels $a_1,a_2,c_1,c_2$ tels que $f(x , y) =  \exp (-a_1(x-c_1)^2 - a_2 (y-c_2)^2 )$. % Appliquer l'algorithme de la décomposition de Gauss et en déduire le signe de $q$
En déduire de  la valeur de 
\[
\iint_{D_2} f(x,y) dxdy\] où $D_2 = \{ (x,y)\in\R^2 | 9(x-2)^2 + \frac {1}{4} (y+2)^2  < 1 \}$.
		%\end{enumerate}
\end{enumerate}


\section{Probabilité}

\exo{}Soit $X$ une variable aléatoire réelle telle que $\E(X^2) < +\infty$. Montrer que 
\[
	\E( (X-\E(X))^2) = \E(X^2) - (\E(X))^2.
\]

\bigskip

\exo{} On vous propose un jeu d'argent : vous lancez trois fois une pièce équilibrée et si vous n'obtenez que des ``pile'' ou que des ``face'', vous empochez $5$ euros. Sinon, c'est à vous de débourser 2 euros.
\begin{enumerate}
	\item  Donner la loi de probabilité de $X$, le gain associé à ce jeu.
	\item  Calculer l'espérance de $X$. Ce jeu vaut-il le coup ?
	\item  Calculer la variance de $X$.
\end{enumerate}

\begin{center}
{\large \sffamily\bfseries RENDRE LE SUJET AVEC LA COPIE. }
\end{center}

\end{document}

