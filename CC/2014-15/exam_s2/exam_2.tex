\documentclass{article}
\makeatletter
%--------------------------------------------------------------------------------

\usepackage[french]{babel}
\usepackage{amsmath}
\usepackage{amsbsy}
\usepackage{amsfonts}
\usepackage{amssymb}
\usepackage{amscd}
\usepackage{amsthm}
\usepackage{mathtools}
\usepackage{eurosym}
\usepackage{nicefrac}

\usepackage{latexsym}
\usepackage[a4paper,hmargin=20mm,vmargin=25mm]{geometry}
\usepackage{dsfont}
\usepackage[utf8]{inputenc}
\usepackage[T1]{fontenc}
\usepackage{lmodern}

\usepackage{multicol}
\usepackage[inline]{enumitem}
\setlist{nosep}
\setlist[itemize,1]{,label=$-$}


\newenvironment{modenumerate}
  {\enumerate\setupmodenumerate}
  {\endenumerate}

\newif\ifmoditem
\newcommand{\setupmodenumerate}{%
  \global\moditemfalse
  \let\origmakelabel\makelabel
  \def\moditem##1{\global\moditemtrue\def\mesymbol{##1}\item}%
  \def\makelabel##1{%
    \origmakelabel{##1\ifmoditem\rlap{\mesymbol}\fi\enspace}%
    \global\moditemfalse}%
}


\usepackage{sectsty}
%\sectionfont{}
\allsectionsfont{\color{astral}\normalfont\sffamily\bfseries\normalsize}

\usepackage{graphicx}
\usepackage{tikz}
\usetikzlibrary{babel}
\usepackage{tikz,tkz-tab}

\usepackage[babel=true, kerning=true]{microtype}


\usepackage{pgfplots}
\usepgfplotslibrary{fillbetween}
\pgfplotsset{compat=newest}
\usepgfplotslibrary{external} 
\tikzexternalize[prefix=./output_latex/]
%\DeclareSymbolFont{RalphSmithFonts}{U}{rsfs}{m}{n}
%\DeclareSymbolFontAlphabet{\mathscr}{RalphSmithFonts}
%\def\mathcal#1{{\mathscr #1}}



\providecommand{\abs}[1]{\left|#1\right|}
\providecommand{\norm}[1]{\left\Vert#1\right\Vert}
\providecommand{\U}{\mathcal{U}}
\providecommand{\R}{\mathbb{R}}
\providecommand{\Cc}{\mathcal{C}}
\providecommand{\reg}[1]{\mathcal{C}^{#1}}
\providecommand{\1}{\mathds{1}}
\providecommand{\N}{\mathbb{N}}
\providecommand{\Z}{\mathbb{Z}}
\providecommand{\p}{\partial}
\providecommand{\one}{\mathds{1}}
\providecommand{\E}{\mathbb{E}}\providecommand{\V}{\mathbb{V}}
\renewcommand{\P}{\mathbb{P}}


%Operateur
\providecommand{\abs}[1]{\left\lvert#1\right\rvert}
\providecommand{\sabs}[1]{\lvert#1\rvert}
\providecommand{\babs}[1]{\bigg\lvert#1\bigg\rvert}
\providecommand{\norm}[1]{\left\lVert#1\right\rVert}
\providecommand{\bnorm}[1]{\bigg\lVert#1\bigg\rVert}
\providecommand{\snorm}[1]{\lVert#1\rVert}
\providecommand{\prs}[1]{\left\langle #1\right\rangle}
\providecommand{\sprs}[1]{\langle #1\rangle}
\providecommand{\bprs}[1]{\bigg\langle #1\bigg\rangle}

\DeclareMathOperator{\deet}{Det}
\DeclareMathOperator{\hess}{Hess}
\DeclareMathOperator{\jac}{Jac}


\newcommand\rst[2]{{#1}_{\restriction_{#2}}}



% generate breakable white space allowing students to write notes.

\usepackage[framemethod=tikz]{mdframed}

\mdfdefinestyle{response}{
	leftmargin=.01\textwidth,
	rightmargin=.01\textwidth,
	linewidth=1pt
	hidealllines=false,
	leftline=true,
	rightline=true,topline=true,bottomline=true,
	skipabove=0pt,
	%innertopmargin=-5pt,
	%innerbottommargin=2pt,
	linecolor=black,
	innerrightmargin=0pt,
	}



\newcommand*{\DivideLengths}[2]{%
  \strip@pt\dimexpr\number\numexpr\number\dimexpr#1\relax*65536/\number\dimexpr#2\relax\relax sp\relax
}

\providecommand{\rep}[1]{$ $ \newline \begin{mdframed}[style=response]  
	
	\vspace*{\DivideLengths{#1}{3cm}cm}
	\pagebreak[1]	
	\vspace*{\DivideLengths{#1}{3cm}cm}
	\pagebreak[1]		
	\vspace*{\DivideLengths{#1}{3cm}cm}   \end{mdframed}}

\providecommand{\blanc}[1]{$ $ \newline 
	
	\vspace*{\DivideLengths{#1}{3cm}cm}
	\pagebreak[1]	
	\vspace*{\DivideLengths{#1}{3cm}cm}
	\pagebreak[3]		
	\vspace*{\DivideLengths{#1}{3cm}cm}}

\usepackage{ifthen}

\newcommand{\eno}[1]{%
	\ifthenelse{\equal{\version}{eno}}{#1}{}%
}
\newcommand{\cor}[1]{%
        \ifthenelse{\equal{\version}{cor}}{
\medskip 

{\small \color{gray} #1}

\medskip 
}{}
}

%------------------------------------------------------------------------------
%\DeclareUnicodeCharacter{00A0}{~}
\makeatother



%-----------------------------------------------------------------------------
\begin{document}
\noindent Université Montpellier 2 \hfill Année 2014-2015

\noindent HLMA 410
 


\bigskip

\begin{center}
{\large \sffamily\bfseries Examen - session 2}
\end{center}

\textit{Durée 2h00. Les documents, la calculatrice, les téléphones portables, tablettes, ordinateurs ne sont pas autorisés. La qualité de la rédaction sera prise en compte.} 

\bigskip
\bigskip

\section{Analyse}

\exo{Question de cours}
	Soit $\Gamma=([a,b],\phi)$ un arc param\'etr\'e dans $\R^n$ de classe $\mathcal C^1$ et $V:\R^n \rightarrow \R^n$ un champ de vecteurs continu.
\begin{enumerate}
	\item Qu'appelle-t-on circulation de $V$ le long de $\Gamma$ ?
\end{enumerate}
			Soit $\theta: [c,d] \rightarrow [a,b]$ un $C^1$-diff\'eomorphisme et $\psi=\phi \circ \theta$. 
		\begin{enumerate}[resume]
	\item Quelle relation y a-t-il entre la circulation de $V$ le long de $\Gamma'=([c,d],\psi)$ et la circulation le long de $\Gamma$ ? 	\item Le-d\'emontrer.
\end{enumerate}

\bigskip


\exo{Question de cours}
Soit $E$ un espace vectoriel de dimension finie et soient $u,v$ deux vecteurs de $E$. 
\begin{enumerate}
\item Donner la définition d'une norme $\norm{\cdot}$ sur $E$.
\item Montrer que 	
	\[
		\abs{ \norm{u} - \norm{v} } \leq \norm{u-v}.
	\]
	{\itshape Indication : on pourra montrer dans un premier temps que $\norm{u} \leq \norm{v-u} + \norm{v}$}
\end{enumerate}

\bigskip

%\exo{}
%On considère l'application 
%\begin{align*}
	%f : \R^3 & \to \R^2 \\
%(x,y,z) & \mapsto (xye^z,\cos(yz))
%\end{align*}
%\begin{enumerate}
	%\item Justifier que $f$ est de classe $\mathcal C^\infty$ sur $\R^3$.
	%\item Donner la différentielle de $f$ au point $(1,2,3)$.
%\end{enumerate}

\bigskip

\exo{} Soit $D$ le domaine borné du plan délimité par la droite $y=2x$ et la parabole $y^2= 4x$.
\begin{enumerate}
\item Calculer l'aire de $D$.
\item Calculer l'intégrale $\iint_D (y-x) dxdy$.
\item Soit $C$ une courbe paramétrée décrivant le bord orienté positivement de $D$ (\ie de sorte à laisser $D$ sur sa gauche). 
	\begin{enumerate}
		\item Déterminer une expression possible pour $C$.
		\item Calculer la circulation du champs de vecteur $(u,v) \mapsto (v^2,uv)$ le long de $C$ de manière directe.
		\item Calculer la circulation du champs de vecteur $(u,v) \mapsto (v^2,uv)$ le long de $C$ avec l'aide de la formule de Green-Riemann.
	\end{enumerate}
\end{enumerate}

\bigskip


\exo{} Soit $f(x,y) = \abs{4x^2 + 9y^2 - 8x + 36y + 39}$.% Le but de cet exerci 
\begin{enumerate}
	\item Déterminer les réels $a_1,a_2,c_1,c_2$ et $b$ tels que $f(x , y) = \abs{a_1(x-c_1)^2 + a_2 (y-c_2)^2 -b}$. % Appliquer l'algorithme de la décomposition de Gauss et en déduire le signe de $q$
	\item \'Etudier la continuité $f$. Sur quel ensemble $f$ est-elle $\mathcal C^\infty$ ? Justifier la réponse.
	\item Donner les points de minimum de $f$ sur $\R^2$. Indiquer aussi la nature de ces points (minimum global ou local).
	\item Calculer le gradient et la Hessienne de $f$ en les points de $\R^2$ pour lesquels ces quantités sont bien définies.
	\item Déterminer alors le(s) point(s) critique(s) de $f$ donner leur nature (minimum/maximum, local/global, point selle,\ldots).
\end{enumerate}


\bigskip


%\exo{}
%Pour tout $(x,y) \in \R^2 \setminus \left\{ (0,0) \right\}$ on note $f(x,y) = e^{-x^2 - y^2}+\frac{xy^2}{x^2 + 2 y^2}$.
%\begin{enumerate}
	%\item Montrer que $f$ se prolonge en une fonction continue $\tilde f$ définie sur $\R^2$.
	%\item En quels points de $\R^2$ la fonction $\tilde f$ est elle différentiable ?
%\end{enumerate}


\section{Probabilité}

\exo{}
Un étudiant tente désespérément de passer son examen de HLMA410. On suppose qu'il a une probabilité $p \in]0,1[$ de réussite à chaque essai. On suppose que les essais sont indépendants et on note $N$ le nombre d'essais nécessaires pour qu'il valide cet unité d'enseignement (UE).

\begin{enumerate}
\item  
	\begin{enumerate}
		\item Donner la loi de $N$ et calculer $\P(N >2$). 
		\item La moyenne de $N$ existe-t-elle ? Si oui la calculer (on demande le détail du calcul, pas uniquement le résultat).
	\end{enumerate}
\item Soit $Z= \min\{3,N\}$.
	\begin{enumerate}
		\item La moyenne de $Z$ existe-t-elle ? Si oui la calculer.
		\item La variance de $Z$ existe-t-elle ? Si oui la calculer.\end{enumerate}
\end{enumerate}

%\exo{} On vous propose un jeu d'argent : vous lancez trois fois une pièce équilibrée et si vous n'obtenez que des ``pile'' ou que des ``face'', vous empochez $5$ euros. Sinon, c'est à vous de débourser 2 euros.
%\begin{enumerate}
	%\item  Donner la fonction de masse de $X$, le gain associé à ce jeu.
	%\item  Calculer l'espérance de $X$. Ce jeu vaut-il le coup ?
	%\item  Calculer la variance de $X$.
%\end{enumerate}

%\bigskip

%\exo{} Un voyage organisé (séjour au ski par exemple) peut accueillir $95$ personnes. La pratique montre que l'on peut estimer à 5\% la probabilité qu'une réservation soit annulée avant le départ. Le voyagiste fait du surbooking et il accepte 100 réservations. On note $A$ le nombre de réservations annulées.
%\begin{enumerate}
	%\item Quelle est la loi de $A$. Calculer sont espérance et sa variance.
	%\item Quelle est la probabilité que le voyagiste soit obligé de refuser un client ayant réservé ?
%\end{enumerate}
\end{document}

