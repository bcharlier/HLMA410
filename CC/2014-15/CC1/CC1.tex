\documentclass[a4paper]{article}
\makeatletter
%--------------------------------------------------------------------------------

\usepackage[frenchb]{babel}

\usepackage{amsmath}
\usepackage{amsbsy}
\usepackage{amsfonts}
\usepackage{amssymb}
\usepackage{amscd}
\usepackage{amsthm}
\usepackage{mathtools}
\usepackage{eurosym}
\usepackage{nicefrac}

\usepackage{latexsym}
\usepackage[a4paper,hmargin=20mm,vmargin=25mm]{geometry}
\usepackage{dsfont}
\usepackage[utf8]{inputenc}
\usepackage[T1]{fontenc}

\usepackage{multicol}
\usepackage[inline]{enumitem}
%\setlist{nosep}
\setlist[itemize,1]{,label=$-$}

\usepackage{sectsty}
%\sectionfont{}
\allsectionsfont{\normalfont\sffamily\bfseries\normalsize}

\usepackage{graphicx}
\usepackage{tikz}

\usepackage{pgfplots}
\usepgfplotslibrary{fillbetween}
\pgfplotsset{compat=newest}
%\usepgfplotslibrary{external} 
%\tikzexternalize[prefix=./output_latex/]
%\DeclareSymbolFont{RalphSmithFonts}{U}{rsfs}{m}{n}
%\DeclareSymbolFontAlphabet{\mathscr}{RalphSmithFonts}
%\def\mathcal#1{{\mathscr #1}}

\newcounter{zut}
\setcounter{zut}{1}
\newcommand{\exo}[1]{\noindent {\sffamily\bfseries Exercice~\thezut. #1} \
		   \addtocounter{zut}{1}}



\providecommand{\abs}[1]{\left|#1\right|}
\providecommand{\norm}[1]{\left\Vert#1\right\Vert}
\providecommand{\U}{\mathcal{U}}
\providecommand{\R}{\mathbb{R}}
\providecommand{\Cc}{\mathcal{C}}
\providecommand{\reg}[1]{\mathcal{C}^{#1}}
\providecommand{\1}{\mathds{1}}
\providecommand{\N}{\mathbb{N}}
\providecommand{\Z}{\mathbb{Z}}
\providecommand{\E}{\mathbb{E}}
\providecommand{\p}{\partial}
\providecommand{\one}{\mathds{1}}
\renewcommand{\P}{\mathbb{P}}


%Operateur
\providecommand{\abs}[1]{\left\lvert#1\right\rvert}
\providecommand{\sabs}[1]{\lvert#1\rvert}
\providecommand{\babs}[1]{\bigg\lvert#1\bigg\rvert}
\providecommand{\norm}[1]{\left\lVert#1\right\rVert}
\providecommand{\bnorm}[1]{\bigg\lVert#1\bigg\rVert}
\providecommand{\snorm}[1]{\lVert#1\rVert}
\providecommand{\prs}[1]{\left\langle #1\right\rangle}
\providecommand{\sprs}[1]{\langle #1\rangle}
\providecommand{\bprs}[1]{\bigg\langle #1\bigg\rangle}

\DeclareMathOperator{\deet}{Det}
\DeclareMathOperator{\vol}{Vol}
\DeclareMathOperator{\aire}{Aire}
\DeclareMathOperator{\hess}{Hess}
\DeclareMathOperator{\var}{Var}

%------------------------------------------------------------------------------
\DeclareUnicodeCharacter{00A0}{~}
\makeatother



%-----------------------------------------------------------------------------
\begin{document}

\noindent Université Montpellier 2 \hfill Année 2014-2015

\noindent HLMA 410
 


\bigskip

\begin{center}
{\large \sffamily\bfseries Contrôle continu 1}
\end{center}

\textit{Durée 1h30. Les documents, la calculatrice, les téléphones portables, tablettes, ordinateurs ne sont pas autorisés. La qualité de la rédaction sera prise en compte.} 

\bigskip
\bigskip


\exo{(Question de cours)} Rappeler la définition d'un produit scalaire puis la définition d'espace euclidien.
\vspace*{5cm}

%\exo{Valeur absolue}
%\begin{enumerate}
%\item 		Rappeler la définition de la valeur absolue.
%\vspace*{3cm}
%\item 		Démontrer que la valeur absolue est bien une norme sur $\R$.
%\vspace*{7cm}
%\item 		Est-ce la seule norme sur $\R$ ?
%\vspace*{5cm}
%\end{enumerate}

\exo{(Une norme sur $\boldsymbol{ \mathbb{R}^2}$)} Soit $N(x,y) =\max \left\{ \sqrt{x^2+y^2} , \abs{x-y} \right\}$ définie pour tout $(x,y)\in\R^2$.
\begin{enumerate}
	\item Montrer que $N$ est une norme sur $\R^2$.
		\newpage
	\item Calculer la norme $N$ des points suivants : $(-\frac 1 2, \frac 1 2)$, $(\frac 1 2, -\frac 1 2) $, $( \frac{\sqrt{2}}{2},\frac{\sqrt{2}}{2}) $ et $( -\frac{\sqrt{2}}{2},-\frac{\sqrt{2}}{2})$. 

		\vspace*{4cm}

	\item Dessiner (en justifiant) la boule unité de $N$ : \\
			\begin{tikzpicture}[scale=1]
				\def\xone{-3.1}
				\def\xtwo{3.1}
				\def\yone{-3.1}
				\def\ytwo{3.1}

% grid
				\draw[step=1cm,help lines,gray!50] (\xone-.1,\yone-.1) grid (\xtwo+.1,\ytwo+.1);
				\draw[thick,->] (\xone-.3, 0) -- (\xtwo+.3, 0) node[right] {$x$};
				\draw[thick,->] (0, \yone-.3) -- (0, \ytwo+.3) node[above] {$y$};

			%	\draw[] (0,0) circle (2cm);
				\draw[] (2,.1) -- (2,-.1) node [below] {1};
				\draw[] (.1,2) -- (-.1,2) node [left] {1};
				\draw[] (-2,.1) -- (-2,-.1) node [below] {-1};
				\draw[] (.1,-2) -- (-.1,-2) node [left] {-1};
			\end{tikzpicture}

\end{enumerate}


\exo{(Géométrie vectorielle)} Dans $\R^3$ muni du repère canonique $(O,i,j,k)$, on considère les points $A=(6,2,4)$, $B=(2,1,1)$ et $C=(\alpha,3,7)$ où $\alpha \in \R$
\begin{enumerate}
\item Déterminer l'ensemble des $\alpha\in\R$ pour lesquelles :
	\begin{enumerate}
		\item les points $A$, $B$ et $C$ sont alignés.
\vspace*{5cm}
		\item le vecteur $\overrightarrow{OC}$ est unitaire.
\vspace*{5cm}
	\end{enumerate}
\item Déterminer une base orthonormale dont le premier vecteur est colinéaire à $\overrightarrow{OA}$.
\vspace{7cm}
\vspace*{3.5cm}

\item Quelle est la distance entre le point de coordonnées $(6,3,7)$ et la droite contenant les points $A$ et $B$.
\vspace*{5cm}
%\item Quelle est la distance entre le point de coordonnées $(1,3,7)$ et le plan contenant les points $O$, $A$ et $B$.
%\vspace*{5cm}
\end{enumerate}


\exo{(Une fonction de plusieurs variables)} Soit la fonction définie par $f(x,y) =\sqrt{1+\frac{x}{y}}$.
\begin{enumerate}
	\item Déterminer le domaine de définition $\mathcal D$ de $f$. Ce domaine est-il ouvert ? est-il fermé ?
		\vspace{7cm}
%\newpage
%	\item Déterminer l'ensemble image de $f$. Cet ensemble est-il ouvert ? est-il fermé ?
\vspace*{4.5cm}
	\item Déterminer les ensembles de niveau de $f$.
\vspace*{9cm}
	\item Représenter le domaine $\mathcal D$ et les ensembles de niveau $\lambda =0$, $\lambda= 1$ et $\lambda = 2$.
		\begin{center}
			%\begin{tikzpicture}
		%\begin{axis}[ylabel style={rotate=-90},xlabel = $x$,ylabel=$y$,domain=-2:2, view={0}{90},width=.45\textwidth, ] 
					%\addplot3[ domain=-5:5,contour gnuplot={levels={1,2,3},labels=false},thick,samples=100] gnuplot {(1+x/y)**.5}; 
%\end{axis} 
			%\end{tikzpicture}
\begin{tikzpicture}[scale=.5]
	\def\xone{-8}
	\def\xtwo{8}
	\def\yone{-8}
	\def\ytwo{8}

% grid
  \draw[step=2cm,help lines,gray!50] (\xone-.2,\yone-.2) grid (\xtwo+.2,\ytwo+.2);
  \draw[thick,->] (\xone-.3, 0) -- (\xtwo+.3, 0) node[right] {$x$};
  \draw[thick,->] (0, \yone-.3) -- (0, \ytwo+.3) node[above] {$y$};

\end{tikzpicture}
		\end{center}
\end{enumerate}

\end{document}
