\documentclass[a4paper]{article}
\makeatletter
%--------------------------------------------------------------------------------

\usepackage[french]{babel}
\usepackage{amsmath}
\usepackage{amsbsy}
\usepackage{amsfonts}
\usepackage{amssymb}
\usepackage{amscd}
\usepackage{amsthm}
\usepackage{mathtools}
\usepackage{eurosym}
\usepackage{nicefrac}

\usepackage{latexsym}
\usepackage[a4paper,hmargin=20mm,vmargin=25mm]{geometry}
\usepackage{dsfont}
\usepackage[utf8]{inputenc}
\usepackage[T1]{fontenc}
\usepackage{lmodern}

\usepackage{multicol}
\usepackage[inline]{enumitem}
\setlist{nosep}
\setlist[itemize,1]{,label=$-$}


\newenvironment{modenumerate}
  {\enumerate\setupmodenumerate}
  {\endenumerate}

\newif\ifmoditem
\newcommand{\setupmodenumerate}{%
  \global\moditemfalse
  \let\origmakelabel\makelabel
  \def\moditem##1{\global\moditemtrue\def\mesymbol{##1}\item}%
  \def\makelabel##1{%
    \origmakelabel{##1\ifmoditem\rlap{\mesymbol}\fi\enspace}%
    \global\moditemfalse}%
}


\usepackage{sectsty}
%\sectionfont{}
\allsectionsfont{\color{astral}\normalfont\sffamily\bfseries\normalsize}

\usepackage{graphicx}
\usepackage{tikz}
\usetikzlibrary{babel}
\usepackage{tikz,tkz-tab}

\usepackage[babel=true, kerning=true]{microtype}


\usepackage{pgfplots}
\usepgfplotslibrary{fillbetween}
\pgfplotsset{compat=newest}
\usepgfplotslibrary{external} 
\tikzexternalize[prefix=./output_latex/]
%\DeclareSymbolFont{RalphSmithFonts}{U}{rsfs}{m}{n}
%\DeclareSymbolFontAlphabet{\mathscr}{RalphSmithFonts}
%\def\mathcal#1{{\mathscr #1}}



\providecommand{\abs}[1]{\left|#1\right|}
\providecommand{\norm}[1]{\left\Vert#1\right\Vert}
\providecommand{\U}{\mathcal{U}}
\providecommand{\R}{\mathbb{R}}
\providecommand{\Cc}{\mathcal{C}}
\providecommand{\reg}[1]{\mathcal{C}^{#1}}
\providecommand{\1}{\mathds{1}}
\providecommand{\N}{\mathbb{N}}
\providecommand{\Z}{\mathbb{Z}}
\providecommand{\p}{\partial}
\providecommand{\one}{\mathds{1}}
\providecommand{\E}{\mathbb{E}}\providecommand{\V}{\mathbb{V}}
\renewcommand{\P}{\mathbb{P}}


%Operateur
\providecommand{\abs}[1]{\left\lvert#1\right\rvert}
\providecommand{\sabs}[1]{\lvert#1\rvert}
\providecommand{\babs}[1]{\bigg\lvert#1\bigg\rvert}
\providecommand{\norm}[1]{\left\lVert#1\right\rVert}
\providecommand{\bnorm}[1]{\bigg\lVert#1\bigg\rVert}
\providecommand{\snorm}[1]{\lVert#1\rVert}
\providecommand{\prs}[1]{\left\langle #1\right\rangle}
\providecommand{\sprs}[1]{\langle #1\rangle}
\providecommand{\bprs}[1]{\bigg\langle #1\bigg\rangle}

\DeclareMathOperator{\deet}{Det}
\DeclareMathOperator{\hess}{Hess}
\DeclareMathOperator{\jac}{Jac}


\newcommand\rst[2]{{#1}_{\restriction_{#2}}}



% generate breakable white space allowing students to write notes.

\usepackage[framemethod=tikz]{mdframed}

\mdfdefinestyle{response}{
	leftmargin=.01\textwidth,
	rightmargin=.01\textwidth,
	linewidth=1pt
	hidealllines=false,
	leftline=true,
	rightline=true,topline=true,bottomline=true,
	skipabove=0pt,
	%innertopmargin=-5pt,
	%innerbottommargin=2pt,
	linecolor=black,
	innerrightmargin=0pt,
	}



\newcommand*{\DivideLengths}[2]{%
  \strip@pt\dimexpr\number\numexpr\number\dimexpr#1\relax*65536/\number\dimexpr#2\relax\relax sp\relax
}

\providecommand{\rep}[1]{$ $ \newline \begin{mdframed}[style=response]  
	
	\vspace*{\DivideLengths{#1}{3cm}cm}
	\pagebreak[1]	
	\vspace*{\DivideLengths{#1}{3cm}cm}
	\pagebreak[1]		
	\vspace*{\DivideLengths{#1}{3cm}cm}   \end{mdframed}}

\providecommand{\blanc}[1]{$ $ \newline 
	
	\vspace*{\DivideLengths{#1}{3cm}cm}
	\pagebreak[1]	
	\vspace*{\DivideLengths{#1}{3cm}cm}
	\pagebreak[3]		
	\vspace*{\DivideLengths{#1}{3cm}cm}}

\usepackage{ifthen}

\newcommand{\eno}[1]{%
	\ifthenelse{\equal{\version}{eno}}{#1}{}%
}
\newcommand{\cor}[1]{%
        \ifthenelse{\equal{\version}{cor}}{
\medskip 

{\small \color{gray} #1}

\medskip 
}{}
}

%------------------------------------------------------------------------------
%\DeclareUnicodeCharacter{00A0}{~}
\makeatother



%-----------------------------------------------------------------------------
\begin{document}

\noindent Université Montpellier 2 \hfill Année 2014-2015

\noindent HLMA 410
 


\bigskip

\begin{center}
{\large \sffamily\bfseries Contrôle continu 2}
\end{center}

\textit{Durée 1h30. Les documents, la calculatrice, les téléphones portables, tablettes, ordinateurs ne sont pas autorisés. La qualité de la rédaction sera prise en compte.} 

\bigskip
\bigskip

\exo{(Question de cours)} Soit $E$ et $F$ deux espaces vectoriels normés et $a\in E$. 
\begin{enumerate}
	\item Donner la définition de la dérivée directionnelle d'une fonction $f:E\to F$. Une fonction qui admet des dérivées partielles en $a$ est-elle nécessairement continue en $a$ (justifier) ?
		\vspace*{5cm}


	\item Donner la définition d'un $\reg{1}$-difféomorphisme de $E$ sur $F$. En donner une caractérisation (autrement dit, énoncer le ``théorème d'inversion globale'').
		\vspace*{7cm}
\end{enumerate}


\exo{Forme quadratique} Les formes quadratiques suivantes sont elles positives? sont elles définies ? :
\begin{enumerate}
	\item $q(x,y,z,t) = 2xz + 2xy +x^2 + 2tx$
\vspace*{5cm}
	\item $q(x,y,z) = -2(x+y)^2 + (x+y+z)^2 + (x+y-z)^2  $
\vspace*{5cm}
	\item $q(x,y)=e^{\sqrt{\pi}} x^2 + \ln(1+e) y^2 - xy$
\vspace*{5cm}
\end{enumerate}



%\exo{Valeur absolue}
%\begin{enumerate}
%\item 		Rappeler la définition de la valeur absolue.
%\vspace*{3cm}
%\item 		Démontrer que la valeur absolue est bien une norme sur $\R$.
%\vspace*{7cm}
%\item 		Est-ce la seule norme sur $\R$ ?
%\vspace*{5cm}
%\end{enumerate}

\exo{(\'Etude de fonctions de plusieurs variables)}Soit $m\in \N\setminus \left\{ 0 \right\}$ et $f : \R^2 \to \R$ la fonction définie comme suit :
\[
	f(x,y) = \begin{cases}
		\frac{x^m y^2}{x^2 + y^2} \text{ si } \R^2 \setminus \{ (0, 0) \} \\
		0 \text{ sinon}.
	\end{cases}
\]

\begin{enumerate}
	\item  Étude de la fonction sur $\R^2 \setminus \{ (0, 0) \}$ :
		\begin{enumerate}
			\item Montrer que $f$ est continue sur $\R^2 \setminus \{ (0,0) \}$ pour tout $m \in \N\setminus \left\{ 0 \right\}$ 
\vspace*{3cm}
			\item Calculer le gradient de $f$ pour $(x, y) \in \R^2 \setminus \{ (0, 0) \}$ pour tout $m \in \N\setminus \left\{ 0 \right\}$ ;
\vspace*{5cm}
			\item Montrer que $f$ est de classe $\reg{1}$ sur $\R^2 \setminus \{ (0, 0) \}$ pour tout $m \in \N\setminus \left\{ 0 \right\}$ ;
\vspace*{3cm}
			\item Que peut-on conclure sur la différentiabilité de $f$ sur $\R^2 \setminus \{ (0, 0) \}$ ?
\vspace*{3cm}
		\end{enumerate}
	\item  Étude de la fonction en $(0, 0)$ :
		\begin{enumerate}
			\item Pour quelles valeurs de $m\in\N\setminus \left\{ 0 \right\}$ la fonction $f$ est-elle continue en $(0, 0)$ ?
\vfill
			\item La fonction $f$ admet-elle des dérivées partielles en $(0, 0)$ pour tout $m\in\N\setminus \left\{ 0 \right\}$ ? si oui, les calculer.
\vspace*{5cm}
			\item Pour quelles valeurs de $m\in\N\setminus \left\{ 0 \right\}$ la fonction $f$ est-elle différentiable en $(0, 0)$ ?
\vspace*{12cm}
			\item Pour quelles valeurs de $m\in\N\setminus \left\{ 0 \right\}$ la fonction $f$ est-elle de classe $\reg{1}$ en $(0, 0)$ ?
\vfill
		\end{enumerate}
\end{enumerate}
%faccanoni 4.39 p108

%\exo{(Différentielle)} Pour $( x,y,z)\in \R^3$ on note : $f(x,y,z) = (\cos(xy) \cos(z),(1+x^2)^{yz} )$. Montrer que la fonction $f$ est différentiable au point $(a,1,0)$ pour tout $a$ de $\R$ puis calculer sa différentielle en ce point.

\end{document}
