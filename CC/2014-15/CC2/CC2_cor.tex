\documentclass[a4paper]{article}
\makeatletter
%--------------------------------------------------------------------------------

\usepackage[french]{babel}
\usepackage{amsmath}
\usepackage{amsbsy}
\usepackage{amsfonts}
\usepackage{amssymb}
\usepackage{amscd}
\usepackage{amsthm}
\usepackage{mathtools}
\usepackage{eurosym}
\usepackage{nicefrac}

\usepackage{latexsym}
\usepackage[a4paper,hmargin=20mm,vmargin=25mm]{geometry}
\usepackage{dsfont}
\usepackage[utf8]{inputenc}
\usepackage[T1]{fontenc}
\usepackage{lmodern}

\usepackage{multicol}
\usepackage[inline]{enumitem}
\setlist{nosep}
\setlist[itemize,1]{,label=$-$}


\newenvironment{modenumerate}
  {\enumerate\setupmodenumerate}
  {\endenumerate}

\newif\ifmoditem
\newcommand{\setupmodenumerate}{%
  \global\moditemfalse
  \let\origmakelabel\makelabel
  \def\moditem##1{\global\moditemtrue\def\mesymbol{##1}\item}%
  \def\makelabel##1{%
    \origmakelabel{##1\ifmoditem\rlap{\mesymbol}\fi\enspace}%
    \global\moditemfalse}%
}


\usepackage{sectsty}
%\sectionfont{}
\allsectionsfont{\color{astral}\normalfont\sffamily\bfseries\normalsize}

\usepackage{graphicx}
\usepackage{tikz}
\usetikzlibrary{babel}
\usepackage{tikz,tkz-tab}

\usepackage[babel=true, kerning=true]{microtype}


\usepackage{pgfplots}
\usepgfplotslibrary{fillbetween}
\pgfplotsset{compat=newest}
\usepgfplotslibrary{external} 
\tikzexternalize[prefix=./output_latex/]
%\DeclareSymbolFont{RalphSmithFonts}{U}{rsfs}{m}{n}
%\DeclareSymbolFontAlphabet{\mathscr}{RalphSmithFonts}
%\def\mathcal#1{{\mathscr #1}}



\providecommand{\abs}[1]{\left|#1\right|}
\providecommand{\norm}[1]{\left\Vert#1\right\Vert}
\providecommand{\U}{\mathcal{U}}
\providecommand{\R}{\mathbb{R}}
\providecommand{\Cc}{\mathcal{C}}
\providecommand{\reg}[1]{\mathcal{C}^{#1}}
\providecommand{\1}{\mathds{1}}
\providecommand{\N}{\mathbb{N}}
\providecommand{\Z}{\mathbb{Z}}
\providecommand{\p}{\partial}
\providecommand{\one}{\mathds{1}}
\providecommand{\E}{\mathbb{E}}\providecommand{\V}{\mathbb{V}}
\renewcommand{\P}{\mathbb{P}}


%Operateur
\providecommand{\abs}[1]{\left\lvert#1\right\rvert}
\providecommand{\sabs}[1]{\lvert#1\rvert}
\providecommand{\babs}[1]{\bigg\lvert#1\bigg\rvert}
\providecommand{\norm}[1]{\left\lVert#1\right\rVert}
\providecommand{\bnorm}[1]{\bigg\lVert#1\bigg\rVert}
\providecommand{\snorm}[1]{\lVert#1\rVert}
\providecommand{\prs}[1]{\left\langle #1\right\rangle}
\providecommand{\sprs}[1]{\langle #1\rangle}
\providecommand{\bprs}[1]{\bigg\langle #1\bigg\rangle}

\DeclareMathOperator{\deet}{Det}
\DeclareMathOperator{\hess}{Hess}
\DeclareMathOperator{\jac}{Jac}


\newcommand\rst[2]{{#1}_{\restriction_{#2}}}



% generate breakable white space allowing students to write notes.

\usepackage[framemethod=tikz]{mdframed}

\mdfdefinestyle{response}{
	leftmargin=.01\textwidth,
	rightmargin=.01\textwidth,
	linewidth=1pt
	hidealllines=false,
	leftline=true,
	rightline=true,topline=true,bottomline=true,
	skipabove=0pt,
	%innertopmargin=-5pt,
	%innerbottommargin=2pt,
	linecolor=black,
	innerrightmargin=0pt,
	}



\newcommand*{\DivideLengths}[2]{%
  \strip@pt\dimexpr\number\numexpr\number\dimexpr#1\relax*65536/\number\dimexpr#2\relax\relax sp\relax
}

\providecommand{\rep}[1]{$ $ \newline \begin{mdframed}[style=response]  
	
	\vspace*{\DivideLengths{#1}{3cm}cm}
	\pagebreak[1]	
	\vspace*{\DivideLengths{#1}{3cm}cm}
	\pagebreak[1]		
	\vspace*{\DivideLengths{#1}{3cm}cm}   \end{mdframed}}

\providecommand{\blanc}[1]{$ $ \newline 
	
	\vspace*{\DivideLengths{#1}{3cm}cm}
	\pagebreak[1]	
	\vspace*{\DivideLengths{#1}{3cm}cm}
	\pagebreak[3]		
	\vspace*{\DivideLengths{#1}{3cm}cm}}

\usepackage{ifthen}

\newcommand{\eno}[1]{%
	\ifthenelse{\equal{\version}{eno}}{#1}{}%
}
\newcommand{\cor}[1]{%
        \ifthenelse{\equal{\version}{cor}}{
\medskip 

{\small \color{gray} #1}

\medskip 
}{}
}

%------------------------------------------------------------------------------
%\DeclareUnicodeCharacter{00A0}{~}
\makeatother



%-----------------------------------------------------------------------------
\begin{document}

\noindent Université Montpellier 2 \hfill Année 2014-2015

\noindent HLMA 410
 


\bigskip

\begin{center}
{\large \sffamily\bfseries Contrôle continu 2}
\end{center}

%\textit{Durée 1h30. Les documents, la calculatrice, les téléphones portables, tablettes, ordinateurs ne sont pas autorisés. La qualité de la rédaction sera prise en compte.} 

\bigskip
\bigskip

\exo{(Question de cours)} Soit $E$ et $F$ deux espaces vectoriels normés et $a\in E$. 


\begin{enumerate}
	\item Donner la définition de la dérivée directionnelle d'une fonction $f:E\to F$. Une fonction qui admet des dérivées partielles en $a$ est-elle nécessairement continue en $a$ (justifier) ?
\medskip
		\begin{enumerate}
			\item Soient $\U$ un ouvert de $E$ et soit $a\in \U$ et $v\in E$ avec $v\neq 0$. On dit que $f$ admet une dérivée en $a$ suivant la direction $v$ si l'application $t\mapsto f(a+tv) $ est dérivable en $0$. Dans ce cas on note :
\[
	D_v f(a) = \lim_{t\to 0} \frac{f(a+tv) - f(a)}{t}.
\]
			\item Non, prendre $f:\R^2 \to \R $ définie par $f(x,y)= \begin{cases}\frac{xy}{x^2+y^2}, &\text{si $(x,y) \neq (0,0)$} \\ 0, & \text{si $(x,y) =(0,0)$ }\end{cases} $ qui admet des dérivées partielles en $a=(0,0)$ mais n'est pas continue en $(0,0)$.
		\end{enumerate}

\medskip

	\item Donner la définition d'un $\reg{1}$-difféomorphisme de $E$ sur $F$. En donner une caractérisation (autrement dit, énoncer le ``théorème d'inversion globale'').

\medskip
		\begin{enumerate}
			\item Soient $\U$ un ouvert de $E$ et $\mathcal V$ un ouvert de $F$. On dit que $f$ est un $\Cc^1$-difféomorphisme de $\U$ vers $\mathcal V$ si $f$ est une bijection de classe $\Cc^1$ de $\U$ sur $\mathcal V$  dont la réciproque $f^{-1}$ est de classe $\Cc^1$ sur $\mathcal V$.
\item Soit $\U$ un ouvert de $E$ et $f:\U \to F$ une application injective de classe $\Cc^1$. Alors $f$ définit un $\Cc^1$ difféomorphisme du $\U$ sur $f(\U)$ si et seulement si $d_a f$ est un isomorphisme pour tout $a\in\U$.
		\end{enumerate}
\medskip
\end{enumerate}


\exo{Forme quadratique} Les formes quadratiques suivantes sont elles positives? sont elles définies ? :
\begin{enumerate}
	\item $q(x,y,z,t) = 2xz + 2xy +x^2 + 2tx$
	\begin{align*}
		q(x,y,z,t)& =  (x^2 + 2x\left( z+y+t \right) +{\color{gray} \left( z+y+t \right)^2} ) - {\color{gray} \left( z+y+t \right)^2} \\
		& = (x+y+z+t)^2 - (z+y+t)^2
	\end{align*}
	N'est pas définie car $q(0,1,0,-1) = 0$ et n'est pas positive car  $ q(3,-1,-1,-1) < 0 < q(1,0,0,0) $.
\medskip
	
\item $q(x,y,z) = -2(x+y)^2 + (x+y+z)^2 + (x+y-z)^2$
	\medskip 

	Attention, la décomposition n'est pas en somme de carrée de formes linéaires indépendantes. Il faut développer :
	\begin{align*}
		q(x,y,z) =  2 z^2
	\end{align*}
	C'est donc clairement un forme quadratique positive. Mais elle n'est pas définie car $q(1,0,0) = 0 $.
\medskip

	\item $q(x,y)=e^{\sqrt{\pi}} x^2 + \ln(1+e) y^2 - xy$
		
		\medskip

On peut utiliser la méthode des mineurs pour éviter des calculs fastidieux. La matrice associée à $q$ est
\[
	\begin{pmatrix}
		e^{\sqrt{\pi}} & -1/2 \\ -1/2 & \ln(1+e) 
	\end{pmatrix}
\]
Avec les notations du cours : on a $\Delta_1 =e^{\sqrt{\pi}} x^2  >0$ et $\Delta_2 = 	e^{\sqrt{\pi}}\ln(1+e) -1/4 >0$ (car $\sqrt{\pi} >0$ implique $e^{\sqrt{\pi}}>1$ et $1+e >e$ implique $\ln(1+e)>1$)  et $q$ est définie positive.
\end{enumerate}

\bigskip


%\exo{Valeur absolue}
%\begin{enumerate}
%\item 		Rappeler la définition de la valeur absolue.
%\vspace*{3cm}
%\item 		Démontrer que la valeur absolue est bien une norme sur $\R$.
%\vspace*{7cm}
%\item 		Est-ce la seule norme sur $\R$ ?
%\vspace*{5cm}
%\end{enumerate}

\exo{(\'Etude de fonctions de plusieurs variables)}Soit $m\in \N\setminus \left\{ 0 \right\}$ et $f : \R^2 \to \R$ la fonction définie comme suit :
\[
	f(x,y) = \begin{cases}
		\frac{x^m y^2}{x^2 + y^2} \text{ si } \R^2 \setminus \{ (0, 0) \} \\
		0 \text{ sinon}.
	\end{cases}
\]

\begin{enumerate}
	\item  Étude de la fonction sur $\R^2 \setminus \{ (0, 0) \}$ :
		\begin{enumerate}
			\item Montrer que $f$ est continue sur $\R^2 \setminus \{ (0,0) \}$ pour tout $m \in \N\setminus \left\{ 0 \right\}$ 
	\medskip 

$f$ est continue sur $\R^2 \setminus \{ (0,0) \}$ car quotient de fonctions continues dont le dénominateur ne s’annule pas.
\medskip

			\item Calculer le gradient de $f$ pour $(x, y) \in \R^2 \setminus \{ (0, 0) \}$ pour tout $m \in \N\setminus \left\{ 0 \right\}$ ;
				
				\medskip

				\begin{align*}
					\frac{\p f}{\p x} (x,y) = \frac{x^{m-1} y^2 \left( mx^2 + my^2 -2x^2\right)}{ (x^2+y^2)^2} \\
					\frac{\p f}{\p y} (x,y) = \frac{2 x^{m+2} y }{ (x^2+y^2)^2}
				\end{align*}
	\medskip		
			\item Montrer que $f$ est de classe $\reg{1}$ sur $\R^2 \setminus \{ (0, 0) \}$ pour tout $m \in \N\setminus \left\{ 0 \right\}$ ;

				\medskip

$f$ est continue et dérivable sur $\R^2 \setminus \{ (0, 0) \}$. Ses dérivées partielles sont continues sur $\R^2 \setminus \{ (0, 0) \}$ car quotients de
fonctions continues dont les dénominateurs ne s'annulent pas. Alors $f$ est de classe $\Cc^1 (\R^2 \setminus \{ (0, 0) \} )$.

\medskip

			\item Que peut-on conclure sur la différentiabilité de $f$ sur $\R^2 \setminus \{ (0, 0) \}$ ?

\medskip

Comme $f$ est de classe $\Cc^1 (\R^2 \setminus \{ (0, 0) \} )$ alors $f$ est différentiable sur sur $\R^2 \setminus \{ (0, 0) \} $.

\medskip
		\end{enumerate}
	\item  Étude de la fonction en $(0, 0)$ :
		\begin{enumerate}
			\item Pour quelles valeurs de $m\in\N\setminus \left\{ 0 \right\}$ la fonction $f$ est-elle continue en $(0, 0)$ ?
				
				\medskip

				Pour que $f$ soit continue en $(0, 0)$ il faut que  $\lim_{(x,y) \to (0,0)} f(x,y) = 0$. En passant en coordonnées polaires  on a 
				\[
					\abs{f(x,y) } \leq r^m 
				\]
Le membre de droite de l'inégalité $r^m \xrightarrow[ (x,y) \to (0,0)]{} 0  $ pour tout $m \in \N^*$. Donc (théorème des gendarmes) $f$ est continue en $(0, 0)$ pour tout $m \in \N^* $.

			\item La fonction $f$ admet-elle des dérivées partielles en $(0, 0)$ pour tout $m\in\N\setminus \left\{ 0 \right\}$ ? si oui, les calculer.
				
				\medskip

Le gradient de $f$ en $(0, 0)$ est le vecteur de composantes
\begin{align*}
	\frac{\p f}{\p x} (0,0) =  \lim_{h\to 0} \frac{f(h,0) - f(0,0)}{h} = 0\\
\frac{\p f}{\p y} (0,0) =  \lim_{k\to 0} \frac{f(0,k) - f(0,0)}{k} = 0
				\end{align*}

			\item Pour quelles valeurs de $m\in\N\setminus \left\{ 0 \right\}$ la fonction $f$ est-elle différentiable en $(0, 0)$ ?
				
				\medskip

				$f$ est différentiable en (0, 0) si et seulement si $\lim_{(h,k) \to (0,0)} \frac{f(h,k) - f(0,0) -	\frac{\p f}{\p x} (0,0) h -	\frac{\p f}{\p y} (0,0)  k}{ \sqrt{h^2 + k^2} } = 0  $.
				On note $r(h,k) = \frac{f(h,k) - f(0,0) -	\frac{\p f}{\p x} (0,0) h -	\frac{\p f}{\p y} (0,0)  k}{ \sqrt{h^2 + k^2} } = \frac{ h^m k^2}{(h^2 + k^2)^{3/2} }$.
				\begin{enumerate}
					\item Si $m=1$ : $r(k,k) = 1/2^{3/2} \neq 0$ et $f$ n'est pas différentiable en $(0,0)$.
					\item Si $m>1$ : $\abs{r(h,k)} \leq r^{m-1}\xrightarrow[ (x,y) \to (0,0)]{} 0 $ et $f$ est bien différentiable en $(0,0)$.
				\end{enumerate}
\medskip

			\item Pour quelles valeurs de $m\in\N\setminus \left\{ 0 \right\}$ la fonction $f$ est-elle de classe $\reg{1}$ en $(0, 0)$ ?
				\begin{enumerate}
					\item si $m = 1$, $f$ n'est pas différentiable en $(0, 0)$ donc elle n'est pas de classe $\reg 1$.
					\item si $m>1$, on vérifie que les dérivées partielles sont continues en $(0,0)$. En passant en coordonnées polaires  on a :
						\[
							\abs{	\frac{\p f}{\p x}(x,y) } \leq m r^{m-1} \text{ et } \abs{	\frac{\p f}{\p y}(x,y) } \leq 2 r^{m-1} 
						\]
						On en déduit (théorème des gendarmes) que $\lim_{(x,y) \to (0,0)} \frac{\p f}{\p x}(x,y) =  0 $ et $\lim_{(x,y) \to (0,0)} \frac{\p f}{\p y}(x,y) =  0 $. La fonction $f$ est donc de classe $\reg 1$ sur $\R^2$.
				\end{enumerate}


		\end{enumerate}
\end{enumerate}
%faccanoni 4.39 p108

%\exo{(Différentielle)} Pour $( x,y,z)\in \R^3$ on note : $f(x,y,z) = (\cos(xy) \cos(z),(1+x^2)^{yz} )$. Montrer que la fonction $f$ est différentiable au point $(a,1,0)$ pour tout $a$ de $\R$ puis calculer sa différentielle en ce point.

\end{document}
