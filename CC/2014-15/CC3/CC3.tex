\documentclass[a4paper]{article}
\makeatletter
%--------------------------------------------------------------------------------

\usepackage[french]{babel}
\usepackage{amsmath}
\usepackage{amsbsy}
\usepackage{amsfonts}
\usepackage{amssymb}
\usepackage{amscd}
\usepackage{amsthm}
\usepackage{mathtools}
\usepackage{eurosym}
\usepackage{nicefrac}

\usepackage{latexsym}
\usepackage[a4paper,hmargin=20mm,vmargin=25mm]{geometry}
\usepackage{dsfont}
\usepackage[utf8]{inputenc}
\usepackage[T1]{fontenc}
\usepackage{lmodern}

\usepackage{multicol}
\usepackage[inline]{enumitem}
\setlist{nosep}
\setlist[itemize,1]{,label=$-$}


\newenvironment{modenumerate}
  {\enumerate\setupmodenumerate}
  {\endenumerate}

\newif\ifmoditem
\newcommand{\setupmodenumerate}{%
  \global\moditemfalse
  \let\origmakelabel\makelabel
  \def\moditem##1{\global\moditemtrue\def\mesymbol{##1}\item}%
  \def\makelabel##1{%
    \origmakelabel{##1\ifmoditem\rlap{\mesymbol}\fi\enspace}%
    \global\moditemfalse}%
}


\usepackage{sectsty}
%\sectionfont{}
\allsectionsfont{\color{astral}\normalfont\sffamily\bfseries\normalsize}

\usepackage{graphicx}
\usepackage{tikz}
\usetikzlibrary{babel}
\usepackage{tikz,tkz-tab}

\usepackage[babel=true, kerning=true]{microtype}


\usepackage{pgfplots}
\usepgfplotslibrary{fillbetween}
\pgfplotsset{compat=newest}
\usepgfplotslibrary{external} 
\tikzexternalize[prefix=./output_latex/]
%\DeclareSymbolFont{RalphSmithFonts}{U}{rsfs}{m}{n}
%\DeclareSymbolFontAlphabet{\mathscr}{RalphSmithFonts}
%\def\mathcal#1{{\mathscr #1}}



\providecommand{\abs}[1]{\left|#1\right|}
\providecommand{\norm}[1]{\left\Vert#1\right\Vert}
\providecommand{\U}{\mathcal{U}}
\providecommand{\R}{\mathbb{R}}
\providecommand{\Cc}{\mathcal{C}}
\providecommand{\reg}[1]{\mathcal{C}^{#1}}
\providecommand{\1}{\mathds{1}}
\providecommand{\N}{\mathbb{N}}
\providecommand{\Z}{\mathbb{Z}}
\providecommand{\p}{\partial}
\providecommand{\one}{\mathds{1}}
\providecommand{\E}{\mathbb{E}}\providecommand{\V}{\mathbb{V}}
\renewcommand{\P}{\mathbb{P}}


%Operateur
\providecommand{\abs}[1]{\left\lvert#1\right\rvert}
\providecommand{\sabs}[1]{\lvert#1\rvert}
\providecommand{\babs}[1]{\bigg\lvert#1\bigg\rvert}
\providecommand{\norm}[1]{\left\lVert#1\right\rVert}
\providecommand{\bnorm}[1]{\bigg\lVert#1\bigg\rVert}
\providecommand{\snorm}[1]{\lVert#1\rVert}
\providecommand{\prs}[1]{\left\langle #1\right\rangle}
\providecommand{\sprs}[1]{\langle #1\rangle}
\providecommand{\bprs}[1]{\bigg\langle #1\bigg\rangle}

\DeclareMathOperator{\deet}{Det}
\DeclareMathOperator{\hess}{Hess}
\DeclareMathOperator{\jac}{Jac}


\newcommand\rst[2]{{#1}_{\restriction_{#2}}}



% generate breakable white space allowing students to write notes.

\usepackage[framemethod=tikz]{mdframed}

\mdfdefinestyle{response}{
	leftmargin=.01\textwidth,
	rightmargin=.01\textwidth,
	linewidth=1pt
	hidealllines=false,
	leftline=true,
	rightline=true,topline=true,bottomline=true,
	skipabove=0pt,
	%innertopmargin=-5pt,
	%innerbottommargin=2pt,
	linecolor=black,
	innerrightmargin=0pt,
	}



\newcommand*{\DivideLengths}[2]{%
  \strip@pt\dimexpr\number\numexpr\number\dimexpr#1\relax*65536/\number\dimexpr#2\relax\relax sp\relax
}

\providecommand{\rep}[1]{$ $ \newline \begin{mdframed}[style=response]  
	
	\vspace*{\DivideLengths{#1}{3cm}cm}
	\pagebreak[1]	
	\vspace*{\DivideLengths{#1}{3cm}cm}
	\pagebreak[1]		
	\vspace*{\DivideLengths{#1}{3cm}cm}   \end{mdframed}}

\providecommand{\blanc}[1]{$ $ \newline 
	
	\vspace*{\DivideLengths{#1}{3cm}cm}
	\pagebreak[1]	
	\vspace*{\DivideLengths{#1}{3cm}cm}
	\pagebreak[3]		
	\vspace*{\DivideLengths{#1}{3cm}cm}}

\usepackage{ifthen}

\newcommand{\eno}[1]{%
	\ifthenelse{\equal{\version}{eno}}{#1}{}%
}
\newcommand{\cor}[1]{%
        \ifthenelse{\equal{\version}{cor}}{
\medskip 

{\small \color{gray} #1}

\medskip 
}{}
}

%------------------------------------------------------------------------------
%\DeclareUnicodeCharacter{00A0}{~}
\makeatother



%-----------------------------------------------------------------------------
\begin{document}

\noindent Université Montpellier 2 \hfill Année 2014-2015

\noindent HLMA 410
 


\bigskip

\begin{center}
{\large \sffamily\bfseries Contrôle continu 3}
\end{center}

\textit{Durée 1h30. Les documents, la calculatrice, les téléphones portables, tablettes, ordinateurs ne sont pas autorisés. La qualité de la rédaction sera prise en compte.} 

\bigskip
\bigskip

\exo{(Question de cours)}Soit $A$ un événement aléatoire. On appelle variable aléatoire indicatrice de $A$ une variable aléatoire $\one_A$ qui vaut $1$ si $A$ est réalisé et $0$ sinon. Soit $A,B,C \subset \Omega$ :
\begin{enumerate}
	\item Exprimer en fonction de $\one_A$ et $\one_B$ les variables $\1_{A\cup B}$, $\1_{A \cap B}$ et $\one_{A\Delta B}$. Faire les démontrations!
		\vspace*{15cm}

	\item Que dire de $A$ et $B$ si $\1_A \leq \1_B$
		\vspace*{3cm}
\end{enumerate}

\newpage

\exo{(Longueur de courbes)} Calculer la longueur des courbes param\'etr\'ees suivantes :
\begin{enumerate}
\item $\gamma(t)=((1-t)^2e^t,2(1-t)e^t)$, $t\in [0,1]$.
	\vspace*{16cm}
\item $\gamma$ est la courbe d'\'equation polaire $r(t)=\sin(t)$, $\theta(t)=t$.
	\vfill
	\vspace*{8cm}
\end{enumerate}


\exo{(Int\'egrale de Gauss)} Pour $R>0$, on pose $D_R=\{ (x,y)\in\R^2, x^2 + y^2\le R^2\}$ et $\Delta_R=[-R,R]\times [-R,R]$.
\begin{enumerate}
\item Montrer que $D_R\subset \Delta_R\subset D_{\sqrt{2}R}$.  En d\'eduire que :
	\[\iint_{D_R} e^{-x^2-y^2}dxdy\le\iint_{\Delta_R} e^{-x^2-y^2}dxdy\le  \iint_{D_{\sqrt{2}R}} e^{-x^2-y^2}dxdy\].

	\vspace*{7cm}


\item En utilisant les coordonn\'ees polaires, calculer $\displaystyle \iint_{D_R} e^{-x^2-y^2}dxdy$.
	\vspace*{7cm}

\item Montrer que $\displaystyle \iint_{\Delta_R} e^{-x^2-y^2}dxdy=\left(\int_{-R}^R e^{-t^2}dt\right)}^2$
	\vspace*{7cm}

\item En d\'eduire la valeur de $\displaystyle \int_{-\infty}^\infty e^{-t^2}dt = \lim_{R\to + \infty} \int_{-R}^R e^{-t^2} dt$.

\end{enumerate}

%\exo{Longueur de la Deltoïde }Considérons la courbe $\Gamma$ paramétrée $\phi : [0; 2\pi ] \to \R^2$ définie par $\phi(t) = (2 \cos(t ) + \cos(2t ), 2\sin(t ) - \sin(2t))$.
%Calculer la longueur de  $\Gamma$.
%\vspace*{12cm}


%\exo{}Calculer l'aire de 
%\[
	%\mathcal D = \left\{ (x,y) \in \R^2 | y<0, x^2 < y^4(y+4) \right\}.
%\]
%\begin{center}
	%\begin{tikzpicture}
		%\begin{axis}
			%\addplot[domain=-4:0,samples=100] { - x^1 * (x+4 )^(.5) };		
			%\addplot[domain=-4:0,samples=100] {  x^1 * (x+4 )^(.5) };		
		%\end{axis}
	%\end{tikzpicture}
%\end{center}
\end{document}
