\documentclass{tp_um}
\makeatletter
%--------------------------------------------------------------------------------

\usepackage[french]{babel}
\usepackage{amsmath}
\usepackage{amsbsy}
\usepackage{amsfonts}
\usepackage{amssymb}
\usepackage{amscd}
\usepackage{amsthm}
\usepackage{mathtools}
\usepackage{eurosym}
\usepackage{nicefrac}

\usepackage{latexsym}
\usepackage[a4paper,hmargin=20mm,vmargin=25mm]{geometry}
\usepackage{dsfont}
\usepackage[utf8]{inputenc}
\usepackage[T1]{fontenc}
\usepackage{lmodern}

\usepackage{multicol}
\usepackage[inline]{enumitem}
\setlist{nosep}
\setlist[itemize,1]{,label=$-$}


\newenvironment{modenumerate}
  {\enumerate\setupmodenumerate}
  {\endenumerate}

\newif\ifmoditem
\newcommand{\setupmodenumerate}{%
  \global\moditemfalse
  \let\origmakelabel\makelabel
  \def\moditem##1{\global\moditemtrue\def\mesymbol{##1}\item}%
  \def\makelabel##1{%
    \origmakelabel{##1\ifmoditem\rlap{\mesymbol}\fi\enspace}%
    \global\moditemfalse}%
}


\usepackage{sectsty}
%\sectionfont{}
\allsectionsfont{\color{astral}\normalfont\sffamily\bfseries\normalsize}

\usepackage{graphicx}
\usepackage{tikz}
\usetikzlibrary{babel}
\usepackage{tikz,tkz-tab}

\usepackage[babel=true, kerning=true]{microtype}


\usepackage{pgfplots}
\usepgfplotslibrary{fillbetween}
\pgfplotsset{compat=newest}
\usepgfplotslibrary{external} 
\tikzexternalize[prefix=./output_latex/]
%\DeclareSymbolFont{RalphSmithFonts}{U}{rsfs}{m}{n}
%\DeclareSymbolFontAlphabet{\mathscr}{RalphSmithFonts}
%\def\mathcal#1{{\mathscr #1}}



\providecommand{\abs}[1]{\left|#1\right|}
\providecommand{\norm}[1]{\left\Vert#1\right\Vert}
\providecommand{\U}{\mathcal{U}}
\providecommand{\R}{\mathbb{R}}
\providecommand{\Cc}{\mathcal{C}}
\providecommand{\reg}[1]{\mathcal{C}^{#1}}
\providecommand{\1}{\mathds{1}}
\providecommand{\N}{\mathbb{N}}
\providecommand{\Z}{\mathbb{Z}}
\providecommand{\p}{\partial}
\providecommand{\one}{\mathds{1}}
\providecommand{\E}{\mathbb{E}}\providecommand{\V}{\mathbb{V}}
\renewcommand{\P}{\mathbb{P}}


%Operateur
\providecommand{\abs}[1]{\left\lvert#1\right\rvert}
\providecommand{\sabs}[1]{\lvert#1\rvert}
\providecommand{\babs}[1]{\bigg\lvert#1\bigg\rvert}
\providecommand{\norm}[1]{\left\lVert#1\right\rVert}
\providecommand{\bnorm}[1]{\bigg\lVert#1\bigg\rVert}
\providecommand{\snorm}[1]{\lVert#1\rVert}
\providecommand{\prs}[1]{\left\langle #1\right\rangle}
\providecommand{\sprs}[1]{\langle #1\rangle}
\providecommand{\bprs}[1]{\bigg\langle #1\bigg\rangle}

\DeclareMathOperator{\deet}{Det}
\DeclareMathOperator{\hess}{Hess}
\DeclareMathOperator{\jac}{Jac}


\newcommand\rst[2]{{#1}_{\restriction_{#2}}}



% generate breakable white space allowing students to write notes.

\usepackage[framemethod=tikz]{mdframed}

\mdfdefinestyle{response}{
	leftmargin=.01\textwidth,
	rightmargin=.01\textwidth,
	linewidth=1pt
	hidealllines=false,
	leftline=true,
	rightline=true,topline=true,bottomline=true,
	skipabove=0pt,
	%innertopmargin=-5pt,
	%innerbottommargin=2pt,
	linecolor=black,
	innerrightmargin=0pt,
	}



\newcommand*{\DivideLengths}[2]{%
  \strip@pt\dimexpr\number\numexpr\number\dimexpr#1\relax*65536/\number\dimexpr#2\relax\relax sp\relax
}

\providecommand{\rep}[1]{$ $ \newline \begin{mdframed}[style=response]  
	
	\vspace*{\DivideLengths{#1}{3cm}cm}
	\pagebreak[1]	
	\vspace*{\DivideLengths{#1}{3cm}cm}
	\pagebreak[1]		
	\vspace*{\DivideLengths{#1}{3cm}cm}   \end{mdframed}}

\providecommand{\blanc}[1]{$ $ \newline 
	
	\vspace*{\DivideLengths{#1}{3cm}cm}
	\pagebreak[1]	
	\vspace*{\DivideLengths{#1}{3cm}cm}
	\pagebreak[3]		
	\vspace*{\DivideLengths{#1}{3cm}cm}}

\usepackage{ifthen}

\newcommand{\eno}[1]{%
	\ifthenelse{\equal{\version}{eno}}{#1}{}%
}
\newcommand{\cor}[1]{%
        \ifthenelse{\equal{\version}{cor}}{
\medskip 

{\small \color{gray} #1}

\medskip 
}{}
}

%------------------------------------------------------------------------------
%\DeclareUnicodeCharacter{00A0}{~}
\makeatother


\newcommand{\miniscule}{\@setfontsize\miniscule{5}{6}}

%\def\version{eno}
\def\version{cor}
%-----------------------------------------------------------------------------

\title{\large \sffamily\bfseries Contrôle Terminal}
\ue{HLMA410}


%-----------------------------------------------------------------------------
\begin{document}

\maketitle
\textit{Durée 2h00. Les documents, la calculatrice, les téléphones portables, tablettes, ordinateurs ne sont pas autorisés. Les exercices sont indépendants. La qualité de la rédaction sera prise en compte.} 

\bigskip
\bigskip
\exo{}On consid\`ere la fonction $ f: \R^2 \to \R$, 
$$
f(x,y) = x \sqrt{ x^2 + y^2}.
$$
\begin{enumerate}
	\item \'Etudier la continuité de $f$ en chaque point de $ \R^2$. 

\cor{
L'application $(x,y)\mapsto \|(x,y)\|$ (où $\| \cdot \|$ désigne la norme euclidienne) est continue. L'application $f$ est continue comme le produit de deux applications continues.
}

	\item Calculer les dérivées partielles de $f$ en chaque point o\`u elles existent. 

		\cor{

La fonction est $\mathcal C^1$ partout sauf peut être en $(0,0)$ (à cause de la racine carrée). On a 
\[\frac{\partial f}{\partial x} (x,y) =
\begin{cases}
	\sqrt{x^2+y^2} +\frac{x^2}{\sqrt{x^2+y^2}}, \text{ si $(x,y) \in \R^2 \setminus (0,0)$} \\
	\lim_{h\to 0} \frac{f(h,0) - f(0,0)}{h} = \lim_{h\to 0} |h| = 0 ,\text{ sinon}
	\end{cases} \]
et
\[\frac{\partial f}{\partial y} (x,y) =
\begin{cases}
	\frac{xy}{\sqrt{x^2+y^2}}, \text{ si $(x,y) \in \R^2 \setminus (0,0)$} \\
	\lim_{h\to 0} \frac{f(0,h) - f(0,0)}{h} = 0 ,\text{ sinon}
	\end{cases} \]

		}

	\item \'Etudier la continuit\'e des d\'eriv\'ees partielles de $f$ sur $ \R ^2 \setminus{(0,0)}$, puis en $(0,0)$.
		
		\cor{		
		Sur $ \R ^2 \setminus{(0,0)}$, les dérivées partielles sont clairement $\mathcal C^0$. Reste à vérifier en l'origne. On a 
		\[
			\left|\frac{\partial f}{\partial x} (x,y) - \frac{\partial f}{\partial x} (0,0) \right| \leq \sqrt{x^2 + y^2} +\frac{{x^2 + y^2} }{\sqrt{x^2 + y^2}} =2 \sqrt{x^2 + y^2}
	\]
	et
	\[
			\left|\frac{\partial f}{\partial y} (x,y)- \frac{\partial f}{\partial y} (0,0) \right| \leq\frac{{x^2 + y^2} }{\sqrt{x^2 + y^2}} =   \sqrt{x^2 + y^2}
	\]
	Ainsi, les dérivée partielles sont continues en l'origine et $f$ est $\mathcal C^1$ sur $\R^2$.
}


	\item \'Etudier la  différentiabilité de $f$ en chaque point de $ \R^2$.

\cor{
Comme $f$ est $\mathcal C^1$ sur $\R^2$, elle est différentiable partout.
}

	\item %Calculer le développement limité de $f$ à l'ordre 1 en $(1,0)$. 
		En déduire une valeur approchée de $f(1.01,0)$. %$d_{(1, 0) } f ( h_1, h_2)$.

		\cor{
		La différentielle de $f$ en $(1,0)$ est l'application linéaire $L:\R^2 \to \R$ qui $(h_1,h_2) \mapsto (1 + 1) h_1 + 0\times h_2 =2 h_1$. On a donc le DL
	\[
	f(1+h_1,h_2) = f(1,0) + L(h_1,h_2) + o(\|(h_1,h_2\|) = 1 + 2h_1 + o(\|(h_1,h_2)\|)
\]
Par suite,   $f(1.01,0) \approx 1+0.02 = 1.02$.
	}
	%\item Sans calculer la hessienne de $f$, déterminer si l'origine est un point de maximum, de minimum, un point selle,
\end{enumerate}

\exo{} On considère le champ de vecteurs 
\begin{align*}
	F: \R^2 & \to \R^2 \\
	(x,y) & \mapsto (2xy+y\cos(xy), x\cos(xy) +x^2-1)
\end{align*}
%\begin{center}
	%\begin{tikzpicture}
		%\begin{axis}[,%xtick=\empty,
			     %%ytick=\empty,
			     %ztick=\empty ,
			     %xlabel=$x$,ylabel=$y$,xlabel=$x$,ylabel=$y$,domain=-2:2,% y domain=-1:1,
			     %view={0}{90},
			     %xmax=2, xmin=-2,
			     %ymax=2, ymin=-2,%axis equal
			%]
			%\addplot3[blue, quiver={u={2*x*y+y*cos(x*y)}, v={x*cos(x*y) +x^2-1}, scale arrows=.1}, -stealth,samples=25] {0};
			%\addplot3[contour gnuplot={number={25}, labels=true},samples=60] gnuplot {sin(x*y) +y* x**2  -y };
		%\end{axis}
	%\end{tikzpicture}
%\end{center}

%\begin{enumerate}
	%\item Trouver une fonction $f:\R^2 \to \R$ de classe $\mathcal C^2$ telle que $F = \nabla f$, où $\nabla f =(\frac{\partial f}{\partial x} , \frac{\partial f}{\partial y})$.
	%\item Déterminer les points critiques de la fonction $f$.
	%\item Calculer la matrice Hessienne de $f$ en tout point de $\R^2$. 
	%\item Déterminer la nature de la matrice Hessienne en chacun des points critiques  (\textit{i.e.} la forme quadratique associée est-elle définie ? est-elle positive ? est elle négative ?)
	%\item Peut-on déduire, de la question précédente, la nature des points critiques (\textit{i.e.} minimum, maximum, strict, global) ? Justifier!
	%\item \'Etudier la restriction de $f$ a la droite $y=0$. Quelle(s) information(s) cela apporte sur la nature des points critiques (\textit{i.e.} minimum, maximum, strict, global) ?
	%\item Soit $\Gamma$ la courbe paramétrée par $\phi : t \mapsto (t,t^2)$ pour $t \in [0,1]$. Calculer  $\int_{\Gamma} \langle F , d\phi \rangle$.
%\end{enumerate}
On admettra provisoirement l'existence d'une fonction $f:\R^2 \to \R$ telle que $F = (\frac{\partial f}{\partial x} , \frac{\partial f}{\partial y})$.
\begin{enumerate}
    \item Déterminer les deux points critiques de la fonction $f$. On note $A$ le point critiques avec une abscisse négative et $B$ le point critiques avec une abscisse positive.
        \cor{
On cherche $(x,y) \in\R^2$ tel que
\begin{align*}
F(x,y) = (0,0) \Leftrightarrow 	\begin{cases}
		y(2x + \cos(xy) ) = 0 \\
x\cos(xy) +x^2-1 =0
	\end{cases}
\end{align*}
Il faut alors considérer les deux cas suivants :
\begin{itemize}
	\item $y=0$ qui donne $x^2 + x -1 =0$.  On a alors deux solutions $ A=(\frac{-1 -\sqrt{5}}{2} ,0)$ et $ B=(\frac{-1 +\sqrt{5}}{2} ,0)$.
	\item $\cos(xy) = -2x$ qui donne $-x^2 -1 = 0$ qui n'a pas de solution.
\end{itemize}
	En résumé, il y a deux points critiques $A$ et $B$.
        }
	\item Calculer la matrice Hessienne de $f$ en tout point de $\R^2$. Vérifier que $\hess_f (A) = - \hess_f (B) = -\sqrt{5} \begin{pmatrix}
			0 & 1 \\ 1 & 0
		\end{pmatrix} $.  
	
        \cor{
		On a 
		\[
			\hess_f (x,y) = \begin{pmatrix}
	2 y - y^2 \sin(xy)	& 2x + \cos(xy) - xy\sin(xy)	 \\
	2x + \cos(xy) - xy\sin(xy)	& -x^2\sin(xy)
		\end{pmatrix}.
	\]

        }
    \item Déterminer la nature de la matrice Hessienne de $f$ en chacun des points critiques  (\textit{i.e.} la forme quadratique associée est-elle définie ? est-elle positive ? est elle négative ?)
        \cor{
		On  a $\hess_f(A) = -\sqrt{5} \begin{pmatrix}
			0 & 1 \\ 1 & 0
		\end{pmatrix}$. Ainsi la forme quadratique 
		\begin{align*}
			Q_A : \R^2 &\to \R \\ (h_1,h_2) &\mapsto (h_1,h_2) \hess_f(A) (h_1,h_2)^t = -2 \sqrt{5} h_1h_2
		\end{align*}
		n'est pas définie car $Q_A(0,1) =0 $, n'est pas positive, n'est pas négative car $Q_A(1,1) < 0 < Q(-1,1)$.

		On a $\hess_f (B) = - \hess_f (A)$ et les mêmes remarques s'appliquent.

        }
	\item Peut-on déduire, de la question précédente, la nature des points critiques de $f$ (\textit{i.e.} minimum, maximum, strict, global) ? Justifier!
        \cor{
On ne peut pas appliquer les ``conditions suffisantes d'ordre 2'' pour caractériser les points car les Hessiennes ne sont pas définies. Mais on peut remarquer que $\det(\hess_f(A)), \det(\hess_f(B)) < 0$ et les points  $A,B$  sont donc des points selles.

        }
%\end{enumerate}
%On étudie maintenant le potentiel $f$ :  
%\begin{enumerate}[resume]
	\item Trouver une fonction $f:\R^2 \to \R$ de classe $\mathcal C^2$ telle que $\nabla f = F$.
        \cor{
		On intègre une première fois en $x$ la fonction $\frac{\partial f}{\partial x}$ : 
\[
	\int (2xy+y\cos(xy)) dx =  y x^2 +\sin(xy) + \varphi_1(y)
\]
où $\varphi_1:\R\to\R $ est $\Cc^1$. Puis on intégre en $y$ la fonction $\frac{\partial f}{\partial y} $ :
\[
\int (x\cos(xy) +x^2-1)  dy =  y x^2 +\sin(xy) -y + \varphi_2(x)
\]
où $\varphi_2:\R\to\R $ est $\Cc^1$.  Ainsi, la fonction  $f(x,y) = \sin(xy) + x^2 y - y$ convient.

        }
%	\item \'Etudier la restriction de $f$ à la droite $y=0$. Quelle(s) information(s) cela apporte sur la nature des points critiques (\textit{i.e.} minimum, maximum, strict, global) ?
	\item Soit $\Gamma$ la courbe paramétrée par $\phi : t \mapsto (t,t^2)$ pour $t \in [0,1]$. Calculer la circulation de $F$  le long de $\Gamma$.
        \cor{
Le champ de vecteur $F$ est un champ de gradient. On a 
		\[\int_{\Gamma} \prs{F,d\phi} = f(\phi(1)) - f(\phi(0)) = f(1,1) - f(0,0)= \sin(1).\]

        }
\end{enumerate}


\exo{}
\begin{enumerate}
	\item Soient $\alpha, \beta, R > 0$. Calculer l'aire de l'ellipse $E \subset R^2$ définie par l'équation
		\[
			\frac{x^2}{\alpha^2} + \frac{y^2}{\beta^2} < R^2.
		\]
		{\it Indication : On pourra utiliser le changement de variables $(u,v) \mapsto (\alpha u, \beta v)$.}

        \cor{
		On pose $D = \left\{ (x,y) \in\R^2 |	\frac{x^2}{\alpha^2} + \frac{y^2}{\beta^2} < R^2  \right\}$  et $ \phi(u,v) = (\alpha u = x, \beta v = y )$.  On a $\Delta = \phi^{-1}(D) = \left\{ (u,v) \in\R^2 |	u^2 + v^2 < R^2  \right\} $. Comme le changement de variable $\phi $ est linéaire,  le déterminant du Jacobien est constant et égal à $\alpha\beta>0$. On a 
\[
    \operatorname{Aire}(E) = \iint_D dx dy = \alpha\beta \iint_\Delta dudv = \alpha\beta \int_0^{2\pi}d\theta \int_0^R rdr = \alpha\beta\pi R^2.
\]
où on a utilisé un changement de variables en coordonnées polaires. Remarquer enfin que si $\alpha = \beta =1$ on retrouve une formule bien connue.

        }
	\item Soit $H_{a,b,c} \subset \R^3$ le solide défini par
\[
	H_{a,b,c} = \left\{  (x,y,z) \in \R^2 | -1 < z <2, \frac{x^2}{a^2} + \frac{y^2}{b^2} - \frac{z^2}{c^2} <1 \right\}
\]
		où $a$, $b$ et $c$ sont des réels strictement positifs. Calculer le volume de $H_{a,b,c}$.
       

        \cor{



On a $ 
	H_{a,b,c} = \left\{  (x,y,z) \in \R^2 | -1 < z <2, \frac{x^2}{a^2} + \frac{y^2}{b^2} < 1+ \frac{z^2}{c^2}  \right\}
$ et la formule d'intégration par tranche donne :
\begin{align*}
    \operatorname{Vol}(H_{a,b,c}) & = \iiint_{H_{a,b,c}} dxdydz = \int_{-1}^2 \operatorname{Aire}(D_z) dz
\end{align*}
où $D_z = \left\{ (x,y) \in\R^2 | \frac{x^2}{a^2} + \frac{y^2}{b^2} < 1 + \frac{z^2}{c^2}.
\right\}$ et $\operatorname{Vol}(D_z) = ab\pi(1 + \frac{z^2}{c^2} )$ d'après la question 1. On a alors
 \begin{align*}
     \operatorname{Vol}(H_{a,b,c}) & = \pi ab\int_{-1}^2 \left(1+\frac{z^2}{c^2}\right)  dz = 3ab\pi\left( 1+ \frac{1}{c^2} \right)
 \end{align*}

        }

	\item On suppose que $a = b = 1$ et $c = 2$. Calculer l'intégrale
		\[
			I = \iiint_{H_{1,1,2}} z e^{x^2+y^2}dxdydz.
		\]
        \cor{

		Faire un changement de coordonnées cylindriques:
\begin{align*}
	I &= \int_{-1}^2 \left( 2\pi \int_0^{\sqrt{1+ \frac{z^2}{4} }}  e^{r^2} rdr \right)zdz \\
	 & =  \pi \int_{-1}^{2} \left[ e^{r^2} \right]_0^{\sqrt{1+ \frac{z^2}{4} }}z dz \\
	 & = \pi \left[  2e^{1+ \frac{z^2}{4} } - \frac{z^2}{2}  \right]_{-1}^{2}  \\
	 & = \pi (2e^2 -2e^{\frac 5 4} - \frac 3 2)
\end{align*}
        }
\end{enumerate}




\end{document}

