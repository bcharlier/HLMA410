\documentclass{tp_um}
\makeatletter
%--------------------------------------------------------------------------------

\usepackage[french]{babel}
\usepackage{amsmath}
\usepackage{amsbsy}
\usepackage{amsfonts}
\usepackage{amssymb}
\usepackage{amscd}
\usepackage{amsthm}
\usepackage{mathtools}
\usepackage{eurosym}
\usepackage{nicefrac}

\usepackage{latexsym}
\usepackage[a4paper,hmargin=20mm,vmargin=25mm]{geometry}
\usepackage{dsfont}
\usepackage[utf8]{inputenc}
\usepackage[T1]{fontenc}
\usepackage{lmodern}

\usepackage{multicol}
\usepackage[inline]{enumitem}
\setlist{nosep}
\setlist[itemize,1]{,label=$-$}


\newenvironment{modenumerate}
  {\enumerate\setupmodenumerate}
  {\endenumerate}

\newif\ifmoditem
\newcommand{\setupmodenumerate}{%
  \global\moditemfalse
  \let\origmakelabel\makelabel
  \def\moditem##1{\global\moditemtrue\def\mesymbol{##1}\item}%
  \def\makelabel##1{%
    \origmakelabel{##1\ifmoditem\rlap{\mesymbol}\fi\enspace}%
    \global\moditemfalse}%
}


\usepackage{sectsty}
%\sectionfont{}
\allsectionsfont{\color{astral}\normalfont\sffamily\bfseries\normalsize}

\usepackage{graphicx}
\usepackage{tikz}
\usetikzlibrary{babel}
\usepackage{tikz,tkz-tab}

\usepackage[babel=true, kerning=true]{microtype}


\usepackage{pgfplots}
\usepgfplotslibrary{fillbetween}
\pgfplotsset{compat=newest}
\usepgfplotslibrary{external} 
\tikzexternalize[prefix=./output_latex/]
%\DeclareSymbolFont{RalphSmithFonts}{U}{rsfs}{m}{n}
%\DeclareSymbolFontAlphabet{\mathscr}{RalphSmithFonts}
%\def\mathcal#1{{\mathscr #1}}



\providecommand{\abs}[1]{\left|#1\right|}
\providecommand{\norm}[1]{\left\Vert#1\right\Vert}
\providecommand{\U}{\mathcal{U}}
\providecommand{\R}{\mathbb{R}}
\providecommand{\Cc}{\mathcal{C}}
\providecommand{\reg}[1]{\mathcal{C}^{#1}}
\providecommand{\1}{\mathds{1}}
\providecommand{\N}{\mathbb{N}}
\providecommand{\Z}{\mathbb{Z}}
\providecommand{\p}{\partial}
\providecommand{\one}{\mathds{1}}
\providecommand{\E}{\mathbb{E}}\providecommand{\V}{\mathbb{V}}
\renewcommand{\P}{\mathbb{P}}


%Operateur
\providecommand{\abs}[1]{\left\lvert#1\right\rvert}
\providecommand{\sabs}[1]{\lvert#1\rvert}
\providecommand{\babs}[1]{\bigg\lvert#1\bigg\rvert}
\providecommand{\norm}[1]{\left\lVert#1\right\rVert}
\providecommand{\bnorm}[1]{\bigg\lVert#1\bigg\rVert}
\providecommand{\snorm}[1]{\lVert#1\rVert}
\providecommand{\prs}[1]{\left\langle #1\right\rangle}
\providecommand{\sprs}[1]{\langle #1\rangle}
\providecommand{\bprs}[1]{\bigg\langle #1\bigg\rangle}

\DeclareMathOperator{\deet}{Det}
\DeclareMathOperator{\hess}{Hess}
\DeclareMathOperator{\jac}{Jac}


\newcommand\rst[2]{{#1}_{\restriction_{#2}}}



% generate breakable white space allowing students to write notes.

\usepackage[framemethod=tikz]{mdframed}

\mdfdefinestyle{response}{
	leftmargin=.01\textwidth,
	rightmargin=.01\textwidth,
	linewidth=1pt
	hidealllines=false,
	leftline=true,
	rightline=true,topline=true,bottomline=true,
	skipabove=0pt,
	%innertopmargin=-5pt,
	%innerbottommargin=2pt,
	linecolor=black,
	innerrightmargin=0pt,
	}



\newcommand*{\DivideLengths}[2]{%
  \strip@pt\dimexpr\number\numexpr\number\dimexpr#1\relax*65536/\number\dimexpr#2\relax\relax sp\relax
}

\providecommand{\rep}[1]{$ $ \newline \begin{mdframed}[style=response]  
	
	\vspace*{\DivideLengths{#1}{3cm}cm}
	\pagebreak[1]	
	\vspace*{\DivideLengths{#1}{3cm}cm}
	\pagebreak[1]		
	\vspace*{\DivideLengths{#1}{3cm}cm}   \end{mdframed}}

\providecommand{\blanc}[1]{$ $ \newline 
	
	\vspace*{\DivideLengths{#1}{3cm}cm}
	\pagebreak[1]	
	\vspace*{\DivideLengths{#1}{3cm}cm}
	\pagebreak[3]		
	\vspace*{\DivideLengths{#1}{3cm}cm}}

\usepackage{ifthen}

\newcommand{\eno}[1]{%
	\ifthenelse{\equal{\version}{eno}}{#1}{}%
}
\newcommand{\cor}[1]{%
        \ifthenelse{\equal{\version}{cor}}{
\medskip 

{\small \color{gray} #1}

\medskip 
}{}
}

%------------------------------------------------------------------------------
%\DeclareUnicodeCharacter{00A0}{~}
\makeatother


\newcommand{\miniscule}{\@setfontsize\miniscule{5}{6}}

%\def\version{eno}
\def\version{cor}
%-----------------------------------------------------------------------------

\title{\large \sffamily\bfseries Contrôle Terminal}
\ue{HLMA410}


%-----------------------------------------------------------------------------
\begin{document}

\maketitle
\textit{Durée 1h30. Les documents, la calculatrice, les téléphones portables, tablettes, ordinateurs ne sont pas autorisés. Les exercices sont indépendants. La qualité de la rédaction sera prise en compte.} 

\bigskip
\bigskip

\exo{} On consid\`ere le domaine $D=\{(x,y,z)\in \R^3\; ;\quad  0\leq z\leq 1, \;\; x^2+y^2\leq z^4\,\}$.
\begin{enumerate}
\item Dessiner $D$.

    \cor{

    On a $D=\{ (x,y,z)\in \R^3 | 0\leq z\leq 1, \underbrace{0\leq \sqrt{x^2 + y^2} \leq z^2}_{=A_{z^2}}\, \}$. On note $A_{z^2}$ le disque de rayon $z^2>0$.  Le domaine $D$ est donc la partie comprise  au dessus du graphe de la fonction $(x,y)\mapsto (x^2+y^2)^{1/4}$:
        \begin{center}
            \begin{tikzpicture}[scale=.5]
                \begin{axis}[,xlabel=$x$,ylabel=$y$,xmin=-1,xmax=1,xmin=-1,ymax=1,zmin=0, zmax=1]%,xtick=\empty,ytick=\empty,ztick=\empty ]
                    % \addplot3[surf,opacity=.7,samples=50, domain=-1:1] gnuplot {( (x**2+y**2) <1)?(x**2 + y**2)**(0.25):3};
                    %\addplot[domain = -pi:pi, parametric, samples = 100] gnuplot {sin(t),cos(t)};
                    \addplot3[surf,opacity=.7,samples=70, domain=-1:1] gnuplot {( (x**2+y**2) <1)?(x**2 + y**2)**(.25):1};
                \end{axis}
            \end{tikzpicture}
        \end{center}
    }

\item Calculer son volume $V$.

    \cor{
     En appliquant la formule de sommation par tranche on a:
    \[
        V = \int_0^1 \Big( \underbrace{\iint_{A_{z^2}} dxdy}_{=\pi z^4}\Big)dz % =\int_0^1 \int_0^{z^2} \int_0^{2\pi}d\theta rdr dz 
        =  \pi \int_0^1  z^4 dz = \frac \pi 5 
    \]
}
\item Calculer son centre de gravit\'e  $G =  \frac{1}{\vol D} \left( \iiint_D x dxdydz, \iiint_D y dxdydz , \iiint_D z dxdydz \right)$.
    % sur l'axe $Oz$: $z_G=\frac{1}{V}\iiint_D z\;dx\, dy\, dz$. 

\cor{


    Le centre de gravité de $D$ appartient clairement à l'axe de révolution $Oz$. On peut se convaincre que les 2 premières coordonnées s'annulent bien, car on a 
    \[
        \iiint_D x dxdydz =  \int_0^1 \left( \iint_{A_{z^2}} x dxdy \right)dz =  \int_0^1 \big( \int_0^{z2} r dr \underbrace{\int_0^{2\pi} \cos \theta d\theta}_{=0} \big ) dz =0, 
    \]
    et
    \[
        \iiint_D y dxdydz =  \int_0^1 \left( \iint_{A_{z^2}} y dxdy \right)dz =  \int_0^1 \big( \int_0^{z^2} r dr \underbrace{\int_0^{2\pi} \sin \theta d\theta}_{=0} \big ) dz =0 .
    \]
Reste à déterminer la coordonnée sur $Oz$:
\[
   \frac 1V \iiint_D z dxdydz = \frac 1V \int_0^1 z \left( \iint_{A_{z^2}} dxdy \right)dz = \frac \pi V \int_0^1 z^5 dz =  \frac 5 6.
\]
Pour conclure on a $G = (0,0,5/6)$.
}
\end{enumerate}




\exo{} Dans cet exercice, étant donné $r>0$, on note:
\[
D_{r}=\left\{(x, y) \in \mathbb{R}^{2} \text { t.q. } x^{2}+y^{2}<r^{2}\right\} \quad \bar{D}_{r}=\left\{(x, y) \in \mathbb{R}^{2} \text { t.q. } x^{2}+y^{2} \leq r^{2}\right\}
\]
Dans la première partie de cet exercice, on se donne $f: D_{2} \rightarrow \mathbb{R}$ et on suppose que $f$ satisfait:
\begin{equation}\label{eq.hyp}\tag{H}
    f \in \mathcal{C}^{2}\left(D_{2}\right), \quad \frac{\partial^{2} f}{\partial x^{2}}(x, y)+\frac{\partial^{2} f}{\partial y^{2}}(x, y)>0 \quad \forall(x, y) \in D_{1}
\end{equation}
\begin{enumerate}
    \item Justifier en une phrase que la restriction de $f$ à $\bar{D}_{1}$ (notée $f_{\bar{D}_{1}}$ dans la suite) est bornée et atteint ses bornes.
        \cor{

        }
    \item On se donne $\left(x_{0}, y_{0}\right) \in D_{1}$.
        \begin{enumerate}
            \item  Donner le développement limité à l'ordre 2 de $f$ en $\left(x_{0}, y_{0}\right)$.
            \item  En considérant $t \mapsto f\left(x_{0}+t, y_{0}\right)$ et $t \mapsto f\left(x_{0}, y_{0}+t\right)$ pour $t$ proche de 0, montrer que l'hypothèse (H) implique que $f_{\bar{D}_{1}}$ ne peut pas atteindre son maximum en $\left(x_{0}, y_{0}\right)$.
        \end{enumerate}
    \item  On suppose maintenant que $f_{\bar{D}_{1}}$ atteint son maximum en $\left(x_{0}, y_{0}\right) \in \bar{D}_{1} \backslash D_{1} .$ On introduit:
        \[
            \begin{aligned}
                f_{\Gamma}: \quad \mathbb{R} & \longrightarrow \mathbb{R} \\
                t & \longmapsto f(\cos (t), \sin (t))
            \end{aligned}
        \]
        \begin{enumerate}
            \item Justifier qu'il existe $t_{0} \in \mathbb{R}$ tel que $\left(x_{0}, y_{0}\right)=\left(\cos \left(t_{0}\right), \sin \left(t_{0}\right)\right)$ puis que $f_{\Gamma}$ admet un maximum en $t_{0}$.
            \item En déduire que
                \[
                    y_{0} \frac{\partial f}{\partial x}\left(x_{0}, y_{0}\right)-x_{0} \frac{\partial f}{\partial y}\left(x_{0}, y_{0}\right)=0
                \]
            \item  Quelle propriété satisfont alors les vecteurs $\nabla f\left(x_{0}, y_{0}\right)$ et $\left(x_{0}, y_{0}\right)$ ?
        \end{enumerate}
\end{enumerate}
Dans la deuxième partie de l'exercice, on pose:
\[
f(x, y)=\frac{y}{(x-4)^{2}+y^{2}}+\frac{1}{2}\left(x^{2}+y^{2}\right), \quad \forall(x, y) \in \mathbb{R}^{2} \backslash\{(4,0)\}
\]
On rappelle qu'on peut utiliser les résultats de la première partie de l'exercice même sans avoir répondu à ces questions.
\begin{enumerate}[resume]
    \item Justifier que $f$ satisfait \eqref{eq.hyp}.
        \item En déduire que $f_{\bar{D}_{1}}$ est bornée et atteint ses bornes puis que sa valeur maximale ne peut pas être atteinte en un $\left(x_{0}, y_{0}\right) \in D_{1}$.
        \item Soit $\left(x_{0}, y_{0}\right) \in \bar{D}_{1} \backslash D_{1}$ tel que $f_{\bar{D}_{1}}$ est maximale en $\left(x_{0}, y_{0}\right)$. Trouver une équation satisfaite par $\left(x_{0}, y_{0}\right) .$ Puis, en utilisant que $y_{0}^{2}=1-x_{0}^{2}$, montrer que $x_{0}=8 / 17$.
\end{enumerate}
\end{enumerate}
\end{document}

