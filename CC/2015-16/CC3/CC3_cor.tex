\documentclass[a4paper]{tp_um}
\makeatletter
%--------------------------------------------------------------------------------

\usepackage[french]{babel}
\usepackage{amsmath}
\usepackage{amsbsy}
\usepackage{amsfonts}
\usepackage{amssymb}
\usepackage{amscd}
\usepackage{amsthm}
\usepackage{mathtools}
\usepackage{eurosym}
\usepackage{nicefrac}

\usepackage{latexsym}
\usepackage[a4paper,hmargin=20mm,vmargin=25mm]{geometry}
\usepackage{dsfont}
\usepackage[utf8]{inputenc}
\usepackage[T1]{fontenc}
\usepackage{lmodern}

\usepackage{multicol}
\usepackage[inline]{enumitem}
\setlist{nosep}
\setlist[itemize,1]{,label=$-$}


\newenvironment{modenumerate}
  {\enumerate\setupmodenumerate}
  {\endenumerate}

\newif\ifmoditem
\newcommand{\setupmodenumerate}{%
  \global\moditemfalse
  \let\origmakelabel\makelabel
  \def\moditem##1{\global\moditemtrue\def\mesymbol{##1}\item}%
  \def\makelabel##1{%
    \origmakelabel{##1\ifmoditem\rlap{\mesymbol}\fi\enspace}%
    \global\moditemfalse}%
}


\usepackage{sectsty}
%\sectionfont{}
\allsectionsfont{\color{astral}\normalfont\sffamily\bfseries\normalsize}

\usepackage{graphicx}
\usepackage{tikz}
\usetikzlibrary{babel}
\usepackage{tikz,tkz-tab}

\usepackage[babel=true, kerning=true]{microtype}


\usepackage{pgfplots}
\usepgfplotslibrary{fillbetween}
\pgfplotsset{compat=newest}
\usepgfplotslibrary{external} 
\tikzexternalize[prefix=./output_latex/]
%\DeclareSymbolFont{RalphSmithFonts}{U}{rsfs}{m}{n}
%\DeclareSymbolFontAlphabet{\mathscr}{RalphSmithFonts}
%\def\mathcal#1{{\mathscr #1}}



\providecommand{\abs}[1]{\left|#1\right|}
\providecommand{\norm}[1]{\left\Vert#1\right\Vert}
\providecommand{\U}{\mathcal{U}}
\providecommand{\R}{\mathbb{R}}
\providecommand{\Cc}{\mathcal{C}}
\providecommand{\reg}[1]{\mathcal{C}^{#1}}
\providecommand{\1}{\mathds{1}}
\providecommand{\N}{\mathbb{N}}
\providecommand{\Z}{\mathbb{Z}}
\providecommand{\p}{\partial}
\providecommand{\one}{\mathds{1}}
\providecommand{\E}{\mathbb{E}}\providecommand{\V}{\mathbb{V}}
\renewcommand{\P}{\mathbb{P}}


%Operateur
\providecommand{\abs}[1]{\left\lvert#1\right\rvert}
\providecommand{\sabs}[1]{\lvert#1\rvert}
\providecommand{\babs}[1]{\bigg\lvert#1\bigg\rvert}
\providecommand{\norm}[1]{\left\lVert#1\right\rVert}
\providecommand{\bnorm}[1]{\bigg\lVert#1\bigg\rVert}
\providecommand{\snorm}[1]{\lVert#1\rVert}
\providecommand{\prs}[1]{\left\langle #1\right\rangle}
\providecommand{\sprs}[1]{\langle #1\rangle}
\providecommand{\bprs}[1]{\bigg\langle #1\bigg\rangle}

\DeclareMathOperator{\deet}{Det}
\DeclareMathOperator{\hess}{Hess}
\DeclareMathOperator{\jac}{Jac}


\newcommand\rst[2]{{#1}_{\restriction_{#2}}}



% generate breakable white space allowing students to write notes.

\usepackage[framemethod=tikz]{mdframed}

\mdfdefinestyle{response}{
	leftmargin=.01\textwidth,
	rightmargin=.01\textwidth,
	linewidth=1pt
	hidealllines=false,
	leftline=true,
	rightline=true,topline=true,bottomline=true,
	skipabove=0pt,
	%innertopmargin=-5pt,
	%innerbottommargin=2pt,
	linecolor=black,
	innerrightmargin=0pt,
	}



\newcommand*{\DivideLengths}[2]{%
  \strip@pt\dimexpr\number\numexpr\number\dimexpr#1\relax*65536/\number\dimexpr#2\relax\relax sp\relax
}

\providecommand{\rep}[1]{$ $ \newline \begin{mdframed}[style=response]  
	
	\vspace*{\DivideLengths{#1}{3cm}cm}
	\pagebreak[1]	
	\vspace*{\DivideLengths{#1}{3cm}cm}
	\pagebreak[1]		
	\vspace*{\DivideLengths{#1}{3cm}cm}   \end{mdframed}}

\providecommand{\blanc}[1]{$ $ \newline 
	
	\vspace*{\DivideLengths{#1}{3cm}cm}
	\pagebreak[1]	
	\vspace*{\DivideLengths{#1}{3cm}cm}
	\pagebreak[3]		
	\vspace*{\DivideLengths{#1}{3cm}cm}}

\usepackage{ifthen}

\newcommand{\eno}[1]{%
	\ifthenelse{\equal{\version}{eno}}{#1}{}%
}
\newcommand{\cor}[1]{%
        \ifthenelse{\equal{\version}{cor}}{
\medskip 

{\small \color{gray} #1}

\medskip 
}{}
}

%------------------------------------------------------------------------------
%\DeclareUnicodeCharacter{00A0}{~}
\makeatother


\ue{HLMA410}
%-----------------------------------------------------------------------------

\title{\large \sffamily\bfseries Contrôle continu 3}

\begin{document}

\maketitle
\textit{Durée 1h30. Les documents, la calculatrice, les téléphones portables, tablettes, ordinateurs ne sont pas autorisés. La qualité de la rédaction sera prise en compte.} 

\bigskip
\bigskip

\exo{(Question de cours)} 
\begin{enumerate}
	\item Soit $q:\R^n \to\R$ une forme quadratique. Donner la définition et les propriétés élémentaires de la forme polaire $B$ de $q$.

\bigskip

Voir cours.

\bigskip

	\item Démontrer la proposition suivante : Soit $\mathcal B = (e_1 ,\cdots , e_n )$ une base de $\R^n$ et $h = h_1 e_1 + \cdots + h_n e_n \in \R^n$. Soit $f : \R^n \to\R^p$ une application différentiable en $a \in \R^n$, alors $d_a f (h) = h_1 \frac{\partial f}{\partial x_1}(a) + \cdots+ h_n \frac{\partial f}{\partial x_n}(a)$.
	%\item	Soit une fonction $f:\R^n \to \R^p$. Donner la définition de la différentiabilité en $a \in \R^p$.

		\bigskip

Voir cours.

\end{enumerate}


\exo{} Soit $\phi$ la forme bilinéaire symétrique sur $\R^3$ définie par
\[
	\phi(x,y) = (x_1 - 2x_2)(y_1-2y_2) + x_2y_2 + (x_2+x_3)(y_2 + y_3)
\]
pour tout $x=(x_1,x_2,x_3)$ et $y=(y_1,y_2,y_3)$.
\begin{enumerate}
	\item Vérifier que $\phi$ est un produit scalaire sur $\R^3$.


\bigskip

		L'application $\phi:\R^3 \times \R^3 \to \R$ est d'après l'énoncé bilinéaire et symétrique. Reste à voir qu'elle est bien définie et positive. Sa matrice est
		\[\phi(x,y) = \begin{pmatrix}
			x_1 & x_2 & x_3
		\end{pmatrix} \underbrace{\begin{pmatrix}
						1 & -2 & 0 \\ -2 & 6 & 1 \\ 0 & 1 & 1 
					\end{pmatrix}}_{=M} \begin{pmatrix}
			y_1\\y_2\\y_3
	\end{pmatrix}.\]
Les déterminants mineurs principaux de $M$ sont $\Delta_1 = 1$, $\Delta_2 = 4$ et $\Delta_3=1$. D'après le critère de Sylvester, $\phi$ est définie positive. C'est un donc bien un produit scalaire.

\bigskip

	\item  On note $\norm{\cdot}_\phi$ la norme associée à $\phi$. Soit $i=(1,0,0)$, $j=(0,1,0)$ et $k =(0,0,1)$. Calculer les coordonnées de 
\[
		e_1 = \frac{i}{\norm{i}_\phi}, \quad  e_2 = \frac{j - \phi(e_1,j)e_1}{\snorm{j - \phi(e_1,j)e_1}_\phi}, \quad  e_3 = \frac{k - \phi(e_1,k)e_1 - \phi(e_2,k)e_2}{\norm{k - \phi(e_1,k)e_1 - \phi(e_2,k)e_2}_\phi}
	\]
\bigskip

On a 
\begin{itemize}
	\item $\snorm{i}^2_\phi = \phi(i,i) = 1$ et \[e_1 = i = (1,0,0)\]
	\item $\phi(j,e_1) = -2$ et $u_2 = j - \phi(e_1,j)e_1 = (2,1,0)$. De plus $\snorm{u_2}^2_\phi = \phi(u_2,u_2) = 2$. Ainsi 
		\[
			e_2 = (2,1,0) / \sqrt{2} = (\sqrt{2} , \sqrt{2} / 2, 0)
		\]
	\item $\phi(k,e_1) = 0$, $\phi(k,e_2) = 1/\sqrt{2}$ et $u_3 = k - \phi(e_1,k)e_1 - \phi(e_2,k)e_2 = (-1,-1/2,1)$. De plus $\snorm{u_3}^2_\phi = \phi(u_3,u_3) = 1/2$. Ainsi 
		\[
			e_3 = (-\sqrt{2} , -\sqrt{2} / 2, \sqrt{2}). 
	\]

\end{itemize}

	\item Vérifier que $(e_1,e_2,e_3)$ est une base orthonormale pour $\phi$.

\bigskip

Le procédé de la question précédente s'appelle orthonormalisation. Vérifions directement que $\phi(e_i,e_j) = 1$ si $i=j$ et $0$ sinon.  Cela revient à vérifier que le produit matriciel suivant est correct
\[
	E^t M E = Id_3.
\]
avec $E = \begin{pmatrix}
	1 & \sqrt 2 & - \sqrt 2 \\ 0 & \sqrt 2 /2 & -\sqrt 2 /2 \\ 0 & 0 & \sqrt 2 
\end{pmatrix}$ est triangulaire (les colonnes de $E$ sont les vecteurs $e_1$, $e_2$ et $e_3$).

\bigskip
	\item Déterminer (sans calcul) la matrice de $\phi$ dans la base $(e_1,e_2,e_3)$.

\bigskip
		Par définition la matrice $A$ de $\phi$ dans la base $(e_1,e_2,e_3)$ est la matrice $A = [ \phi(e_i,e_j) ]_{i,j=1}^3$. Or d'après les deux questions précédente $A = Id_3$.

\end{enumerate}

		
\exo{}Soit la fonction $f:\R^2 \to \R$ définie par $f(x,y) = \frac{xy^2}{x^2 + y^2}$ si $(x,y) \neq (0,0)$ et $f(0,0) = 0$.
\begin{enumerate}
	\item  \'Etudier la continuité de $f$.

\bigskip
La fonction $f$ est continue sur $\mathbb R^2 \setminus\left\{ (0,0) \right\}$ comme somme et produit de fonctions continues (le dénominateur ne s'annulant pas en dehors de l'origine). Reste à étudier en $(0,0)$:
\[
	\abs{f(x,y) - f(0,0) } =  \frac{|x|y^2}{x^2 + y^2} \leq (x^2 + y^2)^{3/2 -1} =  (x^2 + y^2)^{1/2} \xrightarrow[(x,y) \to (0,0)]{} 0
\]
La fonction $f$ est continue sur tout le plan.

\bigskip

	\item Soit $v = (v_1,v_2) \in \R^2$. Montrer que la dérivée directionnelle $D_v f(x,y)$ existe en tout point $(x,y)\in\mathbb R^2 \setminus\left\{ (0,0) \right\}$. Calculer ensuite $D_v f(x,y)$.

\bigskip

La fonction $f$ est différentiable sur $ \mathbb R^2 \setminus\left\{ (0,0) \right\}$ comme somme et produit de fonctions différentiables. Elle admet donc des dérivées directionnelles en tout point différent de l'origine. De plus,
\[
	D_v f(x,y) =  d_{(x,y)} f (v) = \frac{y^4 - x^2y^2}{(x^2 + y^2)^2} v_1 +  \frac{2x^3y}{(x^2 + y^2)^2} v_2.
\]
\bigskip

		
	\item Soit $v = (v_1,v_2)  \in \R^2$. Montrer que la dérivée directionnelle $D_v f(0,0)$ existe. La fonction $f$ est elle différentiable en l'origine?

\bigskip

Par définition on a 
\[
	D_v f(0,0) = \lim_{h\to 0} \frac{f(hv_1,hv_2) - f(0,0) }{h} = \lim_{h\to 0} \frac 1 h \frac{ hv_1 h^2 v_2}{(h^2v_1^2 + h^2 v_2^2)} = \frac{v_1v_2}{v_1^2 + v^2_2} \in \R.
\]
De plus, $\frac{\partial{}f}{\partial x}(0,0) = D_{(1,0)} f (0,0)  = 0$ et $\frac{\partial f}{\partial y}(0,0) = D_{(0,1)} f(0,0) = 0$. Or il existe une dérivée directionnelle non nulle et $f$ ne peut être différentiable en 0. En effet, il existe $(v_1,v_2) \in \R^2$ tel que $D_v f(0,0) \neq v_1\frac{\partial f}{\partial x}(0,0) + v_2\frac{\partial f}{\partial y}(0,0)$.  
\bigskip



\end{enumerate}	
	
	
	\exo{} \begin{enumerate}
	\item Trouver l'équation du plan tangent au graphe de la fonction $(x,y) \mapsto 4x^2 + y^2$, au point $(x_0,y_0) = (1,-1)$.
%$(x_0,y_0) = (-1/8,-1)$.

\bigskip

On trouve $z = 5 + 8(x-1) -2(y+1)$.

	\begin{center}
			\begin{tikzpicture}[scale=.75]
				\begin{axis}[,xlabel=$x$,ylabel=$y$]%,xtick=\empty,ytick=\empty,ztick=\empty ]
					\addplot3[surf,opacity=.7,samples=50,domain=0:2, y domain = -3:1] gnuplot { 5 + 8*(x-1) - 2 *(y+1))};
					\addplot3[surf,opacity=.7,samples=50,domain=-1:3, y domain = -4:3] gnuplot { 4*x**2 + y**2)};
				\end{axis}
			\end{tikzpicture}
		\end{center}

\bigskip

	\item Trouver les points sur le paraboloïde $z=4x^2+y^2$ où le plan tangent est parallèle au plan $x+2y+z=6$.
		
\bigskip

		Le plan  $x+2y+z=6$ est normal au vecteur $(1,2,1)$. Le plan tangent au paraboloïde est normal au vecteur $(\frac{\partial f}{\partial x} , \frac{\partial f}{\partial y} , 1)$. Trouver un plan tangent parallèle revient à résoudre le système:
		\[
			\begin{cases}
				\frac{\partial f}{\partial x}(x,y) = 1\\
				\frac{\partial f}{\partial y}(x,y) = 2
			\end{cases}
		\]
On trouve une unique solution $(x,y) = (-1/8,-1)$. 

\end{enumerate}	
	

\end{document}
