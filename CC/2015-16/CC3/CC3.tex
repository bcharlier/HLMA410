\documentclass[a4paper]{tp_um}
\makeatletter
%--------------------------------------------------------------------------------

\usepackage[french]{babel}
\usepackage{amsmath}
\usepackage{amsbsy}
\usepackage{amsfonts}
\usepackage{amssymb}
\usepackage{amscd}
\usepackage{amsthm}
\usepackage{mathtools}
\usepackage{eurosym}
\usepackage{nicefrac}

\usepackage{latexsym}
\usepackage[a4paper,hmargin=20mm,vmargin=25mm]{geometry}
\usepackage{dsfont}
\usepackage[utf8]{inputenc}
\usepackage[T1]{fontenc}
\usepackage{lmodern}

\usepackage{multicol}
\usepackage[inline]{enumitem}
\setlist{nosep}
\setlist[itemize,1]{,label=$-$}


\newenvironment{modenumerate}
  {\enumerate\setupmodenumerate}
  {\endenumerate}

\newif\ifmoditem
\newcommand{\setupmodenumerate}{%
  \global\moditemfalse
  \let\origmakelabel\makelabel
  \def\moditem##1{\global\moditemtrue\def\mesymbol{##1}\item}%
  \def\makelabel##1{%
    \origmakelabel{##1\ifmoditem\rlap{\mesymbol}\fi\enspace}%
    \global\moditemfalse}%
}


\usepackage{sectsty}
%\sectionfont{}
\allsectionsfont{\color{astral}\normalfont\sffamily\bfseries\normalsize}

\usepackage{graphicx}
\usepackage{tikz}
\usetikzlibrary{babel}
\usepackage{tikz,tkz-tab}

\usepackage[babel=true, kerning=true]{microtype}


\usepackage{pgfplots}
\usepgfplotslibrary{fillbetween}
\pgfplotsset{compat=newest}
\usepgfplotslibrary{external} 
\tikzexternalize[prefix=./output_latex/]
%\DeclareSymbolFont{RalphSmithFonts}{U}{rsfs}{m}{n}
%\DeclareSymbolFontAlphabet{\mathscr}{RalphSmithFonts}
%\def\mathcal#1{{\mathscr #1}}



\providecommand{\abs}[1]{\left|#1\right|}
\providecommand{\norm}[1]{\left\Vert#1\right\Vert}
\providecommand{\U}{\mathcal{U}}
\providecommand{\R}{\mathbb{R}}
\providecommand{\Cc}{\mathcal{C}}
\providecommand{\reg}[1]{\mathcal{C}^{#1}}
\providecommand{\1}{\mathds{1}}
\providecommand{\N}{\mathbb{N}}
\providecommand{\Z}{\mathbb{Z}}
\providecommand{\p}{\partial}
\providecommand{\one}{\mathds{1}}
\providecommand{\E}{\mathbb{E}}\providecommand{\V}{\mathbb{V}}
\renewcommand{\P}{\mathbb{P}}


%Operateur
\providecommand{\abs}[1]{\left\lvert#1\right\rvert}
\providecommand{\sabs}[1]{\lvert#1\rvert}
\providecommand{\babs}[1]{\bigg\lvert#1\bigg\rvert}
\providecommand{\norm}[1]{\left\lVert#1\right\rVert}
\providecommand{\bnorm}[1]{\bigg\lVert#1\bigg\rVert}
\providecommand{\snorm}[1]{\lVert#1\rVert}
\providecommand{\prs}[1]{\left\langle #1\right\rangle}
\providecommand{\sprs}[1]{\langle #1\rangle}
\providecommand{\bprs}[1]{\bigg\langle #1\bigg\rangle}

\DeclareMathOperator{\deet}{Det}
\DeclareMathOperator{\hess}{Hess}
\DeclareMathOperator{\jac}{Jac}


\newcommand\rst[2]{{#1}_{\restriction_{#2}}}



% generate breakable white space allowing students to write notes.

\usepackage[framemethod=tikz]{mdframed}

\mdfdefinestyle{response}{
	leftmargin=.01\textwidth,
	rightmargin=.01\textwidth,
	linewidth=1pt
	hidealllines=false,
	leftline=true,
	rightline=true,topline=true,bottomline=true,
	skipabove=0pt,
	%innertopmargin=-5pt,
	%innerbottommargin=2pt,
	linecolor=black,
	innerrightmargin=0pt,
	}



\newcommand*{\DivideLengths}[2]{%
  \strip@pt\dimexpr\number\numexpr\number\dimexpr#1\relax*65536/\number\dimexpr#2\relax\relax sp\relax
}

\providecommand{\rep}[1]{$ $ \newline \begin{mdframed}[style=response]  
	
	\vspace*{\DivideLengths{#1}{3cm}cm}
	\pagebreak[1]	
	\vspace*{\DivideLengths{#1}{3cm}cm}
	\pagebreak[1]		
	\vspace*{\DivideLengths{#1}{3cm}cm}   \end{mdframed}}

\providecommand{\blanc}[1]{$ $ \newline 
	
	\vspace*{\DivideLengths{#1}{3cm}cm}
	\pagebreak[1]	
	\vspace*{\DivideLengths{#1}{3cm}cm}
	\pagebreak[3]		
	\vspace*{\DivideLengths{#1}{3cm}cm}}

\usepackage{ifthen}

\newcommand{\eno}[1]{%
	\ifthenelse{\equal{\version}{eno}}{#1}{}%
}
\newcommand{\cor}[1]{%
        \ifthenelse{\equal{\version}{cor}}{
\medskip 

{\small \color{gray} #1}

\medskip 
}{}
}

%------------------------------------------------------------------------------
%\DeclareUnicodeCharacter{00A0}{~}
\makeatother


\ue{HLMA410}
%-----------------------------------------------------------------------------

\title{\large \sffamily\bfseries Contrôle continu 3}

\begin{document}

\maketitle
\textit{Durée 1h30. Les documents, la calculatrice, les téléphones portables, tablettes, ordinateurs ne sont pas autorisés. La qualité de la rédaction sera prise en compte.} 

\bigskip
\bigskip

\exo{(Question de cours)} 
\begin{enumerate}
	\item Soit $q:\R^n \to\R$ une forme quadratique. Donner la définition et les propriétés élémentaires de la forme polaire $B$ de $q$.
		\blanc{5cm}
	\item Démontrer la proposition suivante : Soit $\mathcal B = (e_1 ,\cdots , e_n )$ une base de $\R^n$ et $h = h_1 e_1 + \cdots + h_n e_n \in \R^n$. Soit $f : \R^n \to\R^p$ une application différentiable en $a \in \R^n$, alors $d_a f (h) = h_1 \frac{\partial f}{\partial x_1}(a) + \cdots+ h_n \frac{\partial f}{\partial x_n}(a)$.
	%\item	Soit une fonction $f:\R^n \to \R^p$. Donner la définition de la différentiabilité en $a \in \R^p$.
		\blanc{8cm}
\end{enumerate}


\exo{} Soit $\phi$ la forme bilinéaire symétrique sur $\R^3$ définie par
\[
	\phi(x,y) = (x_1 - 2x_2)(y_1-2y_2) + x_2y_2 + (x_2+x_3)(y_2 + y_3)
\]
pour tout $x=(x_1,x_2,x_3)$ et $y=(y_1,y_2,y_3)$.
\begin{enumerate}
	\item Vérifier que $\phi$ est un produit scalaire sur $\R^3$.
		\blanc{8cm}
	\item  On note $\norm{\cdot}_\phi$ la norme associée à $\phi$. Soit $i=(1,0,0)$, $j=(0,1,0)$ et $k =(0,0,1)$. Calculer les coordonnées de 
\[
		e_1 = \frac{i}{\norm{i}_\phi}, \quad  e_2 = \frac{j - \phi(e_1,j)e_1}{\snorm{j - \phi(e_1,j)e_1}_\phi}, \quad  e_3 = \frac{k - \phi(e_1,k)e_1 - \phi(e_2,k)e_2}{\norm{k - \phi(e_1,k)e_1 - \phi(e_2,k)e_2}_\phi}
	\]
		\blanc{15cm}
	\item Vérifier que $(e_1,e_2,e_3)$ est une base orthonormale pour $\phi$.
		\blanc{8cm}
	\item Déterminer (sans calcul) la matrice de $\phi$ dans la base $(e_1,e_2,e_3)$.
		\blanc{8cm}
\end{enumerate}

\pagebreak	
		
\exo{}Soit la fonction $f:\R^2 \to \R$ définie par $f(x,y) = \frac{xy^2}{x^2 + y^2}$ si $(x,y) \neq (0,0)$ et $f(0,0) = 0$.
\begin{enumerate}
	\item  \'Etudier la continuité de $f$.
		\blanc{8cm}
	\item Soit $v \in \R^2$. Montrer que la dérivée directionnelle $D_v f(x,y)$ existe en tout point $(x,y)\in\mathbb R^2 \setminus\left\{ (0,0) \right\}$.
		\blanc{5cm}
	\item Soit $v \in \R^2$. Montrer que la dérivée directionnelle $D_v f(0,0)$ existe. La fonction $f$ est elle différentiable en l'origine?
		\blanc{8cm}
\end{enumerate}	
	
	
	\exo{} \begin{enumerate}
	\item Trouver l'équation du plan tangent au graphe de la fonction $(x,y) \mapsto 4x^2 + y^2$, au point $(x_0,y_0) = (1,-1)$.
%$(x_0,y_0) = (-1/8,-1)$.
	\blanc{8cm}	
	\item Trouver les points sur le paraboloïde $z=4x^2+y^2$ où le plan tangent est parallèle au plan $x+2y+z=6$.
		
\blanc{8cm}
\end{enumerate}	
	

\end{document}
