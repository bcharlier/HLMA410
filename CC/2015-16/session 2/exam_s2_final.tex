\documentclass{tp_um}

\makeatletter
%--------------------------------------------------------------------------------

\usepackage[french]{babel}
\usepackage{amsmath}
\usepackage{amsbsy}
\usepackage{amsfonts}
\usepackage{amssymb}
\usepackage{amscd}
\usepackage{amsthm}
\usepackage{mathtools}
\usepackage{eurosym}
\usepackage{nicefrac}

\usepackage{latexsym}
\usepackage[a4paper,hmargin=20mm,vmargin=25mm]{geometry}
\usepackage{dsfont}
\usepackage[utf8]{inputenc}
\usepackage[T1]{fontenc}
\usepackage{lmodern}

\usepackage{multicol}
\usepackage[inline]{enumitem}
\setlist{nosep}
\setlist[itemize,1]{,label=$-$}


\newenvironment{modenumerate}
  {\enumerate\setupmodenumerate}
  {\endenumerate}

\newif\ifmoditem
\newcommand{\setupmodenumerate}{%
  \global\moditemfalse
  \let\origmakelabel\makelabel
  \def\moditem##1{\global\moditemtrue\def\mesymbol{##1}\item}%
  \def\makelabel##1{%
    \origmakelabel{##1\ifmoditem\rlap{\mesymbol}\fi\enspace}%
    \global\moditemfalse}%
}


\usepackage{sectsty}
%\sectionfont{}
\allsectionsfont{\color{astral}\normalfont\sffamily\bfseries\normalsize}

\usepackage{graphicx}
\usepackage{tikz}
\usetikzlibrary{babel}
\usepackage{tikz,tkz-tab}

\usepackage[babel=true, kerning=true]{microtype}


\usepackage{pgfplots}
\usepgfplotslibrary{fillbetween}
\pgfplotsset{compat=newest}
\usepgfplotslibrary{external} 
\tikzexternalize[prefix=./output_latex/]
%\DeclareSymbolFont{RalphSmithFonts}{U}{rsfs}{m}{n}
%\DeclareSymbolFontAlphabet{\mathscr}{RalphSmithFonts}
%\def\mathcal#1{{\mathscr #1}}



\providecommand{\abs}[1]{\left|#1\right|}
\providecommand{\norm}[1]{\left\Vert#1\right\Vert}
\providecommand{\U}{\mathcal{U}}
\providecommand{\R}{\mathbb{R}}
\providecommand{\Cc}{\mathcal{C}}
\providecommand{\reg}[1]{\mathcal{C}^{#1}}
\providecommand{\1}{\mathds{1}}
\providecommand{\N}{\mathbb{N}}
\providecommand{\Z}{\mathbb{Z}}
\providecommand{\p}{\partial}
\providecommand{\one}{\mathds{1}}
\providecommand{\E}{\mathbb{E}}\providecommand{\V}{\mathbb{V}}
\renewcommand{\P}{\mathbb{P}}


%Operateur
\providecommand{\abs}[1]{\left\lvert#1\right\rvert}
\providecommand{\sabs}[1]{\lvert#1\rvert}
\providecommand{\babs}[1]{\bigg\lvert#1\bigg\rvert}
\providecommand{\norm}[1]{\left\lVert#1\right\rVert}
\providecommand{\bnorm}[1]{\bigg\lVert#1\bigg\rVert}
\providecommand{\snorm}[1]{\lVert#1\rVert}
\providecommand{\prs}[1]{\left\langle #1\right\rangle}
\providecommand{\sprs}[1]{\langle #1\rangle}
\providecommand{\bprs}[1]{\bigg\langle #1\bigg\rangle}

\DeclareMathOperator{\deet}{Det}
\DeclareMathOperator{\hess}{Hess}
\DeclareMathOperator{\jac}{Jac}


\newcommand\rst[2]{{#1}_{\restriction_{#2}}}



% generate breakable white space allowing students to write notes.

\usepackage[framemethod=tikz]{mdframed}

\mdfdefinestyle{response}{
	leftmargin=.01\textwidth,
	rightmargin=.01\textwidth,
	linewidth=1pt
	hidealllines=false,
	leftline=true,
	rightline=true,topline=true,bottomline=true,
	skipabove=0pt,
	%innertopmargin=-5pt,
	%innerbottommargin=2pt,
	linecolor=black,
	innerrightmargin=0pt,
	}



\newcommand*{\DivideLengths}[2]{%
  \strip@pt\dimexpr\number\numexpr\number\dimexpr#1\relax*65536/\number\dimexpr#2\relax\relax sp\relax
}

\providecommand{\rep}[1]{$ $ \newline \begin{mdframed}[style=response]  
	
	\vspace*{\DivideLengths{#1}{3cm}cm}
	\pagebreak[1]	
	\vspace*{\DivideLengths{#1}{3cm}cm}
	\pagebreak[1]		
	\vspace*{\DivideLengths{#1}{3cm}cm}   \end{mdframed}}

\providecommand{\blanc}[1]{$ $ \newline 
	
	\vspace*{\DivideLengths{#1}{3cm}cm}
	\pagebreak[1]	
	\vspace*{\DivideLengths{#1}{3cm}cm}
	\pagebreak[3]		
	\vspace*{\DivideLengths{#1}{3cm}cm}}

\usepackage{ifthen}

\newcommand{\eno}[1]{%
	\ifthenelse{\equal{\version}{eno}}{#1}{}%
}
\newcommand{\cor}[1]{%
        \ifthenelse{\equal{\version}{cor}}{
\medskip 

{\small \color{gray} #1}

\medskip 
}{}
}

%------------------------------------------------------------------------------
%\DeclareUnicodeCharacter{00A0}{~}
\makeatother



%-----------------------------------------------------------------------------

\title{\Large \sffamily\bfseries Examen - Session 2}
\ue{HLMA410}


%-----------------------------------------------------------------------------
\begin{document}

\maketitle

\bigskip
\bigskip

\textit{Durée 3h00. Les documents, la calculatrice, les téléphones portables, tablettes, ordinateurs ne sont pas autorisés. La qualité de la rédaction sera prise en compte. Les exercices sont indépendants.} 

\bigskip
\bigskip

\section{Analyse}

\exo{(Question de cours)} Soit $f:\R^2 \to \R$ une fonction de classe $\mathcal C^1(\R^2)$. Rappeler la définition du gradient de $f$ et la formule liant gradient et différentielle.

%\exo{(Projection de ${\Bbb R}^2$ sur $\Bbb R$)} %Dans cet exercice, il est fortement recommand\'e de dessiner, et de se souvenir qu'un dessin n'est jamais une preuve.  
%%Si $X$ est un espace vectoriel normé, «$B_x(r)$» désigne la boule ouverte de centre $x\in X$ et de rayon $r>0$. On rappelle que pour une application $p:X\rightarrow Y$, et pour $A\subset X$, l'image de $A$ par $p$ est l'ensemble $p(A)=\{ y\in  Y, \exists x\in A, y=p(x)\}=\{ p(x), x \in A\}$. 
%Soit $p:{\Bbb R}^2\rightarrow {\Bbb R}$ la premi\`ere projection ($p(x,y)=x$). On munit ${\Bbb R}^2$ de la norme euclidienne et $\Bbb R$ de la valeur absolue. %Dans cet exercice,
%%\begin{enumerate}
%%\item Montrer que dans $\Bbb R$, $B(a,\epsilon[=]a-\epsilon,a+\epsilon[$.  
%%\item Montrer que $p(B((x,y),\epsilon[)=B(x,\epsilon[$. En d\'eduire que l'image d'un ouvert de ${\Bbb R}^2$ par $p$ est un ouvert de $\Bbb R$.
%%\item Montrer que $F=\{(x,y), xy=1\}$ est un ferm\'e de ${\Bbb R}^2$ mais que $p(F)$ n'est pas ferm\'e.
%%\item 
	%On suppose maintenant que $F$ est une partie de ${\Bbb R}^2$ ferm\'ee et born\'ee et on consid\`ere une suite $x_n\in p(F)$ telle que $x_n\rightarrow x$.
%\begin{enumerate} 
%\item Montrer qu'il existe une suite $y_n$ telle que $(x_n,y_n)\in F$.
%\item Montrer qu'il existe une sous suite $(x_{\phi(n)},y_{\phi(n)})$ convergente.
%\item En d\'eduire que $x\in p(F)$, puis que $p(F)$ est ferm\'e dans $\Bbb R$.
%\end{enumerate}


\exo{} Soit $\phi:\R^2\to\R^2$ l'application définie par 
\[
		\phi(x,y) = (u,v) = (x+y,x-y) 
\]
\begin{enumerate}
	\item Démontrer que $\phi$ est un $\mathcal C^1$ difféomorphisme. Calculer son inverse $\phi^{-1}$.
	\item Calculer le déterminant jacobien de $\phi$ au point $(x,y)$ et le déterminant jacobien de $\phi^{-1}$ au point $(u,v)$.
	\item On considère l'application $f: \R^2 \to \R$  définie par $f(x,y) = e^{x^2 -y^2}$. Soit
		$g = f \circ \phi^{-1}$. Calculer les dérivées partielles de $f$ puis calculer les dérivées partielle de $g$ {\bfseries en utilisant la règle de la chaîne}.
	\item On note $D =\left\{ (x,y) \in \R^2 ,\, x>0,\, (x-1) <y < (1-x) \right\}$. Déterminer le domaine $\Delta$ image de $D$ par $\phi$. Faire un dessin.
	\item En déduire $\int_D f(x,y) dxdy$.
\end{enumerate}

\exo{}On consid\`ere la fonction $ f: \R^2 \to \R$, 
$$
f(x,y) = x \sqrt{ x^2 + y^2}.
$$
\begin{enumerate}
	\item \'Etudier la continuité de $f$ en chaque point de $ \R^2$. 
	\item Calculer les dérivées partielles de $f$ en chaque point o\`u elles existent. 
	\item \'Etudier la continuit\'e des d\'eriv\'ees partielles de $f$ sur $ \R ^2 \setminus{(0,0)}$, puis en $(0,0)$.
	\item \'Etudier la  différentiabilité de $f$ en chaque point de $ \R^2$.
	\item %Calculer le développement limité de $f$ à l'ordre 1 en $(1,0)$.
		En déduire une valeur approchée de $f(1.01,0)$. %$d_{(1, 0) } f ( h_1, h_2)$.
\end{enumerate}

\exo{} Soit $f(x,y) = \abs{4x^2 + 9y^2 - 8x + 36y + 39}$.% Le but de cet exerci 
\begin{enumerate}
	%\item Déterminer les réels $a_1,a_2,c_1,c_2$ et $b$ tels que $f(x , y) = \abs{a_1(x-c_1)^2 + a_2 (y-c_2)^2 -b}$. % Appliquer l'algorithme de la décomposition de Gauss et en déduire le signe de $q$
	\item Déterminer l'ensemble $ N = \{(x,y) \in \R^2,  4x^2 + 9y^2 - 8x + 36y + 39 <0\}$. 
	\item Étudier la continuité $f$ et donner l'ensemble image de $f$. 
	\item Dessiner l'ensemble $L = \{(x,y) \in \R^2 | f(x,y)> 1/2\}$.
	\item Sur quel ensemble $f$ est-elle $\mathcal C^\infty$ ? Justifier succinctement la réponse.
	\item Donner les points de minimum de $f$ sur $\R^2$. Indiquer, en justifiant, la nature de ces points (minimum global ou local).
		\textit{Indication: cette question se traitera sans calcul}
	\item Calculer le gradient et la Hessienne de $f$ en les points de $\R^2$ pour lesquels ces quantités sont bien définies.
	\item Déterminer alors le(s) point(s) critique(s) de $f$ donner leur nature (minimum/maximum, local/global, point selle,\ldots).
	\item Tracer qualitativement le graphe de $f$.
\end{enumerate}





\bigskip


\section{Probabilités}

\exo{(Question de cours)} Soient $X$ et $Y$ deux variables aléatoires réelles discrètes admettant un moment d'ordre 2. Démontrer que l'on a 
\[
	\E(XY) \leq  \sqrt{\E(X^2)\E(Y^2)}.
\]



\exo{(Lois géométriques)} Soit $X$ et $Y$ deux variables aléatoires indépendantes de lois géométriques de paramètre $0<a<1$ et $0<b<1$ respectivement. On note $Z = \min \left\{ X,Y \right\}$. Le but de l'exercice est de déterminer la loi de $Z$.
\begin{enumerate}
	\item Calculer $\P\left( X \leq k \right)$ et $\P\left( Y\leq k\right)$ pour tout $k\in \N^*$.
	\item Calculer alors $\P\left( Z\leq k \right)$ pour tout $k \in \N^*$
	\item Soit $k\in\N^*$ fixé. Exprimer l'évènement $\left\{ Z = k \right\}$ en fonction des évènements du type $\left\{ Z\leq \ell \right\}$, $\ell\in \N^*$.
	\item En déduire que $Z$ est une variable aléatoire de loi géométrique dont on déterminera le paramètre.
\end{enumerate}


\end{document}
