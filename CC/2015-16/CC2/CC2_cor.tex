\documentclass[a4paper]{tp_um}
\makeatletter
%--------------------------------------------------------------------------------

\usepackage[french]{babel}
\usepackage{amsmath}
\usepackage{amsbsy}
\usepackage{amsfonts}
\usepackage{amssymb}
\usepackage{amscd}
\usepackage{amsthm}
\usepackage{mathtools}
\usepackage{eurosym}
\usepackage{nicefrac}

\usepackage{latexsym}
\usepackage[a4paper,hmargin=20mm,vmargin=25mm]{geometry}
\usepackage{dsfont}
\usepackage[utf8]{inputenc}
\usepackage[T1]{fontenc}
\usepackage{lmodern}

\usepackage{multicol}
\usepackage[inline]{enumitem}
\setlist{nosep}
\setlist[itemize,1]{,label=$-$}


\newenvironment{modenumerate}
  {\enumerate\setupmodenumerate}
  {\endenumerate}

\newif\ifmoditem
\newcommand{\setupmodenumerate}{%
  \global\moditemfalse
  \let\origmakelabel\makelabel
  \def\moditem##1{\global\moditemtrue\def\mesymbol{##1}\item}%
  \def\makelabel##1{%
    \origmakelabel{##1\ifmoditem\rlap{\mesymbol}\fi\enspace}%
    \global\moditemfalse}%
}


\usepackage{sectsty}
%\sectionfont{}
\allsectionsfont{\color{astral}\normalfont\sffamily\bfseries\normalsize}

\usepackage{graphicx}
\usepackage{tikz}
\usetikzlibrary{babel}
\usepackage{tikz,tkz-tab}

\usepackage[babel=true, kerning=true]{microtype}


\usepackage{pgfplots}
\usepgfplotslibrary{fillbetween}
\pgfplotsset{compat=newest}
\usepgfplotslibrary{external} 
\tikzexternalize[prefix=./output_latex/]
%\DeclareSymbolFont{RalphSmithFonts}{U}{rsfs}{m}{n}
%\DeclareSymbolFontAlphabet{\mathscr}{RalphSmithFonts}
%\def\mathcal#1{{\mathscr #1}}



\providecommand{\abs}[1]{\left|#1\right|}
\providecommand{\norm}[1]{\left\Vert#1\right\Vert}
\providecommand{\U}{\mathcal{U}}
\providecommand{\R}{\mathbb{R}}
\providecommand{\Cc}{\mathcal{C}}
\providecommand{\reg}[1]{\mathcal{C}^{#1}}
\providecommand{\1}{\mathds{1}}
\providecommand{\N}{\mathbb{N}}
\providecommand{\Z}{\mathbb{Z}}
\providecommand{\p}{\partial}
\providecommand{\one}{\mathds{1}}
\providecommand{\E}{\mathbb{E}}\providecommand{\V}{\mathbb{V}}
\renewcommand{\P}{\mathbb{P}}


%Operateur
\providecommand{\abs}[1]{\left\lvert#1\right\rvert}
\providecommand{\sabs}[1]{\lvert#1\rvert}
\providecommand{\babs}[1]{\bigg\lvert#1\bigg\rvert}
\providecommand{\norm}[1]{\left\lVert#1\right\rVert}
\providecommand{\bnorm}[1]{\bigg\lVert#1\bigg\rVert}
\providecommand{\snorm}[1]{\lVert#1\rVert}
\providecommand{\prs}[1]{\left\langle #1\right\rangle}
\providecommand{\sprs}[1]{\langle #1\rangle}
\providecommand{\bprs}[1]{\bigg\langle #1\bigg\rangle}

\DeclareMathOperator{\deet}{Det}
\DeclareMathOperator{\hess}{Hess}
\DeclareMathOperator{\jac}{Jac}


\newcommand\rst[2]{{#1}_{\restriction_{#2}}}



% generate breakable white space allowing students to write notes.

\usepackage[framemethod=tikz]{mdframed}

\mdfdefinestyle{response}{
	leftmargin=.01\textwidth,
	rightmargin=.01\textwidth,
	linewidth=1pt
	hidealllines=false,
	leftline=true,
	rightline=true,topline=true,bottomline=true,
	skipabove=0pt,
	%innertopmargin=-5pt,
	%innerbottommargin=2pt,
	linecolor=black,
	innerrightmargin=0pt,
	}



\newcommand*{\DivideLengths}[2]{%
  \strip@pt\dimexpr\number\numexpr\number\dimexpr#1\relax*65536/\number\dimexpr#2\relax\relax sp\relax
}

\providecommand{\rep}[1]{$ $ \newline \begin{mdframed}[style=response]  
	
	\vspace*{\DivideLengths{#1}{3cm}cm}
	\pagebreak[1]	
	\vspace*{\DivideLengths{#1}{3cm}cm}
	\pagebreak[1]		
	\vspace*{\DivideLengths{#1}{3cm}cm}   \end{mdframed}}

\providecommand{\blanc}[1]{$ $ \newline 
	
	\vspace*{\DivideLengths{#1}{3cm}cm}
	\pagebreak[1]	
	\vspace*{\DivideLengths{#1}{3cm}cm}
	\pagebreak[3]		
	\vspace*{\DivideLengths{#1}{3cm}cm}}

\usepackage{ifthen}

\newcommand{\eno}[1]{%
	\ifthenelse{\equal{\version}{eno}}{#1}{}%
}
\newcommand{\cor}[1]{%
        \ifthenelse{\equal{\version}{cor}}{
\medskip 

{\small \color{gray} #1}

\medskip 
}{}
}

%------------------------------------------------------------------------------
%\DeclareUnicodeCharacter{00A0}{~}
\makeatother


\ue{HLMA410}
%-----------------------------------------------------------------------------

\title{\large \sffamily\bfseries Contrôle continu 2}

\begin{document}

\maketitle
\textit{Durée 1h30. Les documents, la calculatrice, les téléphones portables, tablettes, ordinateurs ne sont pas autorisés. La qualité de la rédaction sera prise en compte.} 

\bigskip
\bigskip

\exo{(Question de cours)} Soit $E$ un espace vectoriel et $\|\cdot\|, \|\cdot\|': E \to [0,+\infty[$  deux normes équivalentes sur $E$. Montrer que toute suite $(u_{k})_{k\in\N}$  de $E$ qui converge pour la norme $\|\cdot\|$ converge aussi pour la norme $\|\cdot\|'$.

\bigskip 
On utilise les hypothèses: 
	\begin{enumerate}
		\item 	Les deux normes sont équivalentes, ainsi il existe $\alpha, \beta >0$ tels que $ \alpha\snorm{ \cdot }  \leq \snorm{\cdot}' \leq \beta \snorm{\cdot}$.

		\item 	La suite $u_k \to \ell \in E$ quand $k\to +\infty$ pour la nome $\snorm{\cdot}$, ainsi 
			\[
				0 \leq \snorm{u_k - \ell} \to 0, \quad \text{quand $k\to + \infty$}
			\]
	\end{enumerate}

En utilisant 1) et 2) on a 
\[
	0 \leq \snorm{u_k - \ell}' \leq \beta \snorm{u_k - \ell}
\]
Par le théorème des gendarmes, la suite réelle $(\snorm{u_k - \ell}')_k$  converge et sa limite est $\lim_k \beta \snorm{u_k - \ell} = 0$. Autrement dit, $(u_k)_k$ converge vers $\ell\in E$ pour $\snorm{\cdot}'$.

\exo{(Projection de ${\Bbb R}^2$ sur $\Bbb R$)} Dans cet exercice, il est fortement recommand\'e de dessiner, et de se souvenir qu'un dessin n'est jamais une preuve.  Si $X$ est un espace vectoriel normé, «$B(x,r[$» désigne la boule ouverte de centre $x\in X$ et de rayon $r>0$. On rappelle que pour une application $p:X\rightarrow Y$, et pour $A\subset X$, l'image de $A$ par $p$ est l'ensemble $p(A)=\{ y\in  Y, \exists x\in A, y=p(x)\}=\{ p(x), x \in A\}$. 

Soit $p:{\Bbb R}^2\rightarrow {\Bbb R}$ la premi\`ere projection ($p(x,y)=x$). On munit ${\Bbb R}^2$ de la norme euclidienne et $\Bbb R$ de la valeur absolue. %Dans cet exercice,
\begin{enumerate}
\item Montrer que dans $\Bbb R$, $B(a,\epsilon[=]a-\epsilon,a+\epsilon[$.  


			\bigskip		
On a $B(a,\epsilon[=\{x\in {\Bbb R}, d_{\Bbb R}(a,x)<\epsilon\}=\{x\in {\Bbb R}, |x-a|<\epsilon\}=\{x\in {\Bbb R}, -\epsilon<x-a<\epsilon\}=\{x\in {\Bbb R},a- \epsilon<x<a+\epsilon\}=]a-\epsilon,a+\epsilon[$.
	
			

			\bigskip			
			
\item Montrer que $p(B((x,y),\epsilon[)=B(x,\epsilon[$. En d\'eduire que l'image d'un ouvert de ${\Bbb R}^2$ par $p$ est un ouvert de $\Bbb R$.
	
			\bigskip		
			
						
			On remarque que pour $(x,y),(a,b)\in {\Bbb R}²$, $|x-a|\le \sqrt{(x-a)^2+(y-b)^2}= d_{\Bbb R^2}((x,y),(a,b))$. Pour $a=p(a,b)\in p(B((x,y),\epsilon[)$, on a donc $d_{\Bbb R}(x,a)\le d_{\Bbb R^2} ((x,y),(a,b))<\epsilon$ et donc $a\in B(x,\epsilon[$.

D'autre part, pour $a\in B(x,\epsilon[$, on a $d_{\Bbb R^2 }((x,y),(a,y))=\sqrt{(x-a)^2}=|x-a|<\epsilon$ donc $(a,y) \in B((x,y),\epsilon[$ et $p(a,y)=a$ donc $a\in p(B((x,y),\epsilon[)$.

Soit $U$ un ouvert de ${\Bbb R}^2$ et $x\in p(U)$, il existe $(x,y)\in U$ tel que $p(x,y)=x$. Comme $U$ est ouvert, il existe $\epsilon>0$ tel que $B((x,y),\epsilon[\subset U$. On a alors $p(B((x,y),\epsilon[)=B(x,\epsilon[\subset p(U)$.

			
			
			\bigskip
\item Montrer que $F=\{(x,y), xy=1\}$ est un ferm\'e de ${\Bbb R}^2$ mais que $p(F)$ n'est pas ferm\'e.
			
			\bigskip
				
Soit $X_n=(x_n,y_n)$ une suite de $F$ qui converge vers $X=(x,y)$, alors la suite $x_ny_n$ est constante \'egale \`a $1$, et tend vers $xy$, donc $xy=1$ et $X\in F$. L'ensemble $F$ est donc ferm\'e.

D'autre part, $0\notin p(F)$, car pour tout $y\in \Bbb R$, $(0,y)\notin F$. La suite $x_n=1/n=p(1/n,n)\in p(F)$ mais $\lim x_n=0\notin p(F)$, donc $p(F)$ n'est pas ferm\'e.

	
	
%\item On suppose maintenant que $F$ est une partie de ${\Bbb R}^2$ ferm\'ee et born\'e et on consid\`ere une suite $x_n\in p(F)$ telle que $x_n\rightarrow x$.
%\begin{enumerate} 
%\item Montrer qu'il existe une suite $y_n$ telle que $(x_n,y_n)\in F$.
%\item Montrer qu'il existe une sous suite $(x_{\phi(n)},y_{\phi(n)})$ convergente.
%\item En d\'eduire que $x\in p(F)$, puis que $p(F)$ est ferm\'e dans $\Bbb R$.
%\end{enumerate}
\end{enumerate}

		
		
		
		
		
		
\exo{(La deltoïde)}  Soit la courbe paramétrée $\Gamma$ définie par $\begin{cases}x(t)= 2\cos t + \cos 2t \\ y(t) = 2\sin t - \sin 2t \end{cases}$ pour $t\in[-\pi,\pi]$

	\begin{enumerate}
		\item \'Etudier la parité des fonctions $x(\cdot)$ et $y(\cdot)$. Quelle(s) symétrie(s) cela implique-t-il sur le support de la courbe $\Gamma$?

			\bigskip

On a $x(-t) = 2\cos(- t) + \cos (-2t) = 2\cos(t) + \cos (2t) = x(t)$ et la fonction $x$ est paire. De même, $y(-t) =  2\sin (-t) - \sin (-2t)= -2\sin t + \sin 2t = -y(t)$ et la fonction $y$ est impaire. Cela donne une symétrie axiale par rapport à l'axe des abscisses.   	\bigskip


		\item Calculer $\Gamma'$, $\Gamma''$ et $\Gamma'''$.
	\bigskip
On a 
\begin{align*}
		\Gamma'(t) & = 
	\begin{cases}
	x'(t) = -2\sin t - 2 \sin 2t \\
	y'(t) = 2\cos t - 2 \cos 2t
	\end{cases} \\
		\Gamma''(t) & = 
	\begin{cases}
	x''(t) = -2\cos t - 4 \cos 2t\\
	y''(t) = -2\sin t + 4 \sin 2t \\
	\end{cases}\\
		\Gamma'''(t) & = 
	\begin{cases}
x'''(t) = 	2\sin t + 8 \sin 2t \\
y'''(t) = 	-2\cos t +8 \cos 2t
	\end{cases}
\end{align*}



\item Soit $t \in [-\pi,\pi[$. Montrer que $\cos(t) - \cos(2t) = 0$ a trois solutions $t = 0$ et $t=2\pi/3$ et $t= -2\pi/3$. 

			\bigskip
Une solution consiste à se souvenir que $\cos 2t = 2\cos^2 t -1$. Ansi,
\begin{align*}
	\cos t - \cos 2t &= 0 \\
\Leftrightarrow \cos t - 2\cos^2 t +1 &= 0
\end{align*}
On cherchant les racines du polynôme de degré deux, $X \mapsto -2X^2+X+1$, on arrive à  $\cos t = -1/2$ ou $\cos t = 1$. Ce qui donne le résultat escompté (faire un dessin avec un cercle trigonométrique!).

			\bigskip
	\item Calculer la position des points stationnaires. Donner leur nature ainsi que le comportement local de la courbe en leur voisinage (faire un petit dessin à chaque fois).

\bigskip
		La question précédentes donne les temps en lesquels $x'$ s'annule. Reste à vérifier si $y'$ s'annule aussi en ces temps. C'est bien le cas, on a
		\[
			y'(0) = y'(2\pi/3) = y'(-2\pi/3) = 0
		\]
Il y a donc 3 points stationnaires en $t=-2\pi/3,0,2\pi/3$.


\'Etude des points stationnaires:
\begin{enumerate}
	\item $t=0$ on a $\Gamma(0) = \begin{psmallmatrix}
			3\\0
		\end{psmallmatrix}$, $\Gamma''(0) = \begin{psmallmatrix}
			-6\\0
		\end{psmallmatrix} $ et  $\begin{psmallmatrix}
			0\\6
		\end{psmallmatrix}$. 
		Cela donne le DL suivant,
	\[
		\Gamma(0 + h ) = \begin{psmallmatrix}
		 3 - 3h^2 + o(\abs{h}^3)	\\  h^3 + o(\abs{h}^3)
		\end{psmallmatrix}
	\]
		Avec les notations du cours on a $p=2$ et $q=3$. Ainsi, c'est un point de rebroussement de première espèce. 
		\begin{center}\begin{tikzpicture}[scale=1]
			
				\begin{scope}[rotate=0] 
				
				  \draw[->, thick, red] (0,0)--(-2,0) node[left] {${v}$}; 
			    \draw[->, thick, red] (0,0)--(0,2) node[above] {${w}$}; 
		      \draw [>->,>=latex,very thick, color=blue] (-1,-1) .. controls (-0.5,0) and (-0.2,0) .. (-0.05,0) .. controls (-0.1,0.05) and (-0.5,0) .. (-1,1);
	       \fill (0,0) circle (1pt);
       \end{scope}
       \end{tikzpicture}\end{center}
			\item $t=2\pi/3$ on a $\Gamma(2\pi/3) = \begin{psmallmatrix} -3/2 \\ 3\sqrt 3/2
		\end{psmallmatrix}$, $\Gamma''(0) = \begin{psmallmatrix}
					3\\-3\sqrt{3}
		\end{psmallmatrix} $ et  $\begin{psmallmatrix}
					-3\sqrt{3}\\-3
		\end{psmallmatrix}$. 
		Cela donne le DL suivant,
	\[
		\Gamma(0 + h ) =  \frac 1 2 \begin{psmallmatrix}
			- 3  + 3h^2 - \sqrt{3}h^3 o(\abs{h}^3)	\\  3\sqrt 3  -3 \sqrt 3 h^2  - h^3 + o(\abs{h}^3)
		\end{psmallmatrix}
	\]
		Avec les notations du cours on a $p=2$ et $q=3$. Ainsi, c'est un point de rebroussement de première espèce. 
		\begin{center}
			\begin{tikzpicture}[scale=1]
			
			
			  \draw[->, thick, red] (0,0)--(.75,-1.299038106) node[left] {${v}$}; 
		    \draw[->, thick, red] (0,0)--(-1.299038106, -.75) node[above] {${w}$}; 
		    \begin{scope}[rotate=120] 
		      \draw [>->,>=latex,very thick, color=blue] (-1,-1) .. controls (-0.5,0) and (-0.2,0) .. (-0.05,0) .. controls (-0.1,0.05) and (-0.5,0) .. (-1,1);
	      \end{scope}
       \fill (0,0) circle (1pt);
       \end{tikzpicture}\end{center}

	\item $t= -2\pi/3$. C'est l'image par la symétrie axiale du point traité en (b). On a donc:	$\Gamma(-2\pi/3) = \begin{psmallmatrix} -3/2 \\ -3\sqrt 3/2
		\end{psmallmatrix}$, $\Gamma''(0) = \begin{psmallmatrix}
					3\\3\sqrt{3}
		\end{psmallmatrix} $ et  $\begin{psmallmatrix}
					-3\sqrt{3}\\3
		\end{psmallmatrix}$. 
		Cela donne le DL suivant,
	\[
		\Gamma(0 + h ) =  \frac 1 2 \begin{psmallmatrix}
			- 3  + 3h^2 - \sqrt{3}h^3 o(\abs{h}^3)	\\  -3\sqrt 3  +3 \sqrt 3 h^2  + h^3 + o(\abs{h}^3)
		\end{psmallmatrix}
	\]
		Avec les notations du cours on a $p=2$ et $q=3$. Ainsi, c'est un point de rebroussement de première espèce. 
		\begin{center}
			\begin{tikzpicture}[scale=1,yscale  = -1]
		\draw[->, thick, red] (0,0)--(.75,-1.299038106) node[left] {${v}$}; 
		    \draw[->, thick, red] (0,0)--(1.299038106, .75) node[above] {${w}$}; 
		    \begin{scope}[rotate=120] 
		      \draw [<-<,>=latex,very thick, color=blue] (-1,-1) .. controls (-0.5,0) and (-0.2,0) .. (-0.05,0) .. controls (-0.1,0.05) and (-0.5,0) .. (-1,1);
	      \end{scope}
       \fill (0,0) circle (1pt);
       \end{tikzpicture}\end{center}
\end{enumerate}





	%	\item Y a t il des point de la courbe avec une tangente verticale ? horizontal ? de direction $(1,1)$ ?
			%\blanc{6cm}
	
		\item Calculer les tangentes aux points stationnaires et montrer qu'elles s'intersectent toutes en un même et unique point.

%	Il suffit de calculer le point d'intersection de deux des tangentes et de vérifier que la troisème passe bien par ce même point. 
\bigskip
			
Comme la tangente au point $\Gamma(0)$ est l'axe des abscisses, il suffit de calculer où les 2 autres tangentes coupent cet axe. Par exemple, la tangente au point $\Gamma(2\pi/3)$ est 
\[
	t \mapsto 3/2  \begin{psmallmatrix}
		-1 \\  \sqrt{3}
	\end{psmallmatrix} + 3 t \begin{psmallmatrix}
		1 \\ -\sqrt{3}
	\end{psmallmatrix}		
\]
Elle annule sa deuxième coordonnée en $t=1/2$ en passant par l'origine. Par symétrie, on vérifie immédiatement que la troisième tangente passe elle aussi par l'origine. 
\bigskip

		\item Compléter le tableau de variations suivant:
			\bigskip

			On utilise les propriétés de parité de $x$ et de $y$ !
			\begin{center}
				\begin{tabular}{|c|ccccccccc|}
					\hline    $t$       & $-\pi$ & \hspace{1.5cm}  & $-2\pi/3$ & \hspace{1.5cm}  &  0	& \hspace{1.5cm} & $2\pi/3$ & \hspace{1.5cm}   & $\pi$\\[0.3cm]\hline\hline
					signe de $x'(t)$     &     &  $-$ & 0 & $+$  & 0 & $-$ & 0 &  $+$           &	\\[0.4cm]\hline
					variation de $x(t)$ &  $-$1   & $\searrow$    & $-3/2$ & $\nearrow$ & 3 & $\searrow$  & $-3/2$ &     $\nearrow$      & $-1$	\\[0.9cm]\hline\hline
					signe de $y'(t)$   & 0 & $-$    & 0        &$+$ &0 & $+$& 0 & $-$&      0	\\[0.4cm]\hline
					variation de $y(t)$ &   0  &  $\searrow$        & $-3\sqrt{3}/2 $ &  $\nearrow$& 0 &  $\nearrow$&  $3\sqrt{3}/2 $&   $\searrow$  &	0\\[0.9cm]\hline
				\end{tabular}
			\end{center}
		\item  Sur le graphique suivant, tracer la courbe $\Gamma$ ainsi que les tangentes étudiées aux questions précédentes. {\it Indication: on commencera par tracer les tangentes aux points stationnaires (ces points apparaissent déjà sur le graphique).}
				\begin{center}

	\begin{tikzpicture}\pgfplotsset{compat=1.8}
			\begin{axis}[height=10cm,width=10cm,enlargelimits=true,grid=major,  axis lines=center, axis on top, xlabel={$x$}, ylabel={$x$}, zlabel={$y$}, axis equal,view={90}{0},,
				ymin=-3,ymax=3,zmin=-3,zmax=3,xmin=-3,xmax=3]
				\addplot3[grid=both,samples=500, very thick,red, domain = -pi:pi, samples y =0] ({x},{2*cos(deg(x))  +cos(deg(2*x)) },{2* sin(deg(x))  -sin(deg(2*x))  });
				\addplot3[grid=both,samples=500, thin,blue, domain = -pi:.6, samples y =0] ({x}, { -1 + 2*x },{ sqrt(3) *(1 -2*x)  });
				\addplot3[grid=both,samples=500, thin,blue, domain = -pi:.6, samples y =0] ({x}, { -1 + 2*x },{ -sqrt(3) *(1 -2*x)  });
				\addplot3[grid=both,samples=500, thin,blue, domain = -pi:.6, samples y =0] ({x}, { 0 },{ -x  });
			\end{axis}		\end{tikzpicture}
			\end{center}
			%\blanc{8cm}

		\item La courbe $(\Gamma,[-\pi,\pi[)$ est-elle paramétrée par l'abscisse curviligne ?  {\it Indication: c'est la justification qui sera notée.}
				\bigskip

			Non, car la norme de son vecteur dérivée n'est pas constante.  En effet, $\snorm{\Gamma'}^2 = 8 (1 - \cos(3t))= 16\sin^2(3t/2)$ pour tout $t \in ]-\pi,\pi]$.  
				\bigskip

		\item {\bf\sffamily (Question Bonus)} Montrer que  la longueur de $\Gamma$ est 16. {\it Indication: aucun point ne sera donné pour la définition de la longueur, c'est bel et bien le calcul qui sera évalué.}

			\[
				L = 4 \int_{-\pi}^\pi \abs{\sin(3t/2)} dt = 12 \int_0^{2\pi/3} \sin(3t/2) dt = 8 \int_0^{\pi} \sin(t) dt = 8 \left[ \cos(t) \right]_{t=0}^\pi = 16. 
			\]
	\end{enumerate}
 
\end{document}
