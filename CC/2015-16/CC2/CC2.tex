\documentclass[a4paper]{tp_um}
\makeatletter
%--------------------------------------------------------------------------------

\usepackage[french]{babel}
\usepackage{amsmath}
\usepackage{amsbsy}
\usepackage{amsfonts}
\usepackage{amssymb}
\usepackage{amscd}
\usepackage{amsthm}
\usepackage{mathtools}
\usepackage{eurosym}
\usepackage{nicefrac}

\usepackage{latexsym}
\usepackage[a4paper,hmargin=20mm,vmargin=25mm]{geometry}
\usepackage{dsfont}
\usepackage[utf8]{inputenc}
\usepackage[T1]{fontenc}
\usepackage{lmodern}

\usepackage{multicol}
\usepackage[inline]{enumitem}
\setlist{nosep}
\setlist[itemize,1]{,label=$-$}


\newenvironment{modenumerate}
  {\enumerate\setupmodenumerate}
  {\endenumerate}

\newif\ifmoditem
\newcommand{\setupmodenumerate}{%
  \global\moditemfalse
  \let\origmakelabel\makelabel
  \def\moditem##1{\global\moditemtrue\def\mesymbol{##1}\item}%
  \def\makelabel##1{%
    \origmakelabel{##1\ifmoditem\rlap{\mesymbol}\fi\enspace}%
    \global\moditemfalse}%
}


\usepackage{sectsty}
%\sectionfont{}
\allsectionsfont{\color{astral}\normalfont\sffamily\bfseries\normalsize}

\usepackage{graphicx}
\usepackage{tikz}
\usetikzlibrary{babel}
\usepackage{tikz,tkz-tab}

\usepackage[babel=true, kerning=true]{microtype}


\usepackage{pgfplots}
\usepgfplotslibrary{fillbetween}
\pgfplotsset{compat=newest}
\usepgfplotslibrary{external} 
\tikzexternalize[prefix=./output_latex/]
%\DeclareSymbolFont{RalphSmithFonts}{U}{rsfs}{m}{n}
%\DeclareSymbolFontAlphabet{\mathscr}{RalphSmithFonts}
%\def\mathcal#1{{\mathscr #1}}



\providecommand{\abs}[1]{\left|#1\right|}
\providecommand{\norm}[1]{\left\Vert#1\right\Vert}
\providecommand{\U}{\mathcal{U}}
\providecommand{\R}{\mathbb{R}}
\providecommand{\Cc}{\mathcal{C}}
\providecommand{\reg}[1]{\mathcal{C}^{#1}}
\providecommand{\1}{\mathds{1}}
\providecommand{\N}{\mathbb{N}}
\providecommand{\Z}{\mathbb{Z}}
\providecommand{\p}{\partial}
\providecommand{\one}{\mathds{1}}
\providecommand{\E}{\mathbb{E}}\providecommand{\V}{\mathbb{V}}
\renewcommand{\P}{\mathbb{P}}


%Operateur
\providecommand{\abs}[1]{\left\lvert#1\right\rvert}
\providecommand{\sabs}[1]{\lvert#1\rvert}
\providecommand{\babs}[1]{\bigg\lvert#1\bigg\rvert}
\providecommand{\norm}[1]{\left\lVert#1\right\rVert}
\providecommand{\bnorm}[1]{\bigg\lVert#1\bigg\rVert}
\providecommand{\snorm}[1]{\lVert#1\rVert}
\providecommand{\prs}[1]{\left\langle #1\right\rangle}
\providecommand{\sprs}[1]{\langle #1\rangle}
\providecommand{\bprs}[1]{\bigg\langle #1\bigg\rangle}

\DeclareMathOperator{\deet}{Det}
\DeclareMathOperator{\hess}{Hess}
\DeclareMathOperator{\jac}{Jac}


\newcommand\rst[2]{{#1}_{\restriction_{#2}}}



% generate breakable white space allowing students to write notes.

\usepackage[framemethod=tikz]{mdframed}

\mdfdefinestyle{response}{
	leftmargin=.01\textwidth,
	rightmargin=.01\textwidth,
	linewidth=1pt
	hidealllines=false,
	leftline=true,
	rightline=true,topline=true,bottomline=true,
	skipabove=0pt,
	%innertopmargin=-5pt,
	%innerbottommargin=2pt,
	linecolor=black,
	innerrightmargin=0pt,
	}



\newcommand*{\DivideLengths}[2]{%
  \strip@pt\dimexpr\number\numexpr\number\dimexpr#1\relax*65536/\number\dimexpr#2\relax\relax sp\relax
}

\providecommand{\rep}[1]{$ $ \newline \begin{mdframed}[style=response]  
	
	\vspace*{\DivideLengths{#1}{3cm}cm}
	\pagebreak[1]	
	\vspace*{\DivideLengths{#1}{3cm}cm}
	\pagebreak[1]		
	\vspace*{\DivideLengths{#1}{3cm}cm}   \end{mdframed}}

\providecommand{\blanc}[1]{$ $ \newline 
	
	\vspace*{\DivideLengths{#1}{3cm}cm}
	\pagebreak[1]	
	\vspace*{\DivideLengths{#1}{3cm}cm}
	\pagebreak[3]		
	\vspace*{\DivideLengths{#1}{3cm}cm}}

\usepackage{ifthen}

\newcommand{\eno}[1]{%
	\ifthenelse{\equal{\version}{eno}}{#1}{}%
}
\newcommand{\cor}[1]{%
        \ifthenelse{\equal{\version}{cor}}{
\medskip 

{\small \color{gray} #1}

\medskip 
}{}
}

%------------------------------------------------------------------------------
%\DeclareUnicodeCharacter{00A0}{~}
\makeatother


\ue{HLMA410}
%-----------------------------------------------------------------------------

\title{\large \sffamily\bfseries Contrôle continu 2}

\begin{document}

\maketitle
\textit{Durée 1h30. Les documents, la calculatrice, les téléphones portables, tablettes, ordinateurs ne sont pas autorisés. La qualité de la rédaction sera prise en compte.} 

\bigskip
\bigskip

\exo{(Question de cours)} Soit $E$ un espace vectoriel et $\|\cdot\|, \|\cdot\|': E \to [0,+\infty[$  deux normes équivalentes sur $E$. Montrer que toute suite $(u_{k})_{k\in\N}$  de $E$ qui converge pour la norme $\|\cdot\|$ converge aussi pour la norme $\|\cdot\|'$.
		\blanc{8cm}


\exo{(Projection de ${\Bbb R}^2$ sur $\Bbb R$)} Dans cet exercice, il est fortement recommand\'e de dessiner, et de se souvenir qu'un dessin n'est jamais une preuve.  Si $X$ est un espace vectoriel normé, «$B(x,r[$» désigne la boule ouverte de centre $x\in X$ et de rayon $r>0$. On rappelle que pour une application $p:X\rightarrow Y$, et pour $A\subset X$, l'image de $A$ par $p$ est l'ensemble $p(A)=\{ y\in  Y, \exists x\in A, y=p(x)\}=\{ p(x), x \in A\}$. 

Soit $p:{\Bbb R}^2\rightarrow {\Bbb R}$ la premi\`ere projection ($p(x,y)=x$). On munit ${\Bbb R}^2$ de la norme euclidienne et $\Bbb R$ de la valeur absolue. %Dans cet exercice,
\begin{enumerate}
\item Montrer que dans $\Bbb R$, $B(a,\epsilon[=]a-\epsilon,a+\epsilon[$.  
			\blanc{6cm}
\item Montrer que $p(B((x,y),\epsilon[)=B(x,\epsilon[$. En d\'eduire que l'image d'un ouvert de ${\Bbb R}^2$ par $p$ est un ouvert de $\Bbb R$.
			\blanc{10cm}
\item Montrer que $F=\{(x,y), xy=1\}$ est un ferm\'e de ${\Bbb R}^2$ mais que $p(F)$ n'est pas ferm\'e.
			\blanc{10cm}
%\item On suppose maintenant que $F$ est une partie de ${\Bbb R}^2$ ferm\'ee et born\'e et on consid\`ere une suite $x_n\in p(F)$ telle que $x_n\rightarrow x$.
%\begin{enumerate} 
%\item Montrer qu'il existe une suite $y_n$ telle que $(x_n,y_n)\in F$.
%\item Montrer qu'il existe une sous suite $(x_{\phi(n)},y_{\phi(n)})$ convergente.
%\item En d\'eduire que $x\in p(F)$, puis que $p(F)$ est ferm\'e dans $\Bbb R$.
%\end{enumerate}
\end{enumerate}

		
		
		
		
		
		
\exo{(La deltoïde)}  Soit la courbe paramétrée $\Gamma$ définie par $\begin{cases}x(t)= 2\cos t + \cos 2t \\ y(t) = 2\sin t - \sin 2t \end{cases}$ pour $t\in[-\pi,\pi]$

	\begin{enumerate}
		\item \'Etudier la parité des fonctions $x(\cdot)$ et $y(\cdot)$. Quelle(s) symétrie(s) cela implique-t-il sur le support de la courbe $\Gamma$?
			\blanc{5cm}


		\item Calculer $\Gamma'$, $\Gamma''$ et $\Gamma'''$.
			\blanc{8cm}

\item Soit $t \in [-\pi,\pi[$. Montrer que $\cos(t) - \cos(2t) = 0$ a trois solutions $t = 0$ et $t=2\pi/3$ et $t= -2\pi/3$. 
	\blanc{7cm}

	\item Calculer la position des points stationnaires. Donner leur nature ainsi que le comportement local de la courbe en leur voisinage (faire un petit dessin à chaque fois).
			\blanc{19cm}
	%	\item Y a t il des point de la courbe avec une tangente verticale ? horizontal ? de direction $(1,1)$ ?
			%\blanc{6cm}
	
		\item Calculer les tangentes aux points stationnaires et montrer qu'elles s'intersectent toutes en un même et unique point.

			\blanc{7cm}
		
		\item Compléter le tableau de variations suivant:
			\begin{center}
				\begin{tabular}{|c|ccccccccc|}
					\hline    $t$       & $-\pi$ & \hspace{1.5cm}  & $-2\pi/3$ & \hspace{1.5cm}  &  0	& \hspace{1.5cm} & $2\pi/3$ & \hspace{1.5cm}   & $\pi$\\[0.3cm]\hline\hline
					signe de $x'(t)$     &     &   & & & & & &            &	\\[0.4cm]\hline
					variation de $x(t)$ &     &     & & & & & &          &	\\[0.9cm]\hline\hline
					signe de $y'(t)$     &     &         & & & & & &      &	\\[0.4cm]\hline
					variation de $y(t)$ &     &          & & & & & &     &	\\[0.9cm]\hline
				\end{tabular}
			\end{center}
		\item  Sur le graphique suivant, tracer la courbe $\Gamma$ ainsi que les tangentes étudiées aux questions précédentes. {\it Indication: on commencera par tracer les tangentes aux points stationnaires (ces points apparaissent déjà sur le graphique).}
				\begin{center}
				\begin{tikzpicture}[scale=.5]
					\def\xone{-4}
					\def\xtwo{6}
					\def\yone{-8}
					\def\ytwo{8}

% grid
					\draw[step=2cm,help lines,gray!50] (\xone-.2,\yone-.2) grid (\xtwo+.2,\ytwo+.2);
					\draw[thin,->] (\xone-.3, 0) -- (\xtwo+.3, 0) node[right] {$x$};
					\draw[thin,->] (0, \yone-.3) -- (0, \ytwo+.3) node[above] {$y$};
					
					
\draw (6,0) circle (3pt);
%\draw[very thick,->,gray] (6,0) -- (4,0);
\draw (-3,5.1961524) circle (3pt);
%\draw[very thick,->,gray] (-3,5.1961524) -- (-2,3.4641016);
\draw (-3,-5.1961524) circle (3pt);
%\draw[very thick,->,gray] (-3,-5.1961524) -- (-2,-3.4641016);

%\draw[very thick,->,gray] (6,0) -- (4,0);

				\end{tikzpicture}
			\end{center}
			%\blanc{8cm}

		\item La courbe $(\Gamma,[-\pi,\pi[)$ est-elle paramétrée par l'abscisse curviligne ?  {\it Indication: c'est la justification qui sera notée.}

			\blanc{6cm}

		\item {\bf\sffamily (Question Bonus)} Montrer que  la longueur de $\Gamma$ est 16. {\it Indication: aucun point ne sera donné pour la définition de la longueur, c'est bel et bien le calcul qui sera évalué.}

			\blanc{5cm}
	\end{enumerate}
 
\end{document}
