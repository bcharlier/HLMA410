\documentclass{tp_um}

\makeatletter
%--------------------------------------------------------------------------------

\usepackage[french]{babel}
\usepackage{amsmath}
\usepackage{amsbsy}
\usepackage{amsfonts}
\usepackage{amssymb}
\usepackage{amscd}
\usepackage{amsthm}
\usepackage{mathtools}
\usepackage{eurosym}
\usepackage{nicefrac}

\usepackage{latexsym}
\usepackage[a4paper,hmargin=20mm,vmargin=25mm]{geometry}
\usepackage{dsfont}
\usepackage[utf8]{inputenc}
\usepackage[T1]{fontenc}
\usepackage{lmodern}

\usepackage{multicol}
\usepackage[inline]{enumitem}
\setlist{nosep}
\setlist[itemize,1]{,label=$-$}


\newenvironment{modenumerate}
  {\enumerate\setupmodenumerate}
  {\endenumerate}

\newif\ifmoditem
\newcommand{\setupmodenumerate}{%
  \global\moditemfalse
  \let\origmakelabel\makelabel
  \def\moditem##1{\global\moditemtrue\def\mesymbol{##1}\item}%
  \def\makelabel##1{%
    \origmakelabel{##1\ifmoditem\rlap{\mesymbol}\fi\enspace}%
    \global\moditemfalse}%
}


\usepackage{sectsty}
%\sectionfont{}
\allsectionsfont{\color{astral}\normalfont\sffamily\bfseries\normalsize}

\usepackage{graphicx}
\usepackage{tikz}
\usetikzlibrary{babel}
\usepackage{tikz,tkz-tab}

\usepackage[babel=true, kerning=true]{microtype}


\usepackage{pgfplots}
\usepgfplotslibrary{fillbetween}
\pgfplotsset{compat=newest}
\usepgfplotslibrary{external} 
\tikzexternalize[prefix=./output_latex/]
%\DeclareSymbolFont{RalphSmithFonts}{U}{rsfs}{m}{n}
%\DeclareSymbolFontAlphabet{\mathscr}{RalphSmithFonts}
%\def\mathcal#1{{\mathscr #1}}



\providecommand{\abs}[1]{\left|#1\right|}
\providecommand{\norm}[1]{\left\Vert#1\right\Vert}
\providecommand{\U}{\mathcal{U}}
\providecommand{\R}{\mathbb{R}}
\providecommand{\Cc}{\mathcal{C}}
\providecommand{\reg}[1]{\mathcal{C}^{#1}}
\providecommand{\1}{\mathds{1}}
\providecommand{\N}{\mathbb{N}}
\providecommand{\Z}{\mathbb{Z}}
\providecommand{\p}{\partial}
\providecommand{\one}{\mathds{1}}
\providecommand{\E}{\mathbb{E}}\providecommand{\V}{\mathbb{V}}
\renewcommand{\P}{\mathbb{P}}


%Operateur
\providecommand{\abs}[1]{\left\lvert#1\right\rvert}
\providecommand{\sabs}[1]{\lvert#1\rvert}
\providecommand{\babs}[1]{\bigg\lvert#1\bigg\rvert}
\providecommand{\norm}[1]{\left\lVert#1\right\rVert}
\providecommand{\bnorm}[1]{\bigg\lVert#1\bigg\rVert}
\providecommand{\snorm}[1]{\lVert#1\rVert}
\providecommand{\prs}[1]{\left\langle #1\right\rangle}
\providecommand{\sprs}[1]{\langle #1\rangle}
\providecommand{\bprs}[1]{\bigg\langle #1\bigg\rangle}

\DeclareMathOperator{\deet}{Det}
\DeclareMathOperator{\hess}{Hess}
\DeclareMathOperator{\jac}{Jac}


\newcommand\rst[2]{{#1}_{\restriction_{#2}}}



% generate breakable white space allowing students to write notes.

\usepackage[framemethod=tikz]{mdframed}

\mdfdefinestyle{response}{
	leftmargin=.01\textwidth,
	rightmargin=.01\textwidth,
	linewidth=1pt
	hidealllines=false,
	leftline=true,
	rightline=true,topline=true,bottomline=true,
	skipabove=0pt,
	%innertopmargin=-5pt,
	%innerbottommargin=2pt,
	linecolor=black,
	innerrightmargin=0pt,
	}



\newcommand*{\DivideLengths}[2]{%
  \strip@pt\dimexpr\number\numexpr\number\dimexpr#1\relax*65536/\number\dimexpr#2\relax\relax sp\relax
}

\providecommand{\rep}[1]{$ $ \newline \begin{mdframed}[style=response]  
	
	\vspace*{\DivideLengths{#1}{3cm}cm}
	\pagebreak[1]	
	\vspace*{\DivideLengths{#1}{3cm}cm}
	\pagebreak[1]		
	\vspace*{\DivideLengths{#1}{3cm}cm}   \end{mdframed}}

\providecommand{\blanc}[1]{$ $ \newline 
	
	\vspace*{\DivideLengths{#1}{3cm}cm}
	\pagebreak[1]	
	\vspace*{\DivideLengths{#1}{3cm}cm}
	\pagebreak[3]		
	\vspace*{\DivideLengths{#1}{3cm}cm}}

\usepackage{ifthen}

\newcommand{\eno}[1]{%
	\ifthenelse{\equal{\version}{eno}}{#1}{}%
}
\newcommand{\cor}[1]{%
        \ifthenelse{\equal{\version}{cor}}{
\medskip 

{\small \color{gray} #1}

\medskip 
}{}
}

%------------------------------------------------------------------------------
%\DeclareUnicodeCharacter{00A0}{~}
\makeatother



%-----------------------------------------------------------------------------

\title{\Large \sffamily\bfseries Examen - Session 1}
\ue{HLMA410}


%-----------------------------------------------------------------------------
\begin{document}

\maketitle

\bigskip

\textit{Durée 3h00. Les documents, la calculatrice, les téléphones portables, tablettes, ordinateurs ne sont pas autorisés. La qualité de la rédaction sera prise en compte. Les exercices sont indépendants.} 

\bigskip
\bigskip

\section{Analyse}

\exo{} Soit $(E,\langle\cdot,\cdot\rangle)$ un espace euclidien. Montrer que pour tout $u,v\in E$ 
\[
	\abs{\prs{u,v}} \leq \sqrt{\prs{u,u} \prs{v,v}}.
\]

\exo{}
On considère l'application 
\begin{align*}
	f : \R^3 & \to \R^2 \\
(x,y,z) & \mapsto (xye^z,\cos(yz))
\end{align*}
\begin{enumerate}
	\item Justifier que $f$ est de classe $\mathcal C^1$ sur $\R^3$.
	\item Calculer la différentielle $L$ de $f$ au point $(1,2,3)$.
	\item Détermniner l'ensemble $\{ (x,y,z)\in \R^3 | L(x,y,z) = (0,0) \}$. 
\end{enumerate}

\exo{} On considère le domaine 
\[
	D = \left\{ (x,y) \in\R^2 | x <7, 8-x <y<x+1 \right\}
\]
\begin{enumerate}
	\item Dessiner $D$.
	\item Calculer $\iint_D \frac{dxdy}{(x+y)^2}$
\end{enumerate}


\exo{} Pour $(x,y) \in \R^2\setminus\left\{ (0,0) \right\}$, on note 
\[
	f(x,y) = \frac{xy^2}{x^2 + y^2}
\]
\begin{enumerate}
	\item Montrer que $f$ se prolonge en une fonction $\check f$ continue sur $\R^2$.
	\item En quels points de $\R^2$ la fonction $\check f$ est-elle différentiable ?
\end{enumerate}

\exo{} Soit la fonction \[f(x,y) = \exp \left ( -4x^2 - y^2 + 8x +4y -8 \right)\]
définie sur $\R^2$.% 9*(x-2)^2 + (y+2)^2/4
\begin{enumerate}
	\item \'Etudier la régularité $f$ (continuité, différentiabilité, etc\ldots).
	\item Déterminer le signe de  $(x,y)  \mapsto   -4x^2 - y^2 + 8x +4y -8$. En déduire l'ensemble image de $f$.
	\item Soit  \[E_\lambda =  \left\{ (x,y) \in\R^2 | f(x,y) \geq \lambda \right\}.\] Dessiner $E_{e^{-1}}$, $E_{e^{-2}}$ et  $E_{e^{-3}}$. L'ensemble $E_{e^{-1}} \cup E_{e^{-2}} \cup E_{e^{-3}}$ est-il ouvert, fermé, les deux, ni l'un ni l'autre? Faire une démonstration.
	%\item Donner les points de minimums de $f$ sur $\R^2$. Indiquer aussi la nature de ces points (minimum global ou local).
	\item Calculer le gradient de $f$ et le représenter succinctement sur la figure de la question 3.
	\item Calculer la hessienne de $f$.
	\item Déterminer le(s) point(s) critique(s) de $f$ et donner leur nature (minimum, maximum, point selle,\ldots).
	\item On pose $D_1=\left\{ (X,Y)\in\R^2 | X^2 + Y^2 < 1 \right\}$ et on donne $\iint_{D_1} \exp(-X^2 -Y^2 )dXdY  = \pi(1 - e^{-1})$.
	%\item On rappelle que $\iint_{\R^2} \exp(-X^2 -Y^2 )dXdY  = \pi$.
		%\begin{enumerate}
			%\item Déterminer les réels $a_1,a_2,c_1,c_2$ tels que $f(x , y) =  \exp (-a_1(x-c_1)^2 - a_2 (y-c_2)^2 )$. % Appliquer l'algorithme de la décomposition de Gauss et en déduire le signe de $q$
En déduire de  la valeur de 
\[
\iint_{D_2} f(x,y) dxdy\] où $D_2 = \{ (x,y)\in\R^2 | 4(x-1)^2 +  (y-2)^2  < 1 \}$.
		%\end{enumerate}
\end{enumerate}

%\exo{} On considère l'arc paramétré 
%\begin{align*}
	%\phi:[0,1] &\to \R^2
	%t & \mapsto \left( \frac{t}{1+t^4} , \frac{t^3}{1+t^4} \right)
%\end{align*}
%et le champs de vecteurs 
%\begin{align*}
	%\omega:\R^2 &\to \R^2
	%(x,y) &\mapsto (-y, x)
%\end{align*}
%\begin{enumerate}
	%\item Calculer la circulation de $\omega$ le long de $\phi$
	%\item Calculer la circulation de $\omega$ le long du segment $s$ joignant $\phi(0)$ à $\phi(1)$.
	%%\item Montrer que le support de la courbe $\phi$ est incluse dans le domaine $D = \{(x,y)\in\R^2 | y\leq x\}$.
	%%\item Calculer l'aire du domaine délimité par l'arc $\phi$ et le segment $s$ (on admettra que le bord de ce domaine est l'union des support de $\phi$ et de $s$)
%\end{enumerate}





\section{Probabilités}

\exo{}Soit $X$ une variable aléatoire réelle telle que $\E(X^2) < +\infty$. Montrer que 
\[
	\E( (X-\E(X))^2) = \E(X^2) - (\E(X))^2.
\]

\exo{}Sachant que l'on a obtenu 12 fois «face» en 20 lancers d'une pièce équilibrée, calculer la probabilité que:
\begin{enumerate}
	\item le premier lancer ait amené «face»
	\item au moins deux des cinq premiers lancers aient amené «face» (on ne demande pas de calculer la valeur numérique approchée, c'est le raisonnement qui sera évalué).
\end{enumerate}%Dossier 3 MArin

%\exo{} Soit $X$ et $Y$ deux variables aléatoires indépendantes de lois géométriques de paramètre $a$ et $b$ respectivement. On note $Z = \min \left\{ X,Y \right\}$. Le but de l'exercice est de déterminer la loi de $Z$.
%\begin{enumerate}
	%\item Calculer $\P\left( X \leq k \right)$ et $\P\left( Y\leq k\right)$ pour tout $k\in \N^*$.
	%\item Calculer alors $\P\left( Z\leq k \right)$ pour tout $k \in \N^*$
	%\item Soit $k\in\N^*$ fixé. Exprimer l'évènement $\left\{ Z = k \right\}$ en fonction des évènements du type $\left\{ Z\leq \ell \right\}$, $\ell\in \N^*$.
	%\item En déduire la loi de $Z$.
%\end{enumerate}
\end{document}







